\documentclass[10pt,onecolumn,journal,draftclsnofoot]{IEEEtran}
\usepackage[margin=0.75in]{geometry}
\usepackage{listings}
\usepackage{color}
\usepackage{longtable}
\usepackage{graphicx}
\usepackage{float}
\usepackage{tabu}
\usepackage{enumitem}
\usepackage{courier}
\usepackage{hyperref}
\usepackage{parskip}
\definecolor{dkgreen}{rgb}{0,0.6,0}
\definecolor{gray}{rgb}{0.5,0.5,0.5}
\definecolor{mauve}{rgb}{0.58,0,0.82}

\graphicspath{{../images/}}

%Subsection headers to Arabic numerals
%\renewcommand\thesection{\arabic{section}}
%\renewcommand\thesubsection{\thesection.\arabic{subsection}}
%\renewcommand\thesubsubsection{\thesubsection.\arabic{subsubsection}}

%Section headers to Arabic numerals
%\renewcommand\thesectiondis{\arabic{section}}
%\renewcommand\thesubsectiondis{\thesectiondis.\arabic{subsection}}
%\renewcommand\thesubsubsectiondis{\thesubsectiondis.\arabic{subsubsection}}

%Remove numbering from the bibliography section

\lstset{frame=none,
language=C,
columns=flexible,
numberstyle=\tiny\color{gray},
keywordstyle=\color{blue},
commentstyle=\color{dkgreen},
stringstyle=\color{mauve},
breaklines=true,
breakatwhitespace=true,
tabsize=4,
showstringspaces=false,
basicstyle=\ttfamily
}

\setlength{\parindent}{0cm}

\begin{document}

\begin{titlepage}
	\title{Intel Cloud Orchestration Networking\\ Spring Midterm Progress Report}
	\author{Matthew~Johnson,~Cody~Malick,~and~Garrett~Smith\\
		Team 51, Cloud Orchestra}
	\date{\today}
	\markboth{Senior Design, CS 463, Spring 2017}{}
	\maketitle
	\vspace{4cm}
	\begin{abstract}
		\noindent This document outlines the progress of the Cloud
		Orchestration Networking project for Spring 2017. It contains
		a short description of the project's purposes
		and goals, current progress, code samples, current issues,
		and any solutions to those issues. \end{abstract}

\end{titlepage}
\tableofcontents
\clearpage

\section{Project Goals}

Our project is to first switch the Linux-created GRE tunnel implementation in
Ciao to use GRE tunnels created by Open vSwitch. From that point we will switch
the actual tunneling implementation from GRE to VxLAN/nvGRE based on performance
measurements of each on data center networking cards. After this is completed, a
stretch goal is to replace Linux bridges with Open vSwitch switch instances.

These goals changed somewhat by the middle of the Winter term. The primary goal
now is to replace the Linux bridges with Open vSwitch switch instances because
of an assumption that was found to be incorrect. It was discovered that a full
implementation of Open vSwitch was required. Initially, we had planned on using
a third party API, \texttt{libovsdb} to interface with the Open vSwitch management
database.\cite{libovsdb} While providing the necessary functionality, it added undocumented
overhead. Specifically, all bridges and tunnels generated by Ciao had to be known
about in the calling library. After extensive research and discussion with our
client, we aimed to fully implement Open vSwitch into Ciao, rather than use it
to exclusively create tunnels. 

\section{Purpose}

The current implementation of Ciao tightly integrates software defined
networking principles to leverage a limited local awareness of just enough of
the global cloud's state. Tenant overlay networks are used to overcome
traditional hardware networking challenges by using a distributed, stateless,
self-configuring network topology running over dedicated network software
appliances. This design is achieved using Linux-native Global Routing
Encapsulation (GRE) tunnels and Linux bridges, and scales well in an environment
of a few hundred nodes.

While this initial network implementation in Ciao satisfies current simple
networking needs, all innovation around software defined networks has
shifted to the Open vSwitch (OVS) framework. Moving Ciao to OVS will allow
leverage of packet acceleration frameworks like the Data Plane Development Kit
(DPDK) as well as provide support for multiple tunneling protocols such as VxLAN
and nvGRE. VxLAN and nvGRE are equal cost multipath routing (ECMP) friendly,
which could increase network performance overall.
\section{Progress}

\section{Issues}

\section{Remaining Steps}

\bibliographystyle{IEEEtran}
\bibliography{prog}

\end{document}
