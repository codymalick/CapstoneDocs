\documentclass[10pt,onecolumn,journal,draftclsnofoot]{IEEEtran}
\usepackage[margin=0.75in]{geometry}
\usepackage{listings}
\usepackage{color}
\usepackage{longtable}
\usepackage{graphicx}
\usepackage{float}
\usepackage{tabu}
\usepackage{enumitem}
\usepackage{courier}
\usepackage{hyperref}
\usepackage{parskip}
\definecolor{dkgreen}{rgb}{0,0.6,0}
\definecolor{gray}{rgb}{0.5,0.5,0.5}
\definecolor{mauve}{rgb}{0.58,0,0.82}

%Subsection headers to Arabic numerals
%\renewcommand\thesection{\arabic{section}}
%\renewcommand\thesubsection{\thesection.\arabic{subsection}}
%\renewcommand\thesubsubsection{\thesubsection.\arabic{subsubsection}}

%Section headers to Arabic numerals
%\renewcommand\thesectiondis{\arabic{section}}
%\renewcommand\thesubsectiondis{\thesectiondis.\arabic{subsection}}
%\renewcommand\thesubsubsectiondis{\thesubsectiondis.\arabic{subsubsection}}

%Remove numbering from the bibliography section
\usepackage{etoolbox}
\patchcmd{\thebibliography}{\section*}{\section}{}{}
\patchcmd{\thebibliography}{\addcontentsline{toc}{section}{\refname}}{}{}{}

\graphicspath{{../images/}}

\lstset{frame=tb,
language=C,
columns=flexible,
numberstyle=\tiny\color{gray},
keywordstyle=\color{blue},
commentstyle=\color{dkgreen},
stringstyle=\color{mauve},
breaklines=true,
breakatwhitespace=true,
tabsize=4,
showstringspaces=false,
basicstyle=\ttfamily
}

\setlength{\parindent}{0cm}

\begin{document}

\begin{titlepage}
	\title{Intel Cloud Orchestration Networking\\ Design Document}
	%\author{Matthew~Johnson,~Cody~Malick,~and~Garrett~Smith\\
	%	Team 51, Cloud Orchestra}
	\date{\today}
	\markboth{Senior Design, CS 461, Fall 2016}{}
	\maketitle
	\vspace{4cm}
	\begin{abstract}
		\noindent This document outlines the design considerations for
		the implementation of Open vSwitch and other networking
		technologies in the Cloud Integrated Advanced Orchestrator
		(Ciao), Intel Corporation's advanced cloud orchestration
		software. It describes the various techniques, structure, and
		technology choices that will be used in the execution of our
		project.
	\end{abstract}
	\vspace{6cm}
\begin{center}
	Date of Issue: December 2nd, 2016\\
	Issuing Organization: Intel Corporation\\
	Authorship: Matthew~Johnson,~Cody~Malick,~and~Garrett~Smith\\
	Change History: First Draft, 12-02-2016 \\
\end{center}

\end{titlepage}
\tableofcontents
\clearpage

\section{Introduction}

Our project is to first switch the Linux-created GRE tunnel implementation in
Ciao to use GRE tunnels created by Open vSwitch. From that point we will switch
the actual tunneling implementation from GRE to VxLAN/nvGRE based on performance
measurements of each on data center networking cards. After this is completed, a
stretch goal is to replace Linux bridges with Open vSwitch switch instances.
This document outlines the steps, techniques, and methodology we will utilize to
achieve each goal.

\subsection{Purpose}

The current implementation of Ciao tightly integrates software defined
networking principles to leverage a limited local awareness of just enough of
the global cloud's state. Tenant overlay networks are used to overcome
traditional hardware networking challenges by using a distributed, stateless,
self-configuring network topology running over dedicated network software
appliances. This design is achieved using Linux-native Global Routing
Encapsulation (GRE) tunnels and Linux bridges and scales well in an environment
of a few hundred nodes.

While this initial network implementation in Ciao satisfies current simple
networking needs in Ciao, all innovation around software defined networks has
shifted to the Open vSwtich (OVS) framework. Moving Ciao to OVS will allow
leverage of packet acceleration frameworks like the Data Plane Development Kit
(DPDK) as well as provide support for multiple tunneling protocols such as VxLAN
and nvGRE. VxLAN and nvGRE are equal cost multipath routing (ECMP) friendly,
which could increase network performance overall.

\subsection{Scope}

Ciao exists as a cloud orchestrator for cloud clusters. It is inherently Ciao
exists as a cloud orchestrator for cloud clusters. It is inherently necessary
for the separate nodes in the cloud cluster to be able to talk to each other.
Without a reliable and secure software defined network Ciao would have little
purpose. Utilization of Open VSwitch GRE tunnels allows Ciao to become more
scalable and enables the inclusion of packet-acceleration technology such as the
Data Plane Development Kit (DPDK).

\subsection{Context}

Our network mode will exist within Ciao, a cloud orchestrator designed to be
fast and easy to deploy. Ciao is sectioned into three parts, each with
distinctive purposes~\cite{ciao}.

\begin{description}[leftmargin=12em,style=nextline]
	\item[Controller]
		Responsible for policy choices around tenant
		workloads~\cite{ciao}.
	\item[Scheduler]
		The Scheduler implements a "push/pull" scheduling algorithm. In
		response to a controller approved workload instance arriving at
		the scheduler, it finds a first fit among cluster compute nodes
		currently requesting work~\cite{ciao}.
	\item[Launcher]
		The Launcher abstracts the specific launching details for the
		different workload types (eg: virtual machine, container, bare
		metal). Launcher reports compute node statistics to the
		scheduler and controller. It also reports per-instance
		statistics up to controller ~\cite{ciao}.
\end{description}

Our networking mode must facilitate the communication of packets between all
three levels of Ciao, as well as individual compute and network
nodes and the Compute Node Concentrator (CNCI)~\cite{ciaoNetworking}.

\begin{description}[leftmargin=12em,style=nextline]
	\item[Compute Node]
		A compute node typically runs VM and Container workloads for
		multiple tenants~\cite{ciaoNetworking}.
	\item[Network Node]
		A Network Node is used to aggregate network traffic for all
		tenants while still keeping individual tenant traffic isolated
		from all other the tenants using special virtual machines called
		Compute Node Concentrators (CNCIs)~\cite{ciaoNetworking}.
	\item[Compute Node Concentrator (CNCI)]
		CNCIs are Virtual Machines automatically configured by the
		ciao-controller, scheduled by the ciao-scheduler on a need
		basis, when tenant workloads are created~\cite{ciaoNetworking}.
\end{description}

Specifically, the Ciao network components must communicate securely using the
Simple and Secure Node Transfer Protocol (SSNTP). The network node aggregates
traffic between compute nodes while keeping the tenant traffic isolated from
other tenants in the cluster. Network nodes achieve this with CNCIs. A graphic
of the lowest-level of this network configuration shows their relation to each
other.

\begin{figure}[H]
	\caption{Ciao Network Topology~\cite{ciaoNetTopology}}
	\begin{center}
		\makebox[\textwidth]{\includegraphics[width=\textwidth]{ciao-networking.eps}}
	\end{center}
\end{figure}

\subsection{Summary}

We will implement an Open vSwitch Generic Routing Encapsulation (OVS-GRE) mode
in Ciao in order to leverage DPDK and other software defined networking
technology innovations which are dependent on OVS. This document will outline
our design strategy, design views, and design viewpoints for each component of
our solution.

\bibliographystyle{IEEEtran}
\bibliography{design}

\section{Glossary}
\begin{description}[leftmargin=12em,style=nextline]
	\item[Bridge]
		Software or hardware that connects two or more network segments.
	\item[Ciao]
		Ciao is a cloud orchestrator that provides an easy to deploy,
		secure, scalable cloud orchestration system which handles
		virtual machines, containers, and bare metal apps agnostically
		as generic workloads. Implemented in the Go language, it
		separates logic into "controller", "scheduler" and "launcher"
		components which communicate over the "Simple and Secure Node
		Transfer Protocol (SSNTP)"~\cite{ciao}.
	\item[Cloud]
		A huge, amorphous network of servers somewhere~\cite{xkcd908}.
	\item[Cloud Orchestration]
		A networking tool designed to aid in the deployment of multiple
		virtual machines, containers, or bare-metal
		applications~\cite{ciao}.
	\item[Compute Node Concentrator (CNCI)]
		Virtual Machines automatically configured by the
		ciao-controller, scheduled by the ciao-scheduler on a need
		basis, when tenant workloads are created~\cite{ciaoNetworking}.
	\item[Data Plane Developement Kit (DPDK)]
		DPDK is a set of libraries and drivers for fast packet
		processing. It was designed to run on any processors. The first
		supported CPU was Intel x86 and it is now extended to IBM Power
		8, EZchip TILE-Gx and ARM. It runs mostly in Linux
		userland~\cite{dpdk}.
	\item[Equal Cost Multipath Routing (ECMP)]
		Equal cost multipath routing is a routing strategy in which next
		path routing for a packet can occur along one of several
		equal-cost paths to the destination~\cite{rfc2991}.
	\item[Generic Routing Encapsulation (GRE)]
		Encapsulation of an arbitrary network layer protocol so it can
		be sent over another arbitrary network layer
		protocol~\cite{rfc1701}.
	\item[Linux Bridge]
		Configurable software bridge built into the Linux
		kernel~\cite{linuxBridge}.
	\item[Network Node (NN)]
		A Network Node is used to aggregate network traffic for all
		tenants while still keeping individual tenant traffic isolated
		from all other the tenants using special virtual machines called
		Compute Node Concentrators (CNCIs)~\cite{ciaoNetworking}.
	\item[nvGRE]
		Network Virtualization using Generic Routing
		Encapsulation~\cite{rfc7637}.
	\item[Open vSwitch]
		Open source multilayer software switch with support for
		distribution across multiple physical devices~\cite{ovs}.
	\item[OVS]
		Open vSwitch~\cite{ovs}.
	\item[Packet Acceleration]
		Increasing the speed of the processing and transfer of network
		packets.
	\item[Packet Encapsulation]
		Attaching the headers for a network protocol to a packet so it
		can be transmitted using that
		protocol~\cite{networkingTextbook}.
	\item[SSNTP]
		The Simple and Secure Node Transfer Protocol (SSNTP) is a
		custom, fully asynchronous and TLS based application layer
		protocol. All Ciao components communicate with each others over
		SSNTP~\cite{ciaoSSNTP}.
	\item[Tunnel]
		Point to point network connection that encapsulates traffic
		between points~\cite{networkingTextbook}.
	\item[VxLAN]
		Virtual Extensible Local Area Network~\cite{rfc7348}.
\end{description}

\section{Body}
\subsection{Design Stakeholders}
The stakeholders are Intel, and the members of the Oregon State University
capstone group working on the project, Matthew Johnson, Cody Malick and Garrett
Smith.
\subsection{Design Concerns}
The design concerns are the operating systems that need to be supported,
the implementation of Open vSwitch created GRE tunnels,
the implementation of VxLAN tunnels, the implementation of nvGRE tunnels,
gathering performance metrics for the tunneling implementations, replacing
the Linux bridges used by Ciao with Open vSwitch switch instances, logging, and
testing. The capstone group members are stakeholders interested in all of the
design concerns because they will be implementing all of the concerns. Intel is
interested in all of the design concerns because they are the clients and they
will be using Ciao.

%Each viewpoint needs to be identified by name
\subsection{Context Design Viewpoint}
%High-level concerns - Go, Platform, Ciao integration
%Viewpoint Name
%Design Concerns that are the topics of the viewpoint
%Design Elements
%Analytical Methods
%Viewpoint Source
%Rationale for selection

\subsection{Context Design View}
%Conforms to viewpoint
%Has one or more concerns from the concerns section
%No conflicts between the design elements

\subsection{Interface Design Viewpoint}
%OVS-GRE tunneling implementation

\subsection{Interface Design View}

\subsection{Interaction Design Viewpoint}
% VxLAN/nvGRE tunneling protocol implementation

\subsection{Interaction Design View}

\subsection{Resource Design Viewpoint}
% Network performance testing

\subsection{Resource Design View}
%Repeat these sections as needed for each component

\subsection{Design Rationale}

% Don't know where this goes
\section{High-level considerations}
% I don't know where to put this yet

Our software defined network will be written in the Go programming language and
fully integrated in to the Cloud Integrated Advanced Orchestrator
(Ciao)~\cite{ciao}. The Go programming language was selected for
several reasons, including the efficiency of the language regarding both speed
and memory, the concurrency capabilities, and the ease of implementation. Go was
compared against C and Python as alternatives, and prevailed in every criteria
except for availability of the language.

This network mode will be written as a standalone networking mode for Ciao as an
additional option to the standard Linux bridges available now. For this reason
it must be fully integrated with the Ciao networking framework as it currently
exists~\cite{ciaoNetworking}.

\section{Summary}
We have outlined the steps and design strategy we will take for each goal. Our
design methodology is incremental design, starting with the first goal (Open
vSwitch-created GRE tunnels) and incrementing through each feature until all
goals are achieved.

\clearpage
\section{Signatures}
\vspace{2cm}
\begin{flushleft}
\noindent\hspace{0.7cm}\makebox[1.5in]{\hrulefill}~Robert Nesius, Engineering Manager\\
\vspace{1cm}
\hspace{0.7cm}\makebox[1.5in]{\hrulefill}~Matthew Johnson\\
\vspace{1cm}
\hspace{0.7cm}\makebox[1.5in]{\hrulefill}~Garrett Smith\\
\vspace{1cm}
\hspace{0.7cm}\makebox[1.5in]{\hrulefill}~Cody Malick
\end{flushleft}
\end{document}
