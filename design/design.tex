\documentclass[10pt,letterpaper,onecolumn,journal]{IEEEtran}
%draftclsnofoot
\usepackage[margin=0.75in]{geometry}
\usepackage{listings}
\usepackage{color}
\usepackage{longtable}
\usepackage{graphicx}
\usepackage{float}
\usepackage{tabu}
\usepackage{enumitem}
\usepackage{courier}
\usepackage{hyperref}
\definecolor{dkgreen}{rgb}{0,0.6,0}
\definecolor{gray}{rgb}{0.5,0.5,0.5}
\definecolor{mauve}{rgb}{0.58,0,0.82}

%Subsection headers to Arabic numerals
%\renewcommand\thesection{\arabic{section}}
%\renewcommand\thesubsection{\thesection.\arabic{subsection}}
%\renewcommand\thesubsubsection{\thesubsection.\arabic{subsubsection}}

%Section headers to Arabic numerals
%\renewcommand\thesectiondis{\arabic{section}}
%\renewcommand\thesubsectiondis{\thesectiondis.\arabic{subsection}}
%\renewcommand\thesubsubsectiondis{\thesubsectiondis.\arabic{subsubsection}}

%Remove numbering from the bibliography section
\usepackage{etoolbox}
\patchcmd{\thebibliography}{\section*}{\section}{}{}
\patchcmd{\thebibliography}{\addcontentsline{toc}{section}{\refname}}{}{}{}

\graphicspath{{../images/}}

\lstset{frame=tb,
language=C,
columns=flexible,
numberstyle=\tiny\color{gray},
keywordstyle=\color{blue},
commentstyle=\color{dkgreen},
stringstyle=\color{mauve},
breaklines=true,
breakatwhitespace=true,
tabsize=4,
showstringspaces=false,
basicstyle=\ttfamily
}

\setlength{\parindent}{0cm}

\begin{document}
\begin{titlepage}
	\title{CS 461 - Fall 2016 - Design Document}
	\author{Matthew Johnson, Cody Malick, Garrett Smith\\
		Team 51, Cloud Orchestra}
	\date{\today}
	\maketitle
	\vspace{4cm}
	\begin{abstract}
		\noindent This document outlines the design considerations for
		the implementation of Open vSwitch and other networking
		technologies in the Cloud Integrated Advanced Orchestrator
		(Ciao). It describes the various techniques, structure, and
		technology choices that will be used in the execution of our
		project.
	\end{abstract}

\end{titlepage}
\tableofcontents
\clearpage

\section{Introduction}

Our project is to first switch the Linux-created GRE tunnel implementation in
Ciao to use GRE tunnels created by Open vSwitch. From that point we will switch
the actual tunneling implementation from GRE to VxLAN/nvGRE based on performance
measurements of each on data center networking cards. After this is completed, a
stretch goal is to replace Linux bridges with Open vSwitch switch instances.
This document outlines the steps, techniques, and methodology we will utilize to
achieve each goal.
\subsection{Purpose}
\subsection{Scope}
\subsection{Context}
\subsection{Summary}

\bibliographystyle{IEEEtran}
\bibliography{design}

\section{Glossary}
\begin{description}[leftmargin=12em,style=nextline]
	\item[Bridge]
		Software or hardware that connects two or more network segments.
	\item[Ciao]
		Ciao is a cloud orchestrator that provides an easy to deploy,
		secure, scalable cloud orchestration system which handles
		virtual machines, containers, and bare metal apps agnostically
		as generic workloads. Implemented in the Go language, it
		separates logic into "controller", "scheduler" and "launcher"
		components which communicate over the "Simple and Secure Node
		Transfer Protocol (SSNTP)"~\cite{ciao}.
	\item[Cloud]
		A huge, amorphous network of servers somewhere~\cite{xkcd908}.
	\item[Cloud Orchestration]
		A networking tool designed to aid in the deployment of multiple
		virtual machines, containers, or bare-metal
		applications~\cite{ciao}.
	\item[CNCI]
		Virtual Machines automatically configured by the
		ciao-controller, scheduled by the ciao-scheduler on a need
		basis, when tenant workloads are created~\cite{ciaoNetworking}.
	\item[Generic Routing Encapsulation (GRE)]
		Encapsulation of an arbitrary network layer protocol so it can
		be sent over another arbitrary network layer
		protocol~\cite{rfc1701}.
	\item[Linux Bridge]
		Configurable software bridge built into the Linux
		kernel~\cite{linuxBridge}.
	\item[Network Node (NN)]
		A Network Node is used to aggregate network traffic for all
		tenants while still keeping individual tenant traffic isolated
		from all other the tenants using special virtual machines called
		Compute Node Concentrators (CNCIs)~\cite{ciaoNetworking}.
	\item[nvGRE]
		Network Virtualization using Generic Routing
		Encapsulation~\cite{rfc7637}.
	\item[Open vSwitch]
		Open source multilayer software switch with support for
		distribution across multiple physical devices~\cite{ovs}.
	\item[OVS]
		Open vSwitch~\cite{ovs}.
	\item[Packet Acceleration]
		Increasing the speed of the processing and transfer of network
		packets.
	\item[Packet Encapsulation]
		Attaching the headers for a network protocol to a packet so it
		can be transmitted using that
		protocol~\cite{networkingTextbook}.
	\item[SSNTP]
		The Simple and Secure Node Transfer Protocol (SSNTP) is a
		custom, fully asynchronous and TLS based application layer
		protocol. All Ciao components communicate with each others over
		SSNTP~\cite{ciaoSSNTP}.
	\item[Tunnel]
		Point to point network connection that encapsulates traffic
		between points~\cite{networkingTextbook}.
	\item[VxLAN]
		Virtual Extensible Local Area Network~\cite{rfc7348}.
\end{description}

\section{Body}
\subsection{Design Stakeholders}
The stakeholders are Intel, and the members of the Oregon State University 
capstone group working on the project, Matthew Johnson, Cody Malick and Garrett 
Smith.
\subsection{Design Concerns}
The design concerns are the operating systems that need to be supported, 
the implementation of Open vSwitch created GRE tunnels, 
the implementation of VxLAN tunnels, the implementation of nvGRE tunnels, 
gathering performance metrics for the tunneling implementations, replacing 
the Linux bridges used by Ciao with Open vSwitch switch instances, logging, and 
testing. The capstone group members are stakeholders interested in all of the 
design concerns because they will be implementing all of the concerns. Intel is
interested in all of the design concerns because they are the clients and they
will be using Ciao.

%Each viewpoint needs to be identified by name 
\subsection{Design Viewpoint 1}
%Viewpoint Name 
%Design Concerns that are the topics of the viewpoint
%Design Elements
%Analytical Methods
%Viewpoint Source
%Rationale for selection
\subsection{Design View 1}
%Conforms to viewpoint
%Has one or more concerns from the concerns section
%No conflicts between the design elements
\subsection{Design Viewpoints 2}
\subsection{Design View 2}
\subsection{Design Viewpoint for Network Performance Metrics}
\subsection{Design View for Network Performance Metrics
%Repeat these sections as needed for each component

\subsection{Design Rationale}

% Don't know where this goes
\section{High-level considerations}
% I don't know where to put this yet

Our software defined network will be written in the Go programming language and
fully integrated in to the Cloud Integrated Advanced Orchestrator
(Ciao)~\cite{ciao}. The Go programming language was selected for
several reasons, including the efficiency of the language regarding both speed
and memory, the concurrency capabilities, and the ease of implementation. Go was
compared against C and Python as alternatives, and prevailed in every criteria
except for availability of the language.

This network mode will be written as a standalone networking mode for Ciao as an
additional option to the standard Linux bridges available now. For this reason
it must be fully integrated with the Ciao networking framework as it currently
exists~\cite{ciaoNetworking}.

\section{Summary}
We have outlined the steps and design strategy we will take for each goal. Our
design methodology is incremental design, starting with the first goal (Open
vSwitch-created GRE tunnels) and incrementing through each feature until all
goals are achieved.

\end{document}
