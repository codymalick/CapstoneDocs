\documentclass[10pt,onecolumn,journal,draftclsnofoot]{IEEEtran}
\usepackage[margin=0.75in]{geometry}
\usepackage{listings}
\usepackage{color}
\usepackage{longtable}
\usepackage{graphicx}
\usepackage{float}
\usepackage{tabu}
\usepackage{enumitem}
\usepackage{courier}
\usepackage{hyperref}
\usepackage{parskip}
\definecolor{dkgreen}{rgb}{0,0.6,0}
\definecolor{gray}{rgb}{0.5,0.5,0.5}
\definecolor{mauve}{rgb}{0.58,0,0.82}

%Subsection headers to Arabic numerals
%\renewcommand\thesection{\arabic{section}}
%\renewcommand\thesubsection{\thesection.\arabic{subsection}}
%\renewcommand\thesubsubsection{\thesubsection.\arabic{subsubsection}}

%Section headers to Arabic numerals
%\renewcommand\thesectiondis{\arabic{section}}
%\renewcommand\thesubsectiondis{\thesectiondis.\arabic{subsection}}
%\renewcommand\thesubsubsectiondis{\thesubsectiondis.\arabic{subsubsection}}

%Remove numbering from the bibliography section
\usepackage{etoolbox}
\patchcmd{\thebibliography}{\section*}{\section}{}{}
\patchcmd{\thebibliography}{\addcontentsline{toc}{section}{\refname}}{}{}{}

\graphicspath{{../images/}}

\usepackage{booktabs}

\newcommand\addrow[2]{#1 &#2\\ }

\newcommand\addheading[2]{#1 &#2\\ \hline}
\newcommand\tabularhead{\begin{tabular}{lp{15cm}}
	\hline
	}
	\newcommand\addmulrow[2]{
		\begin{minipage}[t][][t]{2.5cm}#1\end{minipage}%
			&\begin{minipage}[t][][t]{15cm}
				\begin{enumerate} #2   \end{enumerate}
			\end{minipage}\\ }

	\newenvironment{usecase}{\tabularhead}
{\hline\end{tabular}}
%\newenvironment{usecase}
%  \addheading{}{}
%  \addrow{}{}
%  \addmulrow{}{}
%\end{usecase}

\lstset{frame=none,
language=C,
columns=flexible,
numberstyle=\tiny\color{gray},
keywordstyle=\color{blue},
commentstyle=\color{dkgreen},
stringstyle=\color{mauve},
breaklines=true,
breakatwhitespace=true,
tabsize=4,
showstringspaces=false,
basicstyle=\ttfamily
}

\setlength{\parindent}{0cm}

\begin{document}

\begin{titlepage}
	\title{Intel Cloud Orchestration Networking\\ Design Document}
	%\author{Matthew~Johnson,~Cody~Malick,~and~Garrett~Smith\\
	%	Team 51, Cloud Orchestra}
	\date{\today}
	\markboth{Senior Design, CS 461, Fall 2016}{}
	\maketitle
	\vspace{4cm}
	\begin{abstract}
		\noindent This document outlines the design considerations for
		the implementation of Open vSwitch and other networking
		technologies in the Cloud Integrated Advanced Orchestrator
		(Ciao), Intel Corporation's advanced cloud orchestration
		software. It describes the various techniques, structure, and
		technology choices that will be used in the execution of our
		project.
	\end{abstract}
	\vspace{6cm}
	\begin{center}
		Date of Issue: December 2nd, 2016\\
		Issuing Organization: Oregon State University\\
		Authorship: Matthew~Johnson,~Cody~Malick,~and~Garrett~Smith\\
		Change History: First Draft, 12-02-2016 \\
	\end{center}

\end{titlepage}
\tableofcontents
\clearpage

\section{Introduction}

Our project is to first switch the Linux-created GRE tunnel implementation in
Ciao to use GRE tunnels created by Open vSwitch. From that point we will switch
the actual tunneling implementation from GRE to VxLAN/nvGRE based on performance
measurements of each on data center networking cards. After this is completed, a
stretch goal is to replace Linux bridges with Open vSwitch switch instances.
This document outlines the steps, techniques, and methodology we will utilize to
achieve each goal.

\subsection{Purpose}

The current implementation of Ciao tightly integrates software defined
networking principles to leverage a limited local awareness of just enough of
the global cloud's state. Tenant overlay networks are used to overcome
traditional hardware networking challenges by using a distributed, stateless,
self-configuring network topology running over dedicated network software
appliances. This design is achieved using Linux-native Global Routing
Encapsulation (GRE) tunnels and Linux bridges, and scales well in an environment
of a few hundred nodes.

While this initial network implementation in Ciao satisfies current simple
networking needs in Ciao, all innovation around software defined networks has
shifted to the Open vSwtich (OVS) framework. Moving Ciao to OVS will allow
leverage of packet acceleration frameworks like the Data Plane Development Kit
(DPDK) as well as provide support for multiple tunneling protocols such as VxLAN
and nvGRE. VxLAN and nvGRE are equal cost multipath routing (ECMP) friendly,
which could increase network performance overall.

\subsection{Scope}

Ciao exists as a cloud orchestrator for cloud clusters. It is inherently necessary
for the separate nodes in the cloud cluster to be able to talk to each other.
Without a reliable and secure software defined network, Ciao would have little
purpose. Utilization of Open vSwitch GRE tunnels allows Ciao to become more
scalable and enables the inclusion of packet-acceleration technology such as the
Data Plane Development Kit (DPDK).

\subsection{Context}

Our networking mode will exist within Ciao, a cloud orchestrator designed to be
fast and easy to deploy. Ciao is sectioned into three parts, each with
distinctive purposes~\cite{ciao}.

\begin{description}[leftmargin=12em,style=nextline]
	\item[Controller]
		Responsible for policy choices around tenant
		workloads~\cite{ciao}.
	\item[Scheduler]
		The Scheduler implements a "push/pull" scheduling algorithm. In
		response to a controller approved workload instance arriving at
		the scheduler, it finds a first fit among cluster compute nodes
		currently requesting work~\cite{ciao}.
	\item[Launcher]
		The Launcher abstracts the specific launching details for the
		different workload types (eg: virtual machine, container, bare
		metal). Launcher reports compute node statistics to the
		scheduler and controller. It also reports per-instance
		statistics to the controller ~\cite{ciao}.
\end{description}

Our networking mode must facilitate the communication of packets between all
three levels of Ciao, as well as individual compute and network
nodes and the Compute Node Concentrator (CNCI)~\cite{ciaoNetworking}.

\begin{description}[leftmargin=12em,style=nextline]
	\item[Compute Node]
		A compute node typically runs VM and Container workloads for
		multiple tenants~\cite{ciaoNetworking}.
	\item[Network Node]
		A Network Node is used to aggregate network traffic for all
		tenants while still keeping individual tenant traffic isolated
		from all other the tenants using special virtual machines called
		Compute Node Concentrators (CNCIs)~\cite{ciaoNetworking}.
	\item[Compute Node Concentrator (CNCI)]
		CNCIs are Virtual Machines automatically configured by the
		ciao-controller, scheduled by the ciao-scheduler on a need
		basis, when tenant workloads are created~\cite{ciaoNetworking}.
\end{description}

Specifically, the Ciao network components must communicate securely using the
Simple and Secure Node Transfer Protocol (SSNTP). The network node aggregates
traffic between compute nodes while keeping the tenant traffic isolated from
other tenants in the cluster. Network nodes achieve this with CNCIs. A graphic
of the lowest-level of this network configuration shows their relation to each
other.

\begin{figure}[H]
	\caption{Ciao Network Topology~\cite{ciaoNetTopology}}
	\begin{center}
		\makebox[\textwidth]{\includegraphics[width=\textwidth]{ciao-networking.eps}}
	\end{center}
\end{figure}

\subsection{Summary}

We will implement an Open vSwitch Generic Routing Encapsulation (OVS-GRE) mode
in Ciao in order to leverage DPDK and other software defined networking
technology innovations which are dependent on OVS. This document will outline
our design strategy, design views, and design viewpoints for each component of
our solution.

\bibliographystyle{IEEEtran}
\bibliography{design}

\section{Glossary}
\begin{description}[leftmargin=12em,style=nextline]
	\item[Bridge]
		Software or hardware that connects two or more network segments.
	\item[Ciao]
		Ciao is a cloud orchestrator that provides an easy to deploy,
		secure, scalable cloud orchestration system which handles
		virtual machines, containers, and bare metal apps agnostically
		as generic workloads. Implemented in the Go language, it
		separates logic into "controller", "scheduler" and "launcher"
		components which communicate over the "Simple and Secure Node
		Transfer Protocol (SSNTP)"~\cite{ciao}.
	\item[Cloud]
		A huge, amorphous network of servers somewhere~\cite{xkcd908}.
	\item[Cloud Orchestration]
		A networking tool designed to aid in the deployment of multiple
		virtual machines, containers, or bare-metal
		applications~\cite{ciao}.
	\item[Compute Node Concentrator (CNCI)]
		Virtual Machines automatically configured by the
		ciao-controller, scheduled by the ciao-scheduler on a need
		basis, when tenant workloads are created~\cite{ciaoNetworking}.
	\item[Data Plane Developement Kit (DPDK)]
		DPDK is a set of libraries and drivers for fast packet
		processing. It was designed to run on any processors. The first
		supported CPU was Intel x86 and it is now extended to IBM Power
		8, EZchip TILE-Gx and ARM. It runs mostly in Linux
		userland~\cite{dpdk}.
	\item[Equal Cost Multipath Routing (ECMP)]
		Equal cost multipath routing is a routing strategy in which next
		path routing for a packet can occur along one of several
		equal-cost paths to the destination~\cite{rfc2991}.
	\item[Generic Routing Encapsulation (GRE)]
		Encapsulation of an arbitrary network layer protocol so it can
		be sent over another arbitrary network layer
		protocol~\cite{rfc1701}.
	\item[Linux Bridge]
		Configurable software bridge built into the Linux
		kernel~\cite{linuxBridge}.
	\item[Network Node (NN)]
		A Network Node is used to aggregate network traffic for all
		tenants while still keeping individual tenant traffic isolated
		from all other the tenants using special virtual machines called
		Compute Node Concentrators (CNCIs)~\cite{ciaoNetworking}.
	\item[nvGRE]
		Network Virtualization using Generic Routing
		Encapsulation~\cite{rfc7637}.
	\item[Open vSwitch]
		Open source multilayer software switch with support for
		distribution across multiple physical devices~\cite{ovs}.
	\item[OVS]
		Open vSwitch~\cite{ovs}.
	\item[Packet Acceleration]
		Increasing the speed of the processing and transfer of network
		packets.
	\item[Packet Encapsulation]
		Attaching the headers for a network protocol to a packet so it
		can be transmitted using that
		protocol~\cite{networkingTextbook}.
	\item[SSNTP]
		The Simple and Secure Node Transfer Protocol (SSNTP) is a
		custom, fully asynchronous and TLS based application layer
		protocol. All Ciao components communicate with each others over
		SSNTP~\cite{ciaoSSNTP}.
	\item[Tunnel]
		Point to point network connection that encapsulates traffic
		between points~\cite{networkingTextbook}.
	\item[VxLAN]
		Virtual Extensible Local Area Network~\cite{rfc7348}.
\end{description}

\section{Body}
\subsection{Design Stakeholders}
The stakeholders are Intel, and the members of the Oregon State University
capstone group working on the project, Matthew Johnson, Cody Malick and Garrett
Smith. Additional stakeholders are the existing and future users of
Ciao.

\subsection{Design Concerns}
The design concerns are the platforms that need to be supported,
the implementation of Open vSwitch created GRE tunnels,
the implementation of VxLAN tunnels, the implementation of nvGRE tunnels,
gathering performance metrics for the tunneling implementations, replacing
the Linux bridges used by Ciao with Open vSwitch switch instances, logging, and
testing. The capstone group members are stakeholders interested in all of the
design concerns because they will be implementing all of the concerns. Intel is
interested in all of the design concerns because they are the clients and they
will be using Ciao.

%Each viewpoint needs to be identified by name
\subsection{Context Design Viewpoint}
%High-level concerns - Go, Platform, Ciao integration
%Viewpoint Name
The context design viewpoint used here is from the IEEE 1016-2009
standard~\cite{ieee1016}.

\subsubsection{Context Design Concerns}

The design concerns for the context viewpoint are related to overall
compatibility both with the platform the software is running on as well as the
software the networking mode is running in. It must also be compatible with
existing networking protocols as used within Ciao already. Other major concerns
include security and performance of the software defined network.

There is one primary use case initiated by a compute node sending a data packet
to another compute node within the SDN.

\begin{usecase}
	\addheading{Actor}{Compute Node 1}
	\addrow{Precondition}{Compute Node 1 attempts to send data over network
	to Compute Node 2.}
	\addrow{Postcondition}{Compute Node 2 receives data sent by Compute Node
	1}
	\addmulrow{Main Path}{
	\item Compute Node 1 encapsulates packet.
	\item Compute Node 1 sends packet to Network Node.
	\item Network Node routes packet via relevant CNCI to Compute Node 2.
	\item Compute Node 2 receives packet and de-encapsulates.\\
	}
\end{usecase}

\subsubsection{Design Entities}

The users of the networking mode we are implementing are the current and future
users of Ciao, system administrators of small to medium enterprises running
private clouds.

The actors of the networking mode are the components of Ciao networking, the
compute nodes, network nodes, and CNCIs. These nodes handle the encapsulation,
routing, and de-encapsulation of packets.

\subsubsection{Design Relationships}

Compute nodes both send and receive packets to other compute nodes. Network
nodes aggregate this tenant network traffic while CNCIs within the network nodes
keep tenant traffic isolated from each other.

\subsubsection{Design Constraints}

Design of the solution must integrate fully with Ciao as well as the Linux
family of operating systems it runs on. This constraint demands specific
technology choices as described below in the Context Design View section.

\subsection{Context Design View}

Compatibility with the rest of the Ciao system is paramount and drives many
design decisions. The network implementation will be done in the Go 
programming language.

Because compatibility with the reset of the Ciao system is paramount, our
software defined network will be written in the Go programming language and
fully integrated in to Ciao. The Go programming language was selected for
several reasons, including the efficiency of the language regarding both speed
and memory, the concurrency capabilities, and the ease of implementation. Go was
compared against C and Python as alternatives, and prevailed in every criteria
except for availability of the language.

This network mode will be written as a standalone networking mode for Ciao as an
additional option to the standard Linux bridges available now. For this reason,
it must be fully integrated with the Ciao networking framework as it currently
exists~\cite{ciaoNetworking}.

Since Ciao targets the Linux family of operating systems, our networking
solution must also support Linux operating systems.

%Cody
\subsection{Interface Design Viewpoint}
%OVS-GRE tunneling implementation
One of the main project goals is switching the GRE tunneling implementation 
from standard GRE tunnels to Open vSwitch generated GRE tunnels. The
implementation of this object will be through designing and building an
interface for Open vSwitch to hook into Ciao. 

\subsubsection{Design Concerns}
Ciao will need to implement an interface for Open vSwitch in order to utilize
the new feature set that OVS supplies. In order to access the OVS Management
Database, we will need to interface with the Open vSwitch Database Management
Protocol\cite{rfc7047}. The following functions of the protocol will need to be
created:\cite{rfc7047}\\

\begin{center}
	\begin{tabular}{| p{3cm} | p{4cm} | p{6cm} |}
		\hline
		Function Name & Parameters & Description \\ \hline
		Insert & \textit{table}~:~required \newline
		\textit{row}~:~required \newline \textit{id}~:~optional &
		The operation inserts specified value into
		table at row. If no id is specified, then a new unique
		id is generated.\\ \hline
		Select & \textit{table}~:~required \newline \textit{where}~:~
		required \newline \textit{columns}~:~optional & Searches
		\textit{table} for rows that match all the conditions specified
		in \textit{where}. If \textit{columns} is not specified, all
		columns from the table are returned.\\ \hline
		Update & \textit{table}~:~required \newline \textit{where}~:~
		required \newline \textit{row}~:~required & Updates specified
		rows in a table. It searches \textit{table} for rows that match
		all conditions specified in \textit{where}.\\ \hline
		Delete & \textit{table}~:~required \newline \textit{where}~:~
		required & Operation deletes all the rows from \textit{table}
		that match all the conditions specified in \textit{where} \\ \hline
	\end{tabular}
\end{center}

\subsubsection{Design Elements}
The primary method of interaction with Open vSwitch is through the Configuration
Management Database, which provides a programmatic interface for updating and
changing OVS on the fly. This will be the primary interface for our
implementation.

\subsection{Interface Design View}
In order to interact with the interface using the Go programming language,
the implementation will use the libovsdb library\cite{libovsdb}. Libovsdb
provides a direct interface for Go to access and modify the OVS Database
configurations. The following is an example function call to call the above
functions:\cite{gosample}\\

\begin{lstlisting}[caption=Example insert operation in the OVS Database]
	// simple insert operation
	insertOp := libovsdb.Operation{
		Op:       "insert",
		Table:    "Bridge",
		Row:      bridge,
		UUIDName: namedUUID,
	}
\end{lstlisting}

Using the above code, slightly altered for each required operation, Go can
be used to insert, select, update, and delete using a standard format.
%Cody
\subsection{Interaction Design Viewpoint}
% VxLAN/nvGRE tunneling protocol implementation
An important part of implementation is choosing a tunneling protocol to
communicate between compute nodes. Currently, standard Generic Routing
Encapsulation is used to create a virtual network. These tunnels are created at
the Ethernet layer, and allow distributed systems to believe they are on the
same physical network as another computer in a different location.

\subsubsection{Design Concerns}
While GRE tunnels get the job done, a new and more modern protocol is needed to
support advanced virtualization features and scaling to large address spaces.

\subsection{Interaction Design View}
VxLAN is a relatively new virtualization standard developed by a few major
players in the network industry. Cisco, VMware, Citrix, and Redhat all worked
together to create this standard to resolve the major problem of massive
virtual networks. VxLAN's primary advantage is that it has a massive address
space, about sixteen million, and that its overall overhead increase is only
fifty bytes~\cite{vxlan}. Speed will be a deciding factor in which interface is
chosen, so a smaller overhead is good.

NvGRE is another relatively new virtualization standard developed in tandem by
Microsoft, HP, Intel, and Dell~\cite{nvgre-info}. NvGRE sports a few of the
same features as VxLAN, the primary difference between the two being the header
field. They use different UDP port numbers and use a different bit to indicate
that encapsulation has occurred~\cite{nvgre}. The primary difference between the
two is going to be overall performance in our testing environment, and in
production.

Part of the implementation is selecting a more advanced tunneling protocol than
standard GRE tunnels. The project will be implementing both nvGRE, and VxLAN,
and testing the performance of both protocols in order to determine the best
fit. Both protocols implement improved and more modern versions of tunneling
software. 

\subsection{Resource Design Viewpoint}
% Network performance testing
%Viewpoint Name
%Design Concerns that are the topics of the viewpoint
%Design Elements
%Analytical Methods
%Viewpoint Source
The resource design viewpoint used here is from the IEEE 1016-2009 
standard~\cite{ieee1016}.
We must provide explicit performance metrics for our SDN implementation.
These metrics will be what we use to determine how our SDN implementation
compares against the preexisting Ciao SDN implementation. 
The performance metrics must have numbers we can use to calculate actual and 
percentage differences in bandwidth and latency.

\subsubsection{Design Concerns}
The resource viewpoint was chosen because it covers the 
concerns of resource utilization and performance for the network and servers 
Ciao is running on.
The specific resources we are interested in when gathering performance metrics 
for are network bandwidth and latency, and server CPU usage.

\subsubsection{Design Elements}
To compare our changes to Ciao with the current implementation, and decide 
which tunneling protocols to use in our final implementation we need to gather 
performance metrics.
We will measure network bandwidth, latency, 
and CPU the usage of the hardware the SDN is running on before we make any 
changes, and using different tunneling protocols.

\subsection{Resource Design View}
%Repeat these sections as needed for each component
To improve network performance we will use either the VxLAN or nvGRE tunneling 
protocol.
We will gather performance metrics for the Ciao SDN when it is using the VxLAN 
and nvGRE tunneling protocols, and decide which protocol to use depending on 
which protocol performs better. Bandwidth and throughput are the most important 
performance metrics.
To measure the performance metrics for the tunneling protocols we will use 
the iperf3~\cite{iperf} and qperf~\cite{qperf} network tools.
We will use iperf3 to measure network bandwidth and 
CPU usage. We will use qperf to measure network latency.

\subsubsection{Design Elements}
\paragraph{iperf3}
The iperf3 tool will be used to gather network bandwidth measurements, and the
CPU usage of machines sending and receiving network traffic.
We can run iperf3 in a Docker container. 
There is a public Docker image we can use 
so we will not need to build a Docker image for it~\cite{iperfdocker}.
The iperf3 tool must be run in a client server configuration.
The docker image supports running iperf3 in either client or server mode.
We will measure network bandwidth between different tenants on the network 
by running the iperf3 docker container in server mode on one tenant, and in 
client mode on other tenants.
We can measure the CPU usage at the same time as the bandwidth because iperf3
will output both at the same time.

\begin{lstlisting}[caption = Running an iperf3 docker container in server mode]
docker run  -it --rm --name=iperf3-server -p 5201:5201 networkstatic/iperf3 -s
\end{lstlisting}

\begin{lstlisting}[caption = Running an iperf3 docker container in client mode]
docker run  -it --rm networkstatic/iperf3 -c 172.17.0.2
\end{lstlisting}

\paragraph{qperf}
The qperf tool will be used to gather network latency measurements.
We can run qperf in a Docker container. There is a public docker image we can 
use so we will not need to build our own qperf docker image 
%~\cite{qperfdocker}.
qperf must be run in a client server configuration. The docker image supports 
running in either client or server mode.
To measure TCP latency the \texttt{tcp\_lat} flag must be passed to the qperf 
client.
To measure UDP latency the \texttt{udp\_lat} flag must be passed to the qperf 
client.

\begin{lstlisting}[caption = Running a qperf docker container in server mode]
docker run -dti -p 4000:4000 -p 4001:4001 arjanschaaf/centos-qperf -lp 4000
\end{lstlisting}

\begin{lstlisting}[caption = Running a qperf docker container in client mode to 
measure TCP and UDP latency]
docker run -ti --rm arjanschaaf/centos-qperf 172.17.0.2 -lp 4000 -ip 4001 tcp_lat udp_lat
\end{lstlisting}

\subsection{Design Rationale}

Our design decisions were based largely around the goal of maintaining
compatibility with Ciao and the platforms that Ciao runs on, as well as
improving the capabilities and performance of Ciao's SDN. Implementation will be
done in Go in order to integrate with the larger system and because the Go
programming language is platform independent~\cite{gogoodfor}.

Open vSwitch was chosen because of the various features offered. OVS has support
for packet acceleration frameworks such as DPDK and advanced SDN tunneling
protocols such as nvGRE and VxLAN which would allow leverage of ECMP routing and
potentially increase network performance.

\clearpage
\section{Signatures}
\vspace{2cm}
\begin{flushleft}
	\noindent\hspace{0.7cm}\makebox[1.5in]{\hrulefill}~Robert Nesius, Engineering Manager\\
	\vspace{1cm}
	\hspace{0.7cm}\makebox[1.5in]{\hrulefill}~Matthew Johnson\\
	\vspace{1cm}
	\hspace{0.7cm}\makebox[1.5in]{\hrulefill}~Garrett Smith\\
	\vspace{1cm}
	\hspace{0.7cm}\makebox[1.5in]{\hrulefill}~Cody Malick
\end{flushleft}
\end{document}
