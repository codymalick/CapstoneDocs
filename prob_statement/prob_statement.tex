\documentclass[10pt,letterpaper,onecolumn,draftclsnofoot]{IEEEtran}
\usepackage[margin=0.75in]{geometry}
\usepackage{listings}
\usepackage{color}
\usepackage{longtable}
\usepackage{tabu}
\definecolor{dkgreen}{rgb}{0,0.6,0}
\definecolor{gray}{rgb}{0.5,0.5,0.5}
\definecolor{mauve}{rgb}{0.58,0,0.82}

\lstset{frame=tb,
  language=C,
  columns=flexible,
  numberstyle=\tiny\color{gray},
  keywordstyle=\color{blue},
  commentstyle=\color{dkgreen},
  stringstyle=\color{mauve},
  breaklines=true,
  breakatwhitespace=true,
  tabsize=4
}

\begin{document}
\begin{titlepage}
  \title{CS 461 - Fall 2016 - Problem Statement}
  \author{Matthew Johnson, Garrett Smith, Cody Malick\\Cloud Orchestration
  Networking}
  \date{October 10, 2016}
  \maketitle
  \vspace{4cm}
  \begin{abstract}
  	\noindent Intel's Cloud Integrated Advanced Orchestrator (Ciao) cloud
	orchestration platform currently uses Linux bridges and Generic Routing
	Encapsulation (GRE) tunnels to implement a software defined network
	(SDN). While the current implementation allows for scalability above
	typical designs, it lacks compatibility with modern tunneling protocols
	such as VxLAN and nvGRE. The proposed solution is to switch the
	implementation from Linux GRE tunnels to Open vSwitch (OVS) Tunnels and
	Linux bridges to Open vSwitch. This will allow Ciao to leverage packet
	acceleration frameworks such as the Data Plane Development Kit (DPDK)
	and provide needed compatibility.
  \end{abstract}
\end{titlepage}

\section*{Problem Definition}
Software Defined Network (SDN) implementations in practice today focus
exclusively on extremely large scale deployments where there are tens of
thousands of datacenter servers. They use a topology that fully connects all
servers with all servers, resulting in an extremely large and unwieldy mesh of
tunnels. Beyond the mesh complexity itself, Address Resolution Protocol (ARP)
table management, endpoint discovery, broadcast loop prevention and broadcast
traffic management are also challenging in this complex topology.

In contrast, Ciao tightly integrates SDN to achieve a simpler overall
implementation leveraging a limited local awareness of just enough of the global
cloud’s state. Tenant overlay networks are used to overcome the above listed
challenges in typical SDNs by using a distributed, stateless, self-configuring
network topology running over dedicated network software appliances. This design
yields a hierarchical SDN overlay without loops and meshes using Linux bridges
interconnected by Linux native GRE tunnels. This has been shown to scale
extremely well in an environment which consists of a few hundred nodes across a
few server racks, which also happens to be the sweet spot of scale when it comes
to most small and medium enterprises running private clouds today.

While this initial network implementation in Ciao uses Linux bridges and GRE
tunnels, all innovation around SDNs currently has shifted to a framework called
Open vSwitch (OVS). Moving from Linux bridges and GRE to OVS-GRE would allow
Ciao to leverage packet acceleration frameworks like DPDK as well as support
multiple tunneling protocols like VxLAN and nvGRE which are equal cost multipath
routing (ECMP) friendly and in some cases are accelerated directly in the
hardware network cards found on most servers.

\section*{Proposed Solution}
This proposed solution is to add an OVS-GRE mode in Ciao. The project will 
involve two phases with an optional third phase, each resulting in a fully 
functional SDN implementation:

\begin{enumerate}
	\item Switch the GRE tunnel implementation to use OVS created GRE 
		tunnels.
	\item Switch the tunneling implementation to VxLAN/nvGRE based on
		performance measurements of VxLAN and nvGRE on data center
		network cards.
	\item Optional: Replace the linux bridges with OVS switch instances.
\end{enumerate}
This solution will allow Ciao to leverage DPDK and other SDN technology
innovations which are dependent on OVS.

At the engineering expo, we expect to demonstrate our solution running on a
minimal cluster of four nodes. We will also show performance metrics from before
and after our implementation, illustrating improvements to the network speed and
scalability.

\section*{Performance Metrics}
The main metric is a functional SDN utilizing OVS, as well as a VxLAN/nvGRE
tunneling implementation selected based on performance measurements. A marked
improvement over current implementation performance is expected. Efficient team
communication is also expected to help overcome obstacles.

\clearpage
\section*{Approval}
\vspace{2cm}
\noindent Kent Helm, Engineering Manager\hspace{0.7cm} \makebox[1.5in]
{\hrulefill}\\\\\\
\noindent Robert Nesius, Engineering Manager\hspace{0.7cm} \makebox[1.5in]
{\hrulefill}\\\\\\

\noindent Matthew Johnson\hspace{0.3cm} \makebox[1.5in]{\hrulefill}\\\\\\
Garrett Smith\hspace{0.7cm} \makebox[1.5in]{\hrulefill}\\\\\\
Cody Malick\hspace{0.7cm} \makebox[1.5in]{\hrulefill}\\\\\\


\end{document}
