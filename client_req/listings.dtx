% \iffalse
%
% Trademarks appear throughout this documentation without any trademark
% symbol, so you can't assume that a name is free. There is no intention
% of infringement; the usage is to the benefit of the trademark owner.
%
%
%  S O F T W A R E   L I C E N S E
% =================================
%
% The files  listings.dtx  and  listings.ins  and all files generated
% from only these two files are referred to as `the listings package'
% or simply `the package'. lstdrvrs.dtx  and the files generated from
% that file are `drivers'.
%
% The listings package is copyright 1996--2004 Carsten Heinz, and
% continued maintenance on the package is copyright 2006--2007 Brooks
% Moses. From 2013 on copyright is Jobst Hoffmann, who is the maintainer
% since july 2013. The drivers are copyright 1997/1998/1999/2000/2001/
% 2002/2003/2004/2006/2007/2013 any individual author listed in the
% driver files.
%
% The listings package and its drivers may be distributed and/or modified
% under the conditions of the LaTeX Project Public License, either version
% 1.3 of this license or (at your option) any later version.
% The latest version of this license is in
%   http://www.latex-project.org/lppl.txt
% and version 1.3 or later is part of all distributions of LaTeX
% version 2003/12/01 or later.
%
% The package has the LPPL maintenance status "maintained".
%
% $Id: listings.dtx 201 2015-06-04 20:25:39Z j_hoffmann $
%
% The Current Maintainer is Jobst Hoffmann <j.hoffmann(at)fh-aachen.de>.
%
% end of software license
%
%
%<*driver>
\documentclass[a4paper]{ltxdoc}
\DisableCrossrefs
\OnlyDescription

\usepackage{lstdoc,textcomp}
\usepackage{mdframed}           % frames for external files
\usepackage{moreverb}           % writing external files
\usepackage{xcolor}             % because of colouring the background

\makeindex

\begin{document}
    \DocInput{listings.dtx}
\end{document}
%</driver>
% \fi
%
%^^A
%^^A  Command/key to aspect relation
%^^A ================================
%^^A
%\lstisaspect[strings]{string,morestring,deletestring,stringstyle,showstringspaces}
%\lstisaspect[comments]{comment,morecomment,deletecomment,commentstyle}
%\lstisaspect[comment styles]{b,d,l,n,s,ib,id,il,in,is}
%\lstisaspect[pod]{printpod,podcomment}
%\lstisaspect[escape]{texcl,escapebegin,escapeend,escapechar,escapeinside,mathescape}
%\lstisaspect[keywords]{sensitive,classoffset,keywords,morekeywords,deletekeywords,keywordstyle,ndkeywords,morendkeywords,deletendkeywords,ndkeywordstyle,keywordsprefix,otherkeywords}
%\lstisaspect[emph]{emph,moreemph,deleteemph,emphstyle}
%\lstisaspect[tex]{texcs,moretexcs,deletetexcs,texcsstyle}
%\lstisaspect[directives]{directives,moredirectives,deletedirectives,directivestyle}
%\lstisaspect[html]{tag,usekeywordsintag,tagstyle,markfirstintag}
%\lstisaspect[keywordcomments]{keywordcomment,morekeywordcomment,deletekeywordcomment,keywordcommentsemicolon}
%\lstisaspect[index]{index,moreindex,deleteindex,indexstyle,\string\lstindexmacro}
%\lstisaspect[procnames]{procnamestyle,indexprocnames,procnamekeys,moreprocnamekeys,deleteprocnamekeys}
%\lstisaspect[style]{style,\string\lstdefinestyle,\string\lst@definestyle,\string\lststylefiles}
%\lstisaspect[language]{language,alsolanguage,defaultdialect,\string\lstalias,\string\lstdefinelanguage,\string\lst@definelanguage,\string\lstloadlanguages,\string\lstlanguagefiles}
%\lstisaspect[formats]{format,fmtindent,\string\lstdefineformat,\string\lst@defineformat,\string\lstformatfiles}
%\lstisaspect[labels]{numbers,numberstyle,numbersep,stepnumber,numberblanklines,firstnumber,\string\thelstnumber,numberfirstline}
%\lstisaspect[lineshape]{xleftmargin,xrightmargin,resetmargins,linewidth,lineskip,breaklines,breakindent,breakautoindent,prebreak,postbreak,breakatwhitespace}
%\lstisaspect[frames]{framexleftmargin,framexrightmargin,framextopmargin,framexbottommargin,backgroundcolor,fillcolor,rulecolor,rulesepcolor,rulesep,framerule,framesep,frameshape,frameround,frame}
%\lstisaspect[make]{makemacrouse}
%\lstisaspect[fancyvrb]{fancyvrb,fvcmdparams,morefvcmdparams}
%\lstisaspect[lgrind]{lgrindef,\string\lstlgrindeffile}
%\lstisaspect[hyper]{hyperref,morehyperref,deletehyperref,hyperanchor,hyperlink}
%\lstisaspect[kernel]{basewidth,fontadjust,columns,flexiblecolumns,identifierstyle,^^A
%   tabsize,showtabs,tab,showspaces,keepspaces,formfeed,SelectCharTable,^^A
%   MoreSelectCharTable,extendedchars,alsoletter,alsodigit,alsoother,excludedelims,^^A
%   literate,basicstyle,print,firstline,lastline,linerange,nolol,captionpos,abovecaptionskip,^^A
%   belowcaptionskip,label,title,caption,\string\lstlistingname,boxpos,float,^^A
%   floatplacement,aboveskip,belowskip,everydisplay,showlines,emptylines,gobble,name,^^A
%   \string\lstname,\string\lstlistlistingname,\string\lstlistoflistings,^^A
%   \string\lstnewenvironment,\string\lstinline,\string\lstinputlisting,lstlisting,^^A
%   \string\lstloadaspects,\string\lstset,\string\thelstlisting,\string\lstaspectfiles,^^A
%   inputencoding,delim,moredelim,deletedelim,upquote,numberbychapter,^^A
%   \string\lstMakeShortInline,\string\lstDeleteShortInline}
%\lstisaspect[doc]{lstsample,lstxsample}^^A environment
%\lstisaspect[experimental]{includerangemarker,rangebeginprefix,rangebeginsuffix,rangeendprefix,rangeendsuffix,rangeprefix,rangesuffix}
%
%^^A
%^^A  The long awaited beginning of documentation
%^^A =============================================
%^^A
%\newbox\abstractbox
%\setbox\abstractbox=\vbox{
%	\begin{abstract}
%	The \packagename{listings} package is a source code printer for \LaTeX.
%	You can typeset stand alone files as well as listings with an environment
%   similar to \texttt{verbatim} as well as you can print code snippets using
%   a command similar to |\verb|.
%	Many parameters control the output and if your preferred programming
%   language isn't already supported, you can make your own definition.
%	\end{abstract}}
%
% \title{\vspace*{-2\baselineskip}The \textsf{Listings} Package}
% \author{Copyright 1996--2004, Carsten Heinz%
%    \\ Copyright 2006--2007, Brooks Moses
%    \\ Copyright 2013--, Jobst Hoffmann
%    \\ Maintainer: Jobst Hoffmann\thanks{Jobst %
%       Hoffmann became the maintainer of the \packagename{listings}
%       package in 2013; see the Preface for details.}~ %
%    \textless\lstemail\textgreater}
% \date{2015/06/04\enspace\enspace Version 1.6\ \box\abstractbox}
% \def\lstemail{\href{mailto:j.hoffmann@fh-aachen.de}{\texttt{j.hoffmann(at)fh-aachen.de}}}
% \ifhyper
%    \hypersetup{pdfsubject=Package guide,pdfauthor=Jobst Hoffmann <j.hoffmann(at)fh-aachen.de>}
% \fi
%
% \csname @twocolumntrue\endcsname
% \maketitle
%^^A \enlargethispage{2\baselineskip}
% \csname @starttoc\endcsname{toc}
% \onecolumn
%
%
% \section*{Preface}
%
% \paragraph{Transition of package maintenance}
% The \TeX\ world lost contact with Carsten Heinz in late 2004, shortly after
% he released version 1.3b of the \packagename{listings} package.  After many
% attempts to reach him had failed, Hendri Adriaens took over maintenance of
% the package in accordance with the LPPL's procedure for abandoned packages.
% He then passed the maintainership of the package to Brooks Moses, who had
% volunteered for the position while this procedure was going through. The
% result is known as listings version 1.4.
%
% This release, version 1.5, is a minor maintenance release since
% I accepted maintainership of the package.  I would like to thank Stephan
% Hennig who supported the Lua language definitions. He is the one who
% asked for the integration of a new language and gave the impetus to me to
% become the maintainer of this package.
%
%
% \paragraph{News and changes}
% Version 1.5 is the fifth bugfix release.  There are no changes
% in this version, but two extensions: support of modern Fortran (2003,
% 2008) and Lua.
%
%
% \vfill
% \paragraph{Thanks}
% There are many people I have to thank for fruitful communication, posting
% their ideas, giving error reports, adding programming languages to
% \texttt{lstdrvrs.dtx}, and so on. Their names are listed in section
% \ref{uClosingAndCredits}.
%
% \paragraph{Trademarks}
% Trademarks appear throughout this documentation without any trademark
% symbol; they are the property of their respective trademark owner.
% There is no intention of infringement; the usage is to the benefit of the
% trademark owner.
%
%
% \clearpage
%
%
% \part{User's guide}
%
%
% \section{Getting started}\label{uGettingStarted}
%
%
% \subsection{A minimal file}\label{uAMinimalFile}
%
% Before using the \packagename{listings} package, you should be familiar with
% the \LaTeX\ typesetting system. You need not to be an expert.
% Here is a minimal file for \packagename{listings}.
% \begin{verbatim}
%    \documentclass{article}
%    \usepackage{listings}
%
%    \begin{document}
%    \lstset{language=Pascal}
%
%      % Insert Pascal examples here.
%
%    \end{document}\end{verbatim}
% Now type in this first example and run it through \LaTeX.
% \begin{advise}
% \item Must I do that really?
%       \advisespace
%       Yes and no. Some books about programming say this is good.
%       What a mistake! Typing takes time---which is wasted if the code is clear to
%       you. And if you need that time to understand what is going on, the
%       author of the book should reconsider the concept of presenting the
%       crucial things---you might want to say that about this guide even---or
%       you're simply inexperienced with programming. If only the latter case
%       applies, you should spend more time on reading (good) books about
%       programming, (good) documentations, and (good) source code from other
%       people. Of course you should also make your own experiments.
%       You will learn a lot. However, running the example through \LaTeX\
%       shows whether the \packagename{listings} package is installed correctly.
% \item The example doesn't work.
%       \advisespace
%       Are the two packages \packagename{listings} and \packagename{keyval}
%       installed on your system? Consult the administration tool of your
%       \TeX\ distribution, your system administrator, the local \TeX\ and
%       \LaTeX\ guides, a \TeX\ FAQ, and section \ref{rInstallation}---in
%       that order. If you've checked \emph{all} these sources and are
%       still helpless, you might want to write a post to a \TeX\ newsgroup
%       like \texttt{comp.text.tex}.
% \item Should I read the software license before using the package?
%       \advisespace
%       Yes, but read this \emph{Getting started} section first to decide
%       whether you are willing to use the package.^^A ;-)
% \end{advise}
%
%
% \subsection{Typesetting listings}
%
% Three types of source codes are supported: code snippets, code segments, and
% listings of stand alone files.  Snippets are placed inside paragraphs and the
% others as separate paragraphs---the difference is the same as between text
% style and display style formulas.
% \begin{advise}
% \item No matter what kind of source you have, if a listing contains national
%       characters like \'e, \L, \"a, or whatever, you must tell the
%       package about it! Section \lstref{uSpecialCharacters} discusses this issue.
% \end{advise}
%
% \paragraph{Code snippets}
% The well-known \LaTeX\ command |\verb| typesets code snippets verbatim.
% The new command |\lstinline| pretty-prints the code, for example
%`\lstinline!var i:integer;!' is typeset by
%`{\rstyle|\lstinline|}|!var i:integer;!|'. The exclamation marks delimit
% the code and can be replaced by any character not in the code;
% |\lstinline$var i:integer;$| gives the same result.
%
% \paragraph{Displayed code}
% The \texttt{lstlisting} environment typesets the enclosed source code. Like
% most examples, the following one shows verbatim \LaTeX\ code on the right
% and the result on the left. You might take the right-hand side, put it into
% the minimal file, and run it through \LaTeX.
% \begin{lstsample}[lstlisting]{}{}
%    \begin{lstlisting}
%    for i:=maxint to 0 do
%    begin
%        { do nothing }
%    end;
%
%    Write('Case insensitive ');
%    WritE('Pascal keywords.');
%    \end{lstlisting}
% \end{lstsample}
% It can't be easier.
% \begin{advise}
% \item That's not true. The name `\texttt{listing}' is shorter.
%       \advisespace
%       Indeed. But other packages already define environments with that name.
%       To be compatible with such packages, all commands and environments of
%       the \packagename{listings} package use the prefix `\texttt{lst}'.
% \end{advise}
% The environment provides an optional argument. It tells the package to
% perform special tasks, for example, to print only the lines 2--5:
% \begin{lstsample}{\lstset{frame=trbl,framesep=0pt}\label{gFirstKey=ValueList}}{}
%    \begin{lstlisting}[firstline=2,
%                       lastline=5]
%    for i:=maxint to 0 do
%    begin
%        { do nothing }
%    end;
%
%    Write('Case insensitive ');
%    WritE('Pascal keywords.');
%    \end{lstlisting}
% \end{lstsample}
% \begin{advise}
% \item Hold on! Where comes the frame from and what is it good for?
%       \advisespace
%       You can put frames around all listings except code snippets.
%       You will learn how later. The frame shows that empty lines at the end
%       of listings aren't printed. This is line 5 in the example.
% \item Hey, you can't drop my empty lines!
%       \advisespace
%       You can tell the package not to drop them:
%       The key `\ikeyname{showlines}' controls these empty lines and is
%       described in section \ref{rTypesettingListings}. Warning: First
%       read ahead on how to use keys in general.
% \item I get obscure error messages when using `\ikeyname{firstline}'.
%       \advisespace
%       That shouldn't happen. Make a bug report as described in section
%       \lstref{uTroubleshooting}.
% \end{advise}
%
% \paragraph{Stand alone files}
% Finally we come to |\lstinputlisting|, the command used to pretty-print
% stand alone files. It has one optional and one file name argument.
% Note that you possibly need to specify the relative path to the file.
% Here now the result is printed below the verbatim code since both together
% don't fit the text width.
% \begin{lstsample}{\lstset{comment=[l]\%,columns=fullflexible}}{\lstset{alsoletter=\\,emph=\\lstinputlisting,emphstyle=\rstyle}\lstaspectindex{\lstinputlisting}{}}
%    \lstinputlisting[lastline=4]{listings.sty}
% \end{lstsample}
% \begin{advise}
% \item The spacing is different in this example.
%       \advisespace
%       Yes. The two previous examples have aligned columns, i.e.~columns with
%       identical numbers have the same horizontal position---this package
%       makes small adjustments only. The columns in the example here are not
%       aligned. This is explained in section \ref{uFixedAndFlexibleColumns}
%       (keyword: full flexible column format).
% \end{advise}
%
% Now you know all pretty-printing commands and environments. It remains
% to learn the parameters which control the work of the \packagename{listings}
% package. This is, however, the main task. Here are some of them.
%
%
% \subsection{Figure out the appearance}\label{gFigureOutTheAppearance}
%
% Keywords are typeset bold, comments in italic shape, and spaces in strings
% appear as \textvisiblespace. You don't like these settings? Look at this:
%\ifcolor
% \begin{lstxsample}[basicstyle,keywordstyle,identifierstyle,commentstyle,stringstyle,showstringspaces]
%    \lstset{% general command to set parameter(s)
%        basicstyle=\small,          % print whole listing small
%        keywordstyle=\color{black}\bfseries\underbar,
%                                    % underlined bold black keywords
%        identifierstyle=,           % nothing happens
%        commentstyle=\color{white}, % white comments
%        stringstyle=\ttfamily,      % typewriter type for strings
%        showstringspaces=false}     % no special string spaces
% \end{lstxsample}
%\else
% \begin{lstxsample}[basicstyle,keywordstyle,identifierstyle,commentstyle,stringstyle,showstringspaces]
%    \lstset{% general command to set parameter(s)
%        basicstyle=\small,          % print whole listing small
%        keywordstyle=\bfseries\underbar,
%                                    % underlined bold keywords
%        identifierstyle=,           % nothing happens
%        commentstyle=\itshape,      % default
%        stringstyle=\ttfamily,      % typewriter type for strings
%        showstringspaces=false}     % no special string spaces
% \end{lstxsample}
%\fi
% \begin{lstsample}{}{}
%    \begin{lstlisting}
%    for i:=maxint to 0 do
%    begin
%        { do nothing }
%    end;
%
%    Write('Case insensitive ');
%    WritE('Pascal keywords.');
%    \end{lstlisting}
% \end{lstsample}
%\ifcolor
% \begin{advise}
% \item You've requested white coloured comments, but I can see the comment
%       on the left side.
%       \advisespace
%       There are a couple of possible reasons:
%       (1) You've printed the documentation on nonwhite paper.
%       (2) If you are viewing this documentation as a \texttt{.dvi}-file, your
%           viewer seems to have problems with colour specials. Try to print
%           the page on white paper.
%       (3) If a printout on white paper shows the comment, the colour
%           specials aren't suitable for your printer or printer driver.
%           Recreate the documentation and try it again---and ensure that
%           the \packagename{color} package is well-configured.
% \end{advise}
%\fi
% The styles use two different kinds of commands. |\ttfamily| and |\bfseries|
% both take no arguments but |\underbar| does; it underlines the following
% argument. In general, the \emph{very last} command may read exactly one
% argument, namely some material the package typesets. There's one exception.
% The last command of \ikeyname{basicstyle} \emph{must not} read any
% tokens---or you will get deep in trouble.
% \begin{advise}
% \item `|basicstyle=\small|' looks fine, but comments look really bad with
%       `|commentstyle=\tiny|' and empty basic style, say.
%       \advisespace
%       Don't use different font sizes in a single listing.
% \item But I really want it!
%       \advisespace
%       No, you don't.
%^^A       The package adjusts internal data after selecting the basic style at
%^^A       the beginning of each listing. This is a problem if you change the
%^^A       font size for comments or strings, for example.
%^^A       Section \ref{rColumnAlignment} shows how to overcome this.
%^^A       But once again: Don't use different font sizes in a single listing
%^^A       unless you really know what you are doing.
% \end{advise}
%
% \paragraph{Warning}\label{wStrikingStyles}
% You should be very careful with striking styles; the recent example is rather
% moderate---it can get horrible. \emph{Always use decent highlighting.}
% Unfortunately it is difficult to give more recommendations since they depend
% on the type of document you're creating. Slides or other presentations often
% require more striking styles than books, for example.
% In the end, it's \emph{you} who have to find the golden mean!
%
%
% \subsection{Seduce to use}\label{gSeduceToUse}
%
% You know all pretty-printing commands and some main parameters. Here now
% comes a small and incomplete overview of other features. The table of
% contents and the index also provide information.
%
% \paragraph{Line numbers}
% are available for all displayed listings, e.g.~tiny numbers on the left, each
% second line, with 5pt distance to the listing:
% \begin{lstxsample}[numbers,numberstyle,stepnumber,numbersep]
%    \lstset{numbers=left, numberstyle=\tiny, stepnumber=2, numbersep=5pt}
% \end{lstxsample}
% \begin{lstsample}{}{}
%    \begin{lstlisting}
%    for i:=maxint to 0 do
%    begin
%        { do nothing }
%    end;
%
%    Write('Case insensitive ');
%    WritE('Pascal keywords.');
%    \end{lstlisting}
% \end{lstsample}
% \begin{advise}
% \item I can't get rid of line numbers in subsequent listings.
%       \advisespace
%       `|numbers=none|' turns them off.
% \item Can I use these keys in the optional arguments?
%       \advisespace
%       Of course. Note that optional arguments modify values for one
%       particular listing only: you change the appearance, step or distance
%       of line numbers for a single listing. The previous values are
%       restored afterwards.
% \end{advise}
% The environment allows you to interrupt your listings: you can end a listing
% and continue it later with the correct line number even if there are other
% listings in between. Read section \ref{uLineNumbers} for a thorough
% discussion.
%
% \paragraph{Floating listings}
% Displayed listings may float:
% \begin{lstsample}{\lstset{frame=tb}}{}
%    \begin{lstlisting}[float,caption=A floating example]
%    for i:=maxint to 0 do
%    begin
%        { do nothing }
%    end;
%
%    Write('Case insensitive ');
%    WritE('Pascal keywords.');
%    \end{lstlisting}
% \end{lstsample}
% Don't care about the parameter \ikeyname{caption} now. And if you put the
% example into the minimal file and run it through \LaTeX, please don't wonder:
% you'll miss the horizontal rules since they are described elsewhere.
% \begin{advise}
% \item \LaTeX's float mechanism allows one to determine the placement of floats.
%       How can I do that with these?
%       \advisespace
%       You can write `|float=tp|', for example.
% \end{advise}
%
% \paragraph{Other features}
% There are still features not mentioned so far: automatic breaking of long
% lines, the possibility to use \LaTeX\ code in listings, automated indexing,
% or personal language definitions.
% One more little teaser? Here you are. But note that the result is not
% produced by the \LaTeX\ code on the right alone. The main parameter is
% hidden.
% \begin{lstsample}{\lstset{literate={:=}{{$\gets$}}1 {<=}{{$\leq$}}1 {>=}{{$\geq$}}1 {<>}{{$\neq$}}1}}{}
%    \begin{lstlisting}
%    if (i<=0) then i := 1;
%    if (i>=0) then i := 0;
%    if (i<>0) then i := 0;
%    \end{lstlisting}
% \end{lstsample}
%
% You're not sure whether you should use \packagename{listings}?
% Read the next section!
%
%
% \subsection{Alternatives}
%
% \begin{advise}
% \item Why do you list alternatives?
%       \advisespace
%       Well, it's always good to know the competitors.^^A :-)
% \item I've read the descriptions below and the \packagename{listings} package
%       seems to incorporate all the features. Why should I use one of the
%       other programs?
%       \advisespace
%       Firstly, the descriptions give a taste and not a complete overview,
%       secondly, \packagename{listings} lacks some properties, and, ultimately,
%       you should use the program matching your needs most precisely.
% \end{advise}
% This package is certainly not the final utility for typesetting source code.
% Other programs do their job very well, if you are not satisfied with
% \packagename{listings}. Some are independent of \LaTeX, others come as
% separate program plus \LaTeX\ package, and others are packages which
% don't pretty-print the source code. The second type includes converters,
% cross compilers, and preprocessors. Such programs create \LaTeX\ files
% you can use in your document or stand alone ready-to-run \LaTeX\ files.
%
% Note that I'm not dealing with any literate programming tools here, which
% could also be alternatives. However, you should have heard of the
% \texttt{WEB} system, the tool Prof.~Donald E.~Knuth developed and made use
% of to document and implement \TeX.
%
% \paragraph{\href{http://www.infres.enst.fr/~demaille/a2ps}{\packagename{a2ps}}}
% started as `ASCII to PostScript' converter, but today you can invoke the
% program with \texttt{--pretty-print=}\meta{language} option. If your
% favourite programming language is not already supported, you can write your
% own so-called style sheet. You can request line numbers, borders, headers,
% multiple pages per sheet, and many more. You can even print symbols like
% $\forall$ or $\alpha$ instead of their verbose forms. If you just want
% program listings and not a document with some listings, this is the best
% choice.
%
% \paragraph{\href{http://www.ctan.org/tex-archive/nonfree/support/lgrind}{\packagename{LGrind}}}
% is a cross compiler and comes with many predefined programming languages.
% For example, you can put the code on the right in your document, invoke
% \packagename{LGrind} with \texttt{-e} option (and file names), and run the
% created file through \LaTeX. You should get a result similar to the
% left-hand side:
% \begin{center}
% \begin{minipage}{0.45\linewidth}
%\iflgrind
%    \LGindent=0pt
%    \LGinlinefalse\LGbegin\lgrinde
%    \L{\LB{\K{for}_\V{i}:=\V{maxint}_\K{to}_\N{0}_\K{do}}}
%    \L{\LB{\K{begin}}}
%    \L{\LB{____\C{}\{_do_nothing_\}\CE{}}}
%    \L{\LB{\K{end};}}
%    \L{\LB{}}
%    \L{\LB{\V{Write}(\S{}{'}Case_insensitive_{'}\SE{});}}
%    \L{\LB{\V{WritE}(\S{}{'}Pascal_keywords.{'}\SE{});}}
%    \endlgrinde\LGend
%\else
%    \packagename{LGrind} not installed.
%\fi
% \end{minipage}
% \begin{minipage}{0.45\linewidth}
% \begin{verbatim}
% %[
% for i:=maxint to 0 do
% begin
%     { do nothing }
% end;
%
% Write('Case insensitive ');
% WritE('Pascal keywords.');
% %]\end{verbatim}
% \end{minipage}
% \end{center}
% If you use |%(| and |%)| instead of |%[| and |%]|, you get a code snippet
% instead of a displayed listing. Moreover you can get line numbers to the
% left or right, use arbitrary \LaTeX\ code in the source code, print symbols
% instead of verbose names, make font setup, and more. You will (have to)
% like it (if you don't like \packagename{listings}).
%
% Note that \packagename{LGrind} contains code with a no-sell license and is
% thus nonfree software.
%
% \paragraph{\href{ftp://axp3.sv.fh-mannheim.de/cvt2latex}{\packagename{cvt2ltx}}}
% is a family of `source code to \LaTeX' converters for C, Objective C, \Cpp,
% IDL and Perl. Different styles, line numbers and other qualifiers can be
% chosen by command-line option. Unfortunately it isn't documented how other
% programming languages can be added.
%
% \paragraph{\href{http://www.ctan.org/tex-archive/support/C++2LaTeX-1_1pl1}{\packagename{\Cpp2\LaTeX}}}
% is a C/\Cpp\ to \LaTeX\ converter. You can specify the fonts for comments,
% directives, keywords, and strings, or the size of a tabulator. But as far as
% I know you can't number lines.
%
% \paragraph{\href{http://www.ctan.org/tex-archive/support/slatex}{\packagename{S\LaTeX}}}
% is a pretty-printing Scheme program (which invokes \LaTeX\ automatically)
% especially designed for Scheme and other Lisp dialects. It supports stand
% alone files, text and display listings, and you can even nest the
% commands/environments if you use \LaTeX\ code in comments, for example.
% Keywords, constants, variables, and symbols are definable and use of
% different styles is possible. No line numbers.
%
% \paragraph{\href{http://www.ctan.org/tex-archive/support/tiny_c2l}{\packagename{tiny\textunderscore c2ltx}}}
% is a C/\Cpp/Java to \LaTeX\ converter based on \packagename{cvt2ltx} (or the
% other way round?). It supports line numbers, block comments, \LaTeX\ code
% in/as comments, and smart line breaking. Font selection and tabulators are
% hard-coded, i.e.~you have to rebuild the program if you want to change the
% appearance.
%
% \paragraph{\href{http://www.ctan.org/tex-archive/macros/latex/contrib/misc}{\packagename{listing}}}
% ---note the missing \packagename{s}---is not a pretty-printer and the
% aphorism about documentation at the end of \texttt{listing.sty} is not
% true.\space ^^A :-)
% It defines |\listoflistings| and a nonfloating environment for listings.
% All font selection and indention must be done by hand. However, it's
% useful if you have another tool doing that work, e.g.~\packagename{LGrind}.
%
% \paragraph{\href{http://www.ctan.org/tex-archive/macros/latex/contrib/alg}{\packagename{alg}}}
% provides essentially the same functionality as \packagename{algorithms}.
% So read the next paragraph and note that the syntax will be different.
%
% \paragraph{\href{http://www.ctan.org/tex-archive/macros/latex/contrib/algorithms}{\packagename{algorithms}}}
% goes a quite different way. You describe an algorithm and the package
% formats it, for example
% \begin{center}
% \begin{minipage}{0.45\linewidth}
%\ifalgorithmicpkg
%    \begin{algorithmic}
%    \IF {$i\leq0$}
%    \STATE $i\gets1$
%    \ELSE\IF {$i\geq0$}
%    \STATE $i\gets0$
%    \ENDIF\ENDIF
%    \end{algorithmic}
%\else
%    \packagename{algorithms} not installed.
%\fi
% \end{minipage}
% \begin{minipage}{0.45\linewidth}
% \begin{verbatim}
%\begin{algorithmic}
%\IF{$i\leq0$}
%\STATE $i\gets1$
%\ELSE\IF{$i\geq0$}
%\STATE $i\gets0$
%\ENDIF\ENDIF
%\end{algorithmic}\end{verbatim}
% \end{minipage}
% \end{center}
% As this example shows, you get a good looking algorithm even from a bad
% looking input. The package provides a lot more constructs like |for|-loops,
% |while|-loops, or comments. You can request line numbers, `ruled', `boxed'
% and floating algorithms, a list of algorithms, and you can customize the
% terms \textbf{if}, \textbf{then}, and so on.
%
% \paragraph{\href{http://www.mimuw.edu.pl/~wolinski/pretprin.html}{\packagename{pretprin}}}
% is a package for pretty-printing texts in formal languages---as the title
% in TUGboat, Volume 19 (1998), No.~3 states. It provides environments which
% pretty-print \emph{and} format the source code. Analyzers for Pascal and
% Prolog are defined; adding other languages is easy---if you are or get a bit
% familiar with automatons and formal languages.
%
% \paragraph{\packagename{alltt}}
% defines an environment similar to \texttt{verbatim} except that |\|, |{| and
% |}| have their usual meanings. This means that you can use commands in the
% verbatims, e.g.~select different fonts or enter math mode.
%
% \paragraph{\href{http://www.ctan.org/tex-archive/macros/latex/contrib/moreverb}{\packagename{moreverb}}}
% requires \packagename{verbatim} and provides verbatim output to a file,
% `boxed' verbatims and line numbers.
%
% \paragraph{\packagename{verbatim}}
% defines an improved version of the standard \texttt{verbatim} environment and
% a command to input files verbatim.
%
% \paragraph{\href{http://www.ctan.org/tex-archive/macros/latex/contrib/fancyvrb}{\packagename{fancyvrb}}}
% is, roughly speaking, a superset of \packagename{alltt},
% \packagename{moreverb}, and \packagename{verbatim}, but many more parameters
% control the output. The package provides frames, line numbers on the left or
% on the right, automatic line breaking (difficult), and more. For example, an
% interface to \packagename{listings} exists, i.e.~you can pretty-print source
% code automatically.
% The package \packagename{fvrb-ex} builds on \packagename{fancyvrb} and
% defines environments to present examples similar to the ones in this guide.
%
%
% \section{The next steps}\label{uTheNextSteps}
%
% Now, before actually using the \packagename{listings} package, you should
% \emph{really} read the software license. It does not cost much time and
% provides information you probably need to know.
%
%
% \subsection{Software license}\label{uSoftwareLicense}
%
% The files \texttt{listings.dtx} and \texttt{listings.ins} and all
% files generated from only these two files are referred to as `the
% \packagename{listings} package' or simply `the package'.
% \texttt{lstdrvrs.dtx} and the files generated from that file are
% `drivers'.
%
% \paragraph{Copyright}
%   The \packagename{listings} package is copyright 1996--2004 Carsten Heinz,
%   and copyright 2006 Brooks Moses.  The drivers are copyright any individual
%   author listed in the driver files.
%
% \paragraph{Distribution and modification}
%   The \packagename{listings} package and its drivers may be distributed
%   and/or modified under the conditions of the LaTeX Project Public License,
%   either version 1.3 of this license or (at your option) any later version.
%   The latest version of this license is in
%      \href{http://www.latex-project.org/lppl.txt}{http://www.latex-project.org/lppl.txt}
%   and version 1.3 or later is part of all distributions of LaTeX version
%  2003/12/01 or later.
%
% \paragraph{Contacts}
%   Read section \lstref{uTroubleshooting} on how to submit a bug report.
%   Send all other comments, ideas, and additional programming languages to
%   \lstemail\ using \texttt{listings} as part of the subject.
%
%
% \subsection{Package loading}\label{uPackageLoading}
%
% As usual in \LaTeX, the package is loaded by
%    |\usepackage[|\meta{options}|]{listings}|,
% where |[|\meta{options}|]| is optional and gives a comma separated list of
% options. Each either loads an additional \packagename{listings} aspect, or
% changes default properties. Usually you don't have to take care of such
% options. But in some cases it could be necessary: if you want to compile
% documents created with an earlier version of this package or if you use
% special features. Here's an incomplete list of possible options.
% \begin{advise}
% \item Where is a list of all of the options?
%       \advisespace
%       In the developer's guide since they were introduced to debug the
%       package more easily. Read section \ref{uHowTos} on how to get that
%       guide.
% \end{advise}
% \begin{description}
% \item[\normalfont\texttt{0.21}]\leavevmode
%
%       invokes a compatibility mode for compiling documents written for
%       \packagename{listings} version 0.21.
%
% \item[\normalfont\texttt{draft}]\leavevmode
%
%       The package prints no stand alone files, but shows the captions and
%       defines the corresponding labels.
%       Note that a global |\documentclass|-option \texttt{draft} is
%       recognized, so you don't need to repeat it as a package option.
%
% \item[\normalfont\texttt{final}]\leavevmode\label{uoption:final}
%
%       Overwrites a global \texttt{draft} option.
%
% \item[\normalfont\texttt{savemem}]\leavevmode
%
%       tries to save some of \TeX's memory. If you switch between languages
%       often, it could also reduce compile time. But all this depends on the
%       particular document and its listings.
% \end{description}
% Note that various experimental features also need explicit loading via
% options. Read the respective lines in section \ref{rExperimentalFeatures}.
%
% \medbreak
% After package loading it is recommend to load all used dialects of programming
% languages with the following command. It is faster to load several languages
% with one command than loading each language on demand.
% \begin{syntax}
% \item {\rstyle\icmdname\lstloadlanguages}\marg{comma separated list of languages}
%
%       Each language is of the form \oarg{dialect}\meta{language}. Without
%       the optional \oarg{dialect} the package loads a default dialect. So
%       write `|[Visual]C++|' if you want Visual \Cpp\ and `|[ISO]C++|' for
%       ISO \Cpp. Both together can be loaded by the command
%       |\lstloadlanguages{[Visual]C++,[ISO]C++}|.
%
%       Table \ref{uPredefinedLanguages} on page \pageref{uPredefinedLanguages}
%       shows all defined languages and their dialects.
% \end{syntax}
%^^A After or even before language loading, you might want to define default
%^^A dialects---just to be independent of configuration files.
%
%
% \subsection{The key=value interface}\label{uTheKey=ValueInterface}
%
% This package uses the \packagename{keyval} package from the
% \packagename{graphics} bundle by David Carlisle. Each parameter is
% controlled by an associated key and a user supplied value. For example,
% \ikeyname{firstline} is a key and |2| a valid value for this key.
%
% The command {\rstyle\icmdname\lstset} gets a comma separated list of
% ``key|=|value'' pairs. The first list with more than a single entry is on
% page \pageref{gFirstKey=ValueList}: |firstline=2,lastline=5|.
% \begin{advise}
% \item So I can write `|\lstset{firstline=2,lastline=5}|' once for all?
%       \advisespace
%       No. `\ikeyname{firstline}' and `\ikeyname{lastline}' belong to a small
%       set of
%       keys which are only used on individual listings. However, your command is
%       not illegal---it has no effect. You have to use these keys inside the
%       optional argument of the environment or input command.
% \item What's about a better example of a key|=|value list?
%       \advisespace
%       There is one in section \ref{gFigureOutTheAppearance}.
% \item `|language=[77]Fortran|' does not work inside an optional argument.
%       \advisespace
%       You must put braces around the value if a value with optional argument
%       is used inside an optional argument. In the case here write
%       `|language={[77]Fortran}|' to select Fortran 77.
% \item If I use the `\ikeyname{language}' key inside an optional argument, the
%       language isn't active when I typeset the next listing.
%       \advisespace
%       All parameters set via `|\lstset|' keep their values up to the end of
%       the current environment or group. Afterwards the previous values are
%       restored. The optional parameters of the two pretty-printing commands
%       and the `\texttt{lstlisting}' environment take effect on the particular
%       listing only, i.e.~values are restored immediately. For example, you
%       can select a main language and change it for special listings.
% \item \icmdname\lstinline\ has an optional argument?
%       \advisespace
%       Yes. And from this fact comes a limitation: you can't use the left
%       bracket `|[|' as delimiter unless you specify at least an empty
%       optional argument as in `|\lstinline[][var i:integer;[|'.
%       If you forget this, you will either get a ``runaway argument'' error
%       from \TeX, or an error message from the \packagename{keyval} package.
% \end{advise}
%
%
% \subsection{Programming languages}\label{uProgrammingLanguages}
%
% You already know how to activate programming languages---at least Pascal.
% An optional parameter selects particular dialects of a language. For example,
% |language=[77]Fortran| selects Fortran 77 and |language=[XSC]Pascal| does the
% same for Pascal XSC. The general form is
%    {\rstyle\ikeyname{language}}|=|\oarg{dialect}\meta{language}.
% If you want to get rid of keyword, comment, and string detection, use
% |language={}| as an argument to |\lstset| or as optional argument.
%
% Table \ref{uPredefinedLanguages} shows all predefined languages and dialects.
% Use the listed names as \meta{language} and \meta{dialect}, respectively. If
% no dialect or `empty' is given in the table, just don't specify a dialect.
% Each underlined dialect is default; it is selected if you leave out
% the optional argument. The predefined defaults are the newest language
% versions or standard dialects.
%^^A
%^^A  Make table of predefined languages.
%^^A
%\let\lstlanguages\empty
%\makeatletter
%\@for\lst@temp:={lstlang1.sty,lstlang2.sty,lstlang3.sty}\do
%    {\IfFileExists\lst@temp{}{\let\lstlanguages\relax}}
%\makeatother
%\ifx\lstlanguages\relax
%    \PackageWarningNoLine{Listings}
%        {Standard drivers not available.\MessageBreak
%         Please check your installation.\MessageBreak
%         Compilation aborted}
%    \csname @@end\expandafter\endcsname
%\fi
%\lstscanlanguages\lstlanguages{lstlang1.sty,lstlang2.sty,lstlang3.sty}{}^^A
%\def\topfigrule{\hrule\kern-0.4pt\relax}^^A
%\let\botfigrule\topfigrule
%\belowcaptionskip=\smallskipamount
% \begin{table}[tbhp]
% \small
% \caption{Predefined languages.
%          Note that some definitions are preliminary, for example HTML and XML.
%          Each underlined dialect is the default dialect.}^^A
%          \label{uPredefinedLanguages}^^A
% \makeatletter
% \setbox\@tempboxa\hbox{^^A
%    \InputIfFileExists{listings.cfg}{\lst@InputCatcodes}{}}^^A
% \lstprintlanguages\lstlanguages
% \end{table}
%^^A
%^^A end of table
%^^A
%\lstset{defaultdialect=[doc]Pascal}^^A restore
% \begin{advise}
% \item How can I define default dialects?
%       \advisespace
%       Check section \ref{rLanguagesAndStyles} for `\keyname{defaultdialect}'.
% \item I have C code mixed with assembler lines. Can \packagename{listings}
%       pretty-print such source code, i.e.~highlight keywords and comments of
%       both languages?
%       \advisespace
%       `\ikeyname{alsolanguage}|=|\oarg{dialect}\meta{language}' selects a
%       language additionally to the active one. So you only have to write a
%       language definition for your assembler dialect, which doesn't interfere
%       with the definition of C, say. Moreover you might want to use the key
%       `\keyname{classoffset}' described in section \ref{rLanguagesAndStyles}.
% \item How can I define my own language?
%       \advisespace
%       This is discussed in section \ref{rLanguageDefinitions}. And if you
%       think that other people could benefit by your definition, you might
%       want to send it to the address in section \ref{uSoftwareLicense}.
%       Then it will be published under the \LaTeX\ Project Public License.
% \end{advise}
% Note that the arguments \meta{language} and \meta{dialect} are case
% insensitive and that spaces have no effect.
%
% There is at least one language (VDM, Vienna Development Language,
% \url{http://www.vdmportal.org}) which is not directly supported by the
% \packagename{listings} package. It needs a package for its own:
% \packagename{vdmlisting}. On the other hand \packagename{vdmlisting} uses
% the \packagename{listings} package and so it should be mentioned in this
% context.
%
%
% \subsubsection{Preferences}\label{uPreferences}
%
% Sometimes authors of language support provide their own configuration
% preferences. These may come either from their personal experience or
% from the settings in an IDE and can be defined as a \packagename{listings}
% style. From version 1.5b of the \packagename{listings} package on these
% styles are provided as files with the name
% |listings-|\meta{language}|.prf|, \meta{language} is the name of the
% supported programming language in lowercase letters.
%
% So if an user of the \packagename{listings} package wants to use these
% preferences, she/he can say for example when using Python
% \begin{quote}
%     |\input{listings-python.prf}|
% \end{quote}
% at the end of her/his |listings.cfg| configuration file as long as the
% file |listings-python.prf| resides in the \TeX{} search path. Of course
% that file can be changed according to the user's preferences.
%
% At the moment there are five such preferences files:
% \begin{enumerate}
%   \item |listings-acm.prf|
%   \item |listings-bash.prf|
%   \item |listings-fortran.prf|
%   \item |listings-lua.prf|
%   \item |listings-python.prf|
% \end{enumerate}
% All contributors are invited to supply more personal preferences.
%
%
% \subsection{Special characters}\label{uSpecialCharacters}
%
%
% \paragraph{Tabulators}
% You might get unexpected output if your sources contain tabulators.
% The package assumes tabulator stops at columns 9, 17, 25, 33, and so on.
% This is predefined via |tabsize=8|. If you change the eight to the number
% $n$, you will get tabulator stops at columns $n+1,2n+1,3n+1,$ and so on.
% \begin{lstsample}[tabsize]{}{}
%    \lstset{tabsize=2}
%    \begin{lstlisting}
%    123456789
%    	{ one tabulator }
%    		{ two tabs }
%    123		{ 123 + two tabs }
%    \end{lstlisting}
% \end{lstsample}
% For better illustration, the left-hand side uses |tabsize=2| but the verbatim
% code |tabsize=4|. Note that |\lstset| modifies the values for all following
% listings in the same environment or group. This is no problem here since the
% examples are typeset inside minipages. If you want to change settings for a
% single listing, use the optional argument.
%
%
% \paragraph{Visible tabulators and spaces}
% One can make spaces and tabulators visible:
% \begin{lstsample}[showspaces,showtabs,tab]{}{}
%    \lstset{showspaces=true,
%            showtabs=true,
%            tab=\rightarrowfill}
%    \begin{lstlisting}
%        for i:=maxint to 0 do
%        begin
%    	{ do nothing }
%        end;
%    \end{lstlisting}
% \end{lstsample}
% If you request \ikeyname{showspaces} but no \ikeyname{showtabs},
% tabulators are converted to visible spaces.
% The default definition of \ikeyname{tab} produces a `wide visible space'
% \lstinline[showtabs]!	!. So you might want to use |$\to$|, |$\dashv$|
% or something else instead.
% \begin{advise}
% \item Some sort of advice: (1) You should really indent lines of source code
%       to make listings more readable. (2) Don't indent some lines with
%       spaces and others via tabulators. Changing the tabulator size (of your
%       editor or pretty-printing tool) completely disturbs the columns.
%       (3) As a consequence, never share your files with differently tab sized
%       people!^^A true only if you use tabulators, just :-)
% \item To make the \LaTeX\ code more readable, I indent the environments'
%       program listings. How can I remove that indention in the output?
%       \advisespace
%       Read `How to gobble characters' in section \ref{uHowTos}.
% \end{advise}
%
%
% \paragraph{Form feeds}
% Another special character is a form feed causing an empty line by default.
% {\rstyle\ikeyname{formfeed}}|=\newpage| would result in a new page every
% form feed. Please note that such definitions (even the default) might get
% in conflict with frames.
%
%
% \paragraph{National characters}
% If you type in such characters directly as characters of codes 128--255 and
% use them also in listings, let the package know it---or you'll get really
% funny results. {\rstyle\ikeyname{extendedchars}}|=true| allows and
% |extendedchars=false| prohibits \packagename{listings} from handling
% extended characters in listings. If you use them, you should load
% \packagename{fontenc}, \packagename{inputenc} and/or
% any other package which defines the characters.
% \begin{advise}
% \item I have problems using \packagename{inputenc} together with
%       \packagename{listings}.
%       \advisespace
%       This could be a compatibility problem. Make a bug report as described
%       in section \lstref{uTroubleshooting}.
% \end{advise}
% The extended characters don't cover Arabic, Chinese, Hebrew, Japanese, and so
% on---specifically, any encoding which uses multiple bytes per character.
%
% Thus, if you use the a package that supports multibyte characters, such as
% the \packagename{CJK} or \packagename {ucs} packages for Chinese and
% UTF-8 characters, you must avoid letting \packagename{listings}
% process the extended characters.  It is generally best to also specify
% |extendedchars=false| to avoid having \packagename{listings} get entangled
% in the other package's extended-character treatment.
%
% If you do have a listing contained within a CJK environment, and want to have
% CJK characters inside the listing, you can place them within a comment that
% escapes to \LaTeX -- see section \ref{rEscapingToLaTeX} for how to do that.
% (If the listing is not inside a CJK environment, you can simply put a small
% CJK environment within the escaped-to-\LaTeX portion of the comment.)
%
% Similarly, if you are using UTF-8 extended characters in a listing, they must
% be placed within an escape to \LaTeX.
%
% Also, section \ref{uNationalCharacters} has a few details on how to work with
% extended characters in the context of $\Lambda$.
%
%
% \subsection{Line numbers}\label{uLineNumbers}
%
% You already know the keys \ikeyname{numbers}, \ikeyname{numberstyle},
% \ikeyname{stepnumber}, and \ikeyname{numbersep} from section
% \ref{gSeduceToUse}. Here now we deal with continued listings.
% You have two options to get consistent line numbering across listings.
%
% \begin{lstsample}[firstnumber]{\lstset{numbers=left,numberstyle=\tiny,stepnumber=2,numbersep=5pt}}{}
%    \begin{lstlisting}[firstnumber=100]
%    for i:=maxint to 0 do
%    begin
%        { do nothing }
%    end;
%
%    \end{lstlisting}
%    And we continue the listing:
%    \begin{lstlisting}[firstnumber=last]
%    Write('Case insensitive ');
%    WritE('Pascal keywords.');
%    \end{lstlisting}
% \end{lstsample}
% In the example, \ikeyname{firstnumber} is initially set to 100; some lines
% later the value is \texttt{last}, which continues the numbering of the last
% listing. Note that the empty line at the end of the first part is not printed
% here, but it counts for line numbering. You should also notice that you can
% write |\lstset{firstnumber=last}| once and get consecutively numbered code
% lines---except you specify something different for a particular listing.
%
% On the other hand you can use |firstnumber=auto| and name your listings.
% Listings with identical names (case sensitive!) share a line counter.
% \begin{lstsample}[name]{\lstset{numbers=left,numberstyle=\tiny,stepnumber=2,numbersep=5pt}}{}
%    \begin{lstlisting}[name=Test]
%    for i:=maxint to 0 do
%    begin
%        { do nothing }
%    end;
%
%    \end{lstlisting}
%    And we continue the listing:
%    \begin{lstlisting}[name=Test]
%    Write('Case insensitive ');
%    WritE('Pascal keywords.');
%    \end{lstlisting}
% \end{lstsample}
% The next |Test| listing goes on with line number {\makeatletter\lstno@Test},
% no matter whether there are other listings in between.
% \begin{advise}
% \item Okay. And how can I get decreasing line numbers?
%       \advisespace
%       Sorry, what?
%       \advisespace
%       Decreasing line numbers as on page \pageref{rDecreasingLabels}.
%       \advisespace
%       May I suggest to demonstrate your individuality by other means?
%       If you differ, you should try a negative `\ikeyname{stepnumber}'
%       (together with `\ikeyname{firstnumber}').
% \end{advise}
%
% Read section \ref{uHowTos} on how to reference line numbers.
%
%
% \subsection{Layout elements}
%
% It's always a good idea to structure the layout by vertical space,
% horizontal lines, or different type sizes and typefaces. The best to stress
% whole listings are---not all at once---colours, frames, vertical space, and
% captions. The latter are also good to refer to listings, of course.
%
% \paragraph{Vertical space}
% The keys {\rstyle\ikeyname{aboveskip}} and {\rstyle\ikeyname{belowskip}}
% control the vertical space above and below displayed listings. Both keys get
% a dimension or skip as value and are initialized to |\medskipamount|.
%
% \paragraph{Frames}
% The key \ikeyname{frame} takes the verbose values \keyvalue{none},
% \keyvalue{leftline}, \keyvalue{topline}, \keyvalue{bottomline},
% \keyvalue{lines} (top and bottom), \keyvalue{single} for single frames, or
% \keyvalue{shadowbox}.
% \begin{lstsample}[frame]{}{}
%    \begin{lstlisting}[frame=single]
%    for i:=maxint to 0 do
%    begin
%        { do nothing }
%    end;
%    \end{lstlisting}
% \end{lstsample}
% \begin{advise}
% \item The rules aren't aligned.
%       \advisespace
%       This could be a bug of this package or a problem with your
%       \texttt{.dvi} driver. \emph{Before} sending a bug report to the package
%       author, modify the parameters described in section \ref{rFrames}
%       heavily. And do this step by step!
%       For example, begin with `|framerule=10mm|'. If the rules are
%       misaligned by the same (small) amount as before, the problem does not
%       come from the rule width. So continue with the next parameter.  Also,
%       Adobe Acrobat sometimes has single-pixel rounding errors which can
%       cause small misalignments at the corners when PDF files are displayed
%       on screen; these are unfortunately normal.
% \end{advise}
% Alternatively you can control the rules at the \texttt{t}op, \texttt{r}ight,
% \texttt{b}ottom, and \texttt{l}eft directly by using the four initial letters
% for single rules and their upper case versions for double rules.
% \begin{lstsample}[frame]{}{}
%    \begin{lstlisting}[frame=trBL]
%    for i:=maxint to 0 do
%    begin
%        { do nothing }
%    end;
%    \end{lstlisting}
% \end{lstsample}
% Note that a corner is drawn if and only if both adjacent rules are requested.
% You might think that the lines should be drawn up to the edge, but what's
% about round corners? The key \ikeyname{frameround} must get exactly four
% characters as value. The first character is attached to the upper right
% corner and it continues clockwise. `\texttt{t}' as character makes the
% corresponding corner round.
% \begin{lstsample}[frameround]{}{}
%    \lstset{frameround=fttt}
%    \begin{lstlisting}[frame=trBL]
%    for i:=maxint to 0 do
%    begin
%        { do nothing }
%    end;
%    \end{lstlisting}
% \end{lstsample}
% Note that \ikeyname{frameround} has been used together with |\lstset| and thus
% the value affects all following listings in the same group or environment.
% Since the listing is inside a \texttt{minipage} here, this is no problem.
% \begin{advise}
% \item Don't use frames all the time, and in particular not with short listings.
%       This would emphasize nothing. Use frames for $10\%$ or even less of
%       your listings, for your most important ones.
% \item If you use frames on floating listings, do you really want frames?
%       \advisespace
%       No, I want to separate floats from text.
%       \advisespace
%       Then it is better to redefine \LaTeX's `|\topfigrule|' and
%       `|\botfigrule|'. For example, you could write
%       `|\renewcommand*\topfigrule{\hrule\kern-0.4pt\relax}|' and make the
%       same definition for |\botfigrule|.
% \end{advise}
%
% \paragraph{Captions}
% Now we come to \ikeyname{caption} and \ikeyname{label}. You might guess
% (correctly) that they can be used in the same manner as \LaTeX's |\caption|
% and |\label| commands, although here it is also possible to have a caption
% regardless of whether or not the listing is in a float:
% \begin{lstsample}[caption,label]{\lstset{xleftmargin=.05\linewidth}}{}
%    \begin{lstlisting}[caption={Useless code},label=useless]
%    for i:=maxint to 0 do
%    begin
%        { do nothing }
%    end;
%    \end{lstlisting}
% \end{lstsample}
% Afterwards you could refer to the listing via |\ref{useless}|. By default
% such a listing gets an entry in the list of listings, which can be printed
% with the command {\rstyle\icmdname\lstlistoflistings}. The key
% {\rstyle\ikeyname{nolol}} suppresses an entry for both the environment or
% the input command. Moreover, you can specify a short caption for the list
% of listings:
%    \keyname{caption}|={|\oarg{short}\meta{long}|}|.
% Note that the whole value is enclosed in braces since an optional value is
% used in an optional argument.
%
% If you don't want the label \texttt{\lstlistingname} plus number, you should
% use \ikeyname{title}:
% \begin{lstsample}[title]{\lstset{xleftmargin=.05\linewidth}}{}
%    \begin{lstlisting}[title={`Caption' without label}]
%    for i:=maxint to 0 do
%    begin
%        { do nothing }
%    end;
%    \end{lstlisting}
% \end{lstsample}
% \begin{advise}
% \item Something goes wrong with `\keyname{title}' in my document: in front of
%       the title is a delimiter.
%       \advisespace
%       The result depends on the document class; some are not compatible.
%       Contact the package author for a work-around.
% \end{advise}
%
% \paragraph{Colours}
% One more element. You need the \packagename{color} package and can then
% request coloured background via
% \ikeyname{backgroundcolor}|=|\meta{color command}.
% \begin{advise}
% \item Great! I love colours.
%       \advisespace
%       Fine, yes, really. And I like to remind you of the warning about
%       striking styles on page \pageref{wStrikingStyles}.
% \end{advise}
%\ifcolor
% \begin{lstxsample}[backgroundcolor]
%    \lstset{backgroundcolor=\color{yellow}}
% \end{lstxsample}
%\else
% \begin{verbatim}
%    color package not installed\end{verbatim}
%\fi
% \begin{lstsample}{}{}
%    \begin{lstlisting}[frame=single,
%                       framerule=0pt]
%    for i:=maxint to 0 do
%    begin
%        j:=square(root(i));
%    end;
%    \end{lstlisting}
% \end{lstsample}
% The example also shows how to get coloured space around the whole listing:
% use a frame whose rules have no width.
%
%
% \subsection{Emphasize identifiers}\label{uEmphasizeIdentifiers}
%
% Recall the pretty-printing commands and environment. |\lstinline| prints
% code snippets, |\lstinputlisting| whole files, and \texttt{lstlisting}
% pieces of code which reside in the \LaTeX\ file. And what are these
% different `types' of source code good for? Well, it just happens that a
% sentence contains a code fragment. Whole files are typically included in or
% as an appendix. Nevertheless some books about programming also include such
% listings in normal text sections---to increase the number of pages.
% Nowadays source code should be shipped on disk or CD-ROM and only the main
% header or interface files should be typeset for reference. So, please, don't
% misuse the \packagename{listings} package. But let's get back to the topic.
%
% Obviously `\texttt{lstlisting} source code' isn't used to make an executable
% program from. Such source code has some kind of educational purpose or even
% didactic.
% \begin{advise}
% \item What's the difference between educational and didactic?
%       \advisespace
%       Something educational can be good or bad, true or false.
%       Didactic is true by definition.^^A :-)
% \end{advise}
% Usually \emph{keywords} are highlighted when the package typesets a piece of
% source code. This isn't necessary for readers who know the programming
% language well. The main matter is the presentation of interface, library or
% other functions or variables. If this is your concern, here come the right
% keys. Let's say, you want to emphasize the functions |square| and |root|,
% for example, by underlining them. Then you could do it like this:
% \begin{lstxsample}[emph,emphstyle]
%    \lstset{emph={square,root},emphstyle=\underbar}
% \end{lstxsample}
% \begin{lstsample}{}{}
%    \begin{lstlisting}
%    for i:=maxint to 0 do
%    begin
%        j:=square(root(i));
%    end;
%    \end{lstlisting}
% \end{lstsample}
% \begin{advise}
% \item Note that the list of identifiers |{square,root}| is enclosed in
%       braces. Otherwise the \packagename{keyval} package would complain
%       about an undefined key \keyname{root} since the comma finishes the
%       key=value pair.
%       Note also that you \emph{must} put braces around the value if you
%       use an optional argument of a key inside an optional argument of a
%       pretty-printing command. Though it is not necessary, the following
%       example uses these braces. They are typically forgotten when they
%       become necessary,
% \end{advise}
%
% Both keys have an optional \meta{class number} argument for multiple
% identifier lists:
%\ifcolor
% \begin{lstxsample}[emph,emphstyle]
%    \lstset{emph={square},      emphstyle=\color{red},
%            emph={[2]root,base},emphstyle={[2]\color{blue}}}
% \end{lstxsample}
%\else
% \begin{lstxsample}[emph,emphstyle]
%    \lstset{emph={square},      emphstyle=\underbar,
%            emph={[2]root,base},emphstyle={[2]\fbox}}
% \end{lstxsample}
%\fi
% \begin{lstsample}{}{}
%    \begin{lstlisting}
%    for i:=maxint to 0 do
%    begin
%        j:=square(root(i));
%    end;
%    \end{lstlisting}
% \end{lstsample}
% \begin{advise}
% \item What is the maximal \meta{class number}?
%       \advisespace
%       $2^{31}-1=2\,147\,483\,647$. But \TeX's memory will exceed before you
%       can define so many different classes.
% \end{advise}
%
% One final hint: Keep the lists of identifiers disjoint. Never use a keyword
% in an `emphasize' list or one name in two different lists. Even if your
% source code is highlighted as expected, there is no guarantee that it is
% still the case if you change the order of your listings or if you use the
% next release of this package.
%
%
%\iffalse
% \subsection{*Listing alignment}\label{uListingAlignment}
%
% The examples are typeset with centered \texttt{minipage}s. That's the reason
% why you can't see that line numbers are printed in the margin. Now we
% separate the minipage margin and the minipage by a vertical rule:
% \begin{lstsample}{\lstset{frame=l,framesep=0pt,numberstyle=\tiny,stepnumber=2,numbersep=5pt}}{}
%    Some text before
%    \begin{lstlisting}
%    for i:=maxint to 0 do
%    begin
%        { do nothing }
%    end;
%    \end{lstlisting}
% \end{lstsample}
% The listing is lined up with the normal text. The parameter \ikeyname{xleftmargin}
% moves the listing to the right (or left if the dimension is negative).
% \begin{lstsample}{\lstset{frame=l,framesep=0pt,numberstyle=\tiny,stepnumber=2,numbersep=5pt}}{}
%    Some text before
%    \begin{lstlisting}[xleftmargin=15pt]
%    for i:=maxint to 0 do
%    begin
%        { do nothing }
%    end;
%    \end{lstlisting}
%
%    \begin{lstlisting}{ }
%    Write('Insensitive');
%    WritE('keywords.');
%    \end{lstlisting}
% \end{lstsample}
% Note again that optional arguments change settings for single listings.
%
% If you use environments like \texttt{itemize} or \texttt{enumerate}, there
% is `natural' indention coming from these environments. By default the
% \packagename{listings} package respects this. But you might use
% |resetmargins=true| (or |false|) to make your own decision. You can use it
% together with |xleftmargin|, of course.
% \begin{advise}
% \item I get heavy overfull |\hbox|es from some listings.
%       \advisespace
%       This comes from long lines in your listings. You have some options
%       to get rid of the overful |\hbox|es. Firstly I recommend to typeset
%       listings in smaller fonts than the surrounding text, for example
%       `|basicstyle=\small|'. Secondly you might want to use the flexible
%       column format. Thirdly you can increase the line width or set it
%       explicitly, refer section \ref{rMarginsAndLineShape}.
%       If all this doesn't help, you might want to change
%       `\ikeyname{basewidth}', but be careful! The two unknown items are
%       explained in the next section.
% \end{advise}
%
% You might need to control the vertical position of listings with the
% \ikeyname{boxpos} key, for example, if you use them in \texttt{minipage} or
% \texttt{tabular} environments. Here `listings' means \texttt{lstlisting} or
% |\lstinputlisting|. As the following example shows, you can even place such
% listings inside paragraphs, but you must force the package to do this by
% enclosing the listing in |\hbox{| and |}|.
% \begin{advise}
% \item Is it good form to use the \TeX-primitive `|\hbox|' in a \LaTeX\
%       document?
%       \advisespace
%       No, it's not. But \LaTeX's `|\mbox|' does not work in this example:
% \end{advise}
% \begin{lstsample}{}{}
%    Here are some multi-line listings inside a paragraph.
%    The `boxpos' key controls their vertical alignment:
%    \hbox{\begin{lstlisting}[boxpos=c]
%    center
%    center
%    \end{lstlisting}}
%    \hbox{\begin{lstlisting}[boxpos=b]
%    bottom baseline
%    bottom baseline
%    \end{lstlisting}}
%    \hbox{\begin{lstlisting}[boxpos=t]
%    top baseline
%    top baseline
%    \end{lstlisting}}
% \end{lstsample}
%\fi
%
%
% \subsection{Indexing}\label{uIndexing}
%
% Indexing is just like emphasizing identifiers---I mean the usage:
% \begin{lstxsample}[index]
%    \lstset{index={square},index={[2]root}}
% \end{lstxsample}
% \begin{lstsample}{}{}
%    \begin{lstlisting}
%    for i:=maxint to 0 do
%    begin
%        j:=square(root(i));
%    end;
%    \end{lstlisting}
% \end{lstsample}
% Of course, you can't see anything here. You will have to look at the index.
% \begin{advise}
% \item Why is the `\ikeyname{index}' key able to work with multiple identifier
%       lists?
%       \advisespace
%       This question is strongly related to the `{\rstyle\ikeyname{indexstyle}}'
%       key. Someone might want to create multiple indexes or want to insert
%       prefixes like `|constants|', `|functions|', `|keywords|', and so on.
%       The `\ikeyname{indexstyle}' key works like the other style keys except
%       that the last token \emph{must} take an argument, namely the
%       (printable form of the) current identifier.
%
%       You can define `|\newcommand\indexkeywords[1]{\index{keywords, #1}}|'
%       and make similar definitions for constant or function names. Then
%       `|indexstyle=[1]\indexkeywords|' might meet your purpose. This becomes
%       easier if you want to create multiple indexes with the
%       \href{http://www.ctan.org/tex-archive/macros/latex/contrib/camel}
%       {\packagename{index}} package.
%       If you have defined appropriate new indexes, it is possible to write
%       `|indexstyle=\index[keywords]|', for example.
%
% \item Let's say, I want to index all keywords. It would be annoying to
%       type in all the keywords again, specifically if the used programming
%       language changes frequently.
%       \advisespace
%       Just read ahead.
% \end{advise}
% The \ikeyname{index} key has in fact two optional arguments. The first is the
% well-known \meta{class number}, the second is a comma separated list of other
% keyword classes whose identifiers are indexed. The indexed identifiers then
% change automatically with the defined keywords---not automagically, it's not
% an illusion.^^A :-)
%
% Eventually you need to know the names of the keyword classes. It's usually
% the key name followed by a class number, for example, |emph2|, |emph3|,
% \ldots, |keywords2| or |index5|. But there is no number for the first order
% classes |keywords|, |emph|, |directives|, and so on.
% \begin{advise}
% \item `|index=[keywords]|' does not work.
%       \advisespace
%       The package can't guess which optional argument you mean. Hence you
%       must specify both if you want to use the second one. You should try
%       `|index=[1][keywords]|'.
% \end{advise}
%
%
% \subsection{Fixed and flexible columns}\label{uFixedAndFlexibleColumns}
%
% The first thing a reader notices---except different styles for keywords,
% etc.---is the column alignment. Arne John Glenstrup invented the flexible
% column format in 1997. Since then some efforts were made to develop this
% branch farther. Currently four column formats are provided: fixed, flexible,
% space-flexible, and full flexible. Take a close look at the following
% examples.
% \begin{center}
% \lstset{style={},language={}}
% \def\sample{\begin{lstlisting}^^J WOMEN\ \ are^^A
%                               ^^J \ \ \ \ \ \ \ MEN^^A
%                               ^^J WOMEN are^^A
%                               ^^J better MEN^^J \end{lstlisting}}
% \begin{tabular}{@{}c@{\qquad\quad}c@{\qquad\quad}c@{\qquad\quad}c@{}}
% {\rstyle\ikeyname{columns}}|=| & \texttt{fixed} & \texttt{flexible} & \texttt{fullflexible}\\
%          & (at {\makeatletter\lst@widthfixed})
%          & (at {\makeatletter\lst@widthflexible})
%          & (at {\makeatletter\lst@widthflexible})\\
% \noalign{\medskip}
%   \lstset{basicstyle=\ttfamily,basewidth=0.51em}\sample
% & \lstset{columns=fixed}\sample
% & \lstset{columns=flexible}\sample
% & \lstset{columns=fullflexible}\sample
% \end{tabular}
% \end{center}
% \begin{advise}
% \item Why are women better men?
%       \advisespace
%       Do you want to philosophize? Well, have I ever said that the
%       statement ``women are better men'' is true? I can't even remember this
%       about ``women are men'' \ldots . ^^A ;-)
% \end{advise}
% In the abstract one can say: The fixed column format ruins the spacing
% intended by the font designer, while the flexible formats ruin the column
% alignment (possibly) intended by the programmer. Common to all is that the
% input characters are translated into a sequence of basic output units like
% \begingroup \lstset{gobble=6,xleftmargin=\leftmargini}
% \makeatletter
%^^A  Make \fbox around each output unit.
% \fboxsep=0pt
% \def\lst@alloverstyle#1{\fbox{\kern-\fboxrule\strut#1}\kern-\fboxrule}
% \begin{lstlisting}[basewidth=1em]
%     if x=y then write('align')
%            else print('align');
% \end{lstlisting}
% Now, the fixed format puts $n$ characters into a box of width $n\times{}
% $`base width', where the base width is {\makeatletter\lst@widthfixed} in the
% example. The format shrinks and stretches the space between the characters
% to make them fit the box. As shown in the example, some character strings look
%    \hbox to 2em{b\hss a\hss d}
% or
%    \hbox to 2em{w\hss o\hss r\hss s\hss e},
% but the output is vertically aligned.
% \endgroup
%
% If you don't need or like this, you should use a flexible format. All
% characters are typeset at their natural width. In particular, they never
% overlap. If a word requires more space than reserved, the rest of the line
% simply moves to the right. The difference between the three formats is that
% the full flexible format cares about nothing else, while the normal flexible
% and space-flexible formats try to fix the column alignment if a character
% string needs less space than `reserved'.  The normal flexible format will
% insert make-up space to fix the alignment at spaces, before and after
% identifiers, and before and after sequences of other characters; the
% space-flexible format will only insert make-up space by stretching
% existing spaces.  In the flexible example above, the two MENs are vertically
% aligned since some space has been inserted in the fourth line to fix the
% alignment. In the full flexible format, the two MENs are not aligned.
%
% Note that both flexible modes printed the two blanks in the first line as a
% single blank, but for different reasons: the normal flexible format fixes
% the column alignment (as would the space-flexible format), and the full
% flexible format doesn't care about the second space.
%
%
% \section{Advanced techniques}\label{uAdvancedTechniques}
%
%
% \subsection{Style definitions}
%
% It is obvious that a pretty-printing tool like this requires some kind of
% language selection and definition. The first has already been described and
% the latter is convered by the next section. However, it is very convenient
% to have the same for printing styles: at a central place of your document
% they can be modified easily and the changes take effect on all listings.
%
% Similar to languages,
%    {\rstyle\ikeyname{style}}|=|\meta{style name}
% activates a previously defined style. A definition is as easy:
%    {\rstyle|\lstdefinestyle|}\marg{style name}\marg{key=value list}.
% Keys not used in such a definition are untouched by the corresponding style
% selection, of course. For example, you could write
% \begin{verbatim}
%   \lstdefinestyle{numbers}
%       {numbers=left, stepnumber=1, numberstyle=\tiny, numbersep=10pt}
%   \lstdefinestyle{nonumbers}
%       {numbers=none}\end{verbatim}
% and switch from listings with line numbers to listings without ones and vice
% versa simply by |style=nonumbers| and |style=numbers|, respectively.
% \begin{advise}
% \item You could even write
%           `|\lstdefinestyle{C++}{language=C++,style=numbers}|'.
%       Style and language names are independent of each other and so might
%       coincide. Moreover it is possible to activate other styles.
%
% \item It's easy to crash the package using styles. Write
%       '|\lstdefinestyle{crash}{style=crash}|' and '|\lstset{style=crash}|'.
%       \TeX's capacity will exceed, sorry [parameter stack size]. Only bad
%       boys use such recursive calls, but only good girls use this package.
%       Thus the problem is of minor interest.^^A :-)
% \end{advise}
%
%
% \subsection{Language definitions}\label{uLanguageDefinitions}
%
% These are like style definitions except for an optional dialect name and an
% optional base language---and, of course, a different command name and
% specialized keys. In the simple case it's
%    {\rstyle|\lstdefinelanguage|}\marg{language name}\marg{key=value list}.
% For many programming languages it is sufficient to specify keywords and
% standard function names, comments, and strings. Let's look at an example.
% \begin{lstxsample}[morekeywords,sensitive,morecomment,morestring]
%    \lstdefinelanguage{rock}
%      {morekeywords={one,two,three,four,five,six,seven,eight,
%          nine,ten,eleven,twelve,o,clock,rock,around,the,tonight},
%       sensitive=false,
%       morecomment=[l]{//},
%       morecomment=[s]{/*}{*/},
%       morestring=[b]",
%      }
% \end{lstxsample}
% \begingroup \csname lst@EndWriteFile\endcsname
% \bigbreak
%
% \noindent
% There isn't much to say about keywords. They are defined like identifiers
% you want to emphasize. Additionally you need to specify whether they are
% case sensitive or not. And yes: you could insert |[2]| in front of the
% keyword \texttt{one} to define the keywords as `second order' and print them
% in |keywordstyle={[2]...}|.
% \begin{advise}
% \item I get a `\texttt{Missing = inserted for }|\ifnum|' error when I select
%       my language.
%       \advisespace
%       Did you forget the comma after `|keywords={...}|'? And if you encounter
%       unexpected characters after selecting a language (or style), you have
%       probably forgotten a different comma or you have given to many
%       arguments to a key, for example, |morecomment=[l]{--}{!}|.
% \end{advise}
%
% So let's turn to comments and strings. Each value starts with a
% \emph{mandatory} \oarg{type} argument followed by a changing number of
% opening and closing delimiters. Note that each delimiter (pair) requires a
% key=value on its own, even if types are equal. Hence, you'll need to insert
% \texttt{morestring=[b]'} if single quotes open and close string or character
% literals in the same way as double quotes do in the example.
%
% Eventually you need to know the types and their numbers of delimiters. The
% reference guide contains full lists, here we discuss only the most common.
% For strings these are {\rstyle\texttt{b}} and {\rstyle\texttt{d}} with one
% delimiter each. This delimiter opens and closes the string and inside a
% string it is either escaped by a \texttt backslash or it is \texttt doubled.
% The comment type {\rstyle\texttt{l}} requires exactly one delimiter, which
% starts a comment on any column. This comment goes up to the end of line.
% The other two most common comment types are {\rstyle\texttt{s}} and
% {\rstyle\texttt{n}} with two delimiters each. The first delimiter opens a
% comment which is terminated by the second delimiter. In contrast to the
% \texttt s-type, \texttt n-type comments can be nested.
% \begin{lstxsample}[b,d,l,s,n]
%    \lstset{morecomment=[l]{//},
%            morecomment=[s]{/*}{*/},
%            morecomment=[n]{(*}{*)},
%            morestring=[b]",
%            morestring=[d]'}
% \end{lstxsample}
% \begin{lstsample}{}{}
%    \begin{lstlisting}
%    "str\"ing "    not a string
%    'str''ing '    not a string
%    // comment line
%    /* comment/**/ not a comment
%    (* nested (**) still comment
%       comment  *) not a comment
%    \end{lstlisting}
% \end{lstsample}
% \begin{advise}
% \item Is it \emph{that} easy?
%       \advisespace
%       Almost. There are some troubles you can run into. For example, if
%       `\texttt{-*}' starts a comment line and `\texttt{-*-}' a string
%       (unlikely but possible), then you must define the shorter delimiter
%       first.
%       Another problem: by default some characters are not allowed inside
%       keywords, for example `\texttt{-}', `\texttt{:}', `\texttt{.}', and
%       so on. The reference guide covers this problem by introducing some
%       more keys, which let you adjust the standard character table
%       appropriately. But note that white space characters are prohibited
%       inside keywords.
% \end{advise}
% Finally remember that this section is only an introduction to language
% definitions. There are more keys and possibilities.
%
%
% \subsection{Delimiters}\label{uDelimiters}
%
% You already know two special delimiter classes: comments and strings.
% However, their full syntax hasn't been described so far. For example,
% \ikeyname{commentstyle} applies to all comments---unless you specify
% something different. The \emph{optional} \oarg{style} argument follows the
% \emph{mandatory} \oarg{type} argument.
%\ifcolor
% \begin{lstxsample}
%    \lstset{morecomment=[l][keywordstyle]{//},
%            morecomment=[s][\color{white}]{/*}{*/}}
% \end{lstxsample}
%\else
% \begin{lstxsample}
%    \lstset{morecomment=[l][keywordstyle]{//},
%            morecomment=[s][\underbar]{/*}{*/}}
% \end{lstxsample}
%\fi
% \begin{lstsample}{}{}
%    \begin{lstlisting}
%    // bold comment line
%    a single /* comment */
%    \end{lstlisting}
% \end{lstsample}
% As you can see, you have the choice between specifying the style explicitly
% by \LaTeX\ commands or implicitly by other style keys. But, you're right,
% some implicitly defined styles have no seperate keys, for example the second
% order keyword style. Here---and never with the number 1---you just append
% the order to the base key: \texttt{keywordstyle2}.
%
% You ask for an application? Here you are: one can define different printing
% styles for `subtypes' of a comment, for example
%\ifcolor
% \begin{lstxsample}
%    \lstset{morecomment=[s][\color{blue}]{/*+}{*/},
%            morecomment=[s][\color{red}]{/*-}{*/}}
% \end{lstxsample}
%\else
% \begin{lstxsample}
%    \lstset{morecomment=[s][\upshape]{/*+}{*/},
%            morecomment=[s][\bfseries]{/*-}{*/}}
% \end{lstxsample}
%\fi
% \begin{lstsample}{\lstset{morecomment=[s]{/*}{*/}}}{}
%    \begin{lstlisting}
%    /*  normal comment */
%    /*+    keep cool   */
%    /*-     danger!    */
%    \end{lstlisting}
% \end{lstsample}
% Here, the comment style is not applied to the second and third line.
% \begin{advise}
% \item Please remember that both `extra' comments must be defined \emph{after}
%       the normal comment, since the delimiter `\texttt{/*}' is a substring of
%       `\texttt{/*+}' and `\texttt{/*-}'.
%
% \item I have another question. Is `\texttt{language=}\meta{different
%       language}' the only way to remove such additional delimiters?
%       \advisespace
%       Call {\rstyle\ikeyname{deletecomment}} and/or
%       {\rstyle\ikeyname{deletestring}} with the same arguments to remove
%       the delimiters (but you don't need to provide the optional style
%       argument).
% \end{advise}
% Eventually, you might want to use the prefix \texttt{i} on any comment type.
% Then the comment is not only invisible, it is completely discarded from the
% output!
% \begin{lstxsample}[is]
%    \lstset{morecomment=[is]{/*}{*/}}
% \end{lstxsample}
% \begin{lstsample}{}{}
%    \begin{lstlisting}
%    begin /* comment */ end
%    begin/* comment */end
%    \end{lstlisting}
% \end{lstsample}
%
% Okay, and now for the real challenges. More general delimiters can be defined
% by the key {\rstyle\ikeyname{moredelim}}. Legal types are {\rstyle\texttt{l}}
% and {\rstyle\texttt{s}}. These types can be preceded by an \texttt{i}, but
% this time \emph{only the delimiters} are discarded from the output. This way
% you can select styles by markers.
% \begin{lstxsample}
%    \lstset{moredelim=[is][\ttfamily]{|}{|}}
% \end{lstxsample}
% \begin{lstsample}{}{}
%    \begin{lstlisting}
%    roman |typewriter|
%    \end{lstlisting}
% \end{lstsample}
% You can even let the package detect keywords, comments, strings, and other
% delimiters inside the contents.
% \begin{lstxsample}
%    \lstset{moredelim=*[s][\itshape]{/*}{*/}}
% \end{lstxsample}
% \begin{lstsample}{}{}
%    \begin{lstlisting}
%    /* begin
%      (* comment *)
%       ' string ' */
%    \end{lstlisting}
% \end{lstsample}
% Moreover, you can force the styles to be applied cumulatively.
% \begin{lstxsample}
%    \lstset{moredelim=**[is][\ttfamily]{|}{|}, % cumulative
%            moredelim=*[s][\itshape]{/*}{*/}}  % not so
% \end{lstxsample}
% \begin{lstsample}{}{}
%    \begin{lstlisting}
%    /* begin
%       ' string '
%       |typewriter| */
%
%    | begin
%     ' string '
%     /*typewriter*/ |
%    \end{lstlisting}
% \end{lstsample}
% Look carefully at the output and note the differences. The second
% \texttt{begin} is not printed in bold typewriter type since standard
% \LaTeX\ has no such font.
%
% This suffices for an introduction. Now go and find some more applications.
%
%
% \subsection{Closing and credits}\label{uClosingAndCredits}
%
% You've seen a lot of keys but you are far away from knowing all of them.
% The next step is the real use of the \packagename{listings} package.
% Please take the following advice. Firstly, look up the known commands and
% keys in the reference guide to get a notion of the notation there. Secondly,
% poke around with these keys to learn some other parameters. Then, hopefully,
% you'll be prepared if you encounter any problems or need some special things.
%
% \begin{advise}
% \item
% There is one question `you' haven't asked all the last pages: who is to
% blame. Carsten Heinz wrote the guides, coded the \packagename{listings}
% package and wrote some language drivers. Brooks Moses currently maintains
% the package.  Other people defined more languages
% or contributed their ideas; many others made bug reports, but only the first
% bug finder is listed.
%^^A
%^^A Thanks for error reports (first bug finder only), new programming
%^^A languages, etc.
%^^A Special thanks for communication which lead to kernel extensions, and to
%^^A Hendri Adriaens for reviving maintenance on the package.
%^^A
% Special thanks go to (alphabetical order)
% \begin{quote}
% \hyphenpenalty=10000\relax \rightskip=0pt plus \linewidth
%   \lstthanks{Hendri~Adriaens}{-},
%   \lstthanks{Andreas~Bartelt}{Andreas.Bartelt@Informatik.Uni-Oldenburg.DE},
%   \lstthanks{Jan~Braun}{Jan.Braun@tu-bs.de},
%   \lstthanks{Denis~Girou}{Denis.Girou@idris.fr},
%   \lstthanks{Arne~John~Glenstrup}{panic@diku.dk},
%   \lstthanks{Frank~Mittelbach}{frank.mittelbach@latex-project.org},
%   \lstthanks{Rolf~Niepraschk}{niepraschk@PTB.DE},
%   \lstthanks{Rui~Oliveira}{rco@di.uminho.pt},
%   \lstthanks{Jens~Schwarzer}{schwarzer@schwarzer.dk}, and
%   \lstthanks{Boris~Veytsman}{boris@plmsc.psu.edu}.
% \end{quote}
% Moreover we wish to thank
% \begin{quote}
% \hyphenpenalty=10000\relax \rightskip=0pt plus \linewidth
%   \lstthanks{Bj{\o}rn~{\AA}dlandsvik}{bjorn@imr.no},
%   \lstthanks{Omair-Inam~Abdul-Matin}{-},
%   \lstthanks{Gaurav~Aggarwal}{gaurav@ics.uci.edu},
%   \lstthanks{Jason~Alexander}{jalex@ea.oac.uci.edu},
%   \lstthanks{Andrei~Alexandrescu}{-},
%   \lstthanks{Holger~Arndt}{-},
%   \lstthanks{Donald~Arseneau}{ASND@erich.triumf.ca},
%   \lstthanks{David~Aspinall}{David.Aspinall@ed.ac.uk},
%   \lstthanks{Frank~Atanassow}{-},
%   \lstthanks{Claus~Atzenbeck}{Claus.Atzenbeck@stud.uni-regensburg.de},
%   \lstthanks{Michael~Bachmann}{-},
%   \lstthanks{Luca~Balzerani}{-},
%   \lstthanks{Peter~Bartke}{bartke@inf.fu-berlin.de} (big thankyou), ^^A beta tester
%   \lstthanks{Heiko~Bauke}{-},
%   \lstthanks{Oliver~Baum}{oli.baum@web.de},
%   \lstthanks{Ralph~Becket}{rbeck@microsoft.com},
%   \lstthanks{Andres~Becerra~Sandoval}{abecerra@univalle.edu.co},
%   \lstthanks{Kai~Below}{below@tu-harburg.de},
%   \lstthanks{Matthias~Bethke}{-},
%   \lstthanks{Javier~Bezos}{javier.bezos@bancoval.es},
%   \lstthanks{Olaf~Trygve~Berglihn}{olafb@pvv.org}, ^^A {1999/11/29}{3-char comment delimiter don't work (Python)}
%   \lstthanks{Geraint~Paul~Bevan}{geraint@users.sf.net},
%   \lstthanks{Peter~Biechele}{peter.biechele@physik.uni-freiburg.de},
%   \lstthanks{Beat~Birkhofer}{beat@birkhofer.ch},
%   \lstthanks{Fr\'ed\'eric~Boulanger}{Frederic.Boulanger@supelec.fr},
%   \lstthanks{Joachim~Breitner}{-},
%   \lstthanks{Martin~Brodbeck}{Martin.Brodbeck@gmx.de},
%   \lstthanks{Walter~E.~Brown}{WB@fnal.gov},
%   \lstthanks{Achim~D.~Brucker}{brucker@informatik.uni-freiburg.de},
%   \lstthanks{J\'an Bu\v{s}a}{-},
%   \lstthanks{Thomas~ten~Cate}{-},
%   \lstthanks{David~Carlisle}{davidc@nag.co.uk},
%   \lstthanks{Bradford~Chamberlain}{brad@cs.washington.edu},
%   \lstthanks{Brian~Christensen}{-},
%   \lstthanks{Neil~Conway}{-},
%   \lstthanks{Patrick~Cousot}{Patrick.Cousot@wanadoo.fr},
%   \lstthanks{Xavier~Cr\'egut}{cregut@enseeiht.fr},
%   \lstthanks{Christopher~Creutzig}{-},
%   \lstthanks{Holger~Danielsson}{dani@fbg.schwerte.de},
%   \lstthanks{Andreas~Deininger}{deininger@uni-kassel.de},
%   \lstthanks{Robert~Denham}{Robert.Denham@dnr.qld.gov.au},
%   \lstthanks{Detlev~Dr\"oge}{droege@informatik.uni-koblenz.de},
%   \lstthanks{Anders~Edenbrandt}{Anders.Edenbrandt@dna.lth.se},
%   \lstthanks{Mark~van~Eijk}{mark@luon.net},
%   \lstthanks{Norbert~Eisinger}{Norbert.Eisinger@informatik.uni-muenchen.de},
%   \lstthanks{Brian~Elmegaard}{-},
%   \lstthanks{Jon~Ericson}{Jon.Ericson@jpl.nasa.gov},
%   \lstthanks{Thomas~Esser}{te@dbs.uni-hannover.de},
%   \lstthanks{Chris~Edwards}{edwch00p@infoscience.otago.ac.nz},
%   \lstthanks{David~John~Evans}{Matrix.Software@dial.pipex.com},
%   \lstthanks{Tanguy~Fautr\'e}{tfautre@pandora.be},
%   \lstthanks{Ulrike~Fischer}{-},
%   \lstthanks{Robert~Frank}{rf7@ukc.ac.uk},
%   \lstthanks{Michael~Franke}{-},
%   \lstthanks{Ignacio~Fern\'andez~Galv\'an}{-},
%   \lstthanks{Martine~Gautier}{-}
%   \lstthanks{Daniel~Gazard}{gazard_d@epita.fr},
%   \lstthanks{Daniel~Gerigk}{Daniel.Gerigk@ePost.de},
%   \lstthanks{Dr.~Christoph~Giess}{-},
%   \lstthanks{KP~Gores}{kp.gores@web.de},
%   \lstthanks{Adam~Grabowski}{adam@mizar.org},
%   \lstthanks{Jean-Philippe~Grivet}{grivet@cnrs-orleans.fr},
%   \lstthanks{Christian~Gudrian}{Christian.Gudrian@kawo1.rwth-aachen.de},
%   \lstthanks{Jonathan~de~Halleux}{dehalleux@auto.ucl.ac.be},
%   \lstthanks{Carsten~Hamm}{carsten.hamm@siemens.com},
%   \lstthanks{Martina~Hansel}{Martina.Hansel@fhtw-berlin.de},
%   \lstthanks{Harald~Harders}{h.harders@tu-bs.de},
%   \lstthanks{Christian~Haul}{haul@dvs1.informatik.tu-darmstadt.de},
%   \lstthanks{Aidan~Philip~Heerdegen}{Aidan.Heerdegen@anu.edu.au},
%   \lstthanks{Jim~Hefferon}{Hefferon9@aol.com},
%   \lstthanks{Heiko~Heil}{info@heiko-heil.de},
%   \lstthanks{J\"urgen~Heim}{heim@astro.uni-tuebingen.de},
%   \lstthanks{Martin~Heller}{-},
%   \lstthanks{Stephan~Hennig}{-},
%   \lstthanks{Alvaro~Herrera}{alvherre@dcc.uchile.cl},
%   \lstthanks{Richard~Hoefter}{hoefter@gmx.de},
%   \lstthanks{Dr.~Jobst~Hoffmann}{HOFFMANN@rz.rwth-aachen.de},
%   \lstthanks{Torben~Hoffmann}{toho@it.dtu.dk},
%   \lstthanks{Morten~H\o gholm}{-},
%   \lstthanks{Berthold~H\"ollmann}{bhoel@starship.python.net},
%   \lstthanks{G\'erard~Huet}{-},
%   \lstthanks{Hermann~H\"uttler}{hermann.huettler@gmx.net},
%   \lstthanks{Ralf~Imh\"auser}{snoopy@tribal.line.org},
%   \lstthanks{R.~Isernhagen}{R.Isernhagen@FH-Wolfenbuettel.DE},
%   \lstthanks{Oldrich~Jedlicka}{ojedlick@students.zcu.cz},
%   \lstthanks{Dirk~Jesko}{jesko@iti.cs.uni-magdeburg.de},
%   \lstthanks{Lo\"\i c~Joly}{-},
%   \lstthanks{Christian~Kaiser}{chk@combit.net},
%   \lstthanks{Bekir~Karaoglu}{karabekirus@yahoo.com},
%   \lstthanks{Marcin~Kasperski}{Marcin.Kasperski@softax.com.pl},
%   \lstthanks{Christian~Kindinger}{chkind@uni-wuppertal.de},
%   \lstthanks{Steffen~Klupsch}{steffen@vlsi.informatik.tu-darmstadt.de},
%   \lstthanks{Markus~Kohm}{-},
%   \lstthanks{Peter~K\"oller}{pkoeller@metaprojekt.de} (big thankyou), ^^A beta tester
%   \lstthanks{Reinhard~Kotucha}{Reinhard.Kotucha@web.de},
%   \lstthanks{Stefan~Lagotzki}{info@lagotzki.de},
%   \lstthanks{Tino~Langer}{langer@tournex.de},
%   \lstthanks{Rene~H.~Larsen}{rhl@traceroute.dk},
%   \lstthanks{Olivier~Lecarme}{ol@i3s.unice.fr},
%   \lstthanks{Thomas~Leduc}{Thomas.Leduc@lsv.ens-cachan.fr},
%   \lstthanks{Dr.~Peter~Leibner}{Peter.Leibner@sta.siemens.de},
%   \lstthanks{Thomas~Leonhardt}{leonhardt@informatik.tu-darmstadt.de} (big thankyou), ^^A beta tester
%   \lstthanks{Magnus~Lewis-Smith}{Magnus.Lewis-Smith@pace.co.uk},
%   \lstthanks{Knut~Lickert}{knut.lickert@gmx.de},
%   \lstthanks{Benjamin~Lings}{-},
%   \lstthanks{Dan~Luecking}{luecking@uark.edu},
%   \lstthanks{Peter~L\"offler}{-},
%   \lstthanks{Markus~Luisser}{-},
%   \lstthanks{Kris~Luyten}{no email available},
%   \lstthanks{Jos\'e~Romildo~Malaquias}{romildo@urano.iceb.ufop.br},
%   \lstthanks{Andreas~Matthias}{amat@kabsi.at},
%   \lstthanks{Patrick~TJ~McPhee}{ptjm@interlog.com},
%   ^^A \lstthanks{Brooks~Moses}{-},
%   \lstthanks{Riccardo~Murri}{riccardo.murri@gmx.it},
%   \lstthanks{Knut~M\"uller}{knut@physik3.gwdg.de},
%   \lstthanks{Svend~Tollak~Munkejord}{svendm@efisms.energy.sintef.no},
%   \lstthanks{Gerd~Neugebauer}{gerd.neugebauer@gmx.de},
%   \lstthanks{Torsten~Neuer}{tneuer@inwise.de},
%   \lstthanks{Enzo~Nicosia}{-},
%   \lstthanks{Michael~Niedermair}{m.g.n@gmx.de},
%   \lstthanks{Xavier~Noria}{fxn@hashref.com},
%   \lstthanks{Heiko~Oberdiek}{oberdiek@ruf.uni-freiburg.de},
%   \lstthanks{Xavier~Olive}{-},
%   \lstthanks{Alessio~Pace}{-},
%   \lstthanks{Markus~Pahlow}{pahlowm@mar.dfo-mpo.gc.ca},
%   \lstthanks{Morten~H.~Pedersen}{mhp@dadlnet.dk},
%   \lstthanks{Xiaobo~Peng}{-},
%   \lstthanks{Zvezdan~V.~Petkovic}{zpetkovic@acm.org},
%   \lstthanks{Michael~Piefel}{piefel@informatik.hu-berlin.de},
%   \lstthanks{Michael~Piotrowski}{mxp@linguistik.uni-erlangen.de},
%   \lstthanks{Manfred~Piringer}{sz0490@rrze.uni-erlangen.de},
%   \lstthanks{Vincent~Poirriez}{Vincent.Poirriez@univ-valenciennes.fr},
%   \lstthanks{Adam~Prugel-Bennett}{apb@ecs.soton.ac.uk},
%   \lstthanks{Ralf~Quast}{rquast@hs.uni-hamburg.de},
%   \lstthanks{Aslak~Raanes}{araanes@ifi.ntnu.no},
%   \lstthanks{Venkatesh~Prasad~Ranganath}{vranganath@cox.net},
%   \lstthanks{Tobias~Rapp}{-},
%   \lstthanks{Jeffrey~Ratcliffe}{-},
%   \lstthanks{Georg~Rehm}{Georg.Rehm@germanistik.uni-giessen.de},
%   \lstthanks{Fermin~Reig}{reig@ics.uci.edu},
%   \lstthanks{Detlef~Reimers}{dreimers@aol.com},
%   \lstthanks{Stephen~Reindl}{stephen.reindl@vodafone.com},
%   \lstthanks{Franz~Rinnerthaler}{-},
%   \lstthanks{Peter~Ruckdeschel}{Peter.Ruckdeschel@uni-bayreuth.de},
%   \lstthanks{Magne~Rudshaug}{magne@ife.no},
%   \lstthanks{Jonathan~Sauer}{jonathan.sauer@gmx.de},
%   \lstthanks{Vespe~Savikko}{vespe@cs.tut.fi},
%   \lstthanks{Mark~Schade}{-},
%   \lstthanks{Gunther~Schmidl}{gschmidl@gmx.at},
%   \lstthanks{Andreas~Schmidt}{-},
%   \lstthanks{Walter~Schmidt}{wschmi@arcor.de},
%   \lstthanks{Christian~Schneider}{-},
%   \lstthanks{Jochen~Schneider}{jschneider@ds3.etech.haw-hamburg.de},
%   \lstthanks{Benjamin~Schubert}{benjamin.schubert@berlin.de},
%   \lstthanks{Sebastian~Schubert}{-},
%   \lstthanks{Uwe~Siart}{uwe.siart@ei.tum.de},
%   \lstthanks{Axel~Sommerfeldt}{axel@sommerfeldt.net},
%   \lstthanks{Richard~Stallman}{-},
%   \lstthanks{Nigel~Stanger}{nstanger@infoscience.otago.ac.nz},
%   \lstthanks{Martin~Steffen}{ms@informatik.uni-kiel.de},
%   \lstthanks{Andreas~Stephan}{Andreas.Stephan@victoria.de},
%   \lstthanks{Stefan~Stoll}{stoll@phys.chem.ethz.ch},
%   \lstthanks{Enrico~Straube}{no email available},
%   \lstthanks{Werner~Struckmann}{struck@ips.cs.tu-bs.de},
%   \lstthanks{Martin~S\"u\ss kraut}{Edon.Myder@web.de},
%   \lstthanks{Gabriel~Tauro}{gabriel@informatik.uni-jena.de},
%   \lstthanks{Winfried~Theis}{theis@statistik.uni-dortmund.de},
%   \lstthanks{Jens~T.~Berger~Thielemann}{jensthi@ifi.uio.no},
%   \lstthanks{William~Thimbleby}{-},
%   \lstthanks{Arnaud~Tisserand}{arnaud.tisserand@ens-lyon.fr},
%   \lstthanks{Jens~Troeger}{-},
%   \lstthanks{Kalle~Tuulos}{kalle.tuulos@nic.fi},
%   \lstthanks{Gregory~Van~Vooren}{Gregory.VanVooren@rug.ac.be},
%   \lstthanks{Timothy~Van~Zandt}{tvz@econ.insead.edu},
%   \lstthanks{J\"org~Viermann}{-},
%   \lstthanks{Thorsten~Vitt}{vitt@informatik.hu-berlin.de},
%   \lstthanks{Herbert~Voss}{voss@perce.de} (big thankyou), ^^A beta tester
%   \lstthanks{Edsko~de~Vries}{devriese@tcd.ie},
%   \lstthanks{Herfried~Karl~Wagner}{hirf@gmx.at},
%   \lstthanks{Dominique~de~Waleffe}{ddw@miscrit.be},
%   \lstthanks{Bernhard~Walle}{-},
%   \lstthanks{Jared~Warren}{warren@cs.queensu.ca},
%   \lstthanks{Michael~Weber}{mweber@informatik.hu-berlin.de},
%   \lstthanks{Sonja~Weidmann}{Sonja.Weidmann@gmx.de},
%   \lstthanks{Andreas~Weidner}{-},
%   \lstthanks{Herbert~Weinhandl}{weinhand@grz08u.unileoben.ac.at},
%   \lstthanks{Robert~Wenner}{robert.wenner@gmx.de},
%   \lstthanks{Michael~Wiese}{wiese@itwm.uni-kl.de},
%   \lstthanks{James~Willans}{-},
%   \lstthanks{J\"orn~Wilms}{wilms@rocinante.colorado.edu},
%   \lstthanks{Kai~Wollenweber}{kai@ece.WPI.EDU},
%   \lstthanks{Ulrich~G.~Wortmann}{uliw@erdw.ethz.ch},
%   \lstthanks{Cameron~H.G.~Wright}{-},
%   \lstthanks{Andrew~Zabolotny}{-}, and
%   \lstthanks{Florian~Z\"ahringer}{-}.
% \end{quote}
% There are probably other people who contributed to this package.
% If I've missed your name, send an email.
% \end{advise}
%
%
% \part{Reference guide}
%
%
% \section{Main reference}\label{rMainReference}
%
% Your first training is completed. Now that you've left the User's guide, the
% friend telling you what to do has gone. Get more practice and become a
% journeyman!^^A :-)
% \begin{advise}
% \item Actually, the friend hasn't gone. There are still some advices, but
%       only from time to time.
% \end{advise}
%
%
% \subsection{How to read the reference}
%
% Commands, keys and environments are presented as follows.
% \begin{syntax}
% \item[1.0,default,hints] \texttt{command}, \texttt{environment} or
%       \keyname{key} with \meta{parameters}
%
%       This field contains the explanation; here we describe the other fields.
%
%       If present, the label in the left margin provides extra information:
%       `\textit{addon}' indicates additionally introduced functionality,
%       `\textit{changed}' a modified key, `\textit{data}' a command just
%       containing data (which is therefore adjustable via |\renewcommand|),
%       and so on. Some keys and functionality are `\emph{bug}'-marked or
%       with a \dag-sign. These features might change in future or could be
%       removed, so use them with care.
%
%       If there is verbatim text touching the right margin, it is the
%       predefined value. Note that some keys default to this value every
%       listing, namely the keys which can be used on individual listings only.
% \end{syntax}
% Regarding the parameters, please keep in mind the following:
% \begin{enumerate}
% \item A list always means a comma separated list. You must put braces around
%       such a list. Otherwise you'll get in trouble with the
%       \packagename{keyval} package; it complains about an undefined key.
% \item You must put parameter braces around the whole value of a key if you
%       use an \oarg{optional argument} of a key inside an optional
%       \oarg{key=value list}:
%       |\begin{lstlisting}[caption=|{\rstyle|{|}|[one]two|{\rstyle|}|}|]|.
% \item Brackets `|[ ]|' usually enclose optional arguments and must be typed
%       in verbatim. Normal brackets `[ ]' always indicate an optional argument
%       and must not be typed in. Thus |[*]| must be typed in exactly as is,
%       but [|*|] just gets |*| if you use this argument.
% \item A vertical rule indicates an alternative, e.g.~^^A
%       \meta{\alternative{true,false}} allows either \texttt{true} or
%       \texttt{false} as arguments.
% \item If you want to enter one of the special characters |{}#%\|, this
%       character must be escaped with a backslash. This means that you must
%       write |\}| for the single character `right brace'---but of course not
%       for the closing paramater character.
% \end{enumerate}
%
%
% \subsection{Typesetting listings}\label{rTypesettingListings}
%
% \begin{syntax}
% \item[0.19] \rcmdname\lstset\marg{key=value list}
%
%       sets the values of the specified keys, see also section
%       \ref{uTheKey=ValueInterface}.
%       The parameters keep their values up to the end of the current group.
%       In contrast, all optional \meta{key=value list}s below modify the
%       parameters for single listings only.
%
% \item[0.18] \rcmdname\lstinline\oarg{key=value list}\meta{character}\meta{source code}\meta{same character}
%
%       works like |\verb| but respects the active language and style. These
%       listings use flexible columns unless requested differently in the
%       optional argument, and do not support frames or background colors.
%       You can write `|\lstinline!var i:integer;!|' and get
%       `\lstinline!var i:integer;!'.
%
%       Since the command first looks ahead for an optional argument, you must
%       provide at least an empty one if you want to use |[| as
%       \meta{character}.
%
%       \dag\ An experimental implementation has been done to support the
%       syntax |\lstinline|\oarg{key=value list}\marg{source code}. Try it if
%       you want and report success and failure. A known limitation is that
%       inside another argument the last source code token must not be an
%       explicit space token---and, of course, using a listing inside another
%       argument is itself experimental, see section
%       \ref{rListingsInsideArguments}.
%
%       Another limitation is that this feature can't be used in cells of a
%       |tabular|-environment. See \section{uListingsArguments} for a
%       workaround.
%
%       See also section \ref{rShortInline} for commands to create short analogs
%       for the |\lstinline| command.
%
% \item[0.15] |\begin{|\texttt{\rstyle lstlisting}|}|\oarg{key=value list}
%
%       \leavevmode\hspace*{-\leftmargini}|\end{|\texttt{\rstyle lstlisting}|}|
%
%       typesets the code in between as a displayed listing.
%
%       In contrast to the environment of the \packagename{verbatim} package,
%       \LaTeX\ code on the same line and after the end of environment is
%       typeset respectively executed.
%
% \item[0.1] \rcmdname\lstinputlisting\oarg{key=value list}\marg{file name}
%
%       typesets the stand alone source code file as a displayed listing.
% \end{syntax}
%
%
% \subsection{Space and placement}
%
% \begin{syntax}
% \item[0.20,floatplacement] \rkeyname{float}|=|[|*|]\meta{subset of \textup{\texttt{tbph}}}\syntaxor\rkeyname{float}
%
%       makes sense on individual displayed listings only and lets them float.
%       The argument controls where \LaTeX\ is \emph{allowed} to put the float:
%       at the top or bottom of the current/next page, on a separate page, or
%       here where the listing is.
%
%       The optional star can be used to get a double-column float in a
%       two-column document.
%
% \item[0.21,tbp] \rkeyname{floatplacement}|=|\meta{place specifiers}
%
%       is used as place specifier if \keyname{float} is used without value.
%
% \item[0.21,\medskipamount] \rkeyname{aboveskip}|=|\meta{dimension}
% \item[0.21,\medskipamount] \rkeyname{belowskip}|=|\meta{dimension}
%
%       define the space above and below displayed listings.
%
% \item[0.17,0pt,\dag] \rkeyname{lineskip}|=|\meta{dimension}
%
%       specifies additional space between lines in listings.
%
% \item[0.18,c,\dag] \rkeyname{boxpos}|=|\meta{\alternative{b,c,t}}
%
%       Sometimes the \packagename{listings} package puts a |\hbox| around a
%       listing---or it couldn't be printed or even processed correctly.
%       The key determines the vertical alignment to the surrounding material:
%       bottom baseline, centered or top baseline.
% \end{syntax}
%
%
% \subsection{The printed range}
%
% \begin{syntax}
% \item[0.12,true] \rkeyname{print}|=|\meta{\alternative{true,false}}\syntaxor\rkeyname{print}
%
%       controls whether an individual displayed listing is typeset. Even if
%       set false, the respective caption is printed and the label is defined.
%
%       Note: If the package is loaded without the \texttt{draft} option, you
%       can use this key together with |\lstset|. In the other case the key
%       can be used to typeset particular listings despite using the
%       \texttt{draft} option.
%
% \item[0.1,1] \rkeyname{firstline}|=|\meta{number}
% \item[0.1,9999999] \rkeyname{lastline}|=|\meta{number}
%
%       can be used on individual listings only. They determine the physical
%       input lines used to print displayed listings.
%
% \item[1.2] \rkeyname{linerange}|={|\meta{first1}\texttt-\meta{last1}\texttt,\meta{first2}\texttt-\meta{last2}\texttt, and so on|}|\label{uoption:linerange}
%
%       can be used on individual listings only. The given line ranges
%       of the listing are displayed. The intervals must be sorted and must
%       not intersect.
%
% \item[0.20,false] \rkeyname{showlines}|=|\meta{\alternative{true,false}}\syntaxor\rkeyname{showlines}
%
%       If true, the package prints empty lines at the end of listings.
%       Otherwise these lines are dropped (but they count for line numbering).
%
% \item[1.0] \rkeyname{emptylines}|=|[|*|]\meta{number}
%
%       sets the maximum of empty lines allowed. If there is a block of more
%       than \meta{number} empty lines, only \meta{number} ones are printed.
%       Without the optional star, line numbers can be disturbed when blank
%       lines are omitted; with the star, the lines keep their original
%       numbers.
%
% \item[0.19,0] \rkeyname{gobble}|=|\meta{number}
%
%       gobbles \meta{number} characters at the beginning of each
%       \emph{environment} code line. This key has no effect on \cs{lstinline}
%       or \cs{lstinputlisting}.
%
%       Tabulators expand to \ikeyname{tabsize} spaces before they are gobbled.
%       Code lines with fewer than \ikeyname{gobble} characters are considered
%       empty.  Never indent the end of environment by more characters.
% \end{syntax}
%
%
% \subsection{Languages and styles}\label{rLanguagesAndStyles}
%
% Please note that the arguments \meta{language}, \meta{dialect}, and
% \meta{style name} are case insensitive and that spaces have no effect.
% \begin{syntax}
% \item[0.18,{{}}] \rkeyname{style}|=|\meta{style name}
%
%       activates the key=value list stored with |\lstdefinestyle|.
%
% \item[0.19] \rcmdname\lstdefinestyle\marg{style name}\marg{key=value list}
%
%       stores the key=value list.
%
% \item[0.17,{{}}] \rkeyname{language}|=|\oarg{dialect}\meta{language}
%
%       activates a (dialect of a) programming language. The `empty' default
%       language detects no keywords, no comments, no strings, and so on; it
%       may be useful for typesetting plain text.
%       If \meta{dialect} is not specified, the package chooses the default
%       dialect, or the empty dialect if there is no default dialect.
%
%       Table \ref{uPredefinedLanguages} on page \pageref{uPredefinedLanguages}
%       lists all languages and dialects provided by \texttt{lstdrvrs.dtx}.
%       The predefined default dialects are underlined.
%
% \item[0.21] \rkeyname{alsolanguage}|=|\oarg{dialect}\meta{language}
%
%       activates a (dialect of a) programming language in addition to the
%       current active one. Note that some language definitions interfere with
%       each other and are plainly incompatible; for instance, if one is case
%       sensitive and the other is not.
%
%       Take a look at the \ikeyname{classoffset} key in section
%       \ref{rFigureOutTheAppearance} if you want to highlight the keywords
%       of the languages differently.
%
% \item[0.19] \rkeyname{defaultdialect}|=|\oarg{dialect}\meta{language}
%
%       defines \meta{dialect} as default dialect for \meta{language}.
%       If you have defined a default dialect other than empty, for example
%       |defaultdialect=[iama]fool|, you can't select the empty dialect, even
%       not with |language=[]fool|.
% \end{syntax}
%
% Finally, here's a small list of language-specific keys.
% \begin{syntax}
% \item[0.19,false,optional] \rkeyname{printpod}|=|\meta{\alternative{true,false}}
%
%       prints or drops PODs in Perl.
%
% \item[0.20,true,{renamed,optional}] \rkeyname{usekeywordsintag}|=|\meta{\alternative{true,false}}\label{uoption:usekeywordsintag}
%
%       The package either use the first order keywords in tags or prints all
%       identifiers inside |<>| in keyword style.
%
% \item[1.1,{{}},optional] \rkeyname{tagstyle}|=|\meta{style}\label{uoption:tagstyle}
%
%       determines the style in which tags and their content is printed.
%
% \item[1.1,false,optional] \rkeyname{markfirstintag}|=|\meta{style}\label{uoption:markfirstintag}
%
%       prints the first name in tags with keyword style.
%
% \item[0.20,true,optional] \rkeyname{makemacrouse}|=|\meta{\alternative{true,false}}
%
%       Make specific: Macro use of identifiers, which are defined as first
%       order keywords, also prints the surrounding |$(| and |)| in keyword
%       style. e.g.~you could get
%           \textbf{\textdollar(}\textbf{strip} \textdollar(BIBS)\textbf{)}.
%       If deactivated you get
%           \textdollar(\textbf{strip} \textdollar(BIBS)).
% \end{syntax}
%
%
% \subsection{Figure out the appearance}\label{rFigureOutTheAppearance}
%
% \begin{syntax}
% \item[0.18,{{}}] \rkeyname{basicstyle}|=|\meta{basic style}
%
%       is selected at the beginning of each listing. You could use
%       |\footnotesize|, |\small|, |\itshape|, |\ttfamily|, or something like
%       that. The last token of \meta{basic style} must not read any following
%       characters.
%
% \item[0.18,{{}}] \rkeyname{identifierstyle}|=|\meta{style}
% \item[0.11,\itshape] \rkeyname{commentstyle}|=|\meta{style}
% \item[0.12,{{}}] \rkeyname{stringstyle}|=|\meta{style}
%
%       determines the style for non-keywords, comments, and strings. The
%       \emph{last} token can be an one-parameter command like |\textbf| or
%       |\underbar|.
%
% \item[0.11,\bfseries,addon] \rkeyname{keywordstyle}|=|\oarg{number}[\textasteriskcentered]\meta{style}\label{roption:keywordstyle}
%
%       is used to print keywords.  The optional \meta{number} argument is the
%       class number to which the style should be applied.
%
%       Add-on: If you use the optional star after the (optional) class number, the
%       keywords are printed uppercase\,---\,even if a language is case
%       sensitive and defines lowercase keywords only. Maybe there should also be an
%       option for lowercase keywords \ldots
%
% \item[0.19,keywordstyle,deprecated] \rkeyname{ndkeywordstyle}|=|\meta{style}
%
%       is equivalent to |keywordstyle=2|\meta{style}.
%
% \item[1.0,0] \rkeyname{classoffset}|=|\meta{number}
%
%       is added to all class numbers before the styles, keywords, identifiers,
%       etc.~are assigned. The example below defines the keywords directly;
%       you could do it indirectly by selecting two different languages.
% \end{syntax}
%\ifcolor
% \begin{lstxsample}
%    \lstset{classoffset=0,
%            morekeywords={one,three,five},keywordstyle=\color{red},
%            classoffset=1,
%            morekeywords={two,four,six},keywordstyle=\color{blue},
%            classoffset=0}% restore default
% \end{lstxsample}
%\else
% \begin{lstxsample}
%    \lstset{classoffset=0,
%            morekeywords={one,three,five},keywordstyle=\itshape,
%            classoffset=1,
%            morekeywords={two,four,six},keywordstyle=\bfseries},
%            classoffset=0}% restore default
% \end{lstxsample}
%\fi
% \begin{lstsample}{}{}
%    \begin{lstlisting}
%    one two three
%    four five six
%    \end{lstlisting}
% \end{lstsample}
%
% \begin{syntax}
% \item[0.20,keywordstyle,{addon,bug,optional}] \rkeyname{texcsstyle}|=|[|*|]\oarg{class number}\meta{style}\label{roption:texcsstyle}
% \item[0.20,keywordstyle,optional] \rkeyname{directivestyle}|=|\meta{style}
%
%       determine the style of \TeX\ control sequences and directives.
%       Note that these keys are present only if you've chosen an appropriate
%       language.
%
%       The optional star of |texcsstyle| also highlights the backslash in
%       front of the control sequence name. Note that this option is set for
%       all |texcs| lists.
%
%       Bug: \texttt{texcs\ldots} interferes with other keyword lists. If, for
%       example, \texttt{emph} contains the word \texttt{foo}, then the control
%       sequence |\foo| will show up in \texttt{emphstyle}.
%
% \item[0.21] \rkeyname{emph}|=|\oarg{number}\marg{identifier list}
% \item[0.21] \rkeyname{moreemph}|=|\oarg{number}\marg{identifier list}
% \item[0.21] \rkeyname{deleteemph}|=|\oarg{number}\marg{identifier list}
% \item[0.21] \rkeyname{emphstyle}|=|\oarg{number}\marg{style}
%
%       respectively define, add or remove the \meta{identifier list} from
%       `emphasize class \meta{number}', or define the style for that class.
%       If you don't give an optional argument, the package assumes
%       \meta{number}$\,=1$.
%
%       These keys are described more detailed in section
%       \ref{uEmphasizeIdentifiers}.
%
% \item[1.0] \rkeyname{delim}|=|[\texttt*[\texttt*]]\texttt[\meta{type}\texttt][\texttt[\meta{style}\texttt]]\meta{delimiter\textup(s\textup)}
% \item[1.0] \rkeyname{moredelim}|=|[\texttt*[\texttt*]]\texttt[\meta{type}\texttt][\texttt[\meta{style}\texttt]]\meta{delimiter\textup(s\textup)}
% \item[1.0] \rkeyname{deletedelim}|=|[\texttt*[\texttt*]]\texttt[\meta{type}\texttt]\meta{delimiter\textup(s\textup)}
%
%       define, add, or remove user supplied delimiters.  (Note that this does
%       not affect strings or comments.)
%
%       In the first two cases \meta{style} is used to print the delimited
%       code (and the delimiters). Here, \meta{style} could be something like
%       |\bfseries| or |\itshape|, or it could refer to other styles via
%       \texttt{keywordstyle}, \texttt{keywordstyle2}, \texttt{emphstyle},
%       etc.
%
%       Supported types are \texttt{l} and \texttt{s}, see the comment keys in
%       section \ref{uLanguageDefinitions} for an explanation. If you use the
%       prefix \texttt i, i.e.~\texttt{il} or \texttt{is}, the delimiters are
%       not printed, which is some kind of invisibility.
%
%       If you use one optional star, the package will detect keywords,
%       comments, and strings inside the delimited code. With both optional
%       stars, aditionally the style is applied cumulatively; see section
%       \ref{uDelimiters}.
% \end{syntax}
%
%
% \subsection{Getting all characters right}
%
% \begin{syntax}
% \item[0.18,true] \rkeyname{extendedchars}|=|\meta{\alternative{true,false}}\syntaxor\rkeyname{extendedchars}
%
%       allows or prohibits extended characters in listings, that means
%       (national) characters of codes 128--255. If you use extended
%       characters, you should load \packagename{fontenc} and/or
%       \packagename{inputenc}, for example.
%
% \item[1.0,{{}}] \rkeyname{inputencoding}|=|\meta{encoding}
%
%       determines the input encoding. The usage of this key requires the
%       \packagename{inputenc} package; nothing happens if it's not loaded.
%
% \item[1.1,false] \rkeyname{upquote}|=|\meta{\alternative{true,false}}\label{uoption:upquote}
%
%       determines whether the left and right quote are printed |`'| or
%       \texttt{\textasciigrave\textquotesingle}.
%       This key requires the \packagename{textcomp} package if true.
%
% \item[0.12,8] \rkeyname{tabsize}|=|\meta{number}
%
%       sets tabulator stops at columns $\meta{number}+1$, $2\cdot\meta{number}+1$, $3\cdot\meta{number}+1$, and so on.
%       Each tabulator in a listing moves the current column to the next
%       tabulator stop.
%
% \item[0.20,false] \rkeyname{showtabs}|=|\meta{\alternative{true,false}}
%
%       make tabulators visible or invisible. A visible tabulator looks like
%       \lstinline[showtabs]!	!, but that can be changed. If you choose
%       invisible tabulators but visible spaces, tabulators are converted to
%       an appropriate number of spaces.
%
% \item[0.20] \rkeyname{tab}|=|\meta{tokens}
%
%       \meta{tokens} is used to print a visible tabulator. You might want to use |$\to$|, |$\mapsto$|, |$\dashv$| or something like that instead of the strange default definition.
%
% \item[0.20,false] \rkeyname{showspaces}|=|\meta{\alternative{true,false}}
%
%       lets all blank spaces appear {\textvisiblespace} or as blank spaces.
%
% \item[0.12,true] \rkeyname{showstringspaces}|=|\meta{\alternative{true,false}}
%
%       lets blank spaces in strings appear {\textvisiblespace} or as blank
%       spaces.
%
% \item[0.19,\bigbreak] \rkeyname{formfeed}|=|\meta{tokens}
%
%       Whenever a listing contains a form feed, \meta{tokens} is executed.
% \end{syntax}
%
%
% \subsection{Line numbers}\label{rLineNumbers}
%
% \begin{syntax}
% \item[1.0,none] \rkeyname{numbers}|=|\meta{\alternative{none,left,right}}
%
%       makes the package either print no line numbers, or put them on the
%       left or the right side of a listing.
%
% \item[0.16,1] \rkeyname{stepnumber}|=|\meta{number}
%
%       All lines with ``line number $\equiv 0$ modulo \meta{number}'' get a
%       line number.
%       If you turn line numbers on and off with \keyname{numbers}, the
%       parameter \keyname{stepnumber} will keep its value. Alternatively you
%       can turn them off via |stepnumber=0| and on with a nonzero number, and
%       keep the value of \keyname{numbers}.
%
% \item[1.1,false] \rkeyname{numberfirstline}|=|\meta{\alternative{true,false}}\label{uoption:numberfirstline}
%
%       The first line of each listing gets numbered (if numbers are on at all)
%       even if the line number is not divisible by \keyname{stepnumber}.
%
% \item[0.16,{{}}] \rkeyname{numberstyle}|=|\meta{style}
%
%       determines the font and size of the numbers.
%
% \item[0.19,10pt] \rkeyname{numbersep}|=|\meta{dimension}
%
%       is the distance between number and listing.
%
% \item[1.0,true] \rkeyname{numberblanklines}|=|\meta{\alternative{true,false}}
%
%       If this is set to false, blank lines get no printed line number.
%
% \item[0.20,auto] \rkeyname{firstnumber}|=|\meta{\alternative{auto,last,\normalfont\meta{number}}}
%
%       \texttt{auto} lets the package choose the first number: a new listing
%       starts with number one, a named listing continues the most recent
%       same-named listing (see below), and a stand alone file begins with
%       the number corresponding to the first input line.
%
%       \texttt{last} continues the numbering of the most recent listing and
%       \meta{number} sets it to the number.
%
% \item[1.0] \rkeyname{name}|=|\meta{name}
%
%       names a listing. Displayed environment-listings with the same name
%       share a line counter if |firstnumber=auto| is in effect.
%
% \item[0.20,\arabic{lstnumber},data] \rcmdname\thelstnumber
%
%       prints the lines' numbers.
% \end{syntax}
% We show an example on how to redefine |\thelstnumber|. But if you test it,
% you won't get the result shown on the left.
% \begin{lstxsample}
%    \renewcommand*\thelstnumber{\oldstylenums{\the\value{lstnumber}}}
% \end{lstxsample}
% \begin{lstsample}{\lstset{stepnumber=-1}\label{rDecreasingLabels}}{}
%    \begin{lstlisting}[numbers=left,
%                       firstnumber=753]
%    begin { empty lines }
%
%
%
%
%
%
%    end; { empty lines }
%    \end{lstlisting}
% \end{lstsample}
%
% \begin{advise}
% \item
% The example shows a sequence $n,n+1,\ldots,n+7$ of 8 three-digit figures such that the sequence contains each digit $0,1,\ldots,9$.
% But 8 is not minimal with that property.
% Find the minimal number and prove that it is minimal.
% How many minimal sequences do exist?
%
% Now look at the generalized problem:
% Let $k\in\{1,\ldots,10\}$ be given.
% Find the minimal number $m\in\{1,\ldots,10\}$ such that there is a sequence $n,{n+1},\ldots,\allowbreak{n+m-1}$ of $m$ $k$-digit figures which contains each digit $\{0,\ldots,9\}$.
% Prove that the number is minimal.
% How many minimal sequences do exist?
%
% If you solve this problem with a computer, write a \TeX\ program!
% \end{advise}
%
%
% \subsection{Captions}
%
% In despite of \LaTeX\ standard behaviour, captions and floats are independent
% from each other here; you can use captions with non-floating listings.
% \begin{syntax}
% \item[0.21] \rkeyname{title}|=|\meta{title text}
%
%       is used for a title without any numbering or label.
%
% \item[0.20] \rkeyname{caption}|={|\oarg{short}\meta{caption text}|}|
%
%       The caption is made of \cs{lstlistingname} followed by a running
%       number, a seperator, and \meta{caption text}. Either the caption text
%       or, if present, \meta{short} will be used for the list of listings.
%
% \item[0.21] \rkeyname{label}|=|\meta{name}
%
%       makes a listing referable via |\ref|\marg{name}.
%
% \item[0.16] \rcmdname\lstlistoflistings
%
%       prints a list of listings. Each entry is with descending priority
%       either the short caption, the caption, the file name or the name of the
%       listing, see also the key \keyname{name} in section \ref{rLineNumbers}.
%
% \item[1.0] \rkeyname{nolol}|=|\meta{\alternative{true,false}}\syntaxor\rkeyname{nolol}
%
%       If true, the listing does not make it into the list of listings.
%
% \item[0.16,Listings,data] \rcmdname\lstlistlistingname
%
%       The header name for the list of listings.
%
% \item[0.20,Listing,data] \rcmdname\lstlistingname
%
%       The caption label for listings.
%
% \item[0.20,\arabic{lstlisting},data] \rcmdname\thelstlisting
%
%       prints the running number of the caption.
%
% \item[1.4,true] \rkeyname{numberbychapter}|=|\meta{\alternative{true,false}}
%
%       If true, and |\thechapter| exists, listings are numbered by chapter.
%       Otherwise, they are numbered sequentially from the beginning of the
%       document.  This key can only be used before |\begin{document}|.
%
% \item[0.19] \rcmdname\lstname
%
%       prints the name of the current listing which is either the file name or
%       the name defined by the \keyname{name} key. This command can be used to
%       define a caption or title template, for example by
%       |\lstset{caption=\lstname}|.
%
% \item[0.20,t] \rkeyname{captionpos}|=|\meta{subset of \textup{\texttt{tb}}}
%
%       specifies the positions of the caption: top and/or bottom of the
%       listing.
%
% \item[0.20,\smallskipamount] \rkeyname{abovecaptionskip}|=|\meta{dimension}
% \item[0.20,\smallskipamount] \rkeyname{belowcaptionskip}|=|\meta{dimension}
%
%       is the vertical space respectively above or below each caption.
% \end{syntax}
%
%
% \subsection{Margins and line shape}\label{rMarginsAndLineShape}
%
% \begin{syntax}
% \item[0.21,\linewidth] \rkeyname{linewidth}|=|\meta{dimension}
%
%       defines the base line width for listings. The following three keys are
%       taken into account additionally.
%
% \item[0.19,0pt] \rkeyname{xleftmargin}|=|\meta{dimension}
% \item[1.0,0pt] \rkeyname{xrightmargin}|=|\meta{dimension}
%
%       The dimensions are used as extra margins on the left and right. Line
%       numbers and frames are both moved accordingly.
%
% \item[0.19,false] \rkeyname{resetmargins}|=|\meta{\alternative{true,false}}
%
%       If true, indention from list environments like \texttt{enumerate} or
%       \texttt{itemize} is reset, i.e.~not used.
%
% \item[0.20,false] \rkeyname{breaklines}|=|\meta{\alternative{true,false}}\syntaxor\rkeyname{breaklines}
%
%       activates or deactivates automatic line breaking of long lines.
%
% \item[1.2,false] \rkeyname{breakatwhitespace}|=|\meta{\alternative{true,false}}\syntaxor\rkeyname{breakatwhitespace}\label{uoption:breakatwhitespace}
%
%       If true, it allows line breaks only at white space.
%
% \item[0.20,{{}}] \rkeyname{prebreak}|=|\meta{tokens}
% \item[0.20,{{}}] \rkeyname{postbreak}|=|\meta{tokens}
%
%       \meta{tokens} appear at the end of the current line respectively at the beginning of the next (broken part of the) line.
%
%       You must not use dynamic space (in particular spaces) since internally we use |\discretionary|.
%       However |\space| is redefined to be used inside \meta{tokens}.
%
% \item[0.20,20pt] \rkeyname{breakindent}|=|\meta{dimension}
%
%       is the indention of the second, third, \ldots\ line of broken lines.
%
% \item[0.20,true] \rkeyname{breakautoindent}|=|\meta{\alternative{true,false}}\syntaxor\rkeyname{breakautoindent}
%
%       activates or deactivates automatic indention of broken lines. This
%       indention is used additionally to \ikeyname{breakindent}, see the
%       example below.
%       Visible spaces or visible tabulators might set this auto
%       indention to zero.
% \end{syntax}
% In the following example we use tabulators to create long lines, but the
% verbatim part uses |tabsize=1|.
% \begin{lstxsample}
%    \lstset{postbreak=\space, breakindent=5pt, breaklines}
% \end{lstxsample}
% \begin{lstsample}{\lstset{string=[d]",tabsize=6}}{\lstset{tabsize=1}\hfuzz=1in}
%    \begin{lstlisting}
%    		"A long string is broken!"
%    			"Another long line."
%    \end{lstlisting}
%
%    \begin{lstlisting}[breakautoindent
%                                 =false]
%    		{ Now auto indention is off. }
%    \end{lstlisting}
% \end{lstsample}
%
%
% \subsection{Frames}\label{rFrames}
%
% \begin{syntax}
% \item[1.0,none] \rkeyname{frame}|=|\meta{\alternative{none,leftline,topline,bottomline,lines,single,shadowbox}}
%
%       draws either no frame, a single line on the left, at the top, at the
%       bottom, at the top and bottom, a whole single frame, or a shadowbox.
%
%       Note that \packagename{fancyvrb} supports the same frame types except
%       \texttt{shadowbox}. The shadow color is \keyname{rulesepcolor}, see
%       below.
%
% \item[0.19,{{}}] \rkeyname{frame}|=|\meta{subset of \textup{\texttt{trblTRBL}}}
%
%		The characters \texttt{trblTRBL} designate lines at the top and
%       bottom of a listing and to lines on the right and left. Upper case
%       characters are used to draw double rules. So |frame=tlrb| draws a
%       single frame and |frame=TL| double lines at the top and on the left.
%
%       Note that frames usually reside outside the listing's space.
%
% \item[0.20,ffff] \rkeyname{frameround}|=|\meta{\alternative{t,f}}\meta{\alternative{t,f}}\meta{\alternative{t,f}}\meta{\alternative{t,f}}
%
%       The four letters designate the top right, bottom right, bottom
%       left and top left corner. In this order. \texttt{t} makes the
%       according corner round. If you use round corners, the rule width is
%       controlled via |\thinlines| and |\thicklines|.
%
%       Note: The size of the quarter circles depends on \keyname{framesep}
%       and is independent of the extra margins of a frame. The size is
%       possibly adjusted to fit \LaTeX's circle sizes.
%
% \item[0.19,3pt] \rkeyname{framesep}|=|\meta{dimension}
% \item[0.19,2pt] \rkeyname{rulesep}|=|\meta{dimension}
%
%		control the space between frame and listing and between double rules.
%
% \item[0.19,0.4pt] \rkeyname{framerule}|=|\meta{dimension}
%
%		controls the width of the rules.
%
% \item[1.0,0pt] \rkeyname{framexleftmargin}|=|\meta{dimension}
% \item[1.0,0pt] \rkeyname{framexrightmargin}|=|\meta{dimension}
% \item[1.0,0pt] \rkeyname{framextopmargin}|=|\meta{dimension}
% \item[1.0,0pt] \rkeyname{framexbottommargin}|=|\meta{dimension}
%
%       are the dimensions which are used additionally to \keyname{framesep}
%       to make up the margin of a frame.
%
% \item[0.21] \rkeyname{backgroundcolor}|=|\meta{color command}
% \item[0.21] \rkeyname{rulecolor}|=|\meta{color command}
% \item[1.0] \rkeyname{fillcolor}|=|\meta{color command}
% \item[1.0] \rkeyname{rulesepcolor}|=|\meta{color command}
%
%       specify the colour of the background, the rules, the space between
%       `text box' and first rule, and of the space between two rules,
%       respectively.
%       Note that the value requires a |\color| command, for example
%       \keyname{rulecolor}|=\color{blue}|.
% \end{syntax}
% \ikeyname{frame} does not work with |fancyvrb=true| or when the package
% internally makes a |\hbox| around the listing! And there are certainly more
% problems with other commands; please take the time to make a (bug) report.
%\ifcolor
% \begin{lstxsample}
%    \lstset{framexleftmargin=5mm, frame=shadowbox, rulesepcolor=\color{blue}}
% \end{lstxsample}
%\else
%    \lstset{framexleftmargin=5mm, frame=shadowbox}
%\fi
% \begin{lstsample}{}{}
%    \begin{lstlisting}[numbers=left]
%    for i:=maxint to 0 do
%    begin
%        { do nothing }
%    end;
%    \end{lstlisting}
% \end{lstsample}
%
% Note here the use of |framexleftmargin| to include the line numbers inside
% the frame.
%
% Do you want exotic frames? Try the following key if you want, for example,
% \begin{lstsample}{\lstset{frameshape={RYRYNYYYY}{yny}{yny}{RYRYNYYYY}}}{}
%    \begin{lstlisting}
%    for i:=maxint to 0 do
%    begin
%        { do nothing }
%    end;
%    \end{lstlisting}
% \end{lstsample}
% \begin{syntax}
% \item[0.20,,\dag] \rkeyname{frameshape}|=|\marg{top shape}\marg{left shape}\marg{right shape}\marg{bottom shape}
%
%       gives you full control over the drawn frame parts.
%       The arguments are not case sensitive.
%
%       Both \meta{left shape} and \meta{right shape} are `left-to-right'
%       \alternative{y,n} character sequences (or empty). Each |y| lets the
%       package draw a rule, otherwise the rule is blank. These vertical rules
%       are drawn `left-to-right' according to the specified shapes.
%       The example above uses |yny|.
%
%       \meta{top shape} and \meta{bottom shape} are `left-rule-right'
%       sequences (or empty). The first `left-rule-right' sequence is attached
%       to the most inner rule, the second to the next, and so on.
%       Each sequence has three characters: `rule' is either |y| or |n|;
%       `left' and `right' are |y|, |n| or |r| (which makes a corner round).
%       The example uses |RYRYNYYYY| for both shapes:
%       |RYR| describes the most inner (top and bottom) frame shape, |YNY|
%       the middle, and |YYY| the most outer.
% \end{syntax}
% To summarize, the example above used
% \begin{verbatim}
%    \lstset{frameshape={RYRYNYYYY}{yny}{yny}{RYRYNYYYY}}\end{verbatim}
% Note that you are not resticted to two or three levels.
% However you'll get in trouble if you use round corners when they are too big.
%
%
% \subsection{Indexing}
%
% \begin{syntax}
% \item[0.19] \rkeyname{index}|=|\oarg{number}\oarg{keyword classes}\marg{identifiers}
% \item[0.21] \rkeyname{moreindex}|=|\oarg{number}\oarg{keyword classes}\marg{identifiers}
% \item[0.21] \rkeyname{deleteindex}|=|\oarg{number}\oarg{keyword classes}\marg{identifiers}
%
%       define, add and remove \meta{identifiers} and \meta{keyword classes}
%       from the index class list \meta{number}. If you don't specify the
%       optional number, the package assumes \meta{number} $=1$.
%
%		Each appearance of the explicitly given identifiers and each appearance
%       of the identifiers of the specified \meta{keyword classes} is indexed.
%       For example, you could write |index=[1][keywords]| to index all
%       keywords. Note that |[1]| is required here---otherwise we couldn't use
%       the second optional argument.
%
% \item[0.19,\lstindexmacro] \rkeyname{indexstyle}|=|\oarg{number}\meta{tokens \textup(one-parameter command\textup)}
%
%       \meta{tokens} actually indexes the identifiers for the list
%       \meta{number}. In contrast to the style keys, \meta{tokens}
%       \emph{must} read exactly one parameter, namely the identifier.
%       Default definition is\icmdname{\lstindexmacro}\vspace*{-\itemsep}
% \begin{verbatim}
%    \newcommand\lstindexmacro[1]{\index{{\ttfamily#1}}}\end{verbatim}
%       \vspace*{-\itemsep}which you shouldn't modify.
%       Define your own indexing commands and use them as argument to this key.
% \end{syntax}
% Section \ref{uIndexing} describes this feature in detail.
%
%
% \subsection{Column alignment}\label{rColumnAlignment}
%
% \begin{syntax}
% \item[1.0,{[c]fixed}] \rkeyname{columns}|=|\oarg{\alternative{c,l,r}}\meta{alignment}
%
%       selects the column alignment.  The \meta{alignment} can be |fixed|,
%       |flexible|, |spaceflexible|, or |fullflexible|; see section
%       \ref{uFixedAndFlexibleColumns} for details.
%
%       The optional |c|, |l|, or |r| controls the horizontal orientation of
%       smallest output units (keywords, identifiers, etc.). The arguments work
%       as follows, where vertical bars visualize the effect:
%           $\vert$\lstinline[columns={[c]fixed}]!listing!$\vert$,
%           $\vert$\lstinline[columns={[l]fixed}]!listing!$\vert$, and
%           $\vert$\lstinline[columns={[r]fixed}]!listing!$\vert$
%       in fixed column mode,
%           $\vert$\lstinline[columns={[c]flexible}]!listing!$\vert$,
%           $\vert$\lstinline[columns={[l]flexible}]!listing!$\vert$, and
%           $\vert$\lstinline[columns={[r]flexible}]!listing!$\vert$
%       with flexible columns, and
%           $\vert$\lstinline[columns={[c]fullflexible}]!listing!$\vert$,
%           $\vert$\lstinline[columns={[l]fullflexible}]!listing!$\vert$, and
%           $\vert$\lstinline[columns={[r]fullflexible}]!listing!$\vert$
%       with space-flexible or full flexible columns (which ignore the
%       optional argument, since they do not add extra space around
%       printable characters).
%
% \item[0.18,false] \rkeyname{flexiblecolumns}|=|\meta{\alternative{true,false}}\syntaxor\rkeyname{flexiblecolumns}
%
%       selects the most recently selected flexible or fixed column format,
%       refer to section \ref{uFixedAndFlexibleColumns}.
%
% \item[0.21,false,\dag] \rkeyname{keepspaces}|=|\meta{\alternative{true,false}}
%
%       |keepspaces=true| tells the package not to drop spaces to fix column
%       alignment and always converts tabulators to spaces.
%
% \item[0.16] \rkeyname{basewidth}|=|\meta{dimension}\syntaxor
% \item[0.18,{{0.6em,0.45em}}] \rkeyname{basewidth}|={|\meta{fixed}|,|\meta{flexible mode}|}|
%
%       sets the width of a single character box for fixed and flexible column
%       mode (both to the same value or individually).
%
% \item[0.20,false] \rkeyname{fontadjust}|=|\meta{\alternative{true,false}}\syntaxor\rkeyname{fontadjust}
%
%       If true the package adjusts the base width every font selection.
%       This makes sense only if \ikeyname{basewidth} is given in font specific
%       units like `em' or `ex'---otherwise this boolean has no effect.
%
%       After loading the package, it doesn't adjust the width every font
%       selection: it looks at \ikeyname{basewidth} each listing and uses the
%       value for the whole listing. This is possibly inadequate if the style
%       keys in section \ref{rFigureOutTheAppearance} make heavy font size
%       changes, see the example below.
%
%       Note that this key might disturb the column alignment and might have an
%       effect on the keywords' appearance!
% \end{syntax}
% \begin{lstsample}{\lstset{basicstyle=\normalsize}}{}
%    \lstset{commentstyle=\scriptsize}
%    \begin{lstlisting}
%    { scriptsize font
%      doesn't look good }
%    for i:=maxint to 0 do
%    begin
%        { do nothing }
%    end;
%    \end{lstlisting}
% \end{lstsample}
% \begin{lstsample}{\lstset{basicstyle=\normalsize,commentstyle=\scriptsize}}{}
%    \begin{lstlisting}[fontadjust]
%    { scriptsize font
%      looks better now }
%    for i:=maxint to 0 do
%    begin
%        { do nothing }
%    end;
%    \end{lstlisting}
% \end{lstsample}
%
%
% \subsection{Escaping to \LaTeX}\label{rEscapingToLaTeX}
%
% \textbf{Note:} {\itshape Any escape to \LaTeX\ may disturb the column
% alignment since the package can't control the spacing there.}
% \begin{syntax}
% \item[0.18,false] \rkeyname{texcl}|=|\meta{\alternative{true,false}}\syntaxor\rkeyname{texcl}
%
%       activates or deactivates \LaTeX\ comment lines. If activated, comment
%       line delimiters are printed as usual, but the comment line text (up to
%       the end of line) is read as \LaTeX\ code and typeset in comment style.
% \end{syntax}
% The example uses \Cpp\ comment lines (but doesn't say how to define them).
% Without |\upshape| we would get \textit{calculate} since the comment style
% is |\itshape|.
% \begin{lstsample}{\lstset{morecomment=[l]//}}{}
%    \begin{lstlisting}[texcl]
%    // \upshape calculate $a_{ij}$
%      A[i][j] = A[j][j]/A[i][j];
%    \end{lstlisting}
% \end{lstsample}
%
% \begin{syntax}
% \item[0.19,false] \rkeyname{mathescape}|=|\meta{\alternative{true,false}}
%
%       activates or deactivates special behaviour of the dollar sign.
%       If activated a dollar sign acts as \TeX's text math shift.
%
%       This key is useful if you want to typeset formulas in listings.
%
% \item[0.19,{{}}] \rkeyname{escapechar}|=|\meta{character}\syntaxor\rkeyname{escapechar}|={}|
%
%       If not empty the given character escapes the user to \LaTeX: all code
%       between two such characters is interpreted as \LaTeX\ code. Note that
%       \TeX's special characters must be entered with a preceding backslash,
%       e.g.~|escapechar=\%|.
%
% \item[0.20,{{}}] \rkeyname{escapeinside}|=|\meta{character}\meta{character}\syntaxor\rkeyname{escapeinside}|={}|
%
%       Is a generalization of \ikeyname{escapechar}. If the value is not
%       empty, the package escapes to \LaTeX\ between the first and second
%       character.
%
% \item[0.20,{{}}] \rkeyname{escapebegin}|=|\meta{tokens}
% \item[0.20,{{}}] \rkeyname{escapeend}|=|\meta{tokens}
%
%       The tokens are executed at the beginning respectively at the end of
%       each escape, in particular for \ikeyname{texcl}.
%       See section \ref{uNationalCharacters} for an application.
% \end{syntax}
%
% \begin{lstsample}{\lstset{morecomment=[l]//}}{}
%    \begin{lstlisting}[mathescape]
%    // calculate $a_{ij}$
%      $a_{ij} = a_{jj}/a_{ij}$;
%    \end{lstlisting}
% \end{lstsample}
%
% \begin{lstsample}{\lstset{morecomment=[l]//}}{}
%    \begin{lstlisting}[escapechar=\%]
%    // calc%ulate $a_{ij}$%
%      %$a_{ij} = a_{jj}/a_{ij}$%;
%    \end{lstlisting}
% \end{lstsample}
%
% \begin{lstsample}{\lstset{morecomment=[l]//}}{}
%    \lstset{escapeinside=`'}
%    \begin{lstlisting}
%    // calc`ulate $a_{ij}$'
%      `$a_{ij} = a_{jj}/a_{ij}$';
%    \end{lstlisting}
% \end{lstsample}
% In the first example the comment line up to $a_{ij}$ has been typeset by the
% \packagename{listings} package in comment style. The $a_{ij}$ itself is
% typeset in `\TeX\ math mode' without comment style. About half of the
% comment line of the second example has been typeset by this package, and
% the rest is in `\LaTeX\ mode'.
%
% To avoid problems with the current and future version of this package:
% \begin{enumerate}
% \item Don't use any commands of the \packagename{listings} package when you
%       have escaped to \LaTeX.
% \item Any environment must start and end inside the same escape.
% \item You might use |\def|, |\edef|, etc., but do not assume that the
%       definitions are present later, unless they are |\global|.
% \item |\if \else \fi|, groups, math shifts |$| and |$$|, \ldots\ must be
%       balanced within each escape.
% \item \ldots
% \end{enumerate}
% Expand that list yourself and mail me about new items.
%
%
% \subsection{Interface to \textsf{fancyvrb}}
%
% The \packagename{fancyvrb} package---fancy verbatims---from Timothy van Zandt
% provides macros for reading, writing and typesetting verbatim code. It has
% some remarkable features the \packagename{listings} package doesn't have.
% (Some are possible, but you must find somebody who will implement them |;-)|.
% \begin{syntax}
% \item[0.19] \rkeyname{fancyvrb}|=|\meta{\alternative{true,false}}
%
%       activates or deactivates the interface. If active, verbatim code is
%       read by \packagename{fancyvrb} but typeset by \packagename{listings},
%       i.e.~with emphasized keywords, strings, comments, and so on.
%       Internally we use a very special definition of |\FancyVerbFormatLine|.
%
%       This interface works with |Verbatim|, |BVerbatim| and |LVerbatim|.
%       But you shouldn't use \packagename{fancyvrb}'s \keyname{defineactive}.
%       (As far as I can see it doesn't matter since it does nothing at all,
%       but for safety \ldots .)
%       If \packagename{fancyvrb} and \packagename{listings} provide similar
%       functionality, you should use \packagename{fancyvrb}'s.
%
% \item[1.1,{\overlay 1}] \rkeyname{fvcmdparams}|=|\meta{command$_1$}\meta{number$_1$}\ldots\label{uoption:fvcmdparams}
% \item[1.1] \rkeyname{morefvcmdparams}|=|\meta{command$_1$}\meta{number$_1$}\ldots\label{uoption:morefvcmdparams}
%
%       If you use \packagename{fancyvrb}'s \keyname{commandchars}, you must
%       tell the \packagename{listings} package how many arguments each command
%       takes. If a command takes no arguments, there is nothing to do.
%
%       The first (third, fifth, \ldots) parameter to the keys is the command
%       and the second (fourth, sixth, \ldots) is the number of arguments
%       that command takes. So, if you want to use |\textcolor{red}{keyword}|
%       with the \packagename{fancyvrb}-\packagename{listings} interface, you
%       should write |\lstset{morefvcmdparams=\textcolor 2}|.
% \end{syntax}
%
% \iffancyvrb
% \begin{lstsample}{}{}
%    \lstset{morecomment=[l]\ }% :-)
%    \fvset{commandchars=\\\{\}}
%
%    \begin{BVerbatim}
%    First verbatim line.
%    \fbox{Second} verbatim line.
%    \end{BVerbatim}
%
%    \par\vspace{72.27pt}
%
%    \lstset{fancyvrb}
%    \begin{BVerbatim}
%    First verbatim line.
%    \fbox{Second} verbatim line.
%    \end{BVerbatim}
%    \lstset{fancyvrb=false}
% \end{lstsample}
% The lines typeset by the \packagename{listings} package are wider since the
% default \ikeyname{basewidth} doesn't equal the width of a single typewriter type
% character. Moreover, note that the first space begins a comment as defined at
% the beginning of the example.
% \else
% \begin{center}
%    \packagename{fancyvrb} seems to be unavailable on your platform, thus the
%    example couldn't be printed here.
% \end{center}
% \fi
%
%
% \subsection{Environments}\label{rEnvironments}
%
% If you want to define your own pretty-printing environments, try the
% following command. The syntax comes from \LaTeX's |\newenvironment|.
% \begin{syntax}
% \item[0.19] \rcmdname\lstnewenvironment\\
%       \marg{name}\oarg{number}\oarg{opt.~default~arg.}\\
%       |{|\meta{starting code}|}|\\
%       |{|\meta{ending code}|}|
% \end{syntax}
% As a simple example we could just select a particular language.
% \begin{lstxsample}
%    \lstnewenvironment{pascal}
%        {\lstset{language=pascal}}
%        {}
% \end{lstxsample}
% \begin{lstsample}{}{}
%    \begin{pascal}
%    for i:=maxint to 0 do
%    begin
%        { do nothing }
%    end;
%    \end{pascal}
% \end{lstsample}
% Doing other things is as easy, for example, using more keys and adding an
% optional argument to adjust settings each listing:
% \begin{verbatim}
%\lstnewenvironment{pascalx}[1][]
%    {\lstset{language=pascal,numbers=left,numberstyle=\tiny,float,#1}}
%    {}\end{verbatim}
%
%
% \subsection{Short Inline Listing Commands}\label{rShortInline}
%
% Short equivalents of |\lstinline| can also be defined, in a manner similar
% to the short verbatim macros provided by \packagename{shortvrb}.
%
% \begin{syntax}
% \item[1.4] \rcmdname\lstMakeShortInline[\oarg{options}]\meta{character}
%
%       defines \meta{character} to be an equivalent of
%       |\lstinline|[\oarg{options}]\meta{character},
%       allowing for a convenient syntax when using lots of inline listings.
%
% \item[1.4] \rcmdname\lstDeleteShortInline\meta{character}
%
%       removes a definition of \meta{character} created by |\lstMakeShortInline|,
%       and returns \meta{character} to its previous meaning.
% \end{syntax}
%
%
% \subsection{Language definitions}\label{rLanguageDefinitions}
%
% You should first read section \ref{uLanguageDefinitions} for an introduction
% to language definitions. Otherwise you're probably unprepared for the full
% syntax of |\lstdefinelanguage|.
% \begin{syntax}
% \item[0.19] \rcmdname\lstdefinelanguage\syntaxnewline[\oarg{dialect}]\marg{language}\syntaxnewline[\oarg{base dialect}\marg{and base language}]\syntaxnewline\marg{key=value list}\syntaxnewline[\oarg{list of required aspects \textup(keywordcomments,texcs,etc.\textup)}]
%
%		defines the (given dialect of the) programming language \meta{language}.
%       If the language definition is based on another definition, you must
%       specify the whole \oarg{base dialect}\marg{and base language}. Note
%       that an empty \meta{base dialect} uses the default dialect!
%
%       The last optional argument should specify all required aspects. This is
%       a delicate point since the aspects are described in the developer's
%       guide. You might use existing languages as templates. For example,
%       ANSI C uses \aspectname{keywords}, \aspectname{comments},
%       \aspectname{strings} and \aspectname{directives}.
%
%       \icmdname{\lst@definelanguage} has the same syntax and is used to
%       define languages in the driver files.
%
% \begin{advise}
% \item Where should I put my language definition?
%       \advisespace
%       If you need the language for one particular document, put it into
%       the preamble of that document. Otherwise create the local file
%       `\texttt{lstlang0.sty}' or add the definition to that file, but use
%       `|\lst@definelanguage|' instead of `|\lstdefinelanguage|'.
%       However, you might want to send the definition to the address in
%       section \ref{uSoftwareLicense}. Then it will be included with the
%       rest of the languages distributed with the package, and published under
%       the \LaTeX\ Project Public License.
% \end{advise}
%
% \item[0.18] \rcmdname\lstalias\marg{alias}\marg{language}
%
%       defines an alias for a programming language. Each \meta{alias} is
%       redirected to the same dialect of \meta{language}.
%       It's also possible to define an alias for one particular dialect only:
%
% \item[0.18] \rcmdname\lstalias\oarg{alias dialect}\marg{alias}\oarg{dialect}\marg{language}
%
%       Here all four parameters are \emph{nonoptional} and an alias with empty
%       \meta{dialect} will select the default dialect. Note that aliases
%       cannot be chained: The two aliases `|\lstalias{foo1}{foo2}|' and
%       `|\lstalias{foo2}{foo3}|' will \emph{not} redirect |foo1| to |foo3|.
% \end{syntax}
% All remaining keys in this section are intended for building language
% definitions. \emph{No other key should be used in such a definition!}
%
%
% \paragraph{Keywords}
% We begin with keyword building keys. Note: {\itshape If you want to enter
% {\upshape|\|, |{|, |}|, |%|, |#|} or {\upshape|&|} as (part of) an argument
% to the keywords below, you must do it with a preceding backslash!}
% \begin{syntax}
% \item[1.0,,{\dag bug}] \rkeyname{keywordsprefix}|=|\meta{prefix}
%
%       All identifiers starting with \meta{prefix} will be printed as first
%       order keywords.
%
%       Bugs: Currently there are several limitations.
%       (1) The prefix is always case sensitive.
%       (2) Only one prefix can be defined at a time.
%       (3) If used `standalone' outside a language definition, the key might
%           work only after selecting a nonempty language (and switching back to
%           the empty language if necessary).
%       (4) The key does not respect the value of \keyname{classoffset} and
%           has no optional class \meta{number} argument.
%
% \item[0.11] \rkeyname{keywords}|=|\oarg{number}\marg{list of keywords}
% \item[0.11] \rkeyname{morekeywords}|=|\oarg{number}\marg{list of keywords}
% \item[0.18] \rkeyname{deletekeywords}|=|\oarg{number}\marg{list of keywords}
%
%       define, add to or remove the keywords from keyword list \meta{number}.
%       The use of \keyname{keywords} is discouraged since it deletes all
%       previously defined keywords in the list and is thus incompatible with
%       the \keyname{alsolanguage} key.
%
%       Please note the keys \ikeyname{alsoletter} and \ikeyname{alsodigit}
%       below if you use unusual charaters in keywords.
%
% \item[0.19,,deprecated] \rkeyname{ndkeywords}|=|\marg{list of keywords}
% \item[0.19,,deprecated] \rkeyname{morendkeywords}|=|\marg{list of keywords}
% \item[0.19,,deprecated] \rkeyname{deletendkeywords}|=|\marg{list of keywords}
%
%       define, add to or remove the keywords from keyword list 2; note that
%       this is equivalent to |keywords=[2]|\ldots etc.
%       The use of \keyname{ndkeywords} is strongly discouraged.
%
% \item[0.19,,{addon,optional}] \rkeyname{texcs}|=|\oarg{class number}\marg{list of control sequences \textup(without backslashes\textup)}
% \item[0.20,,{addon,optional}] \rkeyname{moretexcs}|=|\oarg{class number}\marg{list of control sequences \textup(without backslashes\textup)}
% \item[0.21,,{addon,optional}] \rkeyname{deletetexcs}|=|\oarg{class number}\marg{list of control sequences \textup(without backslashes\textup)}
%
%       Ditto for control sequences in \TeX\ and \LaTeX.
%
% \item[0.18,,optional] \rkeyname{directives}|=|\marg{list of compiler directives}
% \item[0.21,,optional] \rkeyname{moredirectives}|=|\marg{list of compiler directives}
% \item[0.21,,optional] \rkeyname{deletedirectives}|=|\marg{list of compiler directives}
%
%       defines compiler directives in C, \Cpp, Objective-C, and POV.
%
% \item[0.14] \rkeyname{sensitive}|=|\meta{\alternative{true,false}}
%
%       makes the keywords, control sequences, and directives case sensitive
%       and insensitive, respectively. This key affects the keywords, control
%       sequences, and directives only when a listing is processed. In all
%       other situations they are case sensitive, for example,
%       |deletekeywords={save,Test}| removes `save' and `Test', but neither
%       `SavE' nor `test'.
%
% \item[0.19] \rkeyname{alsoletter}|=|\marg{character sequence}
% \item[0.19] \rkeyname{alsodigit}|=|\marg{character sequence}
% \item[0.19] \rkeyname{alsoother}|=|\marg{character sequence}
%
%       All identifiers (keywords, directives, and such) consist of a letter
%       followed by alpha-numeric characters (letters and digits).
%       For example, if you write
%           |keywords={one-two,\#include}|,
%       the minus sign must become a digit and the sharp a letter since the
%       keywords can't be detected otherwise.
%
%       Table \ref{rStdCharTable} show the standard configuration of the
%       \packagename{listings} package. The three keys overwrite the default
%       behaviour. Each character of the sequence becomes a letter, digit
%       and other, respectively.
%
% \item[0.20] \rkeyname{otherkeywords}|=|\marg{keywords}
%
%       Defines keywords that contain other characters, or start with digits.
%       Each given `keyword' is printed in keyword style, but without changing
%       the `letter', `digit' and `other' status of the characters. This key
%       is designed to define keywords like |=>|, |->|, |-->|, |--|, |::|, and
%       so on. If one keyword is a subsequence of another (like |--| and
%       |-->|), you must specify the shorter first.
%
% \item[0.20,,{renamed,optional}] \rkeyname{tag}|=|\meta{character}\meta{character}\syntaxor\rkeyname{tag}|={}|\label{uoption:tag}
%
%       The first order keywords are active only between the first and second
%       character. This key is used for HTML.
% \end{syntax}
%
% \begin{table}[tb]
% \caption{Standard character table}\label{rStdCharTable}
% \begin{tabular}{ll}
% class & characters\\
% \noalign{\smallskip}
% letter & \texttt{A B C D E F G H I J K L M N O P Q R S T U V W X Y Z}\\
%        & \texttt{a b c d e f g h i j k l m n o p q r s t u v w x y z}\\
%        & \texttt{@ \textdollar\ } |_|\\
% digit  & \texttt{0 1 2 3 4 5 6 7 8 9}\\
% other  & \texttt{!\ " \#\ \%\ \&\ ' ( ) * + , - .\ / :\ ; < = > ?}\\
%        & {\catcode`\|=12\texttt{[ \char92\ ] \textasciicircum\ \char123\ | \char125\ \textasciitilde}}\\
% space  & chr(32)\\
% tabulator & chr(9)\\
% form feed & chr(12)\\
% \noalign{\smallskip}
% \end{tabular}
% \par\noindent
% Note: Extended characters of codes 128--255 (if defined) are \emph{currently}
% letters.
% \end{table}
%
%
% \paragraph{Strings}
% \begin{syntax}
% \item[0.12] \rkeyname{string}|=|\oarg{\alternative{b,d,m,bd,s}}\marg{delimiter \textup(character\textup)}
% \item[0.21] \rkeyname{morestring}|=|\oarg{\alternative{b,d,m,bd,s}}\marg{delimiter}
% \item[0.21] \rkeyname{deletestring}|=|\oarg{\alternative{b,d,m,bd,s}}\marg{delimiter}
%
%       define, add to or delete the delimiter from the list of string
%       delimiters. Starting and ending delimiters are the same, i.e.~in the
%       source code the delimiters must match each other.
%
%       The optional argument is the type and controls the how the delimiter
%       itself is represented in a string or character literal: it is escaped by a
%       |b|ackslash, |d|oubled (or both is allowed via |bd|).  Alternately, the
%       type can refer to an unusual form of delimiter: |s|tring delimiters (akin
%       to the |s| comment type) or |m|atlab-style delimiters.  The latter is a
%       special type for Ada and Matlab and possibly other languages where the
%       string delimiters are also used for other purposes.  It is equivalent
%       to |d|, except that a string does not start after a letter, a right
%       parenthesis, a right bracket, or some other characters.
% \end{syntax}
%
%
% \paragraph{Comments}
% \begin{syntax}
% \item[0.13] \rkeyname{comment}|=|\oarg{type}\meta{delimiter\textup(s\textup)}
% \item[0.21] \rkeyname{morecomment}|=|\oarg{type}\meta{delimiter\textup(s\textup)}
% \item[0.21] \rkeyname{deletecomment}|=|\oarg{type}\meta{delimiter\textup(s\textup)}
%
%       Ditto for comments, but some types require more than a single
%       delimiter. The following overview uses \keyname{morecomment} as the
%       example, but the examples apply to \keyname{comment} and \keyname{deletecomment}
%       as well.
%
% \item[0.13] \keyname{morecomment}|=[l]|\meta{delimiter}
%
%       The delimiter starts a comment line, which in general starts with the
%       delimiter and ends at end of line. If the character sequence |//|
%       should start a comment line (like in \Cpp, Comal 80 or Java),
%       |morecomment=[l]//| is the correct declaration. For Matlab it
%       would be |morecomment=[l]\%|---note the preceding backslash.
%
% \item[0.13] \keyname{morecomment}|=[s]|\marg{delimiter}\marg{delimiter}
%
%       Here we have two delimiters. The second ends a comment starting with
%       the first delimiter. If you require two such comments you can use this
%       type twice. C, Java, PL/I, Prolog and SQL all define single comments
%       via |morecomment=[s]{/*}{*/}|, and Algol does it with
%       |morecomment=[s]{\#}{\#}|, which means that the sharp delimits both
%       beginning and end of a single comment.
%
% \item[0.13] \keyname{morecomment}|=[n]|\marg{delimiter}\marg{delimiter}
%
%       is similar to type |s|, but comments can be nested. Identical arguments
%       are not allowed---think a while about it!
%       Modula-2 and Oberon-2 use |morecomment=[n]{(*}{*)}|.
%
% \item[0.18] \keyname{morecomment}|=[f]|\meta{delimiter}
% \item[0.18] \keyname{morecomment}|=[f][commentstyle]|\oarg{n=preceding columns}\meta{delimiter}
%
%       The delimiter starts a comment line if and only if it appears on a
%       fixed column-number, namely if it is in column $n$ (zero based).
%
% \item[0.17,,optional] \rkeyname{keywordcomment}|=|\marg{keywords}
% \item[0.21,,optional] \rkeyname{morekeywordcomment}|=|\marg{keywords}
% \item[0.21,,optional] \rkeyname{deletekeywordcomment}|=|\marg{keywords}
%
%       A keyword comment begins with a keyword and ends with the same keyword.
%       Consider |keywordcomment={comment,co}|. Then
%       `\textbf{comment}\allowbreak\ldots\textbf{comment}' and
%       `\textbf{co}\ldots\textbf{co}' are comments.
%
% \item[0.17,,optional] \rkeyname{keywordcommentsemicolon}|=|\marg{keywords}\marg{keywords}\marg{keywords}
%
%       The definition of a `keyword comment semicolon' requires three keyword
%       lists, e.g.~|{end}{else,end}{comment}|. A semicolon always ends such a
%       comment. Any keyword of the first argument begins a comment and any
%       keyword of the second argument ends it (and a semicolon also);
%       a comment starting with any keyword of the third argument is terminated
%       with the next semicolon only. In the example all possible comments are
%       `\textbf{end}\ldots\textbf{else}', `\textbf{end}\ldots\textbf{end}'
%       (does not start a comment again) and `\textbf{comment}\ldots;' and
%       `\textbf{end}\ldots;'.
%       Maybe a curious definition, but Algol and Simula use such comments.
%
%       Note: The keywords here need not to be a subset of the defined
%       keywords. They won't appear in keyword style if they aren't.
%
% \item[0.17,,optional] \rkeyname{podcomment}|=|\meta{\alternative{true,false}}
%
%       activates or deactivates PODs---Perl specific.
% \end{syntax}
%
%
% \subsection{Installation}\label{rInstallation}
%
% \paragraph{Software installation}
% \begin{enumerate}
% \item Following the \TeX\ directory structure (TDS), you should put the files
%       of the \packagename{listings} package into directories as follows:
%       \begin{center}
%       \begin{tabular}{lcl}
%       \texttt{listings.pdf}&$\to$&\texttt{texmf/doc/latex/listings}\\
%       \texttt{listings.dtx}, \texttt{listings.ins},\\
%       \texttt{listings.ind}, \texttt{lstpatch.sty},\\
%       \texttt{lstdrvrs.dtx}&$\to$&\texttt{texmf/source/latex/listings}
%       \end{tabular}
%       \end{center}
%       Note that you may not have a patch file \texttt{lstpatch.sty}.
%       If you don't use the TDS, simply adjust the directories below.
% \item	Create the directory \texttt{texmf/tex/latex/listings} or, if it exists
%       already, remove all
%       files except \texttt{lst}\meta{whatever}\texttt{0.sty} and
%       \texttt{lstlocal.cfg} from it.
% \item	Change the working directory to \texttt{texmf/source/latex/listings}
%       and run \texttt{listings.ins} through \TeX.
% \item Move the generated files to \texttt{texmf/tex/latex/listings} if this
%       is not already done.
%       \begin{center}
%       \begin{tabular}{lcl}
%       \texttt{listings.sty}, \texttt{lstmisc.sty},
%           &&\qquad(kernel and add-ons)\\
%       \texttt{listings.cfg},
%           &&\qquad(configuration file)\\
%       \texttt{lstlang}\meta{number}\texttt{.sty},
%           &&\qquad(language drivers)\\
%       \texttt{lstpatch.sty}&$\to$&\texttt{texmf/tex/latex/listings}
%       \end{tabular}
%       \end{center}
% \item If your \TeX\ implementation uses a file name database, update it.
% \item If you receive a patch file later on, put it where
%       \texttt{listings.sty} is (and update the file name database).
% \end{enumerate}
% Note that \packagename{listings} requires at least version 1.10 of the
% \packagename{keyval} package included in the \packagename{graphics} bundle by
% David Carlisle.
%
%
% \paragraph{Software configuration}
% Read this only if you encounter problems with the standard configuration or
% if you want the package to suit foreign languages, for example.
%
% Never modify a file from the \packagename{listings} package, in particular
% not the configuration file. Each new installation or new version overwrites
% it. The software license allows modification, but I can't recommend it.
% It's better to create one or more of the files
% \begin{center}
% \begin{tabular}{lcl}
% \texttt{lstmisc0.sty} & for & local add-ons
%                               (see the developer's guide),\\
% \texttt{lstlang0.sty} & for & local language definitions
%                               (see \ref{rLanguageDefinitions}), and\\
% \texttt{lstlocal.cfg} & as  & local configuration file
% \end{tabular}
% \end{center}
% and put them in the same directory as the other \packagename{listings} files.
% These three files are not touched by a new installation unless you remove them.
% If \texttt{lstlocal.cfg} exists, it is loaded after \texttt{listings.cfg}.
% You might want to change one of the following parameters.
% \begin{syntax}
% \item[,,data] \rcmdname\lstaspectfiles\quad contains~\rlap{\texttt{\lstaspectfiles}}
% \item[,,data] \rcmdname\lstlanguagefiles\quad contains~\rlap{\texttt{\lstlanguagefiles}}
%
%       The package uses the specified files to find add-ons and language
%       definitions.
% \end{syntax}
% Moreover, you might want to adjust
%   \icmdname\lstlistlistingname,
%   \icmdname\lstlistingname,
%   \ikeyname{defaultdialect},
%   \icmdname\lstalias, or
%   \icmdname\lstalias
% \ as described in earlier sections.
%
%
% \section{Experimental features}\label{rExperimentalFeatures}
%
% This section describes the more or less unestablished parts of this package.
% It's unlikely that they will all be removed (unless stated explicitly), but
% they are liable to (heavy) changes and improvements. Such features have been
% \dag-marked in the last sections. So, if you find anything \dag-marked here,
% you should be very, very careful.
%
%
% \subsection{Listings inside arguments}\label{rListingsInsideArguments}
%
% There are some things to consider if you want to use |\lstinline| or the
% listing environment inside arguments. Since \TeX\ reads the argument before
% the `\lst-macro' is executed, this package can't do anything to preserve the
% input: spaces shrink to one space, the tabulator and the end of line are
% converted to spaces, \TeX's comment character is not printable, and so on.
% Hence, \emph{you} must work a bit more. You have to put a backslash in front
% of each of the following four characters: |\{}%|. Moreover you must protect
% spaces in the same manner if: (i) there are two or more spaces following each
% other or (ii) the space is the first character in the line.
% That's not enough: Each line must be terminated with a `line feed' |^^J|.
% And you can't escape to \LaTeX\ inside such listings!
%
% The easiest examples are with |\lstinline| since we need no line feed.
% \begin{verbatim}
%\footnote{\lstinline{var i:integer;} and
%          \lstinline!protected\ \ spaces! and
%          \fbox{\lstinline!\\\{\}\%!}}\end{verbatim}
% yields\lstset{language=Pascal}\footnote{\lstinline{var i:integer;} and
%          \lstinline!protected\ \ spaces! and
%          \fbox{\lstinline!\\\{\}\%!}}
% if the current language is Pascal. Note that this example shows another
% experimental feature: use of argument braces as delimiters. This is
% described in section \ref{rTypesettingListings}.
%
% And now an environment example:
% \begin{lstsample}{\lstset{language={}}}{}
%    \fbox{%
%    \begin{lstlisting}^^J
%    \ !"#$\%&'()*+,-./^^J
%    0123456789:;<=>?^^J
%    @ABCDEFGHIJKLMNO^^J
%    PQRSTUVWXYZ[\\]^_^^J
%    `abcdefghijklmno^^J
%    pqrstuvwxyz\{|\}~^^J
%    \end{lstlisting}}
% \end{lstsample}
% \begin{advise}
% \item You might wonder that this feature is still experimental. The reason:
%       You shouldn't use listings inside arguments; it's not always safe.
% \end{advise}
%
%
% \subsection{\dag\ Export of identifiers}\label{rExportOfIdentifiers}
%
% It would be nice to export function or procedure names. In general that's a
% dream so far. The problem is that programming languages use various syntaxes
% for function and procedure declaration or definition. A general interface is
% completely out of the scope of this package---that's the work of a compiler
% and not of a pretty-printing tool. However, it is possible for particular
% languages: in Pascal, for instance, each function or procedure definition and
% variable declaration is preceded by a particular keyword.
% Note that you must request the following keys with the \texttt{procnames} option:
% |\usepackage[procnames]{listings}|.
% \begin{syntax}
% \item[0.19,{{}},{\dag optional}] \rkeyname{procnamekeys}|=|\marg{keywords}
% \item[0.21,,\dag optional] \rkeyname{moreprocnamekeys}|=|\marg{keywords}
% \item[0.21,,\dag optional] \rkeyname{deleteprocnamekeys}|=|\marg{keywords}
%
%		each specified keyword indicates a function or procedure definition.
%		Any identifier following such a keyword appears in `procname' style.
%		For Pascal you might use\vspace{-.5\baselineskip}
% \begin{verbatim}
%    procnamekeys={program,procedure,function}\end{verbatim}
%
% \item[0.19,keywordstyle,\dag optional] \rkeyname{procnamestyle}|=|\meta{style}
%
%		defines the style in which procedure and function names appear.
%
% \item[0.19,false,\dag optional] \rkeyname{indexprocnames}|=|\meta{\alternative{true,false}}
%
%		If activated, procedure and function names are also indexed.
% \end{syntax}
% \begin{TODO}
% The \aspectname{procnames} aspect is unsatisfactory (and has been unchanged
% at least since 2000). It marks and indexes the function definitions so far, but
% it would be possible to mark also the following function calls, for example.
% A key could control whether function names are added to a special keyword
% class, which then appears in `procname' style. But should these names be
% added globally? There are good reasons for both. Of course, we would also
% need a key to reset the name list.
% \end{TODO}
%
%
% \subsection{\dag\ Hyperlink references}\label{rHyperReferences}
%
% This very small aspect must be requested via the \texttt{hyper} option since it
% is experimental. One possibility for the future is to combine this aspect
% with \aspectname{procnames}. Then it should be possible to click on a
% function name and jump to its definition, for example.
% \begin{syntax}
% \item[0.21,,{\dag optional}] \rkeyname{hyperref}|=|\marg{identifiers}
% \item[0.21,,{\dag optional}] \rkeyname{morehyperref}|=|\marg{identifiers}
% \item[0.21,,{\dag optional}] \rkeyname{deletehyperref}|=|\marg{identifiers}
%
%       hyperlink the specified identifiers (via \packagename{hyperref}
%       package). A `click' on such an identifier jumps to the previous
%       occurrence.
%
% \item[0.21,\hyper@@anchor,{\dag optional}] \rkeyname{hyperanchor}|=|\meta{two-parameter macro}
% \item[0.21,\hyperlink,{\dag optional}] \rkeyname{hyperlink}|=|\meta{two-parameter macro}
%
%       set a hyperlink anchor and link, respectively.
%       The defaults are suited for the \packagename{hyperref} package.
% \end{syntax}
%
%
% \subsection{Literate programming}
%
% We begin with an example and hide the crucial key=value list.
% \begin{lstsample}{\lstset{literate={:=}{{$\gets$}}1 {<=}{{$\leq$}}1 {>=}{{$\geq$}}1 {<>}{{$\neq$}}1}}{}
%    \begin{lstlisting}
%    var i:integer;
%
%    if (i<=0) i := 1;
%    if (i>=0) i := 0;
%    if (i<>0) i := 0;
%    \end{lstlisting}
% \end{lstsample}
% Funny, isn't it? We could leave |i := 0| in our listings instead of
% i| |$\gets$| |0, but that's not literate!  ^^A :-)
% Now you might want to know how this has been done. Have a \emph{close}
% look at the following key.
% \begin{syntax}
% \item[0.20,,\dag] \rkeyname{literate}|=|[|*|]\meta{replacement item}\ldots\meta{replacement item}
%
%       First note that there are no commas between the items. Each item
%       consists of three arguments:
%           \marg{replace}\marg{replacement text}\marg{length}.
%       \meta{replace} is the original character sequence.
%       Instead of printing these characters, we use \meta{replacement text},
%       which takes the width of \meta{length} characters in the output.
%
%       Each `printing unit' in \meta{replacement text} \emph{must} be in braces
%       unless it's a single character. For example, you must put braces
%       around |$\leq$|.
%       If you want to replace |<-1->| by |$\leftarrow1\rightarrow$|, the
%       replacement item would be |{<-1->}{{$\leftarrow$}1{$\rightarrow$}}3|.
%       Note the braces around the arrows.
%
%       If one \meta{replace} is a subsequence of another \meta{replace}, you
%       must define the shorter sequence first. For example, |{-}| must be defined
%       before |{--}| and this before |{-->}|.
%
%       The optional star indicates that literate replacements should not be
%       made in strings, comments, and other delimited text.
% \end{syntax}
% In the example above, I've used
% \begin{verbatim}
%  literate={:=}{{$\gets$}}1 {<=}{{$\leq$}}1 {>=}{{$\geq$}}1 {<>}{{$\neq$}}1\end{verbatim}
% \begin{TODO}
% Of course, it's good to have keys for adding and removing single
% \meta{replacement item}s. Maybe the key(s) should work in the same fashion
% as the string and comment definitions, i.e.~one item per key=value.
% This way it would be easier to provide better auto-detection in case of a
% subsequence.
% \end{TODO}
%
%
% \subsection{\textsf{LGrind} definitions}\label{rLGrindDefinitions}
%
% Yes, it's a nasty idea to steal language definitions from other programs.
% Nevertheless, it's possible for the \packagename{LGrind} definition
% file---at least partially. Please note that this file must be found by
% \TeX.
% \begin{syntax}
% \item[0.21,,{optional}] \rkeyname{lgrindef}|=|\meta{language}
%
%       scans the \texttt{lgrindef} language definition file for
%       \meta{language} and activates it if present. Note that not all
%       \packagename{LGrind} capabilities have a \packagename{listings}
%       analogue.
%
%       Note that `Linda' language doesn't work properly since it defines
%       compiler directives with preceding `|#|' as keywords.
%
% \item[0.21,lgrindef.,{data,optional}] \rcmdname\lstlgrindeffile
%
%       contains the (path and) name of the definition file.
% \end{syntax}
%
%
% \subsection{\dag\ Automatic formatting}
%
% \lstloadaspects{formats}^^A
% The automatic source code formatting is far away from being good. First of
% all, there are no general rules on how source code should be formatted. So
% `format definitions' must be flexible. This flexibility requires a complex
% interface, a powerful `format definition' parser, and lots of code lines
% behind the scenes. Currently, format definitions aren't flexible enough
% (possibly not the definitions but the results). A single `format item' has
% the form
% \begin{itemize}\item[]
%     \meta{input chars}|=|\oarg{exceptional chars}\meta{pre}\oarg{\texttt{\string\string}}\meta{post}
% \end{itemize}
% Whenever \meta{input chars} aren't followed by one of the \meta{exceptional
% chars}, formatting is done according to the rest of the value. If |\string|
% isn't specified, the input characters aren't printed (except it's an
% identifier or keyword). Otherwise \meta{pre} is `executed' before printing
% the original character string and \meta{post} afterwards. These two are
% `subsets' of
% \begin{itemize}
% \item |\newline| ---ensuring a new line;
% \item |\space| ---ensuring a whitespace;
% \item |\indent| ---increasing indention;
% \item |\noindent| ---descreasing indention.
% \end{itemize}
% Now we can give an example.\lstaspectindex{\lstdefineformat}{}\lstaspectindex{format}{}
% \begin{lstxsample}
%    \lstdefineformat{C}{%
%        \{=\newline\string\newline\indent,%
%        \}=\newline\noindent\string\newline,%
%        ;=[\ ]\string\space}
% \end{lstxsample}
% \begin{lstsample}{\lstset{language={}}}{}
%    \begin{lstlisting}[format=C]
%    for (int i=0;i<10; i++){/* wait */};
%    \end{lstlisting}
% \end{lstsample}
% Not good. But there is a (too?) simple work-around:
% \begin{lstxsample}
%    \lstdefineformat{C}{%
%        \{=\newline\string\newline\indent,%
%        \}=[;]\newline\noindent\string\newline,%
%        \};=\newline\noindent\string\newline,%
%        ;=[\ ]\string\space}
% \end{lstxsample}
% \begin{lstsample}{\lstset{language={}}}{}
%    \begin{lstlisting}[format=C]
%    for (int i=0;i<10; i++){/* wait */};
%    \end{lstlisting}
% \end{lstsample}
% Sometimes the problem is just to find a suitable format definition.
% Further formatting is complicated.
% Here are only three examples with increasing level of difficulty.
% \begin{enumerate}
% \item Insert horizontal space to separate function/procedure name and
%       following parenthesis or to separate arguments of a function,
%       e.g.~add the space after a comma (if inside function call).
% \item Smart breaking of long lines. Consider long `and/or' expressions.
%       Formatting should follow the logical structure!
% \item Context sensitive formatting rules. It can be annoying if empty
%       or small blocks take three or more lines in the output---think of
%       scrolling down all the time. So it would be nice if the block
%       formatting was context sensitive.
% \end{enumerate}
% Note that this is a very first and clumsy attempt to provide automatic
% formatting---clumsy since the problem isn't trivial. Any ideas are welcome.
% Implementations also. Eventually you should know that you must request format
% definitions at package loading, e.g.~via |\usepackage[formats]{listings}|.
%
% \subsection{Arbitrary linerange markers}\label{rArbitraryLinerangeMarkers}
%
% Instead of using \keyname{linerange} with line numbers, one can use text
% markers. Each such marker consists of a \meta{prefix}, a \meta{text}, and a
% \meta{suffix}. You once (or more) define prefixes and suffixes and then use
% the marker text instead of the line numbers.
% \begin{lstxsample}
%    \lstset{rangeprefix=\{\ ,% curly left brace plus space
%            rangesuffix=\ \}}% space plus curly right brace
% \end{lstxsample}
% \begin{lstsample}{}{}
%    \begin{lstlisting}%
%          [linerange=loop\ 2-end]
%    { loop 1 }
%    for i:=maxint to 0 do
%    begin
%        { do nothing }
%    end;
%    { end }
%    { loop 2 }
%    for i:=maxint to 0 do
%    begin
%        { do nothing }
%    end;
%    { end }
%    \end{lstlisting}
% \end{lstsample}
% Note that \TeX's special characters like the curly braces, the space, the
% percent sign, and such must be escaped with a backslash.
% \begin{syntax}
% \item[1.2] \rkeyname{rangebeginprefix}|=|\meta{prefix}
% \item[1.2] \rkeyname{rangebeginsuffix}|=|\meta{suffix}
% \item[1.2] \rkeyname{rangeendprefix}|=|\meta{prefix}
% \item[1.2] \rkeyname{rangeendsuffix}|=|\meta{suffix}
%
%       define individual prefixes and suffixes for the begin- and end-marker.
%
% \item[1.2] \rkeyname{rangeprefix}|=|\meta{prefix}
% \item[1.2] \rkeyname{rangesuffix}|=|\meta{suffix}
%
%       define identical prefixes and suffixes for the begin- and end-marker.
%
% \item[1.2,true] \rkeyname{includerangemarker}|=|\meta{\alternative{true,false}}
%
%       shows or hides the markers in the output.
% \end{syntax}
% \begin{lstsample}{\lstset{rangeprefix=\{\ ,rangesuffix=\ \}}}{}
%    \begin{lstlisting}%
%          [linerange=loop\ 1-end,
%           includerangemarker=false,
%           frame=single]
%    { loop 1 }
%    for i:=maxint to 0 do
%    begin
%        { do nothing }
%    end;
%    { end }
%    \end{lstlisting}
% \end{lstsample}
%
%
% \subsection{Multicolumn Listings}\label{rMulticolumnListings}
%
% When the \packagename{multicol} package is loaded, it can be used to typeset
% multi-column listings.  These are specified with the |multicols| key.  For
% example:
% \begin{lstsample}{}{}
%    \begin{lstlisting}[multicols=2]
%    if (i < 0)
%      i = 0
%      j = 1
%    end if
%    if (j < 0)
%      j = 0
%    end if
%    \end{lstlisting}
% \end{lstsample}
%
% The multicolumn option is known to fail with some keys.
%
% \begin{advise}
% \item Which keys?
%       \advisespace
%       Unfortunately, I don't know.  Carsten left the code for this option
%       in the version 1.3b patch file with only that cryptic note for
%       documentation.  Bug reports would be welcome, though I don't promise
%       that they're fixable.  ---Brooks
% \end{advise}
%
%
%\iffalse
% \section{Forthcoming ?}
%
% This section is rather rudimentary. It just lists some things I don't want
% to forget.
%
% First of all, I'd like to support even more languages, for example Maple,
% PostScript, and so on. Fortunately my lifetime is limited, so other
% people may do that work. Please (e-)mail me your language definitions.
%
% Then, there are several ideas for the future. Some have already been stated
% as `to do's; some came from other people and are stated below; some more are
% far from being implemented,
%   e.g.~\keyname{linerange}|=|\oarg{inter}\marg{line range list}
% which prints all lines in the range and executes \meta{inter} when omitting
% some code lines. The main problem here are frames and background colours;
% what should happen to them? In fact, the problem is how this can be coded.
% Another idea is to change the background colour (or the basic style) for
% particular code blocks. This, too, is not easy.
%
%^^A Auto-detect whether surplus space (from spaces and tabs) isn't needed to fix
%^^A alignment of wide character combinations like |==| or |<>|.
%^^A
%^^A Make package compatible to calc package.
%^^A
%^^A Rewrite \lst@LAS, \lst@DefDriver, \lst@Require to distinguish loading
%^^A of languages (which don't need base languages at once) and aspects
%^^A (which need required aspects to be loaded).
%
% \lsthelper{Vincent~Poirriez}{1999/11/18}{code examples inside caml comments}:
% Inside caml comments, |[| and |]| should print the code in
% between in basicstyle (or another newly introduced style). Nesting of these
% `code example delimiters' is allowed, e.g.~|(* [[x;y]] *)|.
%
% \lsthelper{Claus~Atzenbeck}{1999/12/03}{`extendedchars=false' doesn't issue
% warning when extended characters are used}: issue warning in final mode if
% \ikeyname{extendedchars}|=false| but extended chars are used.
%
% \lsthelper{Andreas~Matthias}{2000/01/04}{define header/footer to print
% the listing name}: Make the header/footer print the listing name. Some
% people asked for continued captions.
%\fi
%
%
% \part{Tips and tricks}
%
% Note: This part of the documentation is under construction.
% Section \ref{uHowTos} must be sorted by topic and ordered in some way.
% Moreover a new section `Examples' is planned, but not written.
% Lack of time is the main problem \ldots
%
%
% \section{Troubleshooting}\label{uTroubleshooting}
%
% If you're faced with a problem with the \packagename{listings} package, there are
% some steps you should undergo before you make a bug report. First you should
% consult the reference guide to see whether the problem is already known. If not,
% create a \emph{minimal} file which reproduces the problem. Follow these
% instructions:
% \begin{enumerate}
% \item Start from the minimal file in section \ref{uAMinimalFile}.
% \item Add the \LaTeX\ code which causes the problem, but keep it short.
%       In particular, keep the number of additional packages small.
% \item Remove some code from the file (and the according packages) until the
%       problem disappears. Then you've found a crucial piece.
% \item Add this piece of code again and start over with step 3 until all code
%       and all packages are substantial.
% \item You now have a minimal file. Send a bug report to the address on the
%       first page of this documentation and include the minimal file together
%       with the created \texttt{.log}-file. If you use a very special package
%       (i.e.~one not on CTAN), also include the package if its software license
%       allows it.
% \end{enumerate}
%
%
% \section{Bugs and workarounds}\label{uBugsWorkarounds}
%
% \subsection{Listings inside arguments}\label{uListingsArguments}
%
% At the moment it isn't possible to use \verb-\lstinline{...}- in a cell
% of a table\makeatletter\@ifundefined{r@uProcessingInline}{}{%
% (see section \ref{uProcessingInline} on page \pageref{uProcessingInline}
% for more information)},%
% \makeatother%
% but it is possible to define a wrapper macro
% which can be used instead of \verb-\lstinline{...}-:
% \begin{lstsample}[lstlisting]{}{}
%    \newcommand\foo{\lstinline{t}}
%    \newcommand\foobar[2][]{\lstinline[#1]{#2}}
%
%    \begin{tabular}{ll}
%    \foo & a variable\\
%    \foobar[language=java]{int u;} & a declaration
%    \end{tabular}
% \end{lstsample}
%
%
% \subsection{Listings with a background colour and \LaTeX{} escaped
% formulas}
% \label{uListingsBackgroundColour}
%
% If there is any text escaped to \LaTeX{} with some coloured background
% and surrounding frames, then there are gaps in the background as well as
% in the lines making up the frame.
% \begin{lstsample}[lstlisting]{}{}
%    \begin{lstlisting}[language=C, mathescape,
%      backgroundcolor=\color{yellow!10}, frame=tlb]
%    /* the following code computes $\displaystyle\sum_{i=1}^{n}i$ */
%    for (i = 1; i <= limit; i++) {
%      sum += i;
%    }
%    \end{lstlisting}
% \end{lstsample}
%
% At the moment there is only one workaround:
% \begin{itemize}
%   \item Write your code into an external file \meta{filename}.
%   \item Input your code by |\lstinputlisting|\meta{filename} into your
%     document and surround it with a frame generated by |\begin{mdframed}|
%     \ldots{} |\end{mdframed}|.
% \end{itemize}
% \begin{lstsample}[lstlisting]{}{}
%    \begin{verbatimwrite}{temp.c}
%    /* the following code computes $\displaystyle\sum_{i=1}^{n}i$ */
%    for (i = 1; i <= limit; i++) {
%      sum += i;
%    }
%    \end{verbatimwrite}
%    \begin{mdframed}[backgroundcolor=yellow!10, rightline=false]
%      \lstinputlisting[language=C,mathescape,frame={}]{./temp.c}
%    \end{mdframed}
% \end{lstsample}
% For more information about the |verbatimwrite| environment have a look at
% \cite{Fairbairns:moreverb}, the |mdframed| environment is deeply discussed in
% \cite{DanielSchubert:mdframed}.
%
%
% \section{How tos}\label{uHowTos}
%
%
% \subsubsection*{How to reference line numbers}
% Perhaps you want to put |\label{|\meta{whatever}|}| into a \LaTeX\ escape which is
% inside a comment whose delimiters aren't printed?  If you did that, the compiler
% won't see the \LaTeX\ code since it would be inside a comment, and the
% \packagename{listings} package wouldn't print anything since the delimiters would
% be dropped and |\label| doesn't produce any printable output, but you could still
% reference the line number. Well, your wish is granted.
%
% In Pascal, for example, you could make the package recognize the `special'
% comment delimiters |(*@| and |@*)| as begin-escape and end-escape sequences.
% Then you can use this special comment for |\label|s and other things.
% \begin{lstsample}{\lstset{numberstyle=\tiny,stepnumber=2,numbersep=5pt}}{}
%    \lstset{escapeinside={(*@}{@*)}}
%
%    \begin{lstlisting}
%    for i:=maxint to 0 do
%    begin
%        { comment }(*@\label{comment}@*)
%    end;
%    \end{lstlisting}
%    Line \ref{comment} shows a comment.
% \end{lstsample}
% \begin{advise}
% \item Can I use `|(*@|' and `|*)|' instead?
%       \advisespace
%       Yes.
% \item Can I use `|(*|' and `|*)|' instead?
%       \advisespace
%       Sure. If you want this.
% \item Can I use `|{@|' and `|@}|' instead?
%       \advisespace
%       No, never! The second delimiter is not allowed. The character `|@|' is
%       defined to check whether the escape is over. But reading the lonely
%       `end-argument' brace, \TeX\ encounters the error `\texttt{Argument of @
%       has an extra \char125}'. Sorry.
% \item Can I use `|{|' and `|}|' instead?
%       \advisespace
%       No. Again the second delimiter is not allowed. Here now \TeX\ would
%       give you a `\texttt{Runaway argument}' error. Since `|}|' is defined to
%       check whether the escape is over, it won't work as `end-argument' brace.
% \item And how can I use a comment line?
%       \advisespace
%       For example, write `|escapeinside={//*}{\^^M}|'. Here |\^^M| represents
%       the end of line character.
% \end{advise}
%
%
% \subsubsection*{How to gobble characters}
% To make your \LaTeX\ code more readable, you might want to indent your
% \texttt{lstlisting} listings. This indention should not show up in the
% pretty-printed listings, however, so it must be removed. If you indent each code
% line by three characters, you can remove them via |gobble=3|:
% \begin{lstsample}{}{\lstset{showspaces}}
%    \begin{lstlisting}[gobble=3]
%    1  for i:=maxint to 0 do
%     2 begin
%      3    { do nothing }
%    123end;
%
%       Write('Case insensitive ');
%       WritE('Pascal keywords.');
%    \end{lstlisting}
% \end{lstsample}
% Note that empty lines and the beginning and the end of the environment
% need not respect the indention. However, never indent the end by more than
% `\ikeyname{gobble}' characters. Moreover note that tabulators expand to
% |tabsize| spaces before we gobble.
% \begin{advise}
% \item Could I use `\ikeyname{gobble}' together with `|\lstinputlisting|'?
%       \advisespace
%       Yes, but it has no effect.
%
% \item Note that `\ikeyname{gobble}' can also be set via `|\lstset|'.
% \end{advise}
%
%
% \subsubsection*{How to include graphics}
% \lsthelper{Herbert~Weinhandl}{1999/09/06}{listings + eps} found a very easy
% way to include graphics in listings. Thanks for contributing this idea---an
% idea I would never have had.
%
% Some programming languages allow the dollar sign to be part of an identifier.
% But except for intermediate function names or library functions, this
% character is most often unused. The \packagename{listings} package defines
% the \ikeyname{mathescape} key, which lets `|$|' escape to \TeX's math mode.
% This makes the dollar character an excellent candidate for our purpose here:
% use a package which can include a graphic, set \ikeyname{mathescape} true,
% and include the graphic between two dollar signs, which are inside a comment.
%
% The following example is originally from a header file I got from Herbert.
% For the presentation here I use the \texttt{lstlisting} environment and an
% excerpt from the header file. The |\includegraphics| command is from
% David Carlisle's \packagename{graphics} bundle.
% \begin{verbatim}
%   \begin{lstlisting}[mathescape=true]
%   /*
%    $ \includegraphics[height=1cm]{defs-p1.eps} $
%    */
%   typedef struct {
%     Atom_T          *V_ptr;   /* pointer to Vacancy in grid    */
%     Atom_T          *x_ptr;   /* pointer to (A|B) Atom in grid */
%   } ABV_Pair_T;
%   \end{lstlisting}\end{verbatim}
% The result looks pretty good. Unfortunately you can't see it, because the
% graphic wasn't available when the manual was typeset.
%
%
% \subsubsection*{How to get closed frames on each page}
% The package supports closed frames only for listings which don't cross pages.
% If a listing is split on two pages, there is neither a bottom rule at the
% bottom of a page, nor a top rule on the following page. If you insist on
% these rules, you might want to use \texttt{framed.sty} by Donald Arseneau.
% Then you could write
% \begin{verbatim}
%    \begin{framed}
%    \begin{lstlisting}
%      or \lstinputlisting{...}
%    \end{lstlisting}
%    \end{framed}\end{verbatim}
% The package also provides a \texttt{shaded} environment. If you use it, you
% shouldn't forget to define \texttt{shadecolor} with the \packagename{color}
% package.
%
%
% \subsubsection*{How to print national characters with $\Lambda$ and \packagename{listings}}\label{uNationalCharacters}
%
% Apart from typing in national characters directly, you can use the `escape'
% feature described in section \ref{rEscapingToLaTeX}.
% The keys \ikeyname{escapechar}, \ikeyname{escapeinside}, and \ikeyname{texcl}
% allow partial usage of \LaTeX\ code.
%
% Now, if you use $\Lambda$ (Lambda, the \LaTeX\ variant for Omega) and want,
% for example, Arabic comment lines, you need not write |\begin{arab}|
% \ldots\ |\end{arab}| each escaped comment line. This can be automated:
% \begin{verbatim}
%    \lstset{escapebegin=\begin{arab},escapeend=\end{arab}}
%
%    \begin{lstlisting}[texcl]
%    // Replace text by Arabic comment.
%    for (int i=0; i<1; i++) { };
%    \end{lstlisting}\end{verbatim}
% If your programming language doesn't have comment lines, you'll have to use
% \ikeyname{escapechar} or \ikeyname{escapeinside}:
% \begin{verbatim}
%    \lstset{escapebegin=\begin{greek},escapeend=\end{greek}}
%
%    \begin{lstlisting}[escapeinside=`']
%    /* `Replace text by Greek comment.' */
%    for (int i=0; i<1; i++) { };
%    \end{lstlisting}\end{verbatim}
% Note that the delimiters |`| and |'| are essential here. The example doesn't
% work without them. There is a more clever way if the comment delimiters of
% the programming language are single characters, like the braces in Pascal:
% \begin{verbatim}
%    \lstset{escapebegin=\textbraceleft\begin{arab},
%            escapeend=\end{arab}\textbraceright}
%
%    \begin{lstlisting}[escapeinside=\{\}]
%    for i:=maxint to 0 do
%    begin
%        { Replace text by Arabic comment. }
%    end;
%    \end{lstlisting}\end{verbatim}
% Please note that the `interface' to $\Lambda$ is completely untested.
% Reports are welcome!
%
%
% \subsubsection*{How to get bold typewriter type keywords}
% Use the \href{http://www.ctan.org/tex-archive/fonts/luximono}{\packagename{LuxiMono}} package.
%
% \iffalse
% Many people asked for bold typewriter fonts since they aren't included in
% the \LaTeX\ standard distribution. Here now one answer on how to use them
% in spite of that.
% \begin{advise}
% \item Please note that I personally don't regard the following as a good
%       solution. Such a bold typewriter type is too heavy. It would be better
%       to use a light version of \texttt{cmtt} as basic font and \texttt{cmtt}
%       or a \emph{slightly} heavier type for keywords.
%
% \item Why don't you tell us how to use the better solution?
%       \advisespace
%       A light version of \texttt{cmtt} doesn't exist. If it's once available,
%       you can do a similar job as described below.
% \end{advise}
% First of all, you'll need Metafont source files for bold typewriter, e.g.~
% \texttt{cmbtt8.mf}, \texttt{cmbtt9.mf} and \texttt{cmbtt10.mf} from
% \href{ftp://ftp.dante.de/tex-archive/fonts/cm/mf-extra/bold}
%      {CTAN/fonts/cm/mf-extra/bold}.
% Secondly you have to create \texttt{.tfm}-files, i.e.~run the Metafont
% program on these sources. This is possibly done automatically when you use
% the fonts in a document. Finally you must tell \LaTeX\ that you've installed
% bold typewriter fonts. Just use
% \begin{verbatim}
%    \DeclareFontShape{OT1}{cmtt}{bx}{n}
%         {<5><6><7><8>cmbtt8%
%          <9>cmbtt9%
%          <10><10.95>cmbtt10%
%          <12><14.4><17.28><20.74><24.88>cmbtt10%
%          }{}\end{verbatim}
% in the preamble of your document. If you use these fonts often, you might
% want to make a local copy of \texttt{ot1cmtt.fd} and replace the declaration
% there. But note that you're not allowed to distributed the modified file
% under its original name!
% \fi
%
%
% \subsubsection*{How to work with plain text}
% If you want to use \packagename{listings} to set plain text (perhaps with
% line numbers, or like |verbatim| but with line wrapping, or so forth, use
% the empty language: |\lstset{language=}|.
%
%
% \subsubsection*{How to get the developer's guide}
% In the \emph{source directory} of the listings package, i.e.~where
% the \texttt{.dtx} files are, create the file \texttt{ltxdoc.cfg} with the
% following contents.
% \begin{verbatim}
%    \AtBeginDocument{\AlsoImplementation}\end{verbatim}
% Then run \texttt{listings.dtx} through \LaTeX\ twice, run Makeindex (with
% the |-s gind.ist| option), and then run \LaTeX\ one last time on
% \texttt{listings.dtx}. This creates the whole documentation including User's
% guide, Reference guide, Developer's guide, and Implementation.
%
% If you can run the (GNU) make program, executing the command
% \begin{verbatim}
%    make all\end{verbatim}
% or
% \begin{verbatim}
%    make listings-devel\end{verbatim}
% gives the same result---it is called \texttt{listings-devel.pdf}.
%
% \makeatletter
%^^A \def\index@prologue{\section*{Index}\markboth{Index}{Index}}
% \def\index@prologue{\part{Index}\markboth{Index}{Index}}
% \makeatother
%^^A \StopEventually{\lstcheckreference\setcounter{IndexColumns}{2}\PrintIndex}
% \StopEventually{%
% \begin{thebibliography}{MDB01}
%
%     \bibitem[Fai11]{Fairbairns:moreverb}
%       Robin Fairbairns.
%       \newblock{The \textsf{moreverb} package}, 2011.
%
%     \bibitem[DS13]{DanielSchubert:mdframed}
%       Marco Daniel and Elke Schubert.
%       \newblock{The \textsf{mdframed} package}, 2013.
% \end{thebibliography}
% \setcounter{IndexColumns}{2}\PrintIndex}
%
%
% \part{Developer's guide}
%
% First I must apologize for this developer's guide since some parts are not
% explained as well as possible. But note that you are in a pretty good shape:
% this developer's guide exists! ^^A :-)
% You might want to peek into section \ref{dPackageExtensions} before reading
% section \ref{dBasicConcepts}.
%
%
% \section{Basic concepts}\label{dBasicConcepts}
%
% The functionality of the \packagename{listings} package appears to be
% divided into two parts: on the one hand commands which actually typeset
% listings and on the other via |\lstset| adjustable parameters. Both could
% be implemented in terms of \lst-aspects, which are simply collections of
% public keys and commands and internal hooks and definitions. The package
% defines a couple of aspects, in particular the kernel, the main engine.
% Other aspects drive this engine, and language and style definitions tell
% the aspects how to drive. The relations between car, driver and assistant
% driver are exactly reproduced---and I'll be your driving instructor.
%
%
% \subsection{Package loading}\label{dPackageLoading}
%
% Each option in |\usepackage[|\meta{options}|]{listings}| loads an aspect or
% \emph{prevents} the package from loading it if the aspect name is
% \emph{preceded by an exclamation mark}. This mechanism was designed to clear
% up the dependencies of different package parts and to debug the package. For
% this reason there is another option:
% \begin{syntax}
% \item[0.21,,option] \texttt{noaspects}\leavevmode
%
%       deletes the list of aspects to load. Note that, for example, the
%       option lists |0.21,!labels,noaspects| and |noaspects| are essentially
%       the same: the kernel is loaded and no other aspect.
% \end{syntax}
% This is especially useful for aspect-testing since we can load exactly the
% required parts. Note, however, that an aspect is loaded later if a predefined
% programming language requests it. One can load aspects also by hand:
% \begin{syntax}
% \item[0.20] |\lstloadaspects|\marg{comma separated list of aspect names}
%
%       loads the specified aspects if they are not already loaded.
% \end{syntax}
% Here now is a list of all aspects and related keys and commands---in the hope
% that this list is complete.
% \begin{description}
% \hyphenpenalty=10000\relax \rightskip=0pt plus \linewidth\relax
% \item[\aspectname{strings}]\leavevmode
%
%       \lstprintaspectkeysandcmds{strings}
%
% \item[\aspectname{comments}]\leavevmode
%
%       \lstprintaspectkeysandcmds{comments}
%
% \item[\aspectname{pod}]\leavevmode
%
%       \lstprintaspectkeysandcmds{pod}
%
% \item[\aspectname{escape}]\leavevmode
%
%       \lstprintaspectkeysandcmds{escape}
%
% \item[\aspectname{writefile}] requires 1 |\toks|, 1 |\write|
%
%       |\lst@BeginWriteFile|, |\lst@BeginAlsoWriteFile|, |\lst@EndWriteFile|
%
% \item[\aspectname{style}]\leavevmode
%
%       empty style, \lstprintaspectkeysandcmds{style}
%
% \item[\aspectname{language}]\leavevmode
%
%       empty language, \lstprintaspectkeysandcmds{language}
%
% \item[\aspectname{keywords}]\leavevmode
%
%       \lstprintaspectkeysandcmds{keywords}
%
% \item[\aspectname{emph}] requires \aspectname{keywords}
%
%       \lstprintaspectkeysandcmds{emph}
%
% \item[\aspectname{html}] requires \aspectname{keywords}
%
%       \lstprintaspectkeysandcmds{html}
%
% \item[\aspectname{tex}] requires \aspectname{keywords}
%
%       \lstprintaspectkeysandcmds{tex}
%
% \item[\aspectname{directives}] requires \aspectname{keywords}
%
%       \lstprintaspectkeysandcmds{directives}
%
% \item[\aspectname{index}] requires \aspectname{keywords}
%
%       \lstprintaspectkeysandcmds{index}
%
% \item[\aspectname{procnames}] requires \aspectname{keywords}
%
%       \lstprintaspectkeysandcmds{procnames}
%
% \item[\aspectname{keywordcomments}]
%       requires \aspectname{keywords}, \aspectname{comments}
%
%       \lstprintaspectkeysandcmds{keywordcomments}
%
% \item[\aspectname{labels}] requires 2 |\count|
%
%       \lstprintaspectkeysandcmds{labels}
%
% \item[\aspectname{lineshape}] requires 2 |\dimen|
%
%       \lstprintaspectkeysandcmds{lineshape}
%
% \item[\aspectname{frames}] requires \aspectname{lineshape}
%
%       \lstprintaspectkeysandcmds{frames}
%
% \item[\aspectname{make}] requires \aspectname{keywords}
%
%       \lstprintaspectkeysandcmds{make}
%
% \item[\aspectname{doc}] requires \aspectname{writefile} and 1 |\box|
%
%       \lstprintaspectkeysandcmds{doc}
%
% \item[\aspectname{0.21}] defines old keys in terms of the new ones.
% \item[\aspectname{fancyvrb}] requires 1 |\box|
%
%       \lstprintaspectkeysandcmds{fancyvrb}
%
% \item[\aspectname{lgrind}]\leavevmode
%
%       \lstprintaspectkeysandcmds{lgrind}
%
% \item[\aspectname{hyper}] requires \aspectname{keywords}
%
%       \lstprintaspectkeysandcmds{hyper}
% \end{description}
% The kernel allocates 6 |\count|, 4 |\dimen| and 1 |\toks|.
% Moreover it defines the following keys, commands, and environments:
% \begin{itemize}\item[]
% \hyphenpenalty=10000\relax \rightskip=0pt plus \linewidth\relax
%       \lstprintaspectkeysandcmds{kernel}, \keyname{fancyvrb}
% \end{itemize}
%
%
% \subsection{How to define \lst-aspects}\label{dHowToDefineLstAspects}
%
% There are at least three ways to add new functionality: (a) you write an
% aspect of general interest, send it to me, and I'll just paste it into the
% implementation; (b) you write a `local' aspect not of general interest; or
% (c) you have an idea for an aspect and make me writing it. (a) and (b) are
% good choices.^^A :-)
%
% An aspect definition starts with |\lst@BeginAspect| plus arguments and ends
% with the next |\lst@EndAspect|. In particular, aspect definitions can't be
% nested.
% \begin{syntax}
% \item[0.20] |\lst@BeginAspect|[\oarg{list of required aspects}]\marg{aspect name}
% \item[0.20] |\lst@EndAspect|
% \end{syntax}
% The optional list is a comma separated list of required aspect names.
% The complete aspect is not defined in each of the following cases:
% \begin{enumerate}
% \item \meta{aspect name} is empty.
% \item The aspect is already defined.
% \item A required aspect is neither defined nor loadable via
%       |\lstloadaspects|.
% \end{enumerate}
% Consequently you can't define a part of an aspect and later on another part.
% But it is possible to define aspect $A_1$ and later aspect $A_2$ which
% requires $A_1$.
% \begin{advise}
% \item Put local add-ons into `\texttt{lstmisc0.sty}'---this file is searched
%       first by default. If you want to make add-ons for one particular
%       document just replace the surrounding `|\lst@BeginAspect|' and
%       `|\lst@EndAspect|' by `|\makeatletter|' and `|\makeatother|' and use
%       the definitions in the preamble of your document. However, you have to
%       load required aspects on your own.
% \end{advise}
% You can put any \TeX\ material in between the two commands, but note that
% definitions must be |\global| if you need them later---\LaTeX's |\newcommand|
% makes local definitions and can't be preceded by |\global|. So use the
% following commands, |\gdef|, and commands described in later sections.
% \begin{syntax}
% \item[0.20] |\lst@UserCommand|\meta{macro}\meta{parameter text}\marg{replacement text}
%
%       The macro is (mainly) equivalent to |\gdef|. The purpose is to
%       distinguish user commands and internal global definitions.
%
% \item[0.19] |\lst@Key|\marg{key name}\marg{init value}[\oarg{default value}]\marg{definition}
% \item[0.19] |\lst@Key|\marg{key name}|\relax|[\oarg{default value}]\marg{definition}
%
%       defines a key using the \packagename{keyval} package from David
%       Carlisle. \meta{definition} is the replacement text of a macro with
%       one parameter. The argument is either the value from `key=value' or
%       \meta{default value} if no `=value' is given. The helper macros
%       |\lstKV@...| below might simplify \meta{definition}.
%
%       The key is not initialized if the second argument is |\relax|.
%       Otherwise \meta{init value} is the initial value given to the key.
%       Note that we locally switch to |\globalsdefs=1| to ensure that
%       initialization is not effected by grouping.
%
% \item[0.19] |\lst@AddToHook|\marg{name of hook}\marg{\TeX\ material}
%
%       adds \TeX\ material at predefined points. Section \ref{dHooks} lists
%       all hooks and where they are defined respectively executed.
%       |\lst@AddToHook{A}{\csa}| before |\lst@AddToHook{A}{\csb}|
%       \emph{does not} guarantee that |\csa| is executed before |\csb|.
%
% \item[0.20] |\lst@AddToHookExe|\marg{name of hook}\marg{\TeX\ material}
%
%       also executes \meta{\TeX\ material} for initialization. You might use
%       local variables---local in the sense of \TeX\ and/or usual programming
%       languages---but when the code is executed for initialization all
%       assignments are global: we set |\globaldefs| locally to one.
%
% \item[0.20] |\lst@UseHook|\marg{name of hook}
%
%       executes the hook.
% \end{syntax}
% \begin{advise}
% \item Let's look at two examples. The first extends the package by adding
%       some hook-material. If you want status messages, you might write
% \begin{verbatim}
%    \lst@AddToHook{Init}{\message{\MessageBreak Processing listing ...}}
%    \lst@AddToHook{DeInit}{\message{complete.\MessageBreak}}\end{verbatim}
%       The second example introduces two keys to let the user control the
%       messages. The macro |\lst@AddTo| is described in section
%       \ref{dGeneralPurposeMacros}.
% \begin{verbatim}
%   \lst@BeginAspect{message}
%   \lst@Key{message}{Annoying message.}{\gdef\lst@message{#1}}
%   \lst@Key{moremessage}\relax{\lst@AddTo\lst@message{\MessageBreak#1}}
%   \lst@AddToHook{Init}{\typeout{\MessageBreak\lst@message}}
%   \lst@EndAspect\end{verbatim}
%       However, there are certainly aspects which are more useful.
% \end{advise}
% The following macros can be used in the \meta{definition} argument of the
% |\lst@Key| command to evaluate the argument. The additional prefix |KV|
% refers to the \packagename{keyval} package.
% \begin{syntax}
% \item[0.19] |\lstKV@SetIf|\marg{value}\meta{if macro}
%
%       \meta{if macro} becomes |\iftrue| if the first character of
%       \meta{value} equals |t| or |T|. Otherwise it becomes |\iffalse|.
%       Usually you will use |#1| as \meta{value}.
%
% \item[1.0] \cs{lstKV@SwitchCases}\marg{value}\\
%   |{|\meta{string 1}|&|\meta{execute 1}|\\|\\
%   | |\meta{string 2}|&|\meta{execute 2}|\\|\\
%   \hbox to 3em{\hfill\vdots}\\
%   | |\meta{string $n$}|&|\meta{execute $n$}|}|\marg{else}
%
%       Either execute \meta{else} or the \meta{value} matching part.
%
% \item[0.20] |\lstKV@TwoArg|\marg{value}\marg{subdefinition}
% \item[0.20] |\lstKV@ThreeArg|\marg{value}\marg{subdefinition}
% \item[0.20] |\lstKV@FourArg|\marg{value}\marg{subdefinition}
%
%       \meta{subdefinition} is the replacement text of a macro with two,
%       three, and four parameters. We call this macro with the arguments given
%       by \meta{value}. Empty arguments are added if necessary.
%
% \item[0.19] |\lstKV@OptArg|\oarg{default arg.}\marg{value}\marg{subdefinition}
%
%       |[|\meta{default arg.}|]| is \emph{not} optional. \meta{subdefinition}
%       is the replacement text of a macro with parameter text |[##1]##2|.
%       Note that the macro parameter character |#| is doubled since used
%       within another macro. \meta{subdefinition} accesses these arguments
%       via |##1| and |##2|.
%
%       \meta{value} is usually the argument |#1| passed by the
%       \packagename{keyval} package. If \meta{value} has no optional argument,
%       \meta{default arg.} is inserted to provide the arguments to
%       \meta{subdefinition}.
%
% \item[0.21] |\lstKV@XOptArg|\oarg{default arg.}\marg{value}\meta{submacro}
%
%       Same as |\lstKV@OptArg| but the third argument \meta{submacro} is
%       already a definition and not replacement text.
%
% \item[0.20] |\lstKV@CSTwoArg|\marg{value}\marg{subdefinition}
%
%       \meta{value} is a \texttt comma \texttt separated list of one or two
%       arguments. These are given to the subdefinition which is the
%       replacement text of a macro with two parameters. An empty second
%       argument is added if necessary.
% \end{syntax}
% \begin{advise}
% \item One more example. The key `\keyname{sensitive}' belongs to the aspect
%       \aspectname{keywords}. Therefore it is defined in between
%       `|\lst@BeginAspect{keywords}|' and `|\lst@EndAspect|', which is not shown
%       here.
% \begin{verbatim}
%   \lst@Key{sensitive}\relax[t]{\lstKV@SetIf{#1}\lst@ifsensitive}
%   \lst@AddToHookExe{SetLanguage}{\let\lst@ifsensitive\iftrue}\end{verbatim}
%       The last line is equivalent to
% \begin{verbatim}
%   \lst@AddToHook{SetLanguage}{\let\lst@ifsensitive\iftrue}
%   \global\let\lst@ifsensitive\iftrue\end{verbatim}
%       We initialize the variable globally since the user might request an
%       aspect in a group. Afterwards the variable is used locally---there is
%       no |\global| in \meta{\TeX\ material}. Note that we could define and
%       init the key as follows:
% \begin{verbatim}
%   \lst@Key{sensitive}t[t]{\lstKV@SetIf{#1}\lst@ifsensitive}
%   \lst@AddToHook{SetLanguage}{\let\lst@ifsensitive\iftrue}\end{verbatim}
%\end{advise}
%
%
% \subsection{Internal modes}\label{dInternalModes}
%
% You probably know \TeX's conditional commands |\ifhmode|, |\ifvmode|,
% |\ifmmode|, and |\ifinner|. They tell you whether \TeX\ is in (restricted)
% horizontal or (internal) vertical or in (nondisplay) mathematical mode. For
% example, true |\ifhmode| and true |\ifinner| indicate restricted horizontal
% mode, which means that you are in a |\hbox|. The typical user doesn't care
% about such modes; \TeX/\LaTeX\ manages all this. But since you're reading the
% developer's guide, we discuss the analogue for the \packagename{listings}
% package now. It uses modes to distinguish comments from strings, `comment
% lines' from `single comments', and so on.
%
% The package is in `no mode' before reading the source code. In the phase of
% initialization it goes to `processing mode'. Afterwards the mode depends on
% the actual source code. For example, consider the line
% \begin{verbatim}
%    "string" // comment\end{verbatim}
% and assume \texttt{language=C++}. Reading the string delimiter, the package
% enters `string mode' and processes the string. The matching closing delimiter
% leaves the mode, i.e.\ switches back to the general `processing mode'. Coming
% to the two slashes, the package detects a comment line; it therefore enters
% `comment line mode' and outputs the slashes. Usually this mode lasts to the
% end of line.
%
% But with \texttt{textcl=true} the \aspectname{escape} aspect immediately
% leaves `comment line mode', interrupts the current mode sequence, and enters
% `\TeX\ comment line mode'. At the end of line we reenter the previous mode
% sequence `no mode' $\to$ 'processing mode'. This escape to \LaTeX\ works
% since `no mode' implies that \TeX's characters and catcodes are present,
% whereas `processing mode' means that \packagename{listings}' characters and
% catcodes are active.
%
% \begin{table}[htbp]
% \caption{Internal modes}\label{dDefinedInternalModes}
% \def\lsttabspace{\hspace*{1em}\hfill}
% \begin{tabular}{@{}lp{0.56\linewidth}@{}}
% aspect\lsttabspace\meta{mode name} & Usage/We are processing \ldots\\
% \noalign{\smallskip}
% kernel\lsttabspace |\lst@nomode| &
%       If this mode is active, \TeX's `character table' is present; the other
%       implication is not true. Any other mode \emph{may} imply that catcodes
%       and\nobreak/\allowbreak or definitions of characters are changed.
% \\
%       \lsttabspace |\lst@Pmode| &
%       is a general processing mode. If active we are processing a listing,
%       but haven't entered a more special mode.
% \\
%       \lsttabspace |\lst@GPmode| &
%       general purpose mode for language definitions.
% \\
% \aspectname{pod}\lsttabspace |\lst@PODmode| &
%       \ldots~a POD---Perl specific.
% \\
% \aspectname{escape}\lsttabspace |\lst@TeXLmode| &
%       \ldots~a comment line, but \TeX's character table is present---except
%       the EOL character, which is needed to terminate this mode.
% \\
%       \lsttabspace |\lst@TeXmode| &
%       indicates that \TeX's character table is present (except one user
%       specified character, which is needed to terminate this mode).
% \\
% \aspectname{directives}\lsttabspace |\lst@CDmode| &
%       indicates that the current line began with a compiler directive.
% \\
% \aspectname{keywordcomments}\lsttabspace |\lst@KCmode| &
%       \ldots~a keyword comment.
% \\
%       \lsttabspace |\lst@KCSmode| &
%       \ldots~a keyword comment which can be terminated by a semicolon only.
% \\
% \aspectname{html}\lsttabspace |\lst@insidemode| &
%       Active if we are between \texttt{<} and \texttt{>}.
% \\
% \aspectname{make}\lsttabspace |\lst@makemode| &
%       Used to indicate a keyword.
% \end{tabular}
% \end{table}
% Table \ref{dDefinedInternalModes} lists all static modes and which aspects
% they belong to. Most features use dynamically created mode numbers, for
% example all strings and comments. Each aspect may define its own mode(s)
% simply by allocating it/\allowbreak them inside the aspect definition.
% \begin{syntax}
% \item[0.19] |\lst@NewMode|\meta{mode \textup(control sequence\textup)}
%
%       defines a new static mode, which is a nonnegative integer assigned to
%       \meta{mode}. \meta{mode} should have the prefix \texttt{lst@} and
%       suffix \texttt{mode}.
%
% \item[0.21] |\lst@UseDynamicMode|\marg{token\textup(s\textup)}
%
%       inserts a dynamic mode number as argument to the token(s).
%
%       This macro cannot be used to get a mode number when an aspect is
%       loaded or defined. It can only be used every listing in the process
%       of initialization, e.g.~to define comments when the character table
%       is selected.
%
% \item[0.19,,changed] |\lst@EnterMode|\meta{mode}\marg{start tokens}
%
%       opens a group level, enters the mode, and executes \meta{start tokens}.
%
%       Use |\lst@modetrue| in \meta{start tokens} to prohibit future mode
%       changes---except leaving the mode, of course. You must test yourself
%       whether you're allowed to enter, see below.
%
% \item[0.19] |\lst@LeaveMode|
%
%       returns to the previous mode by closing a group level if and only if
%       the current mode isn't |\lst@nomode| already. You must test yourself
%       whether you're allowed to leave a mode, see below.
%
%\iffalse
% \item[0.19] |\lst@LeaveAllModes|
%
%       returns to |\lst@nomode|.
%       This is some kind of emergency macro, so don't use it!
%\fi
%
% \item[0.19] |\lst@InterruptModes|
% \item[0.19] |\lst@ReenterModes|
%
%       The first command returns to |\lst@nomode|, but saves the current mode
%       sequence on a special stack. Afterwards the second macro returns to the
%       previous mode. In between these commands you may enter any mode you
%       want. In particular you can interrupt modes, enter some modes, and say
%       `interrupt modes' again. Then two re-enters will take you back in front
%       of the first `interrupt modes'.
%
%       Remember that |\lst@nomode| implies that \TeX's character table is
%       active.
% \end{syntax}
% Some variables show the internal state of processing. You are allowed to read
% them, but \emph{direct write access is prohibited}. Note: |\lst@ifmode| is
% \emph{not} obsolete since there is no relation between the boolean and the
% current mode. It will happen that we enter a mode without setting
% |\lst@ifmode| true, and we'll set it true without assigning any mode!
% \begin{syntax}
% \item[0.18,,counter] |\lst@mode|
%
%       keeps the current mode number. Use |\ifnum\lst@mode=|\meta{mode name}
%       to test against a mode. Don't modify the counter directly!
%
% \item[0.18,,boolean] |\lst@ifmode|
%
%       No mode change is allowed if this boolean is true---except leaving the
%       current mode. Use |\lst@modetrue| to modify this variable, but do it
%       only in \meta{start tokens}.
%
% \item[1.0,,boolean] |\lst@ifLmode|
%
%       Indicates whether the current mode ends at end of line.
% \end{syntax}
%
%
% \subsection{Hooks}\label{dHooks}
%
% Several problems arise if you want to define an aspect.
% You should and/or must
%   (a) find additional functionality (of general interest) and implement it,
%   (b) create the user interface, and
%   (c) interface with the \packagename{listings} package, i.e.~find correct
%       hooks and insert appropriate \TeX\ material.
% (a) is out of the scope of this developer's guide. The commands |\lstKV@...|
% in section \ref{dHowToDefineLstAspects} might help you with (b). Here now we
% describe all hooks of the \packagename{listings} package.
%
% All hooks are executed inside an overall group. This group starts somewhere
% near the beginning and ends somewhere at the end of each listing. Don't make
% any other assumptions on grouping. So define variables globally if it's
% necessary---and be alert of side effects if you don't use your own groups.
% \begin{syntax}
% \item \hookname{AfterBeginComment}
%
%       is executed after the package has entered comment mode.
%       The starting delimiter is usually typeset when the hook is called.
%
% \item \hookname{BoxUnsafe}
%
%       Contains all material to deactivate all commands and registers which
%       are possibly unsafe inside |\hbox|. It is used whenever the package
%       makes a box around a listing and for \packagename{fancyvrb} support.
%
% \item \hookname{DeInit}
%
%       Called at the very end of a listing but before closing the box from
%       \hookname{BoxUnsafe} or ending a float.
%
% \item \hookname{DetectKeywords}
%
%       This \hookname{Output} subhook is executed if and only if mode changes
%       are allowed, i.e.~if and only if the package doesn't process a comment,
%       string, and so on---see section \ref{dInternalModes}.
%
% \item \hookname{DisplayStyle}
%
%       deactivates/activates features for displaystyle listings.
%
% \item \hookname{EmptyStyle}
%
%       Executed to select the `empty' style---except the user has redefined
%       the style.
%
% \item \hookname{EndGroup}
%
%       Executed whenever the package closes a group, e.g.~at end of comment or
%       string.
%
% \item \hookname{EOL}
%
%       Called at each end of \emph{input} line, right before
%       \hookname{InitVarsEOL}.
%
% \item \hookname{EveryLine}
%
%       Executed at the beginning of each \emph{output} line, i.e.~more than
%       once for broken lines. This hook must not change the horizontal or
%       vertical position.
%
% \item \hookname{EveryPar}
%
%       Executed once for each input line when the output starts. This hook
%       must not change the horizontal or vertical position.
%
%^^A \item \hookname{ExcludeDelims}
%^^A
%^^A       Executed by the \keyname{excludedelims} key before the excluded
%^^A       delimiters are determined.
%^^A
% \item \hookname{ExitVars}
%
%       Executed right before \hookname{DeInit}.
%
% \item \hookname{FontAdjust}
%
%       adjusts font specific internal values (currently |\lst@width| only).
%
% \item \hookname{Init}
%
%       Executed once each listing to initialize things before the character
%       table is changed. It is called after \hookname{PreInit} and before
%       \hookname{InitVars}.
%
% \item \hookname{InitVars}
%
%       Called to init variables each listing.
%
% \item \hookname{InitVarsBOL}
%
%       initializes variables at the beginning of each input line.
%
% \item \hookname{InitVarsEOL}
%
%       updates variables at the end of each input line.
%
% \item \hookname{ModeTrue}
%
%       executed by the package when mode changes become illegal.
%       Here keyword detection is switched off for comments and strings.
%
% \item \hookname{OnEmptyLine}
%
%       executed \emph{before} the package outputs an empty line.
%
% \item \hookname{OnNewLine}
%
%       executed \emph{before} the package starts one or more new lines,
%       i.e.~before saying |\par\noindent\hbox{}| (roughly speaking).
%
% \item \hookname{Output}
%
%       Called before an identifier is printed.
%       If you want a special printing style, modify |\lst@thestyle|.
%
% \item \hookname{OutputBox}
%
%       used inside each output box. Currently it is only used to make the
%       package work together with Lambda---hopefully.
%
% \item \hookname{OutputOther}
%
%       Called before other character strings are printed.
%       If you want a special printing style, modify |\lst@thestyle|.
%
% \item \hookname{PostOutput}
%
%       Called after printing an identifier or any other output unit.
%
% \item \hookname{PostTrackKeywords}
%
%       is a very special \hookname{Init} subhook to insert keyword tests and
%       define keywords on demand.
%       This hook is called after \hookname{TrackKeywords}.
%
% \item \hookname{PreInit}
%
%       Called right before \hookname{Init} hook.
%
% \item \hookname{PreSet}
%
%       Each typesetting command/environment calls this hook to initialize
%       internals before any user supplied key is set.
%
% \item \hookname{SelectCharTable}
%
%       is executed after the package has selected the standard character
%       table. Aspects adjust the character table here and define string and
%       comment delimiters, and such.
%
% \item \hookname{SetFormat}
%
%       Called before internal assignments for setting a format are made.
%       This hook determines which parameters are reset every format selection.
%
% \item \hookname{SetStyle}
%
%       Called before internal assignments for setting a style are made.
%       This hook determines which parameters are reset every style selection.
%
% \item \hookname{SetLanguage}
%
%       Called before internal assignments for setting a language are made.
%       This hook determines which parameters are reset every language
%       selection.
%
% \item \hookname{TextStyle}
%
%       deactivates/activates features for textstyle listings.
%
% \item \hookname{TrackKeywords}
%
%       is a very special \hookname{Init} subhook to insert keyword tests and
%       define keywords on demand.
%       This hook is called before \hookname{PostTrackKeywords}.
% \end{syntax}
%
%
% \subsection{Character tables}\label{dCharacterTables}
%
% Now you know how a car looks like, and you can get a driving license if you
% take some practice. But you will have difficulties if you want to make heavy
% alterations to the car. So let's take a closer look and come to the most
% difficult part: the engine. We'll have a look at the big picture and fill in
% the details step by step. For our purpose it's good to override \TeX's
% character table. First we define a standard character table which contains
% \begin{itemize}
% \item letters: characters identifiers are out of,
% \item digits: characters for identifiers or numerical constants,
% \item spaces: characters treated as blank spaces,
% \item tabulators: characters treated as tabulators,
% \item form feeds: characters treated as form feed characters, and
% \item others: all other characters.
% \end{itemize}
% This character table is altered depending on the current programming language.
% We may define string and comment delimiters or other special characters.
% Table \ref{rStdCharTable} on page \pageref{rStdCharTable} shows the standard
% character table. It can be modified with the keys \keyname{alsoletter},
% \keyname{alsodigit}, and \keyname{alsoother}.
%
% How do these `classes' work together? Let's say that the current character
% string is `|tr|'. Then letter `|y|' simply appends the letter and we get
% `|try|'. The next nonletter (and nondigit) causes the output of the
% characters. Then we collect all coming nonletters until reaching a letter
% again. This causes the output of the nonletters, and so on. Internally each
% character becomes active in the sense of \TeX\ and is defined to do the right
% thing, e.g.~we say
% \begin{verbatim}
%    \def A{\lst@ProcessLetter A}\end{verbatim}
% where the first `|A|' is active and the second has letter catcode 11.
% The macro |\lst@ProcessLetter| gets one token and treats it as a letter.
% The following macros exist, where the last three get no explicit argument.
% \begin{syntax}
% \item[0.18] |\lst@ProcessLetter| \meta{spec.\ token}
% \item[0.18] |\lst@ProcessDigit| \meta{spec.\ token}
% \item[0.18] |\lst@ProcessOther| \meta{spec.\ token}
% \item[0.18] |\lst@ProcessTabulator|
% \item[0.18] |\lst@ProcessSpace|
% \item[0.20] |\lst@ProcessFormFeed|
% \end{syntax}
% \meta{spec.\ token} is supposed to do two things. Usually it expands to
% a printable version of the character. But if |\lst@UM| is equivalent to
% |\@empty|, \meta{spec.\ token} must expand to a \emph{character token}.
% For example, the sharp usually expands to |\#|, which is defined via
% |\chardef| and is not a character token. But if |\lst@UM| is equivalent to
% |\@empty|, the sharp expands to the character `|#|' (catcode 12). Note:
% \emph{Changes to} |\lst@UM| \emph{must be locally.}  However, there should
% be no need to do such basic things yourself. The \packagename{listings}
% package provides advanced macros which use that feature,
% e.g.~|\lst@InstallKeywords| in section \ref{dKeywordsAndWorkingIdentifiers}.
%
% \begin{syntax}
% \item[0.18] |\lst@Def|\marg{character code}\meta{parameter text}\marg{definition}
% \item[0.18] |\lst@Let|\marg{character code}\meta{token}
%
%       defines the specified character respectively assigns \meta{token}.
%       The catcode table if not affected. Be careful if your definition has
%       parameters: it is not safe to read more than one character ahead.
%       Moreover, the argument can be \emph{arbitrary}; somtimes it's the next
%       source code character, sometimes it's some code of the
%       \packagename{listings} package, e.g.~|\relax|, |\@empty|, |\else|,
%       |\fi|, and so on. Therefore don't use \TeX's ord-operator |`| on such
%       an argument, e.g.~don't write |\ifnum`#1=65| to test against `|A|'.
%
%       |\lst@Def| and |\lst@Let| are relatively slow. The real definition of
%       the standard character table differs from the following example, but it
%       could begin with
% \begin{verbatim}
%    \lst@Def{9}{\lst@ProcessTabulator}
%    \lst@Def{32}{\lst@ProcessSpace}
%    \lst@Def{48}{\lst@ProcessDigit 0}
%    \lst@Def{65}{\lst@ProcessLetter A}\end{verbatim}
%
%\iffalse
% \item[0.20] |\lst@activecharstrue|
% \item[0.20] |\lst@activecharsfalse|
%
%       control whether selecting the character table also makes all characters
%       active (standard/extended). This is usually true and therefore default.
%       Only the \packagename{fancyvrb} interface sets it locally false.
%\fi
% \end{syntax}
% That's enough for the moment. Section \ref{dUsefulInternalDefinitions}
% presents advanced definitions to manipulate the character table, in
% particular how to add new comment or string types.
%
%
% \subsection{On the output}
%
% The \packagename{listings} package uses some variables to keep the output
% data. Write access is not recommended. Let's start with the easy ones.
% \begin{syntax}
% \item[0.17,,data] |\lst@lastother|
%
%       equals \meta{spec.\ token} version of the last processed
%       nonidentifier-character. Since programming languages redefine the
%       standard character table, we use the original \meta{spec.\ token}.
%       For example, if a double quote was processed last, |\lst@lastother|
%       is not equivalent to the macro which enters and leaves string mode.
%       It's equivalent to |\lstum@"|, where |"| belongs to the control
%       sequence. Remember that \meta{spec.\ token} expands either to a
%       printable or to a token character.
%
%       |\lst@lastother| is equivalent to |\@empty| if such a character is not
%       available, e.g.~at the beginning of a line. Sometimes an indentifier
%       has already been printed after processing the last `other' character,
%       i.e.~the character is far, far away. In this case |\lst@lastother|
%       equals |\relax|.
%
% \item[0.17] |\lst@outputspace|
%
%       Use this predefined \meta{spec.\ token} (obviously for character code
%       32) to test against |\lst@lastother|.
%
% \item[0.20] |\lstum@backslash|
%
%       Use this predefined \meta{spec.\ token} (for character code 92) to test
%       against |\lst@lastother|. In the replacement text for |\lst@Def| one
%       could write |\ifx| |\lst@lastother| |\lstum@backslash| \ldots\ to test
%       whether the last character has been a backslash.
%
% \item[0.20] |\lst@SaveOutputDef|\marg{character code}\meta{macro}
%
%       Stores the \meta{spec.\ token} corresponding to \meta{character code}
%       in \meta{macro}. This is the only safe way to get a correct meaning to
%       test against |\lst@lastother|, for example
%           |\lst@SaveOutputDef{"5C}\lstum@backslash|.
%
%       You'll get a ``runaway argument'' error if \meta{character code} is not
%       between 33 and 126 (inclusive).
% \end{syntax}
% Now let's turn to the macros dealing a bit more with the output data and
% state.
% \begin{syntax}
% \item[1.0] |\lst@XPrintToken|
%
%       outputs the current character string and resets it. This macro keeps
%       track of all variables described here.
%
% \item[0.18,,token] |\lst@token|
%
%       contains the current character string. Each `character' usually
%       expands to its printable version, but it must expand to a character
%       token if |\lst@UM| is equivalent to |\@empty|.
%
% \item[0.12,,counter] |\lst@length|
%
%       is the length of the current character string.
%
% \item[0.12,,dimension] |\lst@width|
%
%       is the width of a single character box.
%
% \item[0.20,,global dimension] |\lst@currlwidth|
%
%       is the width of so far printed line.
%
% \item[0.18,,global counter] |\lst@column|
% \item[0.12,,global counter] |\lst@pos| (nonpositive)
%
%       |\lst@column|$-$|\lst@pos| is the length of the so far printed line.
%       We use two counters since this simplifies tabulator handling:
%       |\lst@pos| is a nonpositive representative of `length of so far
%       printed line' modulo \keyname{tabsize}.
%       It's usually not the biggest nonpositive representative.
%
% \item[0.20] |\lst@CalcColumn|
%
%       |\@tempcnta| gets |\lst@column| $-$ |\lst@pos| $+$ |\lst@length|.
%       This is the current column number minus one, or the current column
%       number zero based.
%
% \item[0.18,,global dimension] |\lst@lostspace|
%
%       equals `lost' space: desired current line width minus real line width.
%       Whenever this dimension is positive the flexible column format can use
%       this space to fix the column alignment.
% \end{syntax}
%
%
% \section{Package extensions}\label{dPackageExtensions}
%
%
% \subsection{Keywords and working identifiers}\label{dKeywordsAndWorkingIdentifiers}
%
% The \aspectname{keywords} aspect defines two main macros. Their respective
% syntax is shown on the left. On the right you'll find examples how the
% package actually defines some keys.
% \begin{syntax}
% \item[0.21]
%   \cs{lst@InstallFamily}
%
%   \marg{prefix}\syntaxfill \texttt k\\
%   \marg{name}\syntaxfill |{keywords}|\\
%   \marg{style name}\syntaxfill |{keywordstyle}|\\
%   \marg{style init}\syntaxfill |\bfseries|\\
%   \marg{default style name}\syntaxfill |{keywordstyle}|\\
%   \marg{working procedure}\syntaxfill |{}|\\
%   \meta{\alternative{l,o}}\syntaxfill |l|\\
%   \meta{\alternative{d,o}}\syntaxfill |d|
%
%       installs either a keyword or `working' class of identifiers according
%       to whether \meta{working procedure} is empty.
%
%       The three keys \meta{name}, \keyname{more}\meta{name} and
%       \keyname{delete}\meta{name}, and if not empty \meta{style name} are
%       defined. The first order member of the latter one is initialized with
%       \meta{style init} if not equivalent to |\relax|. If the user leaves a
%       class style undefined, \meta{default style name} is used instead.
%       Thus, make sure that this style is always defined. In the example,
%       the first order keywordstyle is set to |\bfseries| and is the default
%       for all other classes.
%
%       If \meta{working procedure} is not empty, this code is executed when
%       reaching such an (user defined) identifier. \meta{working procedure}
%       takes exactly one argument, namely the class number to which the
%       actual identifier belongs to. If the code uses variables and requires
%       values from previous calls, you must define these variables
%       |\global|ly. It's not sure whether working procedures are executed
%       inside a (separate) group or not.
%
%       \texttt l indicates a language key, i.e.~the lists are reset every
%       language selection. \texttt o stands for `other' key.
%       The keyword respectively working test is either installed at the
%       \hookname{DetectKeyword} or \hookname{Output} hook according to
%       \meta{\alternative{d,o}}.
%
% \item[0.20]
%   \cs{lst@InstallKeywords}
%
%   \marg{prefix}\syntaxfill \texttt{cs}\\
%   \marg{name}\syntaxfill |{texcs}|\\
%   \marg{style name}\syntaxfill |{texcsstyle}|\\
%   \marg{style init}\syntaxfill |\relax|\\
%   \marg{default style name}\syntaxfill |{keywordstyle}|\\
%   \marg{working procedure}\syntaxfill see below\\
%   \meta{\alternative{l,o}}\syntaxfill |l|\\
%   \meta{\alternative{d,o}}\syntaxfill |d|
%
%       Same parameters, same functionality with one execption. The macro
%       installs exactly one keyword class and not a whole family. Therefore
%       the argument to \meta{working procedure} is constant (currently empty).
%
%       The working procedure of the example reads as follows.\vspace*{-.5\baselineskip}
% \begin{verbatim}
%    {\ifx\lst@lastother\lstum@backslash
%         \let\lst@thestyle\lst@texcsstyle
%     \fi}\end{verbatim}
%\vspace*{-.5\baselineskip}
%       What does this procedure do? First of all it is called only if a
%       keyword from the user supplied list (or language definition) is found.
%       The procedure now checks for a preceding backslash and sets the output
%       style accordingly.
%
%\iffalse
% \item[0.20] |\lst@InstallTest|\marg{prefix}\syntaxbreak
%       |\lst@|\meta{name}|@list||\lst@|\meta{name}~|\lst@g|\meta{name}|@list||\lst@g|\meta{name}\syntaxbreak
%       |\lst@g|\meta{name}|@sty|~\meta{\alternative{w,s}}\meta{\alternative{d,o}}
%
%       installs a `working identifier' test or keyword style depending on
%       \meta{\alternative{w,s}}. |\lst@g|\meta{name}|@sty| contains the
%       working procedure or style macro. Note that the behaviour of the tests
%       depends on the \texttt{savemem} option.
%       The keyword respectively working test is either installed at the
%       \hookname{DetectKeyword} or \hookname{Output} hook according to
%       \meta{\alternative{d,o}}.
%
%^^A    Either each call of this macro or each different \meta{prefix} inserts
%^^A    another test (depending on the \texttt{savemem} option).
%
%       |\lst@|\meta{name} contains the current identifier list and
%       |\lst@|\meta{name}|@list| a `|\lst@|\meta{$n_i$}|\lst@g|\meta{$n_i$}'
%       sequence of currently used classes. If no other classes are used,
%       this equals |\lst@|\meta{name}|\lst@g|\meta{name}. The global versions
%       |\lst@g|\ldots\ are used to keep track of the keywords.
%       (This description needs improvement.)
%\fi
% \end{syntax}
%
%
% \subsection{Delimiters}
%
% We describe two stages: adding a new delimiter type to an existing class of
% delimiters and writing a new class. Each class has its name; currently exist
% \texttt{Comment}, \texttt{String}, and \texttt{Delim}. As you know, the
% latter and the first both provide the type \texttt l, but there is no string
% which starts with the given delimiter and ends at end of line. So we'll add
% it now!
%
% First of all we extend the list of string types by
% \begin{verbatim}
%    \lst@AddTo\lst@stringtypes{,l}\end{verbatim}
% Then we must provide the macro which takes the user supplied delimiter and
% makes appropriate definitions. The command name consists of the prefix
% |\lst@|, the delimiter name, |DM| for using dynamic modes, and |@| followed
% by the type.
% \begin{verbatim}
%    \gdef\lst@StringDM@l#1#2\@empty#3#4#5{%
%        \lst@CArg #2\relax\lst@DefDelimB{}{}{}#3{#1}{#5\lst@Lmodetrue}}\end{verbatim}
% You can put these three lines into a \texttt{.sty}-file or surround them by
% |\makeatletter| and |\makeatother| in the preamble of a document.
% And that's all!
%{\makeatletter
%\lst@AddTo\lst@stringtypes{,l}
%\gdef\lst@StringDM@l#1#2\@empty#3#4#5{^^A
%   \lst@CArg #2\relax\lst@DefDelimB{}{}{}#3{#1}{#5\lst@Lmodetrue}}
%}
% \begin{lstsample}{}{}
%    \lstset{string=[l]//}
%    \begin{lstlisting}
%    // This is a string.
%    This isn't a string.
%    \end{lstlisting}
% \end{lstsample}
% You want more details, of course. Let's begin with the arguments.
% \begin{itemize}
% \item The first argument \emph{after} |\@empty| is used to start the
%       delimiter. It's provided by the delimiter class.
% \item The second argument \emph{after} |\@empty| is used to end the
%       delimiter. It's also provided by the delimiter class. We didn't
%       need it in the example, see the explanation below.
% \item The third argument \emph{after} |\@empty| is
%       \marg{style}\meta{start tokens}.
%       This with a preceding |\def\lst@currstyle| is used as argument to
%       |\lst@EnterMode|. The delimiter class also provides it. In the
%       example we `extended' |#5| by |\lst@Lmodetrue| (line mode true).
%       The mode automatically ends at end of line, so we didn't need the
%       end-delimiter argument.
% \end{itemize}
% And now for the other arguments. In case of dynamic modes, the first argument
% is the mode number. Then follow the user supplied  delimiter(s) whose number
% must match the remaining arguments up to |\@empty|. For non-dynamic modes,
% you must either allocate a static mode yourself or use a predefined mode
% number. The delimiters then start with the first argument.
%
% Eventually let's look at the replacement text of the macro. The sequence
% |\lst@CArg #2\relax| puts two required arguments after |\lst@DefDelimB|.
% The syntax of the latter macro is
% \begin{syntax}
% \item[1.0] \cs{lst@DefDelimB}
%
%   |{|\meta{1st}\meta{2nd}\marg{rest}|}|\syntaxfill |{//{}}|\\
%   \meta{save 1st}\syntaxfill |\lst@c/0|\\
%   \marg{execute}\syntaxfill|{}|\\
%   \marg{delim~exe~modetrue}\syntaxfill|{}|\\
%   \marg{delim~exe~modefalse}\syntaxfill|{}|\\
%   \meta{start-delimiter macro}\syntaxfill|#3|\\
%   \meta{mode number}\syntaxfill |{#1}|\\
%   |{|\marg{style}\meta{start tokens}|}|\syntaxfill |{#5\lst@Lmodetrue}|
%
%       defines \meta{1st}\meta{2nd}\meta{rest} as starting-delimiter.
%       \meta{execute} is executed when the package comes to \meta{1st}.
%       \meta{delim~exe~modetrue} and \meta{delim~exe~modefalse} are
%       executed only if the whole delimiter \meta{1st}\meta{2nd}\meta{rest}
%       is found. Exactly one of them is called depending on |\lst@ifmode|.
%
%       By default the package enters the mode if the delimiter is found
%       \emph{and} |\lst@ifmode| is false. Internally we make an appropriate
%       definition of |\lst@bnext|, which can be gobbled by placing
%       |\@gobblethree| at the very end of \meta{delim exe modefalse}.
%       One can provide an own definition (and gobble the default).
%
%       \meta{save 1st} must be an undefined macro and is used internally to
%       store the previous meaning of \meta{1st}. The arguments \meta{2nd}
%       and/or \meta{rest} are empty if the delimiter has strictly less than
%       three characters. All characters of \meta{1st}\meta{2nd}\meta{rest}
%       must already be active (if not empty).
%       That's not a problem since the macro |\lst@CArgX| does this job.
%
% \item[1.0] \cs{lst@DefDelimE}
%
%   |{|\meta{1st}\meta{2nd}\marg{rest}|}|\\
%   \meta{save 1st}\\
%   \marg{execute}\\
%   \marg{delim~exe~modetrue}\\
%   \marg{delim~exe~modefalse}\\
%   \meta{end-delimiter macro}\\
%   \meta{mode number}
%
%       Ditto for ending-delimiter with slight differences:
%       \meta{delim~exe~modetrue} and \meta{delim~exe~modefalse} are executed
%       depending on whether |\lst@mode| equals \meta{mode}.
%
%       The package ends the mode if the delimiter is found and |\lst@mode|
%       equals \meta{mode}. Internally we make an appropriate definition of
%       |\lst@enext| (not |\lst@bnext|), which can be gobbled by placing
%       |\@gobblethree| at the very end of \meta{delim exe modetrue}.
%
% \item[1.0] \cs{lst@DefDelimBE}
%
%   followed by the same eight arguments as for |\lst@DefDelimB| and \ldots\\
%   \meta{end-delimiter macro}
%
%       This is a combination of |\lst@DefDelimB| and |\lst@DefDelimE| for the
%       case of starting and ending delimiter being the same.
% \end{syntax}
% We finish the first stage by examining two easy examples.
% \texttt d-type strings are defined by
% \begin{verbatim}
%    \gdef\lst@StringDM@d#1#2\@empty#3#4#5{%
%        \lst@CArg #2\relax\lst@DefDelimBE{}{}{}#3{#1}{#5}#4}\end{verbatim}
% (and an entry in the list of string types).
% Not a big deal. Ditto \texttt d-type comments:
% \begin{verbatim}
%    \gdef\lst@CommentDM@s#1#2#3\@empty#4#5#6{%
%        \lst@CArg #2\relax\lst@DefDelimB{}{}{}#4{#1}{#6}%
%        \lst@CArg #3\relax\lst@DefDelimE{}{}{}#5{#1}}\end{verbatim}
% Here we just need to use both |\lst@DefDelimB| and |\lst@DefDelimE|.
% \goodbreak
%
% So let's get to the second stage. For illustration, here's the definition of
% the \texttt{Delim} class. The respective first argument to the service macro
% makes it delete all delimiters of the class, add the delimiter, or delete
% the particular delimiter only.
% \begin{verbatim}
%    \lst@Key{delim}\relax{\lst@DelimKey\@empty{#1}}
%    \lst@Key{moredelim}\relax{\lst@DelimKey\relax{#1}}
%    \lst@Key{deletedelim}\relax{\lst@DelimKey\@nil{#1}}\end{verbatim}
% The service macro itself calls another macro with appropriate arguments.
% \begin{verbatim}
%    \gdef\lst@DelimKey#1#2{%
%        \lst@Delim{}#2\relax{Delim}\lst@delimtypes #1%
%                    {\lst@BeginDelim\lst@EndDelim}
%            i\@empty{\lst@BeginIDelim\lst@EndIDelim}}\end{verbatim}
% We have to look at those arguments. Above you can see the actual arguments
% for the \texttt{Delim} class, below are the \texttt{Comment} class ones.
% Note that the user supplied value covers the second and third line of
% arguments.
% \begin{syntax}
% \item[0.21,,changed]
%   \cs{lst@Delim}
%
%   \meta{default style macro}\syntaxfill \cs{lst@commentstyle}\\ \relax
%   [\texttt*[\texttt*]]\texttt[\meta{type}\texttt][\texttt[\meta{style}\texttt][\texttt[\meta{type option}\texttt]]]\\
%   \meta{delimiter\textup(s\textup)}\cs{relax}\syntaxfill|#2|\cs{relax}\\
%   \marg{delimiter name}\syntaxfill|{Comment}|\\
%   \meta{delimiter types macro}\syntaxfill\texttt{\cs{lst@commenttypes}}\\
%   \alternative{\cs{@empty},\cs{@nil},\cs{relax}}\syntaxfill|#1|\\
%   \marg{begin- and end-delim macro}\syntaxfill|{|\cs{lst@BeginComment}\cs{lst@EndComment}|}|\\
%   \meta{extra prefix}\syntaxfill |i|\\
%   \meta{extra conversion}\syntaxfill |\@empty|\\
%   \marg{begin- and end-delim macro}\syntaxfill|{|\cs{lst@BeginIComment}\cs{lst@EndIComment}|}|
%
%   Most arguments should be clear. We'll discuss the last four. Both
%   \marg{begin- and end-delim macro} must contain exactly two control
%   sequences, which are given to |\lst@|\meta{name}[|DM|]|@|\meta{type}
%   to begin and end a delimiter. These are the arguments |#3| and |#4| in our
%   first example of |\lst@StringDM@l|. Depending on whether the user chosen
%   type starts with \meta{extra prefix}, the first two or the last control
%   sequences are used.
%
%   By default the package takes the delimiter(s), makes the characters active,
%   and places them after |\lst@|\meta{name}[|DM|]|@|\meta{type}. If the user
%   type starts with \meta{extra prefix}, \meta{extra conversion} might change
%   the definition of |\lst@next| to choose a different conversion. The default
%   is equivalent to |\lst@XConvert| with |\lst@false|.
%
%   Note that \meta{type} never starts with \meta{extra prefix} since it is
%   discarded. The functionality must be fully implemented by choosing a
%   different \marg{begin- and end-delim macro} pair.
% \end{syntax}
% You might need to know the syntaxes of the \meta{begin- and end-delim macro}s.
% They are called as follows.
% \begin{syntax}
% \item[0.21] |\lst@Begin|\meta{whatever}
%
%   \marg{mode}
%   |{|\marg{style}\meta{start tokens}|}|
%   \meta{delimiter}|\@empty|
%
% \item[0.21] |\lst@End|\meta{whatever}
%
%   \marg{mode}
%   \meta{delimiter}|\@empty|
% \end{syntax}
% The existing macros are internally defined in terms of |\lst@DelimOpen| and
% |\lst@DelimClose|, see the implementation.
%
%
% \subsection{Getting the kernel run}
%
% If you want new pretty-printing environments, you should be happy with
% section \ref{rEnvironments}. New commands like |\lstinline| or
% |\lstinputlisting| are more difficult. Roughly speaking you must follow
% these steps.
% \begin{enumerate}
% \item Open a group to make all changes local.
% \item \meta{Do whatever you want.}
% \item Call |\lsthk@PreSet| in any case.
% \item Now you \emph{might } want to (but need not) use |\lstset| to set some
%       new values.
% \item \meta{Do whatever you want.}
% \item Execute |\lst@Init\relax| to finish initialization.
% \item \meta{Do whatever you want.}
% \item Eventually comes the source code, which is processed by the kernel.
%       You must ensure that the characters are either not already read or all
%       active. Moreover \emph{you} must install a way to detect the end of the
%       source code. If you've reached the end, you must \ldots
% \item \ldots\ call |\lst@DeInit| to shutdown the kernel safely.
% \item \meta{Do whatever you want.}
% \item Close the group from the beginning.
% \end{enumerate}
% For example, consider the |\lstinline| command in case of being not inside an
% argument. Then the steps are as follows.
% \begin{enumerate}
% \item |\leavevmode\bgroup| opens a group.
% \item |\def\lst@boxpos{b}| `baseline' aligns the listing.
% \item |\lsthk@PreSet|
% \item |\lstset{flexiblecolumns,#1}| (|#1| is the user provided
%       key=value list)
% \item |\lsthk@TextStyle| deactivates all features not safe here.
% \item |\lst@Init\relax|
% \item |\lst@Def{`#1}{\lst@DeInit\egroup}| installs the `end inline'
%       detection, where |#1| is the next character after |\lstinline|.
%       Moreover chr(13) is redefined to end the fragment in the same way but
%       also issues an error message.
% \item Now comes the source code and \ldots
% \item \ldots\ |\lst@DeInit| (from |\lst@Def| above) ends the code snippet
%       correctly.
% \item Nothing.
% \item |\egroup| (also from |\lst@Def|) closes the group.
% \end{enumerate}
% The real definition is different since we allow source code inside arguments.
% Read also section \ref{iTheInputCommand} if you really want to write
% pretty-printing commands.
%
%
% \section{Useful internal definitions}\label{dUsefulInternalDefinitions}
%
% This section requires an update.
%
%
% \subsection{General purpose macros}\label{dGeneralPurposeMacros}
%
% \begin{syntax}
% \item[0.19] |\lst@AddTo|\meta{macro}\marg{\TeX~material}
%
%       adds \meta{\TeX~material} globally to the contents of \meta{macro}.
%
% \item[0.20] |\lst@Extend|\meta{macro}\marg{\TeX~material}
%
%       calls |\lst@AddTo| after the first token of \meta{\TeX~material} is
%       |\expand|ed|after|. For example, |\lst@Extend \a \b| merges the
%       contents of the two macros and stores it globally in |\a|.
%
% \item[0.19] |\lst@lAddTo|\meta{macro}\marg{\TeX~material}
% \item[0.20] |\lst@lExtend|\meta{macro}\marg{\TeX~material}
%
%       are local versions of |\lst@AddTo| and |\lst@Extend|.
%
% \item[0.18] |\lst@DeleteKeysIn|\meta{macro}\meta{macro \textup(keys to remove\textup)}
%
%       Both macros contain a comma separated list of keys (or keywords). All
%       keys appearing in the second macro are removed (locally) from the first.
%
% \item[0.19] |\lst@ReplaceIn|\meta{macro}\meta{macro \textup(containing replacement list\textup)}
% \item[0.20] |\lst@ReplaceInArg|\meta{macro}\marg{replacement list}
%
%       The replacement list has the form $a_1b_1$\ldots$a_nb_n$, where each
%       $a_i$ and $b_i$ is a character sequence (enclosed in braces if
%       necessary) and may contain macros, but the first token of $b_i$ must
%       not be equivalent to |\@empty|. Each sequence $a_i$ inside the first
%       macro is (locally) replaced by $b_i$.
%       The suffix |Arg| refers to the \emph{braced} second argument instead of
%       a (nonbraced) macro. It's a hint that we get the `real' argument and
%       not a `pointer' to the argument.
%
% \item[0.20] |\lst@IfSubstring|\marg{character sequence}\meta{macro}\marg{then}\marg{else}
%
%       \meta{then} is executed if \meta{character sequence} is a substring of
%       the contents of \meta{macro}. Otherwise \meta{else} is called.
%
% \item[0.12] |\lst@IfOneOf|\meta{character sequence}|\relax|\meta{macro}\marg{then}\marg{else}
%
%       |\relax| terminates the first parameter here since it is faster than
%       enclosing it in braces. \meta{macro} contains a comma separated list
%       of identifiers. If the character sequence is one of these indentifiers,
%       \meta{then} is executed, and otherwise \meta{else}.
%
% \item[0.21] |\lst@Swap|\marg{tok1}\marg{tok2}
%
%       changes places of the following two tokens or arguments \emph{without}
%       inserting braces. For example, |\lst@Swap{abc}{def}| expands to
%       |defabc|.
%
% \item[0.18] |\lst@IfNextChars|\meta{macro}\marg{then}\marg{else}
% \item[0.19] |\lst@IfNextCharsArg|\marg{character sequence}\marg{then}\marg{else}
%
%       Both macros execute either \meta{then} or \meta{else} according to
%       whether the given character sequence respectively the contents of the
%       given macro is found (after the three arguments). Note an important
%       difference between these macros and \LaTeX's |\@ifnextchar|:
%       We remove the characters behind the arguments until it is possible to
%       decide which part must be executed. However, we save these characters
%       in the macro |\lst@eaten|, so they can be inserted using \meta{then} or
%       \meta{else}.
%
% \item[0.19] |\lst@IfNextCharActive|\marg{then}\marg{else}
%
%       executes \meta{then} if next character is active, and \meta{else}
%       otherwise.
%
% \item[0.20] |\lst@DefActive|\meta{macro}\marg{character sequence}
%
%       stores the character sequence in \meta{macro}, but all characters
%       become active. The string \emph{must not} contain a begin group, end
%       group or escape character (|{}\|); it may contain a left brace, right
%       brace or backslash with other meaning (= catcode). This command
%       would be quite surplus if \meta{character sequence} is not already read
%       by \TeX\ since such catcodes can be changed easily. It is explicitly
%       allowed that the charcaters have been read, e.g.~in
%       |\def\test{\lst@DefActive\temp{ABC}}|!
%
%       Note that this macro changes |\lccode|s 0--9 without restoring them.
%
% \item[0.20] |\lst@DefOther|\meta{macro}\marg{character sequence}
%
%       stores \meta{character sequence} in \meta{macro}, but all characters
%       have catcode 12. Moreover all spaces are removed and control sequences
%       are converted to their name without preceding backslash. For example,
%       |\{ Chip \}| leads to |{Chip}| where all catcodes are 12---internally
%       the primitive |\meaning| is used.
%
% \iffalse
% \item[0.19] |\lst@MakeActive|\marg{character sequence}
%
%       stores the character sequence in |\lst@arg| and has the same
%       restrictions as |\lst@DefActive|. If fact, the latter definition uses
%       this macro here.
% \fi
% \end{syntax}
%
%
% \subsection{Character tables manipulated}\label{dCharacterTablesManipulated}
%
% \begin{syntax}
% \item[0.20] |\lst@SaveDef|\marg{character code}\meta{macro}
%
%       Saves the current definition of the specified character in
%       \meta{macro}. You should always save a character definition before you
%       redefine it! And use the saved version instead of writing directly
%       |\lst@Process|\ldots---the character could already be redefined and
%       thus not equivalent to its standard definition.
%
% \item[0.20] |\lst@DefSaveDef|\marg{character code}\meta{macro}\meta{parameter text}\marg{definition}
% \item[0.20] |\lst@LetSaveDef|\marg{character code}\meta{macro}\meta{token}
%
%       combine |\lst@SaveDef| and |\lst@Def| respectively |\lst@Let|.
% \end{syntax}
% Of course I shouldn't forget to mention \emph{where} to alter the character
% table. Hook material at \hookname{SelectCharTable} makes permanent changes,
% i.e.~it effects all languages. The following two keys can be used in any
% language definition and effects the particular language only.
% \begin{syntax}
% \item[0.20] |SelectCharTable=|\meta{\TeX\ code}
% \item[0.20] |MoreSelectCharTable=|\meta{\TeX\ code}
%
%       uses \meta{\TeX\ code} (additionally) to select the character table.
%       The code is executed after the standard character table is selected,
%       but possibly before other aspects make more changes. Since previous
%       meanings are always saved and executed inside the new definition, this
%       should be harmless.
% \end{syntax}
% Here come two rather useless examples. Each point (full stop) will cause a
% message `|.|' on the terminal and in the |.log| file if language |useless| is
% active:
% \begin{verbatim}
%   \lstdefinelanguage{useless}
%       {SelectCharTable=\lst@DefSaveDef{46}% save chr(46) ...
%            \lsts@point             % ... in \lsts@point and ...
%            {\message{.}\lsts@point}% ... use new definition
%       }\end{verbatim}
% If you want to count points, you could write
% \begin{verbatim}
%   \newcount\lst@points % \global
%   \lst@AddToHook{Init}{\global\lst@points\z@}
%   \lst@AddToHook{DeInit}{\message{Number of points: \the\lst@points}}
%   \lstdefinelanguage[2]{useless}
%       {SelectCharTable=\lst@DefSaveDef{46}\lsts@point
%            {\global\advance\lst@points\@ne \lsts@point}
%       }\end{verbatim}
% |% \global| indicates that the allocated counter is used globally. We zero
% the counter at the beginning of each listing, display a message about the
% current value at the end of a listing, and each processed point advances the
% counter by one.
%
% \begin{syntax}
% \item[0.21] |\lst@CArg|\meta{active characters}|\relax|\meta{macro}
%
%       The string of active characters is split into \meta{1st}, \meta{2nd},
%       and \marg{rest}. If one doesn't exist, an empty argument is used. Then
%       \meta{macro} is called with |{|\meta{1st}\meta{2nd}\marg{rest}|}| plus
%       a yet undefined control sequence \meta{save 1st}. This macro is
%       intended to hold the current definition of \meta{1st}, so \meta{1st}
%       can be redefined without loosing information.
%
% \item[0.19] |\lst@CArgX|\meta{characters}|\relax|\meta{macro}
%
%       makes \meta{characters} active before calling |\lst@CArg|.
%
% \item[0.21] |\lst@CDef{|\meta{1st}\meta{2nd}\marg{rest}|}|\meta{save 1st}\marg{execute}\marg{pre}\marg{post}
%
%       should be used in connection with |\lst@CArg| or |\lst@CArgX|, i.e.~as
%       \meta{macro} there. \meta{1st}, \meta{2nd}, and \meta{rest} must be
%       active characters and \meta{save 1st} must be an undefined control
%       sequence.
%
%       Whenever the package reaches the character \meta{1st} (in a listing),
%       \meta{execute} is executed. If the package detects the whole string
%       \meta{1st}\meta{2nd}\meta{rest}, we additionally execute \meta{pre},
%       then the string, and finally \meta{post}.
%
% \item[0.21] |\lst@CDefX|\meta{1st}\meta{2nd}\marg{rest}\meta{save 1st}\marg{execute}\marg{pre}\marg{post}
%
%       Ditto except that we execute \meta{pre} and \meta{post} without the
%       original string if we reach \meta{1st}\meta{2nd}\meta{rest}.
%       This means that the string is replaced by \meta{pre}\meta{post} (with
%       preceding \meta{execute}).
% \end{syntax}
% As the final example, here's the definition of |\lst@DefDelimB|.
% \begin{verbatim}
%    \gdef\lst@DefDelimB#1#2#3#4#5#6#7#8{%
%        \lst@CDef{#1}#2%
%            {#3}%
%            {\let\lst@bnext\lst@CArgEmpty
%             \lst@ifmode #4\else
%                 #5%
%                 \def\lst@bnext{#6{#7}{#8}}%
%             \fi
%             \lst@bnext}%
%            \@empty}\end{verbatim}
% You got it?
%
%
% \part{Implementation}
%
%
% \CheckSum{12365}
%^^A
%^^A Don't index TeX-primitives.
%^^A
% \DoNotIndex{\advance,\afterassignment,\aftergroup,\batchmode,\begingroup}
% \DoNotIndex{\box,\catcode,\char,\chardef,\closeout,\copy,\count,\csname,\def}
% \DoNotIndex{\dimen,\discretionary,\divide,\dp,\edef,\else,\end,\endcsname}
% \DoNotIndex{\endgroup,\endinput,\endlinechar,\escapechar,\everypar}
% \DoNotIndex{\expandafter,\fi,\gdef,\global,\globaldefs,\hbadness,\hbox}
% \DoNotIndex{\hrulefill,\hss,\ht}
% \DoNotIndex{\if,\ifdim,\iffalse,\ifhmode,\ifinner,\ifnum,\ifodd,\iftrue}
% \DoNotIndex{\ifvoid,\ifx,\ignorespaces,\immediate,\input,\jobname,\kern}
% \DoNotIndex{\lccode,\leftskip,\let,\long,\lower,\lowercase,\meaning,\message}
% \DoNotIndex{\multiply,\muskip,\noexpand,\noindent,\openout,\par,\parfillskip}
% \DoNotIndex{\parshape,\parskip,\raise,\read,\relax,\rightskip,\setbox,\skip}
% \DoNotIndex{\string,\the,\toks,\uppercase,\vbox,\vcenter,\vrule,\vtop,\wd}
% \DoNotIndex{\write,\xdef}
%
%^^A
%^^A Don't index LaTeX's private definitions.
%^^A
% \DoNotIndex{\@@end,\@@par,\@M,\@arabic,\@circlefnt,\@currentlabel}
% \DoNotIndex{\@currenvir,\@depth,\@doendpe,\@dottedtocline,\@eha,\@ehc}
% \DoNotIndex{\@empty,\@firstofone,\@firstoftwo,\@float,\@for,\@getcirc}
% \DoNotIndex{\@gobble,\@gobbletwo,\@halfwidth,\@height,\@ifnextchar}
% \DoNotIndex{\@ifundefined,\@ignoretrue,\@makecaption,\@makeother,\@namedef}
% \DoNotIndex{\@ne,\@noligs,\@notprerr,\@onlypreamble,\@secondoftwo,\@spaces}
% \DoNotIndex{\@starttoc,\@totalleftmargin,\@undefined,\@whilenum}
% \DoNotIndex{\@wholewidth,\@width}
% \DoNotIndex{\c@chapter,\c@figure,\c@page,\end@float,\f@family,\filename@area}
% \DoNotIndex{\filename@base,\filename@ext,\filename@parse,\if@twoside}
% \DoNotIndex{\l@ngrel@x,\m@ne,\new@command,\nfss@catcodes,\tw@,\thr@@}
% \DoNotIndex{\z@,\zap@space}
%
%^^A
%^^A Don't index LaTeX's package definitions.
%^^A
% \DoNotIndex{\AtEndOfPackage}
% \DoNotIndex{\CurrentOption,\DeclareOption,\IfFileExists,\InputIfFileExists}
% \DoNotIndex{\MessageBreak,\NeedsTeXFormat,\PackageError,\PackageWarning}
% \DoNotIndex{\ProcessOptions,\ProvidesFile,\ProvidesPackage,\RequirePackage}
%
%^^A
%^^A Don't index LaTeX's public definitions.
%^^A
% \DoNotIndex{\abovecaptionskip,\active,\addcontentsline,\addtocounter,\begin}
% \DoNotIndex{\belowcaptionskip,\bfseries,\bgroup,\bigbreak,\chapter}
% \DoNotIndex{\contentsname,\do,\egroup,\footnotesize,\index,\itshape}
% \DoNotIndex{\linewidth,\llap,\makeatletter,\newbox,\newcommand,\newcount}
% \DoNotIndex{\newcounter,\newdimen,\newtoks,\newwrite,\nointerlineskip}
% \DoNotIndex{\normalbaselines,\normalfont,\numberline,\pretolerance,\protect}
% \DoNotIndex{\qquad,\refstepcounter,\removelastskip,\renewcommand,\rlap}
% \DoNotIndex{\small,\smallbreak,\smallskipamount,\smash,\space,\strut}
% \DoNotIndex{\strutbox,\tableofcontents,\textasciicircum,\textasciitilde}
% \DoNotIndex{\textasteriskcentered,\textbackslash,\textbar,\textbraceleft}
% \DoNotIndex{\textbraceright,\textdollar,\textendash,\textgreater,\textless}
% \DoNotIndex{\textunderscore,\textvisiblespace,\thechapter,\ttdefault}
% \DoNotIndex{\ttfamily,\typeout,\value,\vphantom}
%
%^^A
%^^A Don't index definitions from other packages.
%^^A
% \DoNotIndex{\MakePercentComment,\MakePercentIgnore}
%
%^^A
%^^A Don't index 0.19 definitions.
%^^A
% \DoNotIndex{\listlistingsname,\listoflistings,\lstbox,\lstbox@}
% \DoNotIndex{\lstlanguage@}
%
%^^A
%^^A Don't index 0.20 subdefinitions.
%^^A
% \DoNotIndex{\lst@ATH@,\lst@BOLGobble@,\lst@BOLGobble@@,\lst@CArg@,\lst@CArg@@}
% \DoNotIndex{\lst@CBC@,\lst@CBC@@,\lst@CCECUse@,\lst@CCPutMacro@,\lst@DefLang@}
% \DoNotIndex{\lst@DefLang@@,\lst@DefOther@,\lst@DeleteKeysIn@,\lst@Escape@}
% \DoNotIndex{\lstframe@,\lst@frameH@,\lst@frameL@,\lst@frameR@}
% \DoNotIndex{\lst@FillFixed@,\lst@FindAlias@,\lst@FVConvert@}
% \DoNotIndex{\lst@IfNextChars@,\lst@IfNextChars@@,\lst@InsideConvert@}
% \DoNotIndex{\lst@InstallKeywords@,\lst@Key@,\lst@KeywordTestI@}
% \DoNotIndex{\lst@MakeActive@,\lst@MakeMoreKeywords@}
% \DoNotIndex{\lst@RC@,\lst@RC@@,\lst@ReplaceIn@,\lst@ReplaceInput@}
% \DoNotIndex{\lst@ReportAllocs@,\lst@SKS@,\lst@SKS@@,\lst@UBC@}
% \DoNotIndex{\lst@WorkingTestI@,\lstalias@,\lstalias@@,\lstalso@}
% \DoNotIndex{\lstdefinestyle@,\lstenv@BOLGobble@@}
% \DoNotIndex{\lstenv@ProcessJ@,\lstinline@,\lstKV@OptArg@,\lstKV@SetIf@}
% \DoNotIndex{\lstlang@,\lstnewenvironment@,\lst@outputpos,\lstset@}
%
%
% \section{Overture}
%
% \paragraph{Registers}
% For each aspect, the required numbers of registers are listed in section
% \lstref{dPackageLoading}. Furthermore, the \packagename{keyval} package
% allocates one token register. The macros, boxes and counters
% |\@temp|\ldots|a|/|b|, the dimensions |\@tempdim|\ldots, and the macro
% |\@gtempa| are also used, see the index.
%
% \paragraph{Naming conventions}
% Let's begin with definitions for the user. All these public macros have
% lower case letters and contain |lst|. Private macros and variables use the
% following prefixes (not up-to-date?):
% \begin{itemize}
% \item |\lst@| for a general macro or variable,
% \item |\lstenv@| if it is defined for the listing environment,
% \item |\lsts@| for |s|aved character meanings,
% \item |\lsthk@|\meta{name of hook} holds hook material,
% \item |\lst|\meta{prefix}|@| for various kinds of keywords and working
%       identifiers.
% \item |\lstlang@|\meta{language}|@|\meta{dialect} contains a language and
% \item |\lststy@|\meta{the style} contains style definition,
% \item |\lstpatch@|\meta{aspect} to patch an aspect,
%
% \item |\lsta@|\meta{language}|$|\meta{dialect} contains alias,
% \item |\lsta@|\meta{language} contains alias for all dialects of a language,
% \item |\lstdd@|\meta{language} contains default dialect of a language
%       (if present).
% \end{itemize}
% To distinguish procedure-like macros from data-macros, the name of procedure
% macros use upper case letters with each beginning word, e.g.~|\lst@AddTo|.
% A macro with suffix |@| is the main working-procedure for another definition,
% for example |\lstinputlisting@| does the main work for |\lstinputlisting|.
%
% \paragraph{Preamble}
% All files generated from this \texttt{listings.dtx} will get a header.
%    \begin{macrocode}
%% Please read the software license in listings-1.3.dtx or listings-1.3.pdf.
%%
%% (w)(c) 1996--2004 Carsten Heinz and/or any other author listed
%% elsewhere in this file.
%% (c) 2006 Brooks Moses
%% (c) 2013- Jobst Hoffmann
%%
%% Send comments and ideas on the package, error reports and additional
%% programming languages to Jobst Hoffmann at <j.hoffmann@fh-aachen.de>.
%%
%    \end{macrocode}
%
% \paragraph{Identification}
% All files will have same date and version.
%    \begin{macrocode}
\def\filedate{2015/06/04}
\def\fileversion{1.6}
%    \end{macrocode}
% What we need and who we are.
%    \begin{macrocode}
%<*kernel>
\NeedsTeXFormat{LaTeX2e}
\AtEndOfPackage{\ProvidesPackage{listings}
             [\filedate\space\fileversion\space(Carsten Heinz)]}
%    \end{macrocode}
% \begin{macro}{\lst@CheckVersion}
% can be used by the various driver files to guarantee the correct version.
%    \begin{macrocode}
\def\lst@CheckVersion#1{\edef\reserved@a{#1}%
    \ifx\lst@version\reserved@a \expandafter\@gobble
                          \else \expandafter\@firstofone \fi}
%    \end{macrocode}
%    \begin{macrocode}
\let\lst@version\fileversion
%</kernel>
%    \end{macrocode}
% \end{macro}
% For example by the miscellaneous file
%    \begin{macrocode}
%<*misc>
\ProvidesFile{lstmisc.sty}
             [\filedate\space\fileversion\space(Carsten Heinz)]
\lst@CheckVersion\fileversion
    {\typeout{^^J%
     ***^^J%
     *** This file requires `listings.sty' version \fileversion.^^J%
     *** You have a serious problem, so I'm exiting ...^^J%
     ***^^J}%
     \batchmode \@@end}
%</misc>
%    \end{macrocode}
% or by the dummy patch.
%    \begin{macrocode}
%<*patch>
\ProvidesFile{lstpatch.sty}
             [\filedate\space\fileversion\space(Carsten Heinz)]
\lst@CheckVersion\lst@version{}
%</patch>
%    \end{macrocode}
%    \begin{macrocode}
%<*doc>
\ProvidesPackage{lstdoc}
             [\filedate\space\fileversion\space(Carsten Heinz)]
%</doc>
%    \end{macrocode}
%
% \paragraph{Category codes}
% We define two macros to ensure correct catcodes when we input other files of
% the \packagename{listings} package.
%
% \begin{macro}{\lst@InputCatcodes}
% |@| and |"| become letters. Tabulators and EOLs are ignored; this avoids
% unwanted spaces---in the case I've forgotten a comment character.
%    \begin{macrocode}
%<*kernel>
\def\lst@InputCatcodes{%
    \makeatletter \catcode`\"12%
    \catcode`\^^@\active
    \catcode`\^^I9%
    \catcode`\^^L9%
    \catcode`\^^M9%
    \catcode`\%14%
    \catcode`\~\active}
%    \end{macrocode}
% \end{macro}
%
% \begin{macro}{\lst@RestoreCatcodes}
% To load the kernel, we will change some catcodes and lccodes. We restore them
% at the end of package loading. \lsthelper{Dr.~Jobst~Hoffmann}{2000/11/17}
% {incompatibility with typehtml package} reported an incompatibility with the
% \packagename{typehtml} package, which is resolved by |\lccode`\/`\/| below.
%    \begin{macrocode}
\def\lst@RestoreCatcodes#1{%
    \ifx\relax#1\else
        \noexpand\catcode`\noexpand#1\the\catcode`#1\relax
        \expandafter\lst@RestoreCatcodes
    \fi}
\edef\lst@RestoreCatcodes{%
    \noexpand\lccode`\noexpand\/`\noexpand\/%
    \lst@RestoreCatcodes\"\^^I\^^M\~\^^@\relax
    \catcode12\active}
%    \end{macrocode}
% Now we are ready for
%    \begin{macrocode}
\lst@InputCatcodes
\AtEndOfPackage{\lst@RestoreCatcodes}
%</kernel>
%    \end{macrocode}
% \end{macro}
%
% \paragraph{Statistics}
% \begin{macro}{\lst@GetAllocs}
% \begin{macro}{\lst@ReportAllocs}
% are used to show the allocated registers.
%    \begin{macrocode}
%<*info>
\def\lst@GetAllocs{%
    \edef\lst@allocs{%
        0\noexpand\count\the\count10,1\noexpand\dimen\the\count11,%
        2\noexpand\skip\the\count12,3\noexpand\muskip\the\count13,%
        4\noexpand\box\the\count14,5\noexpand\toks\the\count15,%
        6\noexpand\read\the\count16,7\noexpand\write\the\count17}}
\def\lst@ReportAllocs{%
    \message{^^JAllocs:}\def\lst@temp{none}%
    \expandafter\lst@ReportAllocs@\lst@allocs,\z@\relax\z@,}
\def\lst@ReportAllocs@#1#2#3,{%
    \ifx#2\relax \message{\lst@temp^^J}\else
        \@tempcnta\count1#1\relax \advance\@tempcnta -#3\relax
        \ifnum\@tempcnta=\z@\else
            \let\lst@temp\@empty
            \message{\the\@tempcnta \string#2,}%
        \fi
        \expandafter\lst@ReportAllocs@
    \fi}
\lst@GetAllocs
%    \end{macrocode}
% \end{macro}\end{macro}
% \begingroup
%    \begin{macrocode}
%</info>
%    \end{macrocode}
% \endgroup
%
% \paragraph{Miscellaneous}
% \begin{macro}{\@lst}
% Just a definition to save memory space.
%    \begin{macrocode}
%<*kernel>
\def\@lst{lst}
%</kernel>
%    \end{macrocode}
% \end{macro}
%
%
% \section{General problems}
%
% All definitions in this section belong to the kernel.
%    \begin{macrocode}
%<*kernel>
%    \end{macrocode}
%
%
%^^A \subsection{Quick `if parameter empty'}
%^^A
%^^A There are many situations where you have to look whether a macro parameter is empty.
%^^A We have at least two possibilities to test |#1|, for example:
%^^A \begin{center}
%^^A \begin{minipage}{0.35\linewidth}
%^^A \begin{verbatim}
%^^A\def\test{#1}%
%^^A\ifx \test\empty
%^^A        % #1 is empty
%^^A\else
%^^A        % #1 is not empty
%^^A\fi\end{verbatim}
%^^A \end{minipage}
%^^A \hskip2em\vrule\hskip2em
%^^A \begin{minipage}{0.35\linewidth}
%^^A \begin{verbatim}
%^^A\ifx \empty#1\empty
%^^A        % #1 is empty
%^^A\else
%^^A        % #1 is not empty
%^^A\fi\end{verbatim}
%^^A \end{minipage}
%^^A \end{center}
%^^A where |\empty| is defined by |\def\empty{}|, of course.
%^^A The left variant should be clear and works in any case.
%^^A The right-hand side works correct if and only if the first token of |#1| is
%^^A not equivalent to |\empty|.
%^^A This granted, the |\ifx| is true if and only if |#1| is empty (since |\empty|
%^^A left from |#1| is (not) compared with |\empty| on the right).
%^^A The two |\empty|s might be replaced by any other macro, which is not
%^^A equivalent to the first token of the argument.
%^^A But the definition of that macro shouldn't be too complex since this would
%^^A slow down the |\ifx|.
%^^A The right example needs about $45\%$ of the left's time.
%^^A Note that this \TeX{}nique lost its importance from version 0.18 on and that
%^^A other packages use |!| or |\relax| instead of |\empty|, for example.
%^^A
%^^A This \TeX nique is described in ``The \TeX book'' on page 376.
%
%
% \subsection{Substring tests}\label{iSubstringTests}
%
% \lstset{language=TeX,gobble=4,xleftmargin=20pt,columns=[l]fullflexible,mathescape,keywordstyle=\ttfamily,texcsstyle=\bfseries}
% \let\texverb\lstinline
% \lstnewenvironment{texcode}[1][]{\lstset{#1}}{}
% \lstset{keywords={def,key}}
%
% It's easy to decide whether a given character sequence is a substring of
% another string. For example, for the substring \texverb|def| we could say
% \begin{texcode}
%   \def \lst@temp#1def#2\relax{%
%       \ifx \@empty#2\@empty
%               % "def" is not a substring
%       \else
%               % "def" is a substring
%       \fi}
%
%   \lst@temp $\meta{another\ string}$def\relax
% \end{texcode}
% When \TeX\ passes the arguments |#1| and |#2|, the second is empty if
% and only if \texverb|def| is not a substring. Without the additional
% \texverb|def\relax|, one would get a ``runaway argument'' error if
% \meta{another string} doesn't contain \texverb|def|.
%
% We use substring tests mainly in the special case of an identifier and a
% comma separated list of keys or keywords:
% \begin{texcode}[keywords=key]
%   \def \lst@temp#1,key,#2\relax{%
%       \ifx \@empty#2\@empty
%               % `key' is not a keyword
%       \else
%               % `key' is a keyword
%       \fi}
%
%   \lst@temp,$\meta{list\ of\ keywords}$,key,\relax
% \end{texcode}
% This works very well and is quite fast. But we can reduce run time in the
% case that \texttt{key} is a keyword. Then |#2| takes the rest of the string,
% namely all keywords after \texttt{key}.
% Since \TeX\ inserts |#2| between the \texverb|\@empty|s, it must drop all of
% |#2| except the first character---which is compared with \texverb|\@empty|.
% We can redirect this rest to a third parameter:
% \begin{texcode}[keywords=key]
%   \def \lst@temp#1,key,#2#3\relax{%
%       \ifx \@empty#2%
%               % "key" is not a keyword
%       \else
%               % "key" is a keyword
%       \fi}
%
%   \lst@temp,$\meta{list\ of\ keywords}$,key,\@empty\relax
% \end{texcode}
% That's a bit faster and an improvement for version 0.20.
%
% \begin{macro}{\lst@IfSubstring}
% The implementation should be clear from the discussion above.
%    \begin{macrocode}
\def\lst@IfSubstring#1#2{%
    \def\lst@temp##1#1##2##3\relax{%
        \ifx \@empty##2\expandafter\@secondoftwo
                 \else \expandafter\@firstoftwo \fi}%
    \expandafter\lst@temp#2#1\@empty\relax}
%    \end{macrocode}
% \end{macro}
%
% \begin{macro}{\lst@IfOneOf}
% Ditto.
%    \begin{macrocode}
\def\lst@IfOneOf#1\relax#2{%
    \def\lst@temp##1,#1,##2##3\relax{%
        \ifx \@empty##2\expandafter\@secondoftwo
                 \else \expandafter\@firstoftwo \fi}%
    \expandafter\lst@temp\expandafter,#2,#1,\@empty\relax}
%    \end{macrocode}
% \end{macro}
% \begin{REMOVED}
% One day, if there is need for a case insensitive key(word) test again, we
% can use two |\uppercase|s to normalize the first parameter:
%    \begin{verbatim}
%\def\lst@IfOneOfInsensitive#1\relax#2{%
%    \uppercase{\def\lst@temp##1,#1},##2##3\relax{%
%        \ifx \@empty##2\expandafter\@secondoftwo
%                 \else \expandafter\@firstoftwo \fi}%
%    \uppercase{%
%        \expandafter\lst@temp\expandafter,#2,#1},\@empty\relax}\end{verbatim}
% Here we assume that macro |#2| already contains capital characters only, see
% the definition of |\lst@MakeMacroUppercase| at the very end of section
% \ref{iMakingTests}. If we \emph{must not} assume that, we could simply
% insert an |\expandafter| between the second |\uppercase| and the following
% brace. But this slows down the tests!
% \end{REMOVED}
%
% \begin{macro}{\lst@DeleteKeysIn}
% The submacro does the main work; we only need to expand the second
% macro---the list of keys to remove---and append the terminator |\relax|.
%    \begin{macrocode}
\def\lst@DeleteKeysIn#1#2{%
    \expandafter\lst@DeleteKeysIn@\expandafter#1#2,\relax,}
%    \end{macrocode}
% `Replacing' the very last |\lst@DeleteKeysIn@| by |\lst@RemoveCommas|
% terminates the loop here. Note: The |\@empty| after |#2| ensures that this
% macro also works if |#2| is empty.
%    \begin{macrocode}
\def\lst@DeleteKeysIn@#1#2,{%
    \ifx\relax#2\@empty
        \expandafter\@firstoftwo\expandafter\lst@RemoveCommas
    \else
        \ifx\@empty#2\@empty\else
%    \end{macrocode}
% If we haven't reached the end of the list and if the key is not empty, we
% define a temporary macro which removes all appearances.
%    \begin{macrocode}
            \def\lst@temp##1,#2,##2{%
                ##1%
                \ifx\@empty##2\@empty\else
                    \expandafter\lst@temp\expandafter,%
                \fi ##2}%
            \edef#1{\expandafter\lst@temp\expandafter,#1,#2,\@empty}%
        \fi
    \fi
    \lst@DeleteKeysIn@#1}
%    \end{macrocode}
% \end{macro}
% \begin{OLDDEF}
% The following modification needs about $50\%$ more run time.
% It doesn't use |\edef| and thus also works with |\{| inside |#1|.
% However, we don't need that at the moment.
%    \begin{verbatim}
%            \def\lst@temp##1,#2,##2{%
%                \ifx\@empty##2%
%                    \lst@lAddTo#1{##1}%
%                \else
%                    \lst@lAddTo#1{,##1}%
%                    \expandafter\lst@temp\expandafter,%
%                \fi ##2}%
%            \let\@tempa#1\let#1\@empty
%            \expandafter\lst@temp\expandafter,\@tempa,#2,\@empty\end{verbatim}
% \end{OLDDEF}
%
% \begin{macro}{\lst@RemoveCommas}
% The macro drops commas at the beginning and assigns the new value to |#1|.
%    \begin{macrocode}
\def\lst@RemoveCommas#1{\edef#1{\expandafter\lst@RC@#1\@empty}}
\def\lst@RC@#1{\ifx,#1\expandafter\lst@RC@ \else #1\fi}
%    \end{macrocode}
% \end{macro}
% \begin{OLDDEF}
% The following version works with |\{| inside the macro |#1|.
%    \begin{verbatim}
%\def\lst@RemoveCommas#1{\expandafter\lst@RC@#1\@empty #1}
%\def\lst@RC@#1{%
%    \ifx,#1\expandafter\lst@RC@
%      \else\expandafter\lst@RC@@\expandafter#1\fi}
%\def\lst@RC@@#1\@empty#2{\def#2{#1}}\end{verbatim}
% \end{OLDDEF}
%
% \begin{macro}{\lst@ReplaceIn}
% \begin{macro}{\lst@ReplaceInArg}
% These macros are similar to |\lst@DeleteKeysIn|, except that \ldots
%    \begin{macrocode}
\def\lst@ReplaceIn#1#2{%
    \expandafter\lst@ReplaceIn@\expandafter#1#2\@empty\@empty}
\def\lst@ReplaceInArg#1#2{\lst@ReplaceIn@#1#2\@empty\@empty}
%    \end{macrocode}
% \ldots\space we replace |#2| by |#3| instead of |,#2,| by a single comma
% (which removed the key |#2| above).
%    \begin{macrocode}
\def\lst@ReplaceIn@#1#2#3{%
    \ifx\@empty#3\relax\else
        \def\lst@temp##1#2##2{%
            \ifx\@empty##2%
                \lst@lAddTo#1{##1}%
            \else
                \lst@lAddTo#1{##1#3}\expandafter\lst@temp
            \fi ##2}%
        \let\@tempa#1\let#1\@empty
        \expandafter\lst@temp\@tempa#2\@empty
        \expandafter\lst@ReplaceIn@\expandafter#1%
    \fi}
%    \end{macrocode}
% \end{macro}
% \end{macro}
%
%
% \subsection{Flow of control}
%
% \begin{macro}{\@gobblethree}
% is defined if and only if undefined.
%    \begin{macrocode}
\providecommand*\@gobblethree[3]{}
%    \end{macrocode}
% \end{macro}
%
% \begin{macro}{\lst@GobbleNil}
%    \begin{macrocode}
\def\lst@GobbleNil#1\@nil{}
%    \end{macrocode}
% \end{macro}
%
% \begin{macro}{\lst@Swap}
% is just this:
%    \begin{macrocode}
\def\lst@Swap#1#2{#2#1}
%    \end{macrocode}
% \end{macro}
%
% \begin{macro}{\lst@if}
% \begin{macro}{\lst@true}
% \begin{macro}{\lst@false}
% A general |\if| for temporary use.
%    \begin{macrocode}
\def\lst@true{\let\lst@if\iftrue}
\def\lst@false{\let\lst@if\iffalse}
\lst@false
%    \end{macrocode}
% \end{macro}
% \end{macro}
% \end{macro}
%
% \begin{macro}{\lst@IfNextCharsArg}
% is quite easy: We define a macro and call |\lst@IfNextChars|.
%    \begin{macrocode}
\def\lst@IfNextCharsArg#1{%
    \def\lst@tofind{#1}\lst@IfNextChars\lst@tofind}
%    \end{macrocode}
% \end{macro}
%
% \begin{macro}{\lst@IfNextChars}
% We save the arguments and start a loop.
%    \begin{macrocode}
\def\lst@IfNextChars#1#2#3{%
    \let\lst@tofind#1\def\@tempa{#2}\def\@tempb{#3}%
    \let\lst@eaten\@empty \lst@IfNextChars@}
%    \end{macrocode}
% Expand the characters we are looking for.
%    \begin{macrocode}
\def\lst@IfNextChars@{\expandafter\lst@IfNextChars@@\lst@tofind\relax}
%    \end{macrocode}
% Now we can refine |\lst@tofind| and append the input character |#3| to
% |\lst@eaten|.
%    \begin{macrocode}
\def\lst@IfNextChars@@#1#2\relax#3{%
    \def\lst@tofind{#2}\lst@lAddTo\lst@eaten{#3}%
    \ifx#1#3%
%    \end{macrocode}
% If characters are the same, we either call |\@tempa| or continue the test.
%    \begin{macrocode}
        \ifx\lst@tofind\@empty
            \let\lst@next\@tempa
        \else
            \let\lst@next\lst@IfNextChars@
        \fi
        \expandafter\lst@next
    \else
%    \end{macrocode}
% If the characters are different, we call |\@tempb|.
%    \begin{macrocode}
        \expandafter\@tempb
    \fi}
%    \end{macrocode}
% \end{macro}
%
% \begin{macro}{\lst@IfNextCharActive}
% We compare the character |#3| with its active version |\lowercase{~}|.
% Note that the right brace between |\ifx~| and |#3| ends the |\lowercase|.
% The |\endgroup| restores the |\lccode|.
%    \begin{macrocode}
\def\lst@IfNextCharActive#1#2#3{%
    \begingroup \lccode`\~=`#3\lowercase{\endgroup
    \ifx~}#3%
        \def\lst@next{#1}%
    \else
        \def\lst@next{#2}%
    \fi \lst@next #3}
%    \end{macrocode}
% \end{macro}
%
% \begin{macro}{\lst@for}
% A for-loop with expansion of the loop-variable.  This was improved due to
% a suggestion by \lsthelper{Hendri~Adriaens}{2006/03/31}{speedup of
% \lst@for}.
%    \begin{macrocode}
\def\lst@for#1\do#2{%
  \def\lst@forbody##1{#2}%
  \def\@tempa{#1}%
  \ifx\@tempa\@empty\else\expandafter\lst@f@r#1,\@nil,\fi
}
\def\lst@f@r#1,{%
  \def\@tempa{#1}%
  \ifx\@tempa\@nnil\else\lst@forbody{#1}\expandafter\lst@f@r\fi
}
%    \end{macrocode}
% \end{macro}
%
%
% \subsection{Catcode changes}\label{iCatcodeChanges}
%
% A character gets its catcode right after reading it and \TeX\ has no
% primitive command to change attached catcodes. However, we can replace these
% characters by characters with same ASCII codes and different catcodes.
% It's not the same but suffices since the result is the same.
% Here we treat the very special case that all characters become active.
% If we want \texverb|\lst@arg| to contain an active version of the character
% |#1|, a prototype macro could be
% \begin{texcode}
%   \def\lst@MakeActive#1{\lccode`\~=`#1\lowercase{\def\lst@arg{~}}}
% \end{texcode}
% The |\lowercase| changes the ASCII code of |~| to the one of |#1| since we
% have said that |#1| is the lower case version of |~|.
% Fortunately the |\lowercase| doesn't change the catcode, so we have an active
% version of |#1|.
% Note that |~| is usually active.
%
% \begin{macro}{\lst@MakeActive}
% We won't do this character by character.
% To increase speed we change nine characters at the same time (if nine
% characters are left).
% \begin{TODO}
% This was introduced when the delimiters were converted each listings. Now
% this conversion is done only each language selection. So we might want to
% implement a character by character conversion again to decrease the memory
% usage.
% \end{TODO}
% We get the argument, empty |\lst@arg| and begin a loop.
%    \begin{macrocode}
\def\lst@MakeActive#1{%
    \let\lst@temp\@empty \lst@MakeActive@#1%
    \relax\relax\relax\relax\relax\relax\relax\relax\relax}
%    \end{macrocode}
% There are nine |\relax|es since |\lst@MakeActive@| has nine parameters and we
% don't want any problems in the case that |#1| is empty.
% We need nine active characters now instead of a single |~|.
% We make these catcode changes local and define the coming macro |\global|.
%    \begin{macrocode}
\begingroup
\catcode`\^^@=\active \catcode`\^^A=\active \catcode`\^^B=\active
\catcode`\^^C=\active \catcode`\^^D=\active \catcode`\^^E=\active
\catcode`\^^F=\active \catcode`\^^G=\active \catcode`\^^H=\active
%    \end{macrocode}
% First we |\let| the next operation be |\relax|.
% This aborts our loop for processing all characters (default and possibly
% changed later).
% Then we look if we have at least one character.
% If this is not the case, the loop terminates and all is done.
%    \begin{macrocode}
\gdef\lst@MakeActive@#1#2#3#4#5#6#7#8#9{\let\lst@next\relax
    \ifx#1\relax
    \else \lccode`\^^@=`#1%
%    \end{macrocode}
% Otherwise we say that |^^@|=chr(0) is the lower case version of the first
% character.
% Then we test the second character.
% If there is none, we append the lower case |^^@| to |\lst@temp|.
% Otherwise we say that |^^A|=chr(1) is the lower case version of the second
% character and we test the next argument, and so on.
%    \begin{macrocode}
    \ifx#2\relax
        \lowercase{\lst@lAddTo\lst@temp{^^@}}%
    \else \lccode`\^^A=`#2%
    \ifx#3\relax
        \lowercase{\lst@lAddTo\lst@temp{^^@^^A}}%
    \else \lccode`\^^B=`#3%
    \ifx#4\relax
        \lowercase{\lst@lAddTo\lst@temp{^^@^^A^^B}}%
    \else \lccode`\^^C=`#4%
    \ifx#5\relax
        \lowercase{\lst@lAddTo\lst@temp{^^@^^A^^B^^C}}%
    \else \lccode`\^^D=`#5%
    \ifx#6\relax
        \lowercase{\lst@lAddTo\lst@temp{^^@^^A^^B^^C^^D}}%
    \else \lccode`\^^E=`#6%
    \ifx#7\relax
        \lowercase{\lst@lAddTo\lst@temp{^^@^^A^^B^^C^^D^^E}}%
    \else \lccode`\^^F=`#7%
    \ifx#8\relax
        \lowercase{\lst@lAddTo\lst@temp{^^@^^A^^B^^C^^D^^E^^F}}%
    \else \lccode`\^^G=`#8%
    \ifx#9\relax
        \lowercase{\lst@lAddTo\lst@temp{^^@^^A^^B^^C^^D^^E^^F^^G}}%
%    \end{macrocode}
% If nine characters are present, we append (lower case versions of) nine
% active characters and call this macro again via redefining |\lst@next|.
%    \begin{macrocode}
    \else \lccode`\^^H=`#9%
        \lowercase{\lst@lAddTo\lst@temp{^^@^^A^^B^^C^^D^^E^^F^^G^^H}}%
        \let\lst@next\lst@MakeActive@
    \fi \fi \fi \fi \fi \fi \fi \fi \fi
    \lst@next}
\endgroup
%    \end{macrocode}
% This |\endgroup| restores the catcodes of chr(0)--chr(8), but not the
% catcodes of the characters inside |\lst@MakeActive@| since they are already
% read.
%
% Note: A conversion from an arbitrary `catcode--character code' table back to
% \TeX's catcodes is possible if we test against the character codes (either
% via |\ifnum| or |\ifcase|).
% But control sequences and begin and end group characters definitely need
% some special treatment.
% However I haven't checked the details.
% So just ignore this and don't bother me for this note. :\,--\,)
% \end{macro}
%
% \begin{macro}{\lst@DefActive}
% An easy application of |\lst@MakeActive|.
%    \begin{macrocode}
\def\lst@DefActive#1#2{\lst@MakeActive{#2}\let#1\lst@temp}
%    \end{macrocode}
% \end{macro}
%
% \begin{macro}{\lst@DefOther}
% We use the fact that |\meaning| produces catcode 12 characters except spaces
% stay spaces. |\escapechar| is modified locally to suppress the output of an
% escape character. Finally we remove spaces via \LaTeX's |\zap@space|, which
% was proposed by \lsthelper{Rolf~Niepraschk}{1997/04/24}{use \zap@space}---not
% in this context, but that doesn't matter.
%    \begin{macrocode}
\def\lst@DefOther#1#2{%
    \begingroup \def#1{#2}\escapechar\m@ne \expandafter\endgroup
    \expandafter\lst@DefOther@\meaning#1\relax#1}
\def\lst@DefOther@#1>#2\relax#3{\edef#3{\zap@space#2 \@empty}}
%    \end{macrocode}
% \end{macro}
%
%
%\ifhyper
% \subsection{Applications to \ref*{iCatcodeChanges}}\label{iApplicationsTo}
%\else
% \subsection{Applications to \ref{iCatcodeChanges}}\label{iApplicationsTo}
%\fi
%
% If an environment is used inside an argument, the listing is already read and
% we can do nothing to preserve the catcodes.
% However, under certain circumstances the environment can be used inside an
% argument---that's at least what I've said in the User's guide.
% And now I have to work for it coming true.
% Moreover we define an analogous conversion macro for the
% \packagename{fancyvrb} mode.
% \begin{syntax}
% \item[0.19] |\lst@InsideConvert{|\meta{\TeX\ material \textup(already read\textup)}|}|
%
%       \emph{appends} a verbatim version of the argument to |\lst@arg|, but all
%       appended characters are active. Since it's not a character to character
%       conversion, `verbatim' needs to be explained. All characters can be
%       typed in as they are except |\|, |{|, |}| and |%|. If you want one of
%       these, you must write |\\|, |\{|, |\}| and |\%| instead.
%       If two spaces should follow each other, the second (third, fourth,
%       \ldots) space must be entered with a preceding backslash.
%
% \item[0.19] |\lst@XConvert{|\meta{\TeX\ material \textup(already read\textup)}|}|
%
%       \emph{appends} a `verbatim' version of the argument to |\lst@arg|.
%       Here \TeX\ material is allowed to be put inside argument braces like
%       |{(*}{*)}|. The contents of these arguments are converted, the braces
%       stay as curly braces.
%
%       If |\lst@if| is true, each second argument is treated differently.
%       Only the first character (of the delimiter) becomes active.
% \end{syntax}
%
% \begin{macro}{\lst@InsideConvert}
% If \texttt{mathescape} is not on, we call (near the end of this definition) a
% submacro similar to |\zap@space| to replaced single spaces by active spaces.
% Otherwise we check whether the code contains a pair |$...$| and call the
% appropriate macro.
%    \begin{macrocode}
\def\lst@InsideConvert#1{%
   \lst@ifmathescape
      \lst@InsideConvert@e#1$\@nil
      \lst@if
         \lst@InsideConvert@ey#1\@nil
      \else
         \lst@InsideConvert@#1 \@empty
         \expandafter\@gobbletwo
      \fi
      \expandafter\lst@next
   \else
      \lst@InsideConvert@#1 \@empty
   \fi}
\begingroup \lccode`\~=`\ \relax \lowercase{%
%    \end{macrocode}
% We make |#1| active and append these characters (plus an active space) to
% |\lst@arg|.
% If we haven't found the end |\@empty| of the input, we continue the process.
%    \begin{macrocode}
\gdef\lst@InsideConvert@#1 #2{%
    \lst@MakeActive{#1}%
    \ifx\@empty#2%
        \lst@lExtend\lst@arg{\lst@temp}%
    \else
        \lst@lExtend\lst@arg{\lst@temp~}%
        \expandafter\lst@InsideConvert@
    \fi #2}
%    \end{macrocode}
% Finally we end the |\lowercase| and close a group.
%    \begin{macrocode}
}\endgroup
%    \end{macrocode}
% The next definition has been used above to check for |$...$| and the following
% one keeps the math contents from being converted. This feature was requested by
% \lsthelper{Dr.~Jobst~Hoffmann}{}{}.
%    \begin{macrocode}
\def\lst@InsideConvert@e#1$#2\@nil{%
   \ifx\@empty#2\@empty \lst@false \else \lst@true \fi}
\def\lst@InsideConvert@ey#1$#2$#3\@nil{%
   \lst@InsideConvert@#1 \@empty
   \lst@lAddTo\lst@arg{%
      \lst@ifdropinput\else
         \lst@TrackNewLines\lst@OutputLostSpace \lst@XPrintToken
         \setbox\@tempboxa=\hbox\bgroup$\lst@escapebegin
         #2%
         \lst@escapeend$\egroup \lst@CalcLostSpaceAndOutput
         \lst@whitespacefalse
      \fi}%
   \def\lst@next{\lst@InsideConvert{#3}}%
}
%    \end{macrocode}
% \end{macro}
%
% \begin{macro}{\lst@XConvert}
% Check for an argument \ldots
%    \begin{macrocode}
\def\lst@XConvert{\@ifnextchar\bgroup \lst@XConvertArg\lst@XConvert@}
%    \end{macrocode}
% \ldots, convert the argument, add it together with group delimiters to
% |\lst@arg|, and we continue the conversion.
%    \begin{macrocode}
\def\lst@XConvertArg#1{%
    {\lst@false \let\lst@arg\@empty
     \lst@XConvert#1\@nil
     \global\let\@gtempa\lst@arg}%
    \lst@lExtend\lst@arg{\expandafter{\@gtempa}}%
    \lst@XConvertNext}
%    \end{macrocode}
% Having no |\bgroup|, we look whether we've found the end of the input, and
% convert one token ((non)active character or control sequence) and continue.
%    \begin{macrocode}
\def\lst@XConvert@#1{%
    \ifx\@nil#1\else
        \begingroup\lccode`\~=`#1\lowercase{\endgroup
        \lst@lAddTo\lst@arg~}%
        \expandafter\lst@XConvertNext
    \fi}
\def\lst@XConvertNext{%
    \lst@if \expandafter\lst@XConvertX
      \else \expandafter\lst@XConvert \fi}
%    \end{macrocode}
% Now we make only the first character active.
%    \begin{macrocode}
\def\lst@XConvertX#1{%
    \ifx\@nil#1\else
        \lst@XConvertX@#1\relax
        \expandafter\lst@XConvert
    \fi}
\def\lst@XConvertX@#1#2\relax{%
    \begingroup\lccode`\~=`#1\lowercase{\endgroup
    \lst@XCConvertX@@~}{#2}}
\def\lst@XCConvertX@@#1#2{\lst@lAddTo\lst@arg{{#1#2}}}
%    \end{macrocode}
% \end{macro}
%
%
% \subsection{Driver file handling*}
%
% The \packagename{listings} package is split into several driver files,
% miscellaneous (= aspect) files, and one kernel file.
% All these files can be loaded partially and on demand---except the kernel
% which provides this functionality.
% \begin{syntax}
% \item[0.21] |\lst@Require|\marg{name}\marg{prefix}\marg{feature list}\meta{alias}\meta{file list macro}
%
%       tries to load all items of \meta{feature list} from the files
%       listed in \meta{file list macro}.
%       Each item has the form [\oarg{sub}]\meta{feature}.
%       |\lst@if| equals |\iftrue| if and only if all items were loadable.
%
%       The macro \meta{alias} gets an item as argument and must define
%       appropriate versions of |\lst@oalias| and |\lst@malias|. In fact
%       the feature associated with these definitions is loaded. You can
%       use \meta{alias}=|\lst@NoAlias| for no substitution.
%
%       \meta{prefix} identifies the type internally and \meta{name} is used
%       for messages.
%
%       For example, |\lstloadaspects| uses the following arguments where |#1|
%       is the list of aspects: |{aspects}|\allowbreak|a|\allowbreak
%       |{#1}|\allowbreak|\lst@NoAlias|\allowbreak|\lstaspectfiles|.
%
% \item[0.20] |\lst@DefDriver|\marg{name}\marg{prefix}\meta{interface macro}|\if|\alternative{true,false}
%
%
%
% \item[0.21] |\lst@IfRequired|\oarg{sub}\marg{feature}\marg{then}\marg{else}
%
%       is used inside a driver file by the aspect, language, or whatever
%       else defining commands. \meta{then} is executed if and only if
%       \oarg{sub}\marg{feature} has been requested via |\lst@Require|.
%       Otherwise \meta{else} is executed---which is also the case for
%       subsequent calls with the same \oarg{sub}\marg{feature}.
%
%       \meta{then} and \meta{else} may use |\lst@prefix| (read access only).
%
%       |\lst@BeginAspect| in section \ref{iAspectCommands} and |\lst@DefDriver|
%       serve as examples.
% \end{syntax}
%
% \begin{macro}{\lst@Require}
% Initialize variables (if required items aren't empty), \ldots
%    \begin{macrocode}
\def\lst@Require#1#2#3#4#5{%
    \begingroup
    \aftergroup\lst@true
    \ifx\@empty#3\@empty\else
        \def\lst@prefix{#2}\let\lst@require\@empty
%    \end{macrocode}
% \ldots\space and for each nonempty item: determine alias and add it to
% |\lst@require| if it isn't loaded.
%    \begin{macrocode}
        \edef\lst@temp{\expandafter\zap@space#3 \@empty}%
        \lst@for\lst@temp\do{%
          \ifx\@empty##1\@empty\else \lstKV@OptArg[]{##1}{%
            #4[####1]{####2}%
            \@ifundefined{\@lst\lst@prefix @\lst@malias $\lst@oalias}%
            {\edef\lst@require{\lst@require,\lst@malias $\lst@oalias}}%
            {}}%
          \fi}%
%    \end{macrocode}
% Init things and input files if and as long as it is necessary.
%    \begin{macrocode}
        \global\let\lst@loadaspects\@empty
        \lst@InputCatcodes
        \ifx\lst@require\@empty\else
            \lst@for{#5}\do{%
                \ifx\lst@require\@empty\else
                    \InputIfFileExists{##1}{}{}%
                \fi}%
        \fi
%    \end{macrocode}
% Issue error and call |\lst@false| (after closing the local group) if some
% items weren't loadable.
%    \begin{macrocode}
        \ifx\lst@require\@empty\else
            \PackageError{Listings}{Couldn't load requested #1}%
            {The following #1s weren't loadable:^^J\@spaces
             \lst@require^^JThis may cause errors in the sequel.}%
            \aftergroup\lst@false
        \fi
%    \end{macrocode}
% Request aspects.
%    \begin{macrocode}
        \ifx\lst@loadaspects\@empty\else
            \lst@RequireAspects\lst@loadaspects
        \fi
    \fi
    \endgroup}
%    \end{macrocode}
% \end{macro}
%
% \begin{macro}{\lst@IfRequired}
% uses |\lst@IfOneOf| and adds some code to \meta{then} part:
% delete the now loaded item from the list and define
% |\lst|\meta{prefix}|@|\meta{feature}|$|\meta{sub}.
%    \begin{macrocode}
\def\lst@IfRequired[#1]#2{%
    \lst@NormedDef\lst@temp{[#1]#2}%
    \expandafter\lst@IfRequired@\lst@temp\relax}
\def\lst@IfRequired@[#1]#2\relax#3{%
    \lst@IfOneOf #2$#1\relax\lst@require
        {\lst@DeleteKeysIn@\lst@require#2$#1,\relax,%
         \global\expandafter\let
             \csname\@lst\lst@prefix @#2$#1\endcsname\@empty
         #3}}
%    \end{macrocode}
% \end{macro}
%
% \begin{macro}{\lst@require}
%    \begin{macrocode}
\let\lst@require\@empty
%    \end{macrocode}
% \end{macro}
%
% \begin{macro}{\lst@NoAlias}
% just defines |\lst@oalias| and |\lst@malias|.
%    \begin{macrocode}
\def\lst@NoAlias[#1]#2{%
    \lst@NormedDef\lst@oalias{#1}\lst@NormedDef\lst@malias{#2}}
%    \end{macrocode}
% \end{macro}
%
% \begin{macro}{\lst@LAS}
%    \begin{macrocode}
\gdef\lst@LAS#1#2#3#4#5#6#7{%
    \lst@Require{#1}{#2}{#3}#4#5%
    #4#3%
    \@ifundefined{lst#2@\lst@malias$\lst@oalias}%
        {\PackageError{Listings}%
         {#1 \ifx\@empty\lst@oalias\else \lst@oalias\space of \fi
          \lst@malias\space undefined}%
         {The #1 is not loadable. \@ehc}}%
        {#6\csname\@lst#2@\lst@malias $\lst@oalias\endcsname #7}}
%    \end{macrocode}
% \end{macro}
%
% \begin{macro}{\lst@RequireAspects}
% \begin{macro}{\lstloadaspects}
% make use of the just developped definitions.
%    \begin{macrocode}
\def\lst@RequireAspects#1{%
    \lst@Require{aspect}{asp}{#1}\lst@NoAlias\lstaspectfiles}
\let\lstloadaspects\lst@RequireAspects
%    \end{macrocode}
% \end{macro}
% \end{macro}
%
% \begin{macro}{\lstaspectfiles}
% This macro is defined if and only if it's undefined yet.
%    \begin{macrocode}
\@ifundefined{lstaspectfiles}
    {\newcommand\lstaspectfiles{lstmisc0.sty,lstmisc.sty}}{}
%    \end{macrocode}
% \end{macro}
%
% \begin{macro}{\lst@DefDriver}
% Test the next character and reinsert the arguments.
%    \begin{macrocode}
\gdef\lst@DefDriver#1#2#3#4{%
    \@ifnextchar[{\lst@DefDriver@{#1}{#2}#3#4}%
                 {\lst@DefDriver@{#1}{#2}#3#4[]}}
%    \end{macrocode}
% We set |\lst@if| locally true if the item has been requested.
%    \begin{macrocode}
\gdef\lst@DefDriver@#1#2#3#4[#5]#6{%
    \def\lst@name{#1}\let\lst@if#4%
    \lst@NormedDef\lst@driver{\@lst#2@#6$#5}%
    \lst@IfRequired[#5]{#6}{\begingroup \lst@true}%
                           {\begingroup}%
    \lst@setcatcodes
    \@ifnextchar[{\lst@XDefDriver{#1}#3}{\lst@DefDriver@@#3}}
%    \end{macrocode}
% Note that |\lst@XDefDriver| takes optional `base' arguments, but eventually
% calls |\lst@DefDriver@@|. We define the item (in case of need), and
% |\endgroup| resets some catcodes and |\lst@if|, i.e.~|\lst@XXDefDriver| knows
% whether called by a public or internal command.
%    \begin{macrocode}
\gdef\lst@DefDriver@@#1#2{%
    \lst@if
        \global\@namedef{\lst@driver}{#1{#2}}%
    \fi
    \endgroup
    \@ifnextchar[\lst@XXDefDriver\@empty}
%    \end{macrocode}
% We get the aspect argument, and (if not empty) load the aspects immediately
% if called by a public command or extend the list of required aspects or
% simply ignore the argument if the item leaves undefined.
%    \begin{macrocode}
\gdef\lst@XXDefDriver[#1]{%
    \ifx\@empty#1\@empty\else
        \lst@if
            \lstloadaspects{#1}%
        \else
            \@ifundefined{\lst@driver}{}%
            {\xdef\lst@loadaspects{\lst@loadaspects,#1}}%
        \fi
    \fi}
%    \end{macrocode}
% We insert an additional `also'key=value pair.
%    \begin{macrocode}
\gdef\lst@XDefDriver#1#2[#3]#4#5{\lst@DefDriver@@#2{also#1=[#3]#4,#5}}
%    \end{macrocode}
% \end{macro}
%
%
% \subsection{Aspect commands}\label{iAspectCommands}
%
% This section contains commands used in defining `\lst-aspects'.
% \begin{macro}{\lst@UserCommand}
% is mainly equivalent to |\gdef|.
%    \begin{macrocode}
%<!info>\let\lst@UserCommand\gdef
%<info>\def\lst@UserCommand#1{\message{\string#1,}\gdef#1}
%    \end{macrocode}
% \end{macro}
%
% \begin{macro}{\lst@BeginAspect}
% A straight-forward implementation:
%    \begin{macrocode}
\newcommand*\lst@BeginAspect[2][]{%
    \def\lst@curraspect{#2}%
    \ifx \lst@curraspect\@empty
        \expandafter\lst@GobbleAspect
    \else
%    \end{macrocode}
% If \meta{aspect name} is not empty, there are certain other conditions not to
% define the aspect (as described in section \ref{dHowToDefineLstAspects}).
%    \begin{macrocode}
%<!info>        \let\lst@next\@empty
%<info>        \def\lst@next{%
%<info>            \message{^^JDefine lst-aspect `#2':}\lst@GetAllocs}%
        \lst@IfRequired[]{#2}%
            {\lst@RequireAspects{#1}%
             \lst@if\else \let\lst@next\lst@GobbleAspect \fi}%
            {\let\lst@next\lst@GobbleAspect}%
        \expandafter\lst@next
    \fi}
%    \end{macrocode}
% \end{macro}
%
% \begin{macro}{\lst@EndAspect}
% finishes an aspect definition.
%    \begin{macrocode}
\def\lst@EndAspect{%
    \csname\@lst patch@\lst@curraspect\endcsname
%<info>    \lst@ReportAllocs
    \let\lst@curraspect\@empty}
%    \end{macrocode}
% \end{macro}
%
% \begin{macro}{\lst@GobbleAspect}
% drops all code up to the next |\lst@EndAspect|.
%    \begin{macrocode}
\long\def\lst@GobbleAspect#1\lst@EndAspect{\let\lst@curraspect\@empty}
%    \end{macrocode}
% \end{macro}
%
% \begin{macro}{\lst@Key}
% The command simply defines the key. But we must take care of an optional
% parameter and the initialization argument |#2|.
%    \begin{macrocode}
\def\lst@Key#1#2{%
%<info>    \message{#1,}%
    \@ifnextchar[{\lstKV@def{#1}{#2}}%
                 {\def\lst@temp{\lst@Key@{#1}{#2}}
                  \afterassignment\lst@temp
                  \global\@namedef{KV@\@lst @#1}####1}}
%    \end{macrocode}
% Now comes a renamed and modified copy from a \packagename{keyval} macro:
% We need global key definitions.
%    \begin{macrocode}
\def\lstKV@def#1#2[#3]{%
    \global\@namedef{KV@\@lst @#1@default\expandafter}\expandafter
        {\csname KV@\@lst @#1\endcsname{#3}}%
    \def\lst@temp{\lst@Key@{#1}{#2}}\afterassignment\lst@temp
    \global\@namedef{KV@\@lst @#1}##1}
%    \end{macrocode}
% We initialize the key if the first token of |#2| is not |\relax|.
%    \begin{macrocode}
\def\lst@Key@#1#2{%
    \ifx\relax#2\@empty\else
        \begingroup \globaldefs\@ne
        \csname KV@\@lst @#1\endcsname{#2}%
        \endgroup
    \fi}
%    \end{macrocode}
% \end{macro}
%
% \begin{macro}{\lst@UseHook}
% is very, very, \ldots, very (hundreds of times) easy.
%    \begin{macrocode}
\def\lst@UseHook#1{\csname\@lst hk@#1\endcsname}
%    \end{macrocode}
% \end{macro}
%
% \begin{macro}{\lst@AddToHook}
% \begin{macro}{\lst@AddToHookExe}
% \begin{macro}{\lst@AddToHookAtTop}
% All use the same submacro.
%    \begin{macrocode}
\def\lst@AddToHook{\lst@ATH@\iffalse\lst@AddTo}
\def\lst@AddToHookExe{\lst@ATH@\iftrue\lst@AddTo}
\def\lst@AddToHookAtTop{\lst@ATH@\iffalse\lst@AddToAtTop}
%    \end{macrocode}
% If and only if the boolean value is true, the hook material is executed
% globally.
%    \begin{macrocode}
\long\def\lst@ATH@#1#2#3#4{%
    \@ifundefined{\@lst hk@#3}{%
%<info>        \message{^^Jnew hook `#3',^^J}%
        \expandafter\gdef\csname\@lst hk@#3\endcsname{}}{}%
    \expandafter#2\csname\@lst hk@#3\endcsname{#4}%
    \def\lst@temp{#4}%
    #1% \iftrue|false
        \begingroup \globaldefs\@ne \lst@temp \endgroup
    \fi}
%    \end{macrocode}
% \end{macro}
% \end{macro}
% \end{macro}
%
% \begin{macro}{\lst@AddTo}
% Note that the definition is global!
%    \begin{macrocode}
\long\def\lst@AddTo#1#2{%
    \expandafter\gdef\expandafter#1\expandafter{#1#2}}
%    \end{macrocode}
% \end{macro}
%
% \begin{macro}{\lst@AddToAtTop}
% We need a couple of |\expandafter|s now. Simply note that we have\\
%   {\small\hspace*{2em}|\expandafter\gdef\expandafter#1\expandafter{\lst@temp|
%    $\langle$\textit{contents of }|#1|$\rangle$|}|}\\
% after the `first phase' of expansion.
%    \begin{macrocode}
\def\lst@AddToAtTop#1#2{\def\lst@temp{#2}%
    \expandafter\expandafter\expandafter\gdef
    \expandafter\expandafter\expandafter#1%
    \expandafter\expandafter\expandafter{\expandafter\lst@temp#1}}
%    \end{macrocode}
% \end{macro}
%
% \begin{macro}{\lst@lAddTo}
% A local version of |\lst@AddTo| \ldots
%    \begin{macrocode}
\def\lst@lAddTo#1#2{\expandafter\def\expandafter#1\expandafter{#1#2}}
%    \end{macrocode}
% \end{macro}
%
% \begin{macro}{\lst@Extend}
% \begin{macro}{\lst@lExtend}
% \ldots\space and here we expand the first token of the second argument first.
%    \begin{macrocode}
\def\lst@Extend#1#2{%
    \expandafter\lst@AddTo\expandafter#1\expandafter{#2}}
\def\lst@lExtend#1#2{%
    \expandafter\lst@lAddTo\expandafter#1\expandafter{#2}}
%    \end{macrocode}
% \begin{TODO}
% This should never be changed to
%    \begin{verbatim}
%    \def\lst@Extend#1{%
%        \expandafter\lst@AddTo\expandafter#1\expandafter}
%    \def\lst@lExtend#1{%
%        \expandafter\lst@lAddTo\expandafter#1}\end{verbatim}
% The first is not equivalent in case that the second argument is a single
% (= non-braced) control sequence, and the second isn't in case of a braced
% second argument.
% \end{TODO}
% \end{macro}
% \end{macro}
%
%
% \subsection{Interfacing with \textsf{keyval}}
%
% The \packagename{keyval} package passes the value via the one and only
% paramater |#1| to the definition part of the key macro. The following
% commands may be used to analyse the value. Note that we need at least version
% 1.10 of the \packagename{keyval} package. Note also that the package removes
% a naming conflict with AMS classes---reported by \lsthelper{Ralf~Quast}
% {1998/01/08}{\keywords conflicts with AMS classes}.
% \begingroup
%    \begin{macrocode}
\RequirePackage{keyval}[1997/11/10]
%    \end{macrocode}
% \endgroup
%
% \begin{macro}{\lstKV@TwoArg}
% \begin{macro}{\lstKV@ThreeArg}
% \begin{macro}{\lstKV@FourArg}
% Define temporary macros and call with given arguments |#1|. We add empty
% arguments for the case that the user doesn't provide enough.
%    \begin{macrocode}
\def\lstKV@TwoArg#1#2{\gdef\@gtempa##1##2{#2}\@gtempa#1{}{}}
\def\lstKV@ThreeArg#1#2{\gdef\@gtempa##1##2##3{#2}\@gtempa#1{}{}{}}
\def\lstKV@FourArg#1#2{\gdef\@gtempa##1##2##3##4{#2}\@gtempa#1{}{}{}{}}
%    \end{macrocode}
% There's one question: What are the global definitions good for? |\lst@Key|
% might set |\globaldefs| to one and possibly calls this macro. That's the
% reason why we use global definitions here and below.
% \end{macro}
% \end{macro}
% \end{macro}
%
% \begin{macro}{\lstKV@OptArg}
% We define the temporary macro |\@gtempa| and insert default argument if
% necessary.
%    \begin{macrocode}
\def\lstKV@OptArg[#1]#2#3{%
    \gdef\@gtempa[##1]##2{#3}\lstKV@OptArg@{#1}#2\@}
\def\lstKV@OptArg@#1{\@ifnextchar[\lstKV@OptArg@@{\lstKV@OptArg@@[#1]}}
\def\lstKV@OptArg@@[#1]#2\@{\@gtempa[#1]{#2}}
%    \end{macrocode}
% \end{macro}
%
% \begin{macro}{\lstKV@XOptArg}
% Here |#3| is already a definition with at least two parameters whose first
% is enclosed in brackets.
%    \begin{macrocode}
\def\lstKV@XOptArg[#1]#2#3{%
    \global\let\@gtempa#3\lstKV@OptArg@{#1}#2\@}
%    \end{macrocode}
% \end{macro}
%
% \begin{macro}{\lstKV@CSTwoArg}
% Just define temporary macro and call it.
%    \begin{macrocode}
\def\lstKV@CSTwoArg#1#2{%
    \gdef\@gtempa##1,##2,##3\relax{#2}%
    \@gtempa#1,,\relax}
%    \end{macrocode}
% \end{macro}
%
% \begin{macro}{\lstKV@SetIf}
% We simply test the lower case first character of |#1|.
%    \begin{macrocode}
\def\lstKV@SetIf#1{\lstKV@SetIf@#1\relax}
\def\lstKV@SetIf@#1#2\relax#3{\lowercase{%
    \expandafter\let\expandafter#3%
        \csname if\ifx #1t}true\else false\fi\endcsname}
%    \end{macrocode}
% \end{macro}
%
% \begin{macro}{\lstKV@SwitchCases}
% is implemented as a substring test.
%    \begin{macrocode}
\def\lstKV@SwitchCases#1#2#3{%
    \def\lst@temp##1\\#1&##2\\##3##4\@nil{%
        \ifx\@empty##3%
            #3%
        \else
            ##2%
        \fi
    }%
    \lst@temp\\#2\\#1&\\\@empty\@nil}
%    \end{macrocode}
% \end{macro}
%
% \begin{macro}{\lstset}
% Finally this main user interface macro.
% We change catcodes for reading the argument.
%    \begin{macrocode}
\lst@UserCommand\lstset{\begingroup \lst@setcatcodes \lstset@}
\def\lstset@#1{\endgroup \ifx\@empty#1\@empty\else\setkeys{lst}{#1}\fi}
%    \end{macrocode}
% \end{macro}
%
% \begin{macro}{\lst@setcatcodes}
% contains all catcode changes for |\lstset|. The equal-sign has been added
% after a bug report by \lsthelper{Bekir~Karaoglu}{2003/09/16}{keyval problems
% with [turkish]{babel}}---babel's active equal sign clashes with keyval's
% usage. |\catcode`\"=12\relax| has been removed after a bug report by
% \lsthelper{Heiko~Bauke}{2004/06/27}{listings und ngerman}\,---\,hopefully
% this introduces no other bugs.
%    \begin{macrocode}
\def\lst@setcatcodes{\makeatletter \catcode`\==12\relax}
%    \end{macrocode}
% \begin{TODO}
% Change more catcodes?
% \end{TODO}
% \end{macro}
%
%
% \subsection{Internal modes}
%
% \begin{macro}{\lst@NewMode}
% We simply use |\chardef| for a mode definition. The counter |\lst@mode|
% mainly keeps the current mode number. But it is also used to advance the
% number in the macro |\lst@newmode|---we don't waste another counter.
%    \begin{macrocode}
\def\lst@NewMode#1{%
    \ifx\@undefined#1%
        \lst@mode\lst@newmode\relax \advance\lst@mode\@ne
        \xdef\lst@newmode{\the\lst@mode}%
        \global\chardef#1=\lst@mode
        \lst@mode\lst@nomode
    \fi}
%    \end{macrocode}
% \end{macro}
%
% \begin{macro}{\lst@mode}
% \begin{macro}{\lst@nomode}
% We allocate the counter and the first mode.
%    \begin{macrocode}
\newcount\lst@mode
\def\lst@newmode{\m@ne}% init
\lst@NewMode\lst@nomode % init (of \lst@mode :-)
%    \end{macrocode}
% \end{macro}
% \end{macro}
%
% \begin{macro}{\lst@UseDynamicMode}
% For dynamic modes we must not use the counter |\lst@mode| (since possibly
% already valued). |\lst@dynamicmode| substitutes |\lst@newmode| and is a local
% definition here, \ldots
%    \begin{macrocode}
\def\lst@UseDynamicMode{%
    \@tempcnta\lst@dynamicmode\relax \advance\@tempcnta\@ne
    \edef\lst@dynamicmode{\the\@tempcnta}%
    \expandafter\lst@Swap\expandafter{\expandafter{\lst@dynamicmode}}}
%    \end{macrocode}
% \ldots\ initialized each listing with the current `value' of |\lst@newmode|.
%    \begin{macrocode}
\lst@AddToHook{InitVars}{\let\lst@dynamicmode\lst@newmode}
%    \end{macrocode}
% \end{macro}
%
% \begin{macro}{\lst@EnterMode}
% Each mode opens a group level, stores the mode number and execute mode
% specific tokens. Moreover we keep all these changes in mind (locally) and
% adjust internal variables if the user wants it.
%    \begin{macrocode}
\def\lst@EnterMode#1#2{%
    \bgroup \lst@mode=#1\relax #2%
    \lst@FontAdjust
    \lst@lAddTo\lst@entermodes{\lst@EnterMode{#1}{#2}}}
%    \end{macrocode}
%    \begin{macrocode}
\lst@AddToHook{InitVars}{\let\lst@entermodes\@empty}
\let\lst@entermodes\@empty % init
%    \end{macrocode}
% The initialization has been added after a bug report from
% \lsthelper{Herfried~Karl~Wagner}{2002/05/11}{undefined control sequence
% \lst@entermodes}.
% \end{macro}
%
% \begin{macro}{\lst@LeaveMode}
% We simply close the group and call |\lsthk@EndGroup| if and only if the
% current mode is not |\lst@nomode|.
%    \begin{macrocode}
\def\lst@LeaveMode{%
    \ifnum\lst@mode=\lst@nomode\else
        \egroup \expandafter\lsthk@EndGroup
    \fi}
%    \end{macrocode}
%    \begin{macrocode}
\lst@AddToHook{EndGroup}{}% init
%    \end{macrocode}
% \end{macro}
%
% \begin{macro}{\lst@InterruptModes}
% We put the current mode sequence on a stack and leave all modes.
%    \begin{macrocode}
\def\lst@InterruptModes{%
    \lst@Extend\lst@modestack{\expandafter{\lst@entermodes}}%
    \lst@LeaveAllModes}
%    \end{macrocode}
%    \begin{macrocode}
\lst@AddToHook{InitVars}{\global\let\lst@modestack\@empty}
%    \end{macrocode}
% \end{macro}
%
% \begin{macro}{\lst@ReenterModes}
% If the stack is not empty, we leave all modes and pop the topmost element
% (which is the last element of |\lst@modestack|).
%    \begin{macrocode}
\def\lst@ReenterModes{%
    \ifx\lst@modestack\@empty\else
        \lst@LeaveAllModes
        \global\let\@gtempa\lst@modestack
        \global\let\lst@modestack\@empty
        \expandafter\lst@ReenterModes@\@gtempa\relax
    \fi}
\def\lst@ReenterModes@#1#2{%
    \ifx\relax#2\@empty
%    \end{macrocode}
% If we've reached |\relax|, we've also found the last element: we execute |#1|
% and gobble |{#2}|=|{\relax}| after |\fi|.
%    \begin{macrocode}
        \gdef\@gtempa##1{#1}%
        \expandafter\@gtempa
    \else
%    \end{macrocode}
% Otherwise we just add the element to |\lst@modestack| and continue the loop.
%    \begin{macrocode}
        \lst@AddTo\lst@modestack{{#1}}%
        \expandafter\lst@ReenterModes@
    \fi
    {#2}}
%    \end{macrocode}
% \end{macro}
%
% \begin{macro}{\lst@LeaveAllModes}
% Leaving all modes means closing groups until the mode equals |\lst@nomode|.
%    \begin{macrocode}
\def\lst@LeaveAllModes{%
    \ifnum\lst@mode=\lst@nomode
        \expandafter\lsthk@EndGroup
    \else
        \expandafter\egroup\expandafter\lst@LeaveAllModes
    \fi}
%    \end{macrocode}
% We need that macro to end a listing correctly.
%    \begin{macrocode}
\lst@AddToHook{ExitVars}{\lst@LeaveAllModes}
%    \end{macrocode}
% \end{macro}
%
% \begin{macro}{\lst@Pmode}
% \begin{macro}{\lst@GPmode}
% The `processing' and the general purpose mode.
%    \begin{macrocode}
\lst@NewMode\lst@Pmode
\lst@NewMode\lst@GPmode
%    \end{macrocode}
% \end{macro}
% \end{macro}
%
% \begin{macro}{\lst@modetrue}
% The usual macro to value a boolean except that we also execute a hook.
%    \begin{macrocode}
\def\lst@modetrue{\let\lst@ifmode\iftrue \lsthk@ModeTrue}
\let\lst@ifmode\iffalse % init
\lst@AddToHook{ModeTrue}{}% init
%    \end{macrocode}
% \end{macro}
%
% \begin{macro}{\lst@ifLmode}
% Comment lines use a static mode. It terminates at end of line.
%    \begin{macrocode}
\def\lst@Lmodetrue{\let\lst@ifLmode\iftrue}
\let\lst@ifLmode\iffalse % init
\lst@AddToHook{EOL}{\@whilesw \lst@ifLmode\fi \lst@LeaveMode}
%    \end{macrocode}
% \end{macro}
%
%
% \subsection{Divers helpers}
%
% \begin{macro}{\lst@NormedDef}
% works like |\def| (without any parameters!) but normalizes the replacement
% text by making all characters lower case and stripping off spaces.
%    \begin{macrocode}
\def\lst@NormedDef#1#2{\lowercase{\edef#1{\zap@space#2 \@empty}}}
%    \end{macrocode}
% \end{macro}
%
% \begin{macro}{\lst@NormedNameDef}
% works like |\global\@namedef| (again without any parameters!) but normalizes
% both the macro name and the replacement text.
%    \begin{macrocode}
\def\lst@NormedNameDef#1#2{%
    \lowercase{\edef\lst@temp{\zap@space#1 \@empty}%
    \expandafter\xdef\csname\lst@temp\endcsname{\zap@space#2 \@empty}}}
%    \end{macrocode}
% \end{macro}
%
% \begin{macro}{\lst@GetFreeMacro}
% Initialize |\@tempcnta| and |\lst@freemacro|, \ldots
%    \begin{macrocode}
\def\lst@GetFreeMacro#1{%
    \@tempcnta\z@ \def\lst@freemacro{#1\the\@tempcnta}%
    \lst@GFM@}
%    \end{macrocode}
% \ldots\space and either build the control sequence or advance the counter and
% continue.
%    \begin{macrocode}
\def\lst@GFM@{%
    \expandafter\ifx \csname\lst@freemacro\endcsname \relax
        \edef\lst@freemacro{\csname\lst@freemacro\endcsname}%
    \else
        \advance\@tempcnta\@ne
        \expandafter\lst@GFM@
    \fi}
%    \end{macrocode}
% \end{macro}
%
% \begin{macro}{\lst@gtempboxa}
%    \begin{macrocode}
\newbox\lst@gtempboxa
%    \end{macrocode}
%    \begin{macrocode}
%</kernel>
%    \end{macrocode}
% \end{macro}
%
%
% \section{Doing output}
%
%
% \subsection{Basic registers and keys}
%
%    \begin{macrocode}
%<*kernel>
%    \end{macrocode}
%
% \paragraph{The current character string}
% is kept in a token register and a counter holds its length.
% Here we define the macros to put characters into the output queue.
%
% \begin{macro}{\lst@token}
% \begin{macro}{\lst@length}
% are allocated here. Quite a useful comment, isn't it?
%    \begin{macrocode}
\newtoks\lst@token \newcount\lst@length
%    \end{macrocode}
% \end{macro}
% \end{macro}
%
% \begin{macro}{\lst@ResetToken}
% \begin{macro}{\lst@lastother}
% The two registers get empty respectively zero at the beginning of each line.
% After receiving a report from \lsthelper{Claus~Atzenbeck}{1999/11/24}{HTML:
% output unit repeated after >}---I removed such a bug many times---I decided
% to reset these registers in the \hookname{EndGroup} hook, too.
%    \begin{macrocode}
\def\lst@ResetToken{\lst@token{}\lst@length\z@}
%    \end{macrocode}
%    \begin{macrocode}
\lst@AddToHook{InitVarsBOL}{\lst@ResetToken \let\lst@lastother\@empty}
\lst@AddToHook{EndGroup}{\lst@ResetToken \let\lst@lastother\@empty}
%    \end{macrocode}
% The macro |\lst@lastother| will be equivalent to the last `other' character,
% which leads us to |\lst@ifletter|.
% \end{macro}
% \end{macro}
%
% \begin{macro}{\lst@ifletter}
% indicates whether the token contains an identifier or other characters.
%    \begin{macrocode}
\def\lst@lettertrue{\let\lst@ifletter\iftrue}
\def\lst@letterfalse{\let\lst@ifletter\iffalse}
\lst@AddToHook{InitVars}{\lst@letterfalse}
%    \end{macrocode}
% \end{macro}
%
% \begin{macro}{\lst@Append}
% puts the argument into the output queue.
%    \begin{macrocode}
\def\lst@Append#1{\advance\lst@length\@ne
                  \lst@token=\expandafter{\the\lst@token#1}}
%    \end{macrocode}
% \end{macro}
%
% \begin{macro}{\lst@AppendOther}
% Depending on the current state, we first output the character string as an
% identifier. Then we save the `argument' via |\futurelet| and call the macro
% |\lst@Append| to do the rest.
%    \begin{macrocode}
\def\lst@AppendOther{%
    \lst@ifletter \lst@Output\lst@letterfalse \fi
    \futurelet\lst@lastother\lst@Append}
%    \end{macrocode}
% \end{macro}
%
% \begin{macro}{\lst@AppendLetter}
% We output a non-identifier string if necessary and call |\lst@Append|.
%    \begin{macrocode}
\def\lst@AppendLetter{%
    \lst@ifletter\else \lst@OutputOther\lst@lettertrue \fi
    \lst@Append}
%    \end{macrocode}
% \end{macro}
%
% \begin{macro}{\lst@SaveToken}
% \begin{macro}{\lst@RestoreToken}
% If a group end appears and ruins the character string, we can use these
% macros to save and restore the contents. |\lst@thestyle| is the current
% printing style and must be saved and restored, too.
%    \begin{macrocode}
\def\lst@SaveToken{%
    \global\let\lst@gthestyle\lst@thestyle
    \global\let\lst@glastother\lst@lastother
    \xdef\lst@RestoreToken{\noexpand\lst@token{\the\lst@token}%
                           \noexpand\lst@length\the\lst@length\relax
                           \noexpand\let\noexpand\lst@thestyle
                                        \noexpand\lst@gthestyle
                           \noexpand\let\noexpand\lst@lastother
                                        \noexpand\lst@glastother}}
%    \end{macrocode}
% Now -- that means after a bug report by \lsthelper{Rolf~Niepraschk}
% {2002/04/12}{\RequirePackage is missing keywordstyle when near the top of
% a page} -- |\lst@lastother| is also saved and restored.
% \end{macro}
% \end{macro}
%
% \begin{macro}{\lst@IfLastOtherOneOf}
% Finally, this obvious implementation.
%    \begin{macrocode}
\def\lst@IfLastOtherOneOf#1{\lst@IfLastOtherOneOf@ #1\relax}
\def\lst@IfLastOtherOneOf@#1{%
    \ifx #1\relax
        \expandafter\@secondoftwo
    \else
        \ifx\lst@lastother#1%
            \lst@IfLastOtherOneOf@t
        \else
            \expandafter\expandafter\expandafter\lst@IfLastOtherOneOf@
        \fi
    \fi}
\def\lst@IfLastOtherOneOf@t#1\fi\fi#2\relax{\fi\fi\@firstoftwo}
%    \end{macrocode}
% \end{macro}
%
%
% \paragraph{The current position}
% is either the dimension |\lst@currlwidth|, which is the horizontal position
% without taking the current character string into account, or it's the current
% column starting with number 0. This is |\lst@column| $-$ |\lst@pos| $+$
% |\lst@length|. Moreover we have |\lst@lostspace| which is the difference
% between the current and the desired line width. We define macros to insert
% this lost space.
%
% \begin{macro}{\lst@currlwidth}
% \begin{macro}{\lst@column}
% \begin{macro}{\lst@pos}
% the current line width and two counters.
%    \begin{macrocode}
\newdimen\lst@currlwidth % \global
\newcount\lst@column \newcount\lst@pos % \global
\lst@AddToHook{InitVarsBOL}
    {\global\lst@currlwidth\z@ \global\lst@pos\z@ \global\lst@column\z@}
%    \end{macrocode}
% \end{macro}
% \end{macro}
% \end{macro}
%
% \begin{macro}{\lst@CalcColumn}
% sets |\@tempcnta| to the current column.
% Note that |\lst@pos| will be nonpositive.
%    \begin{macrocode}
\def\lst@CalcColumn{%
            \@tempcnta\lst@column
    \advance\@tempcnta\lst@length
    \advance\@tempcnta-\lst@pos}
%    \end{macrocode}
% \end{macro}
%
% \begin{macro}{\lst@lostspace}
% Whenever this dimension is positive we can insert space. A negative `lost
% space' means that the printed line is wider than expected.
%    \begin{macrocode}
\newdimen\lst@lostspace % \global
\lst@AddToHook{InitVarsBOL}{\global\lst@lostspace\z@}
%    \end{macrocode}
% \end{macro}
%
% \begin{macro}{\lst@UseLostSpace}
% We insert space and reset it if and only if |\lst@lostspace| is positive.
%    \begin{macrocode}
\def\lst@UseLostSpace{\ifdim\lst@lostspace>\z@ \lst@InsertLostSpace \fi}
%    \end{macrocode}
% \end{macro}
%
% \begin{macro}{\lst@InsertLostSpace}
% \begin{macro}{\lst@InsertHalfLostSpace}
% Ditto, but insert even if negative. |\lst@Kern| will be defined very soon.
%    \begin{macrocode}
\def\lst@InsertLostSpace{%
    \lst@Kern\lst@lostspace \global\lst@lostspace\z@}
\def\lst@InsertHalfLostSpace{%
    \global\lst@lostspace.5\lst@lostspace \lst@Kern\lst@lostspace}
%    \end{macrocode}
% \end{macro}
% \end{macro}
%
%
% \paragraph{Column widths}
% Here we deal with the width of a single column, which equals the width of a
% single character box. Keep in mind that there are fixed and flexible column
% formats.
%
% \begin{macro}{\lst@width}
% \begin{lstkey}{basewidth}
% \keyname{basewidth} assigns the values to macros and tests whether they are
% negative.
%    \begin{macrocode}
\newdimen\lst@width
\lst@Key{basewidth}{0.6em,0.45em}{\lstKV@CSTwoArg{#1}%
    {\def\lst@widthfixed{##1}\def\lst@widthflexible{##2}%
     \ifx\lst@widthflexible\@empty
         \let\lst@widthflexible\lst@widthfixed
     \fi
     \def\lst@temp{\PackageError{Listings}%
                                {Negative value(s) treated as zero}%
                                \@ehc}%
     \let\lst@error\@empty
     \ifdim \lst@widthfixed<\z@
         \let\lst@error\lst@temp \let\lst@widthfixed\z@
     \fi
     \ifdim \lst@widthflexible<\z@
         \let\lst@error\lst@temp \let\lst@widthflexible\z@
     \fi
     \lst@error}}
%    \end{macrocode}
% We set the dimension in a special hook.
%    \begin{macrocode}
\lst@AddToHook{FontAdjust}
    {\lst@width=\lst@ifflexible\lst@widthflexible
                          \else\lst@widthfixed\fi \relax}
%    \end{macrocode}
% \end{lstkey}
% \end{macro}
%
% \begin{lstkey}{fontadjust}
% \begin{macro}{\lst@FontAdjust}
% This hook is controlled by a switch and is always executed at
% \hookname{InitVars}.
%    \begin{macrocode}
\lst@Key{fontadjust}{false}[t]{\lstKV@SetIf{#1}\lst@iffontadjust}
\def\lst@FontAdjust{\lst@iffontadjust \lsthk@FontAdjust \fi}
%    \end{macrocode}
%    \begin{macrocode}
\lst@AddToHook{InitVars}{\lsthk@FontAdjust}
%    \end{macrocode}
% \end{macro}
% \end{lstkey}
%
%
% \subsection{Low- and mid-level output}
%
% \paragraph{Doing the output}
% means putting the character string into a box register, updating all internal
% data, and eventually giving the box to \TeX.
%
% \begin{macro}{\lst@OutputBox}
% \begin{macro}{\lst@alloverstyle}
% The lowest level is the output of a box register.
% Here we use |\box#1| as argument to |\lst@alloverstyle|.
%    \begin{macrocode}
\def\lst@OutputBox#1{\lst@alloverstyle{\box#1}}
%    \end{macrocode}
% \begin{ALTERNATIVE}
% Instead of |\global\advance\lst@currlwidth| |\wd|\meta{box number} in
% both definitions |\lst@Kern| and |\lst@CalcLostSpaceAndOutput|, we could
% also advance the dimension here. But I decided not to do so since it
% simplifies possible redefinitions of |\lst@OutputBox|: we need not to care
% about |\lst@currlwidth|.
% \end{ALTERNATIVE}
%    \begin{macrocode}
\def\lst@alloverstyle#1{#1}% init
%    \end{macrocode}
% \end{macro}
% \end{macro}
%
% \begin{macro}{\lst@Kern}
% has been used to insert `lost space'.
% It must not use |\@tempboxa| since that \ldots
%    \begin{macrocode}
\def\lst@Kern#1{%
    \setbox\z@\hbox{{\lst@currstyle{\kern#1}}}%
    \global\advance\lst@currlwidth \wd\z@
    \lst@OutputBox\z@}
%    \end{macrocode}
% \end{macro}
%
% \begin{macro}{\lst@CalcLostSpaceAndOutput}
% \ldots\space is used here.
% We keep track of |\lst@lostspace|, |\lst@currlwidth| and |\lst@pos|.
%    \begin{macrocode}
\def\lst@CalcLostSpaceAndOutput{%
    \global\advance\lst@lostspace \lst@length\lst@width
    \global\advance\lst@lostspace-\wd\@tempboxa
    \global\advance\lst@currlwidth \wd\@tempboxa
    \global\advance\lst@pos -\lst@length
%    \end{macrocode}
% Before |\@tempboxa| is output, we insert space if there is enough lost space.
% This possibly invokes |\lst@Kern| via `insert half lost space', which is the
% reason for why we mustn't use |\@tempboxa| above. By redefinition we prevent
% |\lst@OutputBox| from using any special style in |\lst@Kern|.
%    \begin{macrocode}
    \setbox\@tempboxa\hbox{\let\lst@OutputBox\box
        \ifdim\lst@lostspace>\z@ \lst@leftinsert \fi
        \box\@tempboxa
        \ifdim\lst@lostspace>\z@ \lst@rightinsert \fi}%
%    \end{macrocode}
% Finally we can output the new box.
%    \begin{macrocode}
    \lst@OutputBox\@tempboxa \lsthk@PostOutput}
%    \end{macrocode}
%    \begin{macrocode}
\lst@AddToHook{PostOutput}{}% init
%    \end{macrocode}
% \end{macro}
%
% \begin{macro}{\lst@OutputToken}
% Now comes a mid-level definition.
% Here we use |\lst@token| to set |\@tempboxa| and eventually output the box.
% We take care of font adjustment and special output styles.
% Yet unknown macros are defined in the following subsections.
%    \begin{macrocode}
\def\lst@OutputToken{%
    \lst@TrackNewLines \lst@OutputLostSpace
    \lst@ifgobbledws
        \lst@gobbledwhitespacefalse
        \lst@@discretionary
    \fi
    \lst@CheckMerge
    {\lst@thestyle{\lst@FontAdjust
     \setbox\@tempboxa\lst@hbox
        {\lsthk@OutputBox
         \lst@lefthss
         \expandafter\lst@FillOutputBox\the\lst@token\@empty
         \lst@righthss}%
     \lst@CalcLostSpaceAndOutput}}%
    \lst@ResetToken}
%    \end{macrocode}
%    \begin{macrocode}
\lst@AddToHook{OutputBox}{}% init
%    \end{macrocode}
%    \begin{macrocode}
\def\lst@gobbledwhitespacetrue{\global\let\lst@ifgobbledws\iftrue}
\def\lst@gobbledwhitespacefalse{\global\let\lst@ifgobbledws\iffalse}
\lst@AddToHookExe{InitBOL}{\lst@gobbledwhitespacefalse}% init
%    \end{macrocode}
% \end{macro}
%
%
% \paragraph{Delaying the output}
% means saving the character string somewhere and pushing it back when
% neccessary. We may also attach the string to the next output box without
% affecting style detection: both will be printed in the style of the upcoming
% output. We will call this `merging'.
%
% \begin{macro}{\lst@Delay}
% \begin{macro}{\lst@Merge}
% To delay or merge |#1|, we process it as usual and simply save the state
% in macros. For delayed characters we also need the currently `active'
% output routine. Both definitions first check whether there are already
% delayed or `merged' characters.
%    \begin{macrocode}
\def\lst@Delay#1{%
    \lst@CheckDelay
    #1%
    \lst@GetOutputMacro\lst@delayedoutput
    \edef\lst@delayed{\the\lst@token}%
    \edef\lst@delayedlength{\the\lst@length}%
    \lst@ResetToken}
%    \end{macrocode}
%    \begin{macrocode}
\def\lst@Merge#1{%
    \lst@CheckMerge
    #1%
    \edef\lst@merged{\the\lst@token}%
    \edef\lst@mergedlength{\the\lst@length}%
    \lst@ResetToken}
%    \end{macrocode}
% \end{macro}
% \end{macro}
%
% \begin{macro}{\lst@MergeToken}
% Here we put the things together again.
%    \begin{macrocode}
\def\lst@MergeToken#1#2{%
    \advance\lst@length#2%
    \lst@lExtend#1{\the\lst@token}%
    \expandafter\lst@token\expandafter{#1}%
    \let#1\@empty}
%    \end{macrocode}
% \end{macro}
%
% \begin{macro}{\lst@CheckDelay}
% We need to print delayed characters. The mode depends on the current output
% macro. If it equals the saved definition, we put the delayed characters in
% front of the character string (we merge them) since there has been no
% letter-to-other or other-to-letter leap. Otherwise we locally reset the
% current character string, merge this empty string with the delayed one,
% and output it.
%    \begin{macrocode}
\def\lst@CheckDelay{%
    \ifx\lst@delayed\@empty\else
        \lst@GetOutputMacro\@gtempa
        \ifx\lst@delayedoutput\@gtempa
            \lst@MergeToken\lst@delayed\lst@delayedlength
        \else
            {\lst@ResetToken
             \lst@MergeToken\lst@delayed\lst@delayedlength
             \lst@delayedoutput}%
            \let\lst@delayed\@empty
        \fi
    \fi}
%    \end{macrocode}
% \end{macro}
%
% \begin{macro}{\lst@CheckMerge}
% All this is easier for |\lst@merged|.
%    \begin{macrocode}
\def\lst@CheckMerge{%
    \ifx\lst@merged\@empty\else
        \lst@MergeToken\lst@merged\lst@mergedlength
    \fi}
%    \end{macrocode}
%    \begin{macrocode}
\let\lst@delayed\@empty % init
\let\lst@merged\@empty % init
%    \end{macrocode}
% \end{macro}
%
%
% \subsection{Column formats}
%
% It's time to deal with fixed and flexible column modes.
% A couple of open definitions are now filled in.
%
% \begin{macro}{\lst@column@fixed}
% switches to the fixed column format. The definitions here control how the
% output of the above definitions looks like.
%    \begin{macrocode}
\def\lst@column@fixed{%
    \lst@flexiblefalse
    \lst@width\lst@widthfixed\relax
    \let\lst@OutputLostSpace\lst@UseLostSpace
    \let\lst@FillOutputBox\lst@FillFixed
    \let\lst@hss\hss
    \def\lst@hbox{\hbox to\lst@length\lst@width}}
%    \end{macrocode}
% \end{macro}
%
% \begin{macro}{\lst@FillFixed}
% Filling up a fixed mode box is easy.
%    \begin{macrocode}
\def\lst@FillFixed#1{#1\lst@FillFixed@}
%    \end{macrocode}
% While not reaching the end (|\@empty| from above), we insert dynamic space,
% output the argument and call the submacro again.
%    \begin{macrocode}
\def\lst@FillFixed@#1{%
    \ifx\@empty#1\else \lst@hss#1\expandafter\lst@FillFixed@ \fi}
%    \end{macrocode}
% \end{macro}
%
% \begin{macro}{\lst@column@flexible}
% The first flexible format.
%    \begin{macrocode}
\def\lst@column@flexible{%
    \lst@flexibletrue
    \lst@width\lst@widthflexible\relax
    \let\lst@OutputLostSpace\lst@UseLostSpace
    \let\lst@FillOutputBox\@empty
    \let\lst@hss\@empty
    \let\lst@hbox\hbox}
%    \end{macrocode}
% \end{macro}
%
% \begin{macro}{\lst@column@fullflexible}
% This column format inserts no lost space except at the beginning of a line.
%    \begin{macrocode}
\def\lst@column@fullflexible{%
    \lst@column@flexible
    \def\lst@OutputLostSpace{\lst@ifnewline \lst@UseLostSpace\fi}%
    \let\lst@leftinsert\@empty
    \let\lst@rightinsert\@empty}
%    \end{macrocode}
% \end{macro}
%
% \begin{macro}{\lst@column@spaceflexible}
% This column format only inserts lost space by stretching (invisible)
% existing spaces; it does not insert lost space between identifiers
% and other characters where the original does not have a space.  It
% was suggested by \lsthelper{Andrei~Alexandrescu}{-}{2007-02-26}.
%    \begin{macrocode}
\def\lst@column@spaceflexible{%
    \lst@column@flexible
    \def\lst@OutputLostSpace{%
      \lst@ifwhitespace
        \ifx\lst@outputspace\lst@visiblespace
        \else
          \lst@UseLostSpace
        \fi
      \else
        \lst@ifnewline \lst@UseLostSpace\fi
      \fi}%
    \let\lst@leftinsert\@empty
    \let\lst@rightinsert\@empty}
%    \end{macrocode}
% \end{macro}
%
% Thus, we have the column formats. Now we define macros to use them.
%
% \begin{macro}{\lst@outputpos}
% This macro sets the `output-box-positioning' parameter (the old key
% \keyname{outputpos}). We test for |l|, |c| and |r|.
% The fixed formats use |\lst@lefthss| and |\lst@righthss|, whereas the
% flexibles need |\lst@leftinsert| and |\lst@rightinsert|.
%    \begin{macrocode}
\def\lst@outputpos#1#2\relax{%
    \def\lst@lefthss{\lst@hss}\let\lst@righthss\lst@lefthss
    \let\lst@rightinsert\lst@InsertLostSpace
    \ifx #1c%
        \let\lst@leftinsert\lst@InsertHalfLostSpace
    \else\ifx #1r%
        \let\lst@righthss\@empty
        \let\lst@leftinsert\lst@InsertLostSpace
        \let\lst@rightinsert\@empty
    \else
        \let\lst@lefthss\@empty
        \let\lst@leftinsert\@empty
        \ifx #1l\else \PackageWarning{Listings}%
            {Unknown positioning for output boxes}%
        \fi
    \fi\fi}
%    \end{macrocode}
% \end{macro}
%
% \begin{macro}{\lst@ifflexible}
% indicates the column mode but does not distinguish between different fixed
% or flexible modes.
%    \begin{macrocode}
\def\lst@flexibletrue{\let\lst@ifflexible\iftrue}
\def\lst@flexiblefalse{\let\lst@ifflexible\iffalse}
%    \end{macrocode}
% \end{macro}
%
% \begin{lstkey}{columns}
% This is done here: check optional parameter and then build the control
% sequence of the column format.
%    \begin{macrocode}
\lst@Key{columns}{[c]fixed}{\lstKV@OptArg[]{#1}{%
    \ifx\@empty##1\@empty\else \lst@outputpos##1\relax\relax \fi
    \expandafter\let\expandafter\lst@arg
                                \csname\@lst @column@##2\endcsname
%    \end{macrocode}
% We issue a warning or save the definition for later.
%    \begin{macrocode}
    \lst@arg
    \ifx\lst@arg\relax
        \PackageWarning{Listings}{Unknown column format `##2'}%
    \else
        \lst@ifflexible
            \let\lst@columnsflexible\lst@arg
        \else
            \let\lst@columnsfixed\lst@arg
        \fi
    \fi}}
%    \end{macrocode}
%    \begin{macrocode}
\let\lst@columnsfixed\lst@column@fixed % init
\let\lst@columnsflexible\lst@column@flexible % init
%    \end{macrocode}
% \end{lstkey}
%
% \begin{lstkey}{flexiblecolumns}
% Nothing else but a key to switch between the last flexible and fixed mode.
%    \begin{macrocode}
\lst@Key{flexiblecolumns}\relax[t]{%
    \lstKV@SetIf{#1}\lst@ifflexible
    \lst@ifflexible \lst@columnsflexible
              \else \lst@columnsfixed \fi}
%    \end{macrocode}
% \end{lstkey}
%
%
% \subsection{New lines}
%
% \begin{macro}{\lst@newlines}
% This counter holds the number of `new lines' (cr+lf) we have to perform.
%    \begin{macrocode}
\newcount\lst@newlines
\lst@AddToHook{InitVars}{\global\lst@newlines\z@}
\lst@AddToHook{InitVarsBOL}{\global\advance\lst@newlines\@ne}
%    \end{macrocode}
% \end{macro}
%
% \begin{macro}{\lst@NewLine}
% This is how we start a new line: begin new paragraph and output an empty
% box. If low-level definition |\lst@OutputBox| just gobbles the box , we
% don't start a new line. This is used to drop the whole output.
%    \begin{macrocode}
\def\lst@NewLine{%
    \ifx\lst@OutputBox\@gobble\else
        \par\noindent \hbox{}%
    \fi
    \global\advance\lst@newlines\m@ne
    \lst@newlinetrue}
%    \end{macrocode}
% Define |\lst@newlinetrue| and reset if after output.
%    \begin{macrocode}
\def\lst@newlinetrue{\global\let\lst@ifnewline\iftrue}
\lst@AddToHookExe{PostOutput}{\global\let\lst@ifnewline\iffalse}% init
%    \end{macrocode}
% \end{macro}
%
% \begin{macro}{\lst@TrackNewLines}
% If |\lst@newlines| is positive, we execute the hook and insert the
% new lines.
%    \begin{macrocode}
\def\lst@TrackNewLines{%
    \ifnum\lst@newlines>\z@
        \lsthk@OnNewLine
        \lst@DoNewLines
    \fi}
\lst@AddToHook{OnNewLine}{}% init
%    \end{macrocode}
% \end{macro}
%
% \begin{lstkey}{emptylines}
% \lsthelper{Adam~Prugel-Bennett}{2001/02/19}{spacing of empty lines} asked for
% such a key---if I didn't misunderstood him. We check for the optional star
% and set |\lst@maxempty| and switch.
%    \begin{macrocode}
\lst@Key{emptylines}\maxdimen{%
    \@ifstar{\lst@true\@tempcnta\@gobble#1\relax\lst@GobbleNil}%
            {\lst@false\@tempcnta#1\relax\lst@GobbleNil}#1\@nil
    \advance\@tempcnta\@ne
    \edef\lst@maxempty{\the\@tempcnta\relax}%
    \let\lst@ifpreservenumber\lst@if}
%    \end{macrocode}
% \end{lstkey}
%
% \begin{macro}{\lst@DoNewLines}
% First we take care of |\lst@maxempty| and then of the remaining empty lines.
%    \begin{macrocode}
\def\lst@DoNewLines{
    \@whilenum\lst@newlines>\lst@maxempty \do
        {\lst@ifpreservenumber
            \lsthk@OnEmptyLine
            \global\advance\c@lstnumber\lst@advancelstnum
         \fi
         \global\advance\lst@newlines\m@ne}%
    \@whilenum \lst@newlines>\@ne \do
        {\lsthk@OnEmptyLine \lst@NewLine}%
    \ifnum\lst@newlines>\z@ \lst@NewLine \fi}
\lst@AddToHook{OnEmptyLine}{}% init
%    \end{macrocode}
% \end{macro}
%
%
% \subsection{High-level output}
%
% \begin{lstkey}{identifierstyle}
% A simple key.
%    \begin{macrocode}
\lst@Key{identifierstyle}{}{\def\lst@identifierstyle{#1}}
\lst@AddToHook{EmptyStyle}{\let\lst@identifierstyle\@empty}
%    \end{macrocode}
% \end{lstkey}
%
% \begin{macro}{\lst@GotoTabStop}
% Here we look whether the line already contains printed characters.
% If true, we output a box with the width of a blank space.
%    \begin{macrocode}
\def\lst@GotoTabStop{%
    \ifnum\lst@newlines=\z@
        \setbox\@tempboxa\hbox{\lst@outputspace}%
        \setbox\@tempboxa\hbox to\wd\@tempboxa{{\lst@currstyle{\hss}}}%
        \lst@CalcLostSpaceAndOutput
%    \end{macrocode}
% It's probably not clear why it is sufficient to output a single space to go
% to the next tabulator stop. Just note that the space lost by this process is
% `lost space' in the sense above and therefore will be inserted before the
% next characters are output.
%    \begin{macrocode}
    \else
%    \end{macrocode}
% Otherwise (no printed characters) we only need to advance |\lst@lostspace|,
% which is inserted by |\lst@OutputToken| above, and update the column.
%    \begin{macrocode}
        \global\advance\lst@lostspace \lst@length\lst@width
        \global\advance\lst@column\lst@length \lst@length\z@
    \fi}
%    \end{macrocode}
% Note that this version works also in flexible column mode.
% In fact, it's mainly the flexible version of \packagename{listings} 0.20.
% \begin{TODO}
% Use |\lst@ifnewline| instead of |\ifnum\lst@newlines=\z@|?
% \end{TODO}
% \end{macro}
%
% \begin{macro}{\lst@OutputOther}
% becomes easy with the previous definitions.
%    \begin{macrocode}
\def\lst@OutputOther{%
    \lst@CheckDelay
    \ifnum\lst@length=\z@\else
        \let\lst@thestyle\lst@currstyle
        \lsthk@OutputOther
        \lst@OutputToken
    \fi}
%    \end{macrocode}
%    \begin{macrocode}
\lst@AddToHook{OutputOther}{}% init
\let\lst@currstyle\relax % init
%    \end{macrocode}
% \end{macro}
%
% \begin{macro}{\lst@Output}
% We might use identifier style as default.
%    \begin{macrocode}
\def\lst@Output{%
    \lst@CheckDelay
    \ifnum\lst@length=\z@\else
        \ifx\lst@currstyle\relax
            \let\lst@thestyle\lst@identifierstyle
        \else
            \let\lst@thestyle\lst@currstyle
        \fi
        \lsthk@Output
        \lst@OutputToken
    \fi
    \let\lst@lastother\relax}
%    \end{macrocode}
% Note that |\lst@lastother| becomes equivalent to |\relax| and not equivalent
% to |\@empty| as everywhere else. I don't know whether this will be important
% in the future or not.
%    \begin{macrocode}
\lst@AddToHook{Output}{}% init
%    \end{macrocode}
% \end{macro}
%
% \begin{macro}{\lst@GetOutputMacro}
% Just saves the output macro to be used.
%    \begin{macrocode}
\def\lst@GetOutputMacro#1{%
    \lst@ifletter \global\let#1\lst@Output
            \else \global\let#1\lst@OutputOther\fi}
%    \end{macrocode}
% \end{macro}
%
% \begin{macro}{\lst@PrintToken}
% outputs the current character string in letter or nonletter mode.
%    \begin{macrocode}
\def\lst@PrintToken{%
    \lst@ifletter \lst@Output \lst@letterfalse
            \else \lst@OutputOther \let\lst@lastother\@empty \fi}
%    \end{macrocode}
% \end{macro}
%
% \begin{macro}{\lst@XPrintToken}
% is a special definition to print also merged characters.
%    \begin{macrocode}
\def\lst@XPrintToken{%
    \lst@PrintToken \lst@CheckMerge
    \ifnum\lst@length=\z@\else \lst@PrintToken \fi}
%    \end{macrocode}
% \end{macro}
%
%
% \subsection{Dropping the whole output}
%
% \begin{macro}{\lst@BeginDropOutput}
% It's sometimes useful to process a part of a listing as usual, but to drop
% the output. This macro does the main work and gets one argument, namely the
% internal mode it enters. We save |\lst@newlines|, restore it |\aftergroup|
% and redefine one macro, namely |\lst@OutputBox|. After a bug report from
% \lsthelper{Gunther~Schmidl}{2002/02/27}{collapsing empty lines don't work
% with printpod=false}
%    \begin{macrocode}
\def\lst@BeginDropOutput#1{%
    \xdef\lst@BDOnewlines{\the\lst@newlines}%
    \global\let\lst@BDOifnewline\lst@ifnewline
    \lst@EnterMode{#1}%
        {\lst@modetrue
         \let\lst@OutputBox\@gobble
         \aftergroup\lst@BDORestore}}
%    \end{macrocode}
% Restoring the date is quite easy:
%    \begin{macrocode}
\def\lst@BDORestore{%
    \global\lst@newlines\lst@BDOnewlines
    \global\let\lst@ifnewline\lst@BDOifnewline}
%    \end{macrocode}
% \end{macro}
%
% \begin{macro}{\lst@EndDropOutput}
% is equivalent to |\lst@LeaveMode|.
%    \begin{macrocode}
\let\lst@EndDropOutput\lst@LeaveMode
%    \end{macrocode}
%    \begin{macrocode}
%</kernel>
%    \end{macrocode}
% \end{macro}
%
%
% \subsection{Writing to an external file}
%
% \begin{aspect}{writefile}
% Now it would be good to know something about character classes since we need
% to access the true input characters, for example a tabulator and not the
% spaces it `expands' to.
%    \begin{macrocode}
%<*misc>
\lst@BeginAspect{writefile}
%    \end{macrocode}
%
% \begin{macro}{\lst@WF}
% \begin{macro}{\lst@WFtoken}
% The contents of the token will be written to file.
%    \begin{macrocode}
\newtoks\lst@WFtoken % global
\lst@AddToHook{InitVarsBOL}{\global\lst@WFtoken{}}
%    \end{macrocode}
%    \begin{macrocode}
\newwrite\lst@WF
\global\let\lst@WFifopen\iffalse % init
%    \end{macrocode}
% \end{macro}
% \end{macro}
%
% \begin{macro}{\lst@WFWriteToFile}
% To do this, we have to expand the contents and then expand this via |\edef|.
% Empty |\lst@UM| ensures that special characters (underscore, dollar, etc.)
% are written correctly.
%    \begin{macrocode}
\gdef\lst@WFWriteToFile{%
  \begingroup
   \let\lst@UM\@empty
   \expandafter\edef\expandafter\lst@temp\expandafter{\the\lst@WFtoken}%
   \immediate\write\lst@WF{\lst@temp}%
  \endgroup
  \global\lst@WFtoken{}}
%    \end{macrocode}
% \end{macro}
%
% \begin{macro}{\lst@WFAppend}
% Similar to |\lst@Append| but uses |\lst@WFtoken|.
%    \begin{macrocode}
\gdef\lst@WFAppend#1{%
    \global\lst@WFtoken=\expandafter{\the\lst@WFtoken#1}}
%    \end{macrocode}
% \end{macro}
%
% \begin{macro}{\lst@BeginWriteFile}
% \begin{macro}{\lst@BeginAlsoWriteFile}
% use different macros for |\lst@OutputBox| (not) to drop the output.
%    \begin{macrocode}
\gdef\lst@BeginWriteFile{\lst@WFBegin\@gobble}
\gdef\lst@BeginAlsoWriteFile{\lst@WFBegin\lst@OutputBox}
%    \end{macrocode}
% \end{macro}
% \end{macro}
%
% \begin{macro}{\lst@WFBegin}
% Here \ldots
%    \begin{macrocode}
\begingroup \catcode`\^^I=11
\gdef\lst@WFBegin#1#2{%
    \begingroup
    \let\lst@OutputBox#1%
%    \end{macrocode}
% \ldots\space we have to update |\lst@WFtoken| and \ldots
%    \begin{macrocode}
    \def\lst@Append##1{%
        \advance\lst@length\@ne
        \expandafter\lst@token\expandafter{\the\lst@token##1}%
        \ifx ##1\lst@outputspace \else
            \lst@WFAppend##1%
        \fi}%
    \lst@lAddTo\lst@PreGotoTabStop{\lst@WFAppend{^^I}}%
    \lst@lAddTo\lst@ProcessSpace{\lst@WFAppend{ }}%
%    \end{macrocode}
% \ldots\space need different `EOL' and `DeInit' definitions to write the
% token register to file.
%    \begin{macrocode}
    \let\lst@DeInit\lst@WFDeInit
    \let\lst@MProcessListing\lst@WFMProcessListing
%    \end{macrocode}
% Finally we open the file if necessary.
%    \begin{macrocode}
    \lst@WFifopen\else
        \immediate\openout\lst@WF=#2\relax
        \global\let\lst@WFifopen\iftrue
        \@gobbletwo\fi\fi
    \fi}
\endgroup
%    \end{macrocode}
% \end{macro}
%
% \begin{macro}{\lst@EndWriteFile}
% closes the file and restores original definitions.
%    \begin{macrocode}
\gdef\lst@EndWriteFile{%
    \immediate\closeout\lst@WF \endgroup
    \global\let\lst@WFifopen\iffalse}
%    \end{macrocode}
% \end{macro}
%
% \begin{macro}{\lst@WFMProcessListing}
% \begin{macro}{\lst@WFDeInit}
% write additionally |\lst@WFtoken| to external file.
%    \begin{macrocode}
\global\let\lst@WFMProcessListing\lst@MProcessListing
\global\let\lst@WFDeInit\lst@DeInit
\lst@AddToAtTop\lst@WFMProcessListing{\lst@WFWriteToFile}
\lst@AddToAtTop\lst@WFDeInit{%
    \ifnum\lst@length=\z@\else \lst@WFWriteToFile \fi}
%    \end{macrocode}
% \end{macro}
% \end{macro}
%
%    \begin{macrocode}
\lst@EndAspect
%</misc>
%    \end{macrocode}
% \end{aspect}
%
%
% \section{Character classes}\label{iCharacterClasses}
%
% In this section, we define how the basic character classes do behave, before
% turning over to the selection of character tables and how to specialize
% characters.
%
%
% \subsection{Letters, digits and others}
%
%    \begin{macrocode}
%<*kernel>
%    \end{macrocode}
%
% \begin{macro}{\lst@ProcessLetter}
% We put the letter, which is not a whitespace, into the output queue.
%    \begin{macrocode}
\def\lst@ProcessLetter{\lst@whitespacefalse \lst@AppendLetter}
%    \end{macrocode}
% \end{macro}
%
% \begin{macro}{\lst@ProcessOther}
% Ditto.
%    \begin{macrocode}
\def\lst@ProcessOther{\lst@whitespacefalse \lst@AppendOther}
%    \end{macrocode}
% \end{macro}
%
% \begin{macro}{\lst@ProcessDigit}
% A digit appends the character to the current character string. But we must
% use the right macro. This allows digits to be part of an identifier or
% a numerical constant.
%    \begin{macrocode}
\def\lst@ProcessDigit{%
    \lst@whitespacefalse
    \lst@ifletter \expandafter\lst@AppendLetter
            \else \expandafter\lst@AppendOther\fi}
%    \end{macrocode}
% \end{macro}
%
% \begin{macro}{\lst@ifwhitespace}
% indicates whether the last processed character has been white space.
%    \begin{macrocode}
\def\lst@whitespacetrue{\global\let\lst@ifwhitespace\iftrue}
\def\lst@whitespacefalse{\global\let\lst@ifwhitespace\iffalse}
\lst@AddToHook{InitVarsBOL}{\lst@whitespacetrue}
%    \end{macrocode}
% \end{macro}
%
%
% \subsection{Whitespaces}
%
% Here we have to take care of two things: dropping empty lines at the end of
% a listing and the different column formats. Both use |\lst@lostspace|. Lines
% containing only tabulators and spaces should be viewed as empty. In order to
% achieve this, tabulators and spaces at the beginning of a line don't output
% any characters but advance |\lst@lostspace|. Whenever this dimension is
% positive we insert that space before the character string is output. Thus,
% if there are only tabulators and spaces, the line is `empty' since we
% haven't done any output.
%
% We have to do more for flexible columns. Whitespaces can fix the column
% alignment: if the real line is wider than expected, a tabulator is at least
% one space wide; all remaining space fixes the alignment. If there are two or
% more space characters, at least one is printed; the others fix the column
% alignment.
%
%
% \paragraph{Tabulators}
% are processed in three stages. You have already seen the last stage
% |\lst@GotoTabStop|. The other two calculate the necessary width and take care
% of visible tabulators and spaces.
%
% \begin{lstkey}{tabsize}
% We check for a legal argument before saving it. Default tabsize is 8 as
% proposed by \lsthelper{Rolf~Niepraschk}{1997/04/24}{tabsize=8}.
%    \begin{macrocode}
\lst@Key{tabsize}{8}
    {\ifnum#1>\z@ \def\lst@tabsize{#1}\else
         \PackageError{Listings}{Strict positive integer expected}%
         {You can't use `#1' as tabsize. \@ehc}%
     \fi}
%    \end{macrocode}
% \end{lstkey}
%
% \begin{lstkey}{showtabs}
% \begin{lstkey}{tab}
% Two more user keys for tab control.
%    \begin{macrocode}
\lst@Key{showtabs}f[t]{\lstKV@SetIf{#1}\lst@ifshowtabs}
\lst@Key{tab}{\kern.06em\hbox{\vrule\@height.3ex}%
              \hrulefill\hbox{\vrule\@height.3ex}}
    {\def\lst@tab{#1}}
%    \end{macrocode}
% \end{lstkey}
% \end{lstkey}
%
% \begin{macro}{\lst@ProcessTabulator}
% A tabulator outputs the preceding characters, which decrements |\lst@pos| by
% the number of printed characters.
%    \begin{macrocode}
\def\lst@ProcessTabulator{%
    \lst@XPrintToken \lst@whitespacetrue
%    \end{macrocode}
% Then we calculate how many columns we need to reach the next tabulator stop:
% we add |\lst@tabsize| until |\lst@pos| is strict positive. In other words,
% |\lst@pos| is the column modulo |tabsize| and we're looking for a positive
% representative. We assign it to |\lst@length| and reset |\lst@pos| in the
% submacro.
%    \begin{macrocode}
    \global\advance\lst@column -\lst@pos
    \@whilenum \lst@pos<\@ne \do
        {\global\advance\lst@pos\lst@tabsize}%
    \lst@length\lst@pos
    \lst@PreGotoTabStop}
%    \end{macrocode}
% \end{macro}
%
% \begin{macro}{\lst@PreGotoTabStop}
% Visible tabs print |\lst@tab|.
%    \begin{macrocode}
\def\lst@PreGotoTabStop{%
    \lst@ifshowtabs
        \lst@TrackNewLines
        \setbox\@tempboxa\hbox to\lst@length\lst@width
            {{\lst@currstyle{\hss\lst@tab}}}%
        \lst@CalcLostSpaceAndOutput
    \else
%    \end{macrocode}
% If we are advised to keep spaces, we insert the correct number of them.
%    \begin{macrocode}
        \lst@ifkeepspaces
            \@tempcnta\lst@length \lst@length\z@
            \@whilenum \@tempcnta>\z@ \do
                {\lst@AppendOther\lst@outputspace
                 \advance\@tempcnta\m@ne}%
            \lst@OutputOther
        \else
            \lst@GotoTabStop
        \fi
    \fi
    \lst@length\z@ \global\lst@pos\z@}
%    \end{macrocode}
% \end{macro}
%
%
% \paragraph{Spaces}
% are implemented as described at the beginning of this subsection. But first
% we define some user keys.
%
% \begin{macro}{\lst@outputspace}
% \begin{macro}{\lst@visiblespace}
% The first macro is a default definition, \ldots
%    \begin{macrocode}
\def\lst@outputspace{\ }
\def\lst@visiblespace{\lst@ttfamily{\char32}\textvisiblespace}
%    \end{macrocode}
% \end{macro}
% \end{macro}
%
% \begin{lstkey}{showspaces}
% \begin{lstkey}{keepspaces}
% \ldots\space which is modified on user's request.
%    \begin{macrocode}
\lst@Key{showspaces}{false}[t]{\lstKV@SetIf{#1}\lst@ifshowspaces}
\lst@Key{keepspaces}{false}[t]{\lstKV@SetIf{#1}\lst@ifkeepspaces}
\lst@AddToHook{Init}
    {\lst@ifshowspaces
         \let\lst@outputspace\lst@visiblespace
         \lst@keepspacestrue
     \fi}
\def\lst@keepspacestrue{\let\lst@ifkeepspaces\iftrue}
%    \end{macrocode}
% \end{lstkey}
% \end{lstkey}
%
% \begin{macro}{\lst@ProcessSpace}
% We look whether spaces fix the column alignment or not. In the latter case
% we append a space; otherwise \ldots
% \lsthelper{Andrei~Alexandrescu}{-}{2007/02/27} tested the |spaceflexible|
% column setting and found a bug that resulted from |\lst@PrintToken| and
% |\lst@whitespacetrue| being out of order here.
%    \begin{macrocode}
\def\lst@ProcessSpace{%
    \lst@ifkeepspaces
        \lst@PrintToken
        \lst@whitespacetrue
        \lst@AppendOther\lst@outputspace
        \lst@PrintToken
    \else \ifnum\lst@newlines=\z@
%    \end{macrocode}
% \ldots\space we append a `special space' if the line isn't empty.
%    \begin{macrocode}
        \lst@AppendSpecialSpace
    \else \ifnum\lst@length=\z@
%    \end{macrocode}
% If the line is empty, we check whether there are characters in the output
% queue. If there are no characters we just advance |\lst@lostspace|.
% Otherwise we append the space.
%    \begin{macrocode}
            \global\advance\lst@lostspace\lst@width
            \global\advance\lst@pos\m@ne
            \lst@whitespacetrue
        \else
            \lst@AppendSpecialSpace
        \fi
    \fi \fi}
%    \end{macrocode}
% Note that this version works for fixed and flexible column output.
% \end{macro}
%
% \begin{macro}{\lst@AppendSpecialSpace}
% If there are at least two white spaces, we output preceding characters and
% advance |\lst@lostspace| to avoid alignment problems. Otherwise we append
% a space to the current character string.  Also, |\lst@whitespacetrue| has
% been moved after |\lst@PrintToken| so that the token-printer can correctly
% check whether it is printing whitespace or not; this was preventing the
% |spaceflexible| column setting from working correctly.
%    \begin{macrocode}
\def\lst@AppendSpecialSpace{%
    \lst@ifwhitespace
        \lst@PrintToken
        \global\advance\lst@lostspace\lst@width
        \global\advance\lst@pos\m@ne
        \lst@gobbledwhitespacetrue
    \else
        \lst@PrintToken
        \lst@whitespacetrue
        \lst@AppendOther\lst@outputspace
        \lst@PrintToken
    \fi}
%    \end{macrocode}
% \end{macro}
%
%
% \paragraph{Form feeds}
% has been introduced after communication with
% \lsthelper{Jan~Braun}{1998/04/27}{formfeed}.
%
% \begin{lstkey}{formfeed}
% let the user make adjustments.
%    \begin{macrocode}
\lst@Key{formfeed}{\bigbreak}{\def\lst@formfeed{#1}}
%    \end{macrocode}
% \end{lstkey}
%
% \begin{macro}{\lst@ProcessFormFeed}
% Here we execute some macros according to whether a new line has already
% begun or not. No |\lst@EOLUpdate| is used in the else branch
% anymore---\lsthelper{Kalle~Tuulos}{2001/01/14}{form feed gobbles following
% output unit} sent the bug report.
%    \begin{macrocode}
\def\lst@ProcessFormFeed{%
    \lst@XPrintToken
    \ifnum\lst@newlines=\z@
        \lst@EOLUpdate \lsthk@InitVarsBOL
    \fi
    \lst@formfeed
    \lst@whitespacetrue}
%    \end{macrocode}
% \end{macro}
%
%
% \subsection{Character tables}\label{iCharacterTables}
%
%
% \subsubsection{The standard table}
%
% The standard character table is selected by |\lst@SelectStdCharTable|, which
% expands to a token sequence
%    \ldots|\def| |A{\lst@ProcessLetter| |A}|\ldots\space
% where the first A is active and the second has catcode 12. We use the
% following macros to build the character table.
% \begin{syntax}
% \item[0.19] |\lst@CCPut|\meta{class macro}\meta{$c_1$}\ldots\meta{$c_k$}|\z@|
%
%       extends the standard character table by the characters with codes
%       \meta{$c_1$}\ldots\meta{$c_k$} making each character use
%       \meta{class macro}. All these characters must be printable via
%       |\char|\meta{$c_i$}.
%
% \item[0.20] |\lst@CCPutMacro|\meta{class$_1$}\meta{$c_1$}\meta{definition$_1$}\ldots|\@empty\z@\@empty|
%
%       also extends the standard character table: the character \meta{$c_i$}
%       will use \meta{class$_i$} and is printed via \meta{definition$_i$}.
%       These definitions must be \meta{spec. token}s in the sense of section
%       \ref{dCharacterTables}.
% \end{syntax}
%
% \begin{macro}{\lst@Def}
% \begin{macro}{\lst@Let}
% For speed we won't use these helpers too often.
%    \begin{macrocode}
\def\lst@Def#1{\lccode`\~=#1\lowercase{\def~}}
\def\lst@Let#1{\lccode`\~=#1\lowercase{\let~}}
%    \end{macrocode}
% \end{macro}
% \end{macro}
%
% \begingroup
% The definition of the space below doesn't hurt anything. But other aspects,
% for example \aspectname{lineshape} and \aspectname{formats}, redefine also
% the macro |\space|. Now, if \LaTeX\ calls |\try@load@fontshape|, the |.log|
% messages would show some strange things since \LaTeX\ uses |\space| in these
% messages. The following addition ensures that |\space| expands to a space
% and not to something different. This was one more bug reported by
% \lsthelper{Denis~Girou}{1999/09/16}{bad font info message with breaklines}.
%    \begin{macrocode}
\lst@AddToAtTop{\try@load@fontshape}{\def\space{ }}
%    \end{macrocode}
% \endgroup
%
% \begin{macro}{\lst@SelectStdCharTable}
% The first three standard characters. |\lst@Let| has been replaced by
% |\lst@Def| after a bug report from \lsthelper{Chris~Edwards}{2002/02/15}
% {tabulators show up with firstline>1}.
%    \begin{macrocode}
\def\lst@SelectStdCharTable{%
    \lst@Def{9}{\lst@ProcessTabulator}%
    \lst@Def{12}{\lst@ProcessFormFeed}%
    \lst@Def{32}{\lst@ProcessSpace}}
%    \end{macrocode}
% \end{macro}
%
% \begin{macro}{\lst@CCPut}
% The first argument gives the character class, then follow the codes.
%    \begin{macrocode}
\def\lst@CCPut#1#2{%
    \ifnum#2=\z@
        \expandafter\@gobbletwo
    \else
        \lccode`\~=#2\lccode`\/=#2\lowercase{\lst@CCPut@~{#1/}}%
    \fi
    \lst@CCPut#1}
\def\lst@CCPut@#1#2{\lst@lAddTo\lst@SelectStdCharTable{\def#1{#2}}}
%    \end{macrocode}
% Now we insert more standard characters.
%    \begin{macrocode}
\lst@CCPut \lst@ProcessOther
    {"21}{"22}{"28}{"29}{"2B}{"2C}{"2E}{"2F}
    {"3A}{"3B}{"3D}{"3F}{"5B}{"5D}
    \z@
\lst@CCPut \lst@ProcessDigit
    {"30}{"31}{"32}{"33}{"34}{"35}{"36}{"37}{"38}{"39}
    \z@
\lst@CCPut \lst@ProcessLetter
    {"40}{"41}{"42}{"43}{"44}{"45}{"46}{"47}
    {"48}{"49}{"4A}{"4B}{"4C}{"4D}{"4E}{"4F}
    {"50}{"51}{"52}{"53}{"54}{"55}{"56}{"57}
    {"58}{"59}{"5A}
         {"61}{"62}{"63}{"64}{"65}{"66}{"67}
    {"68}{"69}{"6A}{"6B}{"6C}{"6D}{"6E}{"6F}
    {"70}{"71}{"72}{"73}{"74}{"75}{"76}{"77}
    {"78}{"79}{"7A}
    \z@
%    \end{macrocode}
% \end{macro}
%
% \begin{macro}{\lst@CCPutMacro}
% Now we come to a delicate point. The characters not inserted yet aren't
% printable (|_|, |$|, \ldots) or aren't printed well (|*|, |-|, \ldots) if we
% enter these characters. Thus we use proper macros to print the characters.
% Works perfectly. The problem is that the current character string is
% printable for speed, for example |_| is already replaced by a macro version,
% but the new keyword tests need the original characters.
%
% The solution: We define |\def _{\lst@ProcessLetter\lst@um_}| where the first
% underscore is active and the second belongs to the control sequence.
% Moreover we have |\def\lst@um_{\lst@UM _}| where the second underscore has
% the usual meaning. Now the keyword tests can access the original character
% simply by making |\lst@UM| empty. The default definition gets the following
% token and builds the control sequence |\lst@um_@|, which we'll define to
% print the character. Easy, isn't it?^^A ;-)
%
% The following definition does all this for us. The first parameter gives the
% character class, the second the character code, and the last the definition
% which actually prints the character. We build the names |\lst@um_| and
% |\lst@um_@| and give them to a submacro.
%    \begin{macrocode}
\def\lst@CCPutMacro#1#2#3{%
    \ifnum#2=\z@ \else
        \begingroup\lccode`\~=#2\relax \lccode`\/=#2\relax
        \lowercase{\endgroup\expandafter\lst@CCPutMacro@
            \csname\@lst @um/\expandafter\endcsname
            \csname\@lst @um/@\endcsname /~}#1{#3}%
        \expandafter\lst@CCPutMacro
    \fi}
%    \end{macrocode}
% The arguments are now |\lst@um_|, |\lst@um_@|, nonactive character, active
% character, character class and printing definition. We add |\def _{|
% |\lst@ProcessLetter| |\lst@um_}| to |\lst@SelectStdCharTable| (and similarly
% other special characters), define |\def\lst@um_{\lst@UM _}| and |\lst@um_@|.
%    \begin{macrocode}
\def\lst@CCPutMacro@#1#2#3#4#5#6{%
    \lst@lAddTo\lst@SelectStdCharTable{\def#4{#5#1}}%
    \def#1{\lst@UM#3}%
    \def#2{#6}}
%    \end{macrocode}
% The default definition of |\lst@UM|:
%    \begin{macrocode}
\def\lst@UM#1{\csname\@lst @um#1@\endcsname}
%    \end{macrocode}
% And all remaining standard characters.
%    \begin{macrocode}
\lst@CCPutMacro
    \lst@ProcessOther {"23}\#
    \lst@ProcessLetter{"24}\textdollar
    \lst@ProcessOther {"25}\%
    \lst@ProcessOther {"26}\&
    \lst@ProcessOther {"27}{\lst@ifupquote \textquotesingle
                                     \else \char39\relax \fi}
    \lst@ProcessOther {"2A}{\lst@ttfamily*\textasteriskcentered}
    \lst@ProcessOther {"2D}{\lst@ttfamily{-{}}{$-$}}
    \lst@ProcessOther {"3C}{\lst@ttfamily<\textless}
    \lst@ProcessOther {"3E}{\lst@ttfamily>\textgreater}
    \lst@ProcessOther {"5C}{\lst@ttfamily{\char92}\textbackslash}
    \lst@ProcessOther {"5E}\textasciicircum
    \lst@ProcessLetter{"5F}{\lst@ttfamily{\char95}\textunderscore}
    \lst@ProcessOther {"60}{\lst@ifupquote \textasciigrave
                                     \else \char96\relax \fi}
    \lst@ProcessOther {"7B}{\lst@ttfamily{\char123}\textbraceleft}
    \lst@ProcessOther {"7C}{\lst@ttfamily|\textbar}
    \lst@ProcessOther {"7D}{\lst@ttfamily{\char125}\textbraceright}
    \lst@ProcessOther {"7E}\textasciitilde
    \lst@ProcessOther {"7F}-
    \@empty\z@\@empty
%    \end{macrocode}
% \end{macro}
%
% \begin{macro}{\lst@ttfamily}
% What is this ominous macro? It prints either the first or the second
% argument. In |\ttfamily| it ensures that |----| is typeset |----| and not
% $-$$-$$-$$-$ as in version 0.17. Bug encountered by
% \lsthelper{Dr.~Jobst~Hoffmann}{1998/03/30}{|\lst@minus| and |\ttfamily|}.
% Furthermore I added |\relax| after receiving an error report from
% \lsthelper{Magnus~Lewis-Smith}{1999/08/06}{! Bad character code (920).}
%    \begin{macrocode}
\def\lst@ttfamily#1#2{\ifx\f@family\ttdefault#1\relax\else#2\fi}
%    \end{macrocode}
% |\ttdefault| is defined |\long|, so the |\ifx| doesn't work since |\f@family|
% isn't |\long|! We go around this problem by redefining |\ttdefault| locally:
%    \begin{macrocode}
\lst@AddToHook{Init}{\edef\ttdefault{\ttdefault}}
%    \end{macrocode}
% \end{macro}
%
% \begin{lstkey}{upquote}
% is used above to decide which quote to print. We print an error message if
% the necessary \packagename{textcomp} commands are not available. This key
% has been added after an email from \lsthelper{Frank~Mittelbach}{2003/06/18}
% {listings and upquote}.
%    \begin{macrocode}
\lst@Key{upquote}{false}[t]{\lstKV@SetIf{#1}\lst@ifupquote
    \lst@ifupquote
       \@ifundefined{textasciigrave}%
          {\let\KV@lst@upquote\@gobble
           \lstKV@SetIf f\lst@ifupquote \@gobble\fi
           \PackageError{Listings}{Option `upquote' requires `textcomp'
            package.\MessageBreak The option has been disabled}%
          {Add \string\usepackage{textcomp} to your preamble.}}%
          {}%
    \fi}
%    \end{macrocode}
% If an \packagename{upquote} package is loaded, the \keyname{upquote} option
% is enabled by default.
%    \begin{macrocode}
\AtBeginDocument{%
  \@ifpackageloaded{upquote}{\RequirePackage{textcomp}%
                             \lstset{upquote}}{}%
  \@ifpackageloaded{upquote2}{\lstset{upquote}}{}}
%    \end{macrocode}
% \end{lstkey}
%
% \begin{macro}{\lst@ifactivechars}
% A simple switch.
%    \begin{macrocode}
\def\lst@activecharstrue{\let\lst@ifactivechars\iftrue}
\def\lst@activecharsfalse{\let\lst@ifactivechars\iffalse}
\lst@activecharstrue
%    \end{macrocode}
% \end{macro}
%
% \begin{macro}{\lst@SelectCharTable}
% We select the standard character table and switch to active catcodes.
%    \begin{macrocode}
\def\lst@SelectCharTable{%
    \lst@SelectStdCharTable
    \lst@ifactivechars
        \catcode9\active \catcode12\active \catcode13\active
        \@tempcnta=32\relax
        \@whilenum\@tempcnta<128\do
            {\catcode\@tempcnta\active\advance\@tempcnta\@ne}%
    \fi
    \lst@ifec \lst@DefEC \fi
%    \end{macrocode}
% The following line and the according macros below have been added after a
% bug report from \lsthelper{Fr\'ed\'eric~Boulanger}{2001/02/27}{ligatures}.
% The assignment to |\do@noligs| was changed to |\do| after a bug report from
% \lsthelper{Peter~Ruckdeschel}{2002/04/12}{problems with simultanous use of
% seminar.sty and listings.sty}. This bugfix was kindly provided by
% \lsthelper{Timothy~Van~Zandt}{2002/04/13}{Re: ...}.
%    \begin{macrocode}
    \let\do\lst@do@noligs \verbatim@nolig@list
%    \end{macrocode}
% There are two ways to adjust the standard table: inside the hook or with
% |\lst@DeveloperSCT|. We use these macros and initialize the backslash if
% necessary. |\lst@DefRange| has been moved outside the hook after a bug report
% by \lsthelper{Michael~Bachmann}{2004/07/21}{Keine label-Referenzierung
% m\"oglich...}.
%    \begin{macrocode}
    \lsthk@SelectCharTable
    \lst@DeveloperSCT
	\lst@DefRange
    \ifx\lst@Backslash\relax\else
        \lst@LetSaveDef{"5C}\lsts@backslash\lst@Backslash
    \fi}
%    \end{macrocode}
% \end{macro}
%
% \begin{lstkey}{SelectCharTable}
% \begin{lstkey}{MoreSelectCharTable}
% The keys to adjust |\lst@DeveloperSCT|.
%    \begin{macrocode}
\lst@Key{SelectCharTable}{}{\def\lst@DeveloperSCT{#1}}
\lst@Key{MoreSelectCharTable}\relax{\lst@lAddTo\lst@DeveloperSCT{#1}}
%    \end{macrocode}
%    \begin{macrocode}
\lst@AddToHook{SetLanguage}{\let\lst@DeveloperSCT\@empty}
%    \end{macrocode}
% \end{lstkey}
% \end{lstkey}
%
% \begin{macro}{\lst@do@noligs}
% To prevent ligatures, this macro inserts the token |\lst@NoLig| in front of
% |\lst@Process|\meta{whatever}\meta{spec.~token}. This is done by
% |\verbatim@nolig@list| for certain characters. Note that the submacro is
% a special kind of a local |\lst@AddToAtTop|. The submacro definition was
% fixed thanks to \lsthelper{Peter~Bartke}{2002/04/10}{bad `noligs' handling}.
%    \begin{macrocode}
\def\lst@do@noligs#1{%
    \begingroup \lccode`\~=`#1\lowercase{\endgroup
    \lst@do@noligs@~}}
\def\lst@do@noligs@#1{%
    \expandafter\expandafter\expandafter\def
    \expandafter\expandafter\expandafter#1%
    \expandafter\expandafter\expandafter{\expandafter\lst@NoLig#1}}
%    \end{macrocode}
% \end{macro}
%
% \begin{macro}{\lst@NoLig}
% When this extra macro is processed, it adds |\lst@nolig| to the output queue
% without increasing its length. For keyword detection this must expand to
% nothing if |\lst@UM| is empty.
%    \begin{macrocode}
\def\lst@NoLig{\advance\lst@length\m@ne \lst@Append\lst@nolig}
\def\lst@nolig{\lst@UM\@empty}%
%    \end{macrocode}
% But the usual meaning of |\lst@UM| builds the following control sequence,
% which prevents ligatures in the manner of \LaTeX's |\do@noligs|.
%    \begin{macrocode}
\@namedef{\@lst @um@}{\leavevmode\kern\z@}
%    \end{macrocode}
% \end{macro}
%
% \begin{macro}{\lst@SaveOutputDef}
% To get the \meta{spec.~token} meaning of character |#1|, we look for |\def|
% `active character |#1|' in |\lst@SelectStdCharTable|, get the replacement
% text, strip off the character class via |\@gobble|, and assign the meaning.
% Note that you get a ``runaway argument'' error if an illegal \meta{character
% code}=|#1| is used.
%    \begin{macrocode}
\def\lst@SaveOutputDef#1#2{%
    \begingroup \lccode`\~=#1\relax \lowercase{\endgroup
    \def\lst@temp##1\def~##2##3\relax}{%
        \global\expandafter\let\expandafter#2\@gobble##2\relax}%
    \expandafter\lst@temp\lst@SelectStdCharTable\relax}
%    \end{macrocode}
% \end{macro}
%
% \begin{macro}{\lstum@backslash}
% A commonly used character.
%    \begin{macrocode}
\lst@SaveOutputDef{"5C}\lstum@backslash
%    \end{macrocode}
% \end{macro}
%
%
% \subsubsection{National characters}
%
% \begin{lstkey}{extendedchars}
% The user key to activate extended characters 128--255.
%    \begin{macrocode}
\lst@Key{extendedchars}{true}[t]{\lstKV@SetIf{#1}\lst@ifec}
%    \end{macrocode}
% \end{lstkey}
%
% \begin{macro}{\lst@DefEC}
% Currently each character in the range 128--255 is treated as a letter.
%    \begin{macrocode}
\def\lst@DefEC{%
    \lst@CCECUse \lst@ProcessLetter
      ^^80^^81^^82^^83^^84^^85^^86^^87^^88^^89^^8a^^8b^^8c^^8d^^8e^^8f%
      ^^90^^91^^92^^93^^94^^95^^96^^97^^98^^99^^9a^^9b^^9c^^9d^^9e^^9f%
      ^^a0^^a1^^a2^^a3^^a4^^a5^^a6^^a7^^a8^^a9^^aa^^ab^^ac^^ad^^ae^^af%
      ^^b0^^b1^^b2^^b3^^b4^^b5^^b6^^b7^^b8^^b9^^ba^^bb^^bc^^bd^^be^^bf%
      ^^c0^^c1^^c2^^c3^^c4^^c5^^c6^^c7^^c8^^c9^^ca^^cb^^cc^^cd^^ce^^cf%
      ^^d0^^d1^^d2^^d3^^d4^^d5^^d6^^d7^^d8^^d9^^da^^db^^dc^^dd^^de^^df%
      ^^e0^^e1^^e2^^e3^^e4^^e5^^e6^^e7^^e8^^e9^^ea^^eb^^ec^^ed^^ee^^ef%
      ^^f0^^f1^^f2^^f3^^f4^^f5^^f6^^f7^^f8^^f9^^fa^^fb^^fc^^fd^^fe^^ff%
      ^^00}
%    \end{macrocode}
% \end{macro}
%
% \begin{macro}{\lst@CCECUse}
% Reaching end of list (|^^00|) we terminate the loop.
% Otherwise we do the same as in |\lst@CCPut| if the character is not active.
% But if the character is active, we save the meaning before redefinition.
%    \begin{macrocode}
\def\lst@CCECUse#1#2{%
    \ifnum`#2=\z@
        \expandafter\@gobbletwo
    \else
        \ifnum\catcode`#2=\active
            \lccode`\~=`#2\lccode`\/=`#2\lowercase{\lst@CCECUse@#1~/}%
        \else
            \lst@ifactivechars \catcode`#2=\active \fi
            \lccode`\~=`#2\lccode`\/=`#2\lowercase{\def~{#1/}}%
        \fi
    \fi
    \lst@CCECUse#1}
%    \end{macrocode}
% We save the meaning as mentioned. Here we must also use the `|\lst@UM|
% construction' since extended characters could often appear in words =
% identifiers. Bug reported by \lsthelper{Denis~Girou}{1999/07/26}
% {incompatibility with inputenc}.
%    \begin{macrocode}
\def\lst@CCECUse@#1#2#3{%
    \expandafter\def\csname\@lst @EC#3\endcsname{\lst@UM#3}%
    \expandafter\let\csname\@lst @um#3@\endcsname #2%
    \edef#2{\noexpand#1%
            \expandafter\noexpand\csname\@lst @EC#3\endcsname}}
%    \end{macrocode}
% \lsthelper{Daniel~Gerigk}{2001/10/25}{extendedchars do not work} and
% \lsthelper{Heiko~Oberdiek}{2001/10/26}{extendedchars do not work: um@\#3@
% must be @um\#3@} reported an error and a solution, respectively.
% \end{macro}
%
%
% \subsubsection{Catcode problems}
%
% \begin{macro}{\lst@nfss@catcodes}
% \lsthelper{Anders~Edenbrandt}{1997/04/22}{preload of .fd files} found a bug
% with \texttt{.fd}-files. Since we change catcodes and these files are read
% on demand, we must reset the catcodes before the files are input. We use a
% local redefinition of |\nfss@catcodes|.
%    \begin{macrocode}
\lst@AddToHook{Init}
    {\let\lsts@nfss@catcodes\nfss@catcodes
     \let\nfss@catcodes\lst@nfss@catcodes}
%    \end{macrocode}
% The |&|-character had turned into |\&| after a bug report by \lsthelper
% {David~Aspinall}{2003/07/17}{loading of .fd file inside tabular produces
% error}.
%    \begin{macrocode}
\def\lst@nfss@catcodes{%
    \lst@makeletter
        ABCDEFGHIJKLMNOPQRSTUVWXYZabcdefghijklmnopqrstuvwxyz\relax
    \@makeother (\@makeother )\@makeother ,\@makeother :\@makeother\&%
    \@makeother 0\@makeother 1\@makeother 2\@makeother 3\@makeother 4%
    \@makeother 5\@makeother 6\@makeother 7\@makeother 8\@makeother 9%
    \@makeother =\lsts@nfss@catcodes}
%    \end{macrocode}
% The investigation of a bug reported by \lsthelper{Christian~Gudrian}
% {2000/11/16}{problems with mathpple} showed that the equal sign needs
% to have `other' catcode, as assigned above.
% \lsthelper{Svend~Tollak~Munkejord}{2002/04/17}{package incompatible with
%  Lucida .fd files} reported problems with Lucida .fd-files, while
% \lsthelper{Heiko~Oberdiek}{2002/04/17}{Re: listings fails with Lucida
% font} analysed the bug, which above led to the line starting with
% |\@makeaother (|.
%
% The name of |\lst@makeletter| is an imitation of \LaTeX's |\@makeother|.
%    \begin{macrocode}
\def\lst@makeletter#1{%
    \ifx\relax#1\else\catcode`#111\relax \expandafter\lst@makeletter\fi}
%    \end{macrocode}
% \end{macro}
%
% \begin{lstkey}{useoutput}
% \begin{macro}{\output}
% Another problem was first reported by \lsthelper{Marcin~Kasperski}
% {1999/04/28}{listings spoil toc}. It is also catcode related and
% \lsthelper{Donald~Arseneau}{1999/05/13}{comp.text.tex Re: delayed write and
% catcode changes} let me understand it. The point is that \TeX\ seems to use
% the \emph{currently} active catcode table when it writes non-|\immediate|
% |\write|s to file and not the catcodes involved when \emph{reading} the
% characters.
% So a section heading |\L a| was written |\La| if a listing was split on two
% pages since a non-standard catcode table was in use when writing |\La| to
% file, the previously attached catcodes do not matter. One more bug was that
% accents in page headings or footers were lost when a listing was split on
% two pages. \lsthelper{Denis~Girou}{1999/08/03}{Accents lost in heading if
% listing split on two pages} found this latter bug. A similar problem with
% the tilde was reported by \lsthelper{Thorsten~Vitt}{2001/06/25}{fancyhdr +
% listings crossing pages ==> ~ in header, not space}.
%
% We can choose between three possibilities.
% \lsthelper{Donald~Arseneau}{2006/09/14}{cannot select output routine 1}
% noted a bug here in the |\ifcase| argument.
%    \begin{macrocode}
\lst@Key{useoutput}{2}{\edef\lst@useoutput{\ifcase0#1 0\or 1\else 2\fi}}
%    \end{macrocode}
% The first does not modify the existing output routine.
%    \begin{macrocode}
\lst@AddToHook{Init}
{\edef\lst@OrgOutput{\the\output}%
\ifcase\lst@useoutput\relax
\or
%    \end{macrocode}
% The second possibility is as follows: We interrupt the current modes---in
% particular |\lst@Pmode| with modified catcode table---, call the original
% output routine and reenter the mode. This must be done with a little care.
% First we have to close the group which \TeX\ opens at the beginning of the
% output routine. A single |\egroup| gives an `unbalanced output routine'
% error. But |\expandafter\egroup| works. Again it was
% \lsthelper{Donald~Arseneau}{2001/01/10}{comp.text.tex Re: \output puzzle}
% who gave the explaination: The |\expandafter| set the token type of |\bgroup|
% to |backed_up|, which prevents \TeX's from recovering from an unbalanced
% output routine. \lsthelper{Heiko~Oberdiek}{2001/01/05}{comp.text.tex Re:
% \output puzzle} reported that |\csname| |egroup||\endcsname| does the trick,
% too.
%
% However, since \TeX\ checks the contents of |\box| 255 when we close the
% group (`output routine didn't use all of |\box| 255'), we have to save it
% temporaryly.
%    \begin{macrocode}
 \output{\global\setbox\lst@gtempboxa\box\@cclv
         \expandafter\egroup
%    \end{macrocode}
% Now we can interrupt the mode, but we have to save the current character
% string and the current style.
%    \begin{macrocode}
         \lst@SaveToken
     \lst@InterruptModes
%    \end{macrocode}
% We restore the contents, use the original output routine, and \ldots
%    \begin{macrocode}
     \setbox\@cclv\box\lst@gtempboxa
     \bgroup\lst@OrgOutput\egroup
%    \end{macrocode}
% \ldots\space open a group matching the |}| which \TeX\ inserts at the end of
% the output routine. We reenter modes and restore the character string and
% style |\aftergroup|. Moreover we need to reset |\pagegoal|---added after a
% bug report by \lsthelper{Jochen~Schneider}{2002/03/09}{de.comp.text.tex:
% Problem mit Listings-Paket 1.0-Beta; unmotivated pagebreak with preceding
% float}.
%    \begin{macrocode}
     \bgroup
     \aftergroup\pagegoal\aftergroup\vsize
     \aftergroup\lst@ReenterModes\aftergroup\lst@RestoreToken}%
\else
%    \end{macrocode}
% The third option is to restore all catcodes and meanings inside a modified
% output routine and to call the original routine afterwards.
%    \begin{macrocode}
 \output{\lst@RestoreOrigCatcodes
         \lst@ifec \lst@RestoreOrigExtendedCatcodes \fi
         \lst@OrgOutput}%
\fi}
%    \end{macrocode}
% Note that this output routine isn't used too often. It is executed only if
% it's possible that a listing is split on two pages: if a listing ends at
% the bottom or begins at the top of a page, or if a listing is really split.
% \end{macro}
% \end{lstkey}
%
% \begin{macro}{\lst@GetChars}
% \begin{macro}{\lst@ScanChars}
% \begin{lstkey}{rescanchars}
% To make the third |\output|-option work, we have to scan the catcodes and
% also the meanings of active characters:
%    \begin{macrocode}
\def\lst@GetChars#1#2#3{%
    \let#1\@empty
    \@tempcnta#2\relax \@tempcntb#3\relax
    \loop \ifnum\@tempcnta<\@tempcntb\relax
        \lst@lExtend#1{\expandafter\catcode\the\@tempcnta=}%
        \lst@lExtend#1{\the\catcode\@tempcnta\relax}%
        \ifnum\the\catcode\@tempcnta=\active
            \begingroup\lccode`\~=\@tempcnta
            \lowercase{\endgroup
            \lst@lExtend#1{\expandafter\let\expandafter~\csname
                                    lstecs@\the\@tempcnta\endcsname}%
            \expandafter\let\csname lstecs@\the\@tempcnta\endcsname~}%
        \fi
        \advance\@tempcnta\@ne
    \repeat}
%    \end{macrocode}
% As per a bug report by \lsthelper{Benjamin~Lings}{2004/10/15}%
% {\usepackage{xy,listings} yields: "Forbidden control sequence...."}, we
% deactivate |\outer| definition of |^^L| temporarily (inside and outside
% of |\lst@ScanChars|) and restore the catcode at end of package via the
% |\lst@RestoreCatcodes| command.
%    \begin{macrocode}
\begingroup \catcode12=\active\let^^L\@empty
\gdef\lst@ScanChars{%
  \let\lsts@ssL^^L%
  \def^^L{\par}%
    \lst@GetChars\lst@RestoreOrigCatcodes\@ne {128}%
  \let^^L\lsts@ssL
    \lst@GetChars\lst@RestoreOrigExtendedCatcodes{128}{256}}
\endgroup
%    \end{macrocode}
% The scan can be issued by hand and at the beginning of a document.
%    \begin{macrocode}
\lst@Key{rescanchars}\relax{\lst@ScanChars}
\AtBeginDocument{\lst@ScanChars}
%    \end{macrocode}
% \end{lstkey}
% \end{macro}
% \end{macro}
%
%
% \subsubsection{Adjusting the table}
%
% We begin with modifiers for the basic character classes.
%
% \begin{lstkey}{alsoletter}
% \begin{lstkey}{alsodigit}
% \begin{lstkey}{alsoother}
% The macros |\lst@also|\ldots\space will hold |\def|\meta{char}|{|\ldots|}|
% sequences, which adjusts the standard character table.
%    \begin{macrocode}
\lst@Key{alsoletter}\relax{%
    \lst@DoAlso{#1}\lst@alsoletter\lst@ProcessLetter}
\lst@Key{alsodigit}\relax{%
    \lst@DoAlso{#1}\lst@alsodigit\lst@ProcessDigit}
\lst@Key{alsoother}\relax{%
    \lst@DoAlso{#1}\lst@alsoother\lst@ProcessOther}
%    \end{macrocode}
% This is done at \hookname{SelectCharTable} and every language selection
% the macros get empty.
%    \begin{macrocode}
\lst@AddToHook{SelectCharTable}
    {\lst@alsoother \lst@alsodigit \lst@alsoletter}
\lst@AddToHookExe{SetLanguage}% init
    {\let\lst@alsoletter\@empty
     \let\lst@alsodigit\@empty
     \let\lst@alsoother\@empty}
%    \end{macrocode}
% The service macro starts a loop and \ldots
%    \begin{macrocode}
\def\lst@DoAlso#1#2#3{%
    \lst@DefOther\lst@arg{#1}\let#2\@empty
    \expandafter\lst@DoAlso@\expandafter#2\expandafter#3\lst@arg\relax}
\def\lst@DoAlso@#1#2#3{%
    \ifx\relax#3\expandafter\@gobblethree \else
%    \end{macrocode}
% \ldots\space while not reaching |\relax| we use the \TeX nique from
% |\lst@SaveOutputDef| to replace the class by |#2|. Eventually we append
% the new definition to |#1|.
%    \begin{macrocode}
        \begingroup \lccode`\~=`#3\relax \lowercase{\endgroup
        \def\lst@temp##1\def~##2##3\relax{%
            \edef\lst@arg{\def\noexpand~{\noexpand#2\expandafter
                                         \noexpand\@gobble##2}}}}%
        \expandafter\lst@temp\lst@SelectStdCharTable\relax
        \lst@lExtend#1{\lst@arg}%
    \fi
    \lst@DoAlso@#1#2}
%    \end{macrocode}
% \end{lstkey}
% \end{lstkey}
% \end{lstkey}
%
% \begin{macro}{\lst@SaveDef}
% \begin{macro}{\lst@DefSaveDef}
% \begin{macro}{\lst@LetSaveDef}
% These macros can be used in language definitions to make special changes.
% They save the definition and define or assign a new one.
%    \begin{macrocode}
\def\lst@SaveDef#1#2{%
    \begingroup \lccode`\~=#1\relax \lowercase{\endgroup\let#2~}}
\def\lst@DefSaveDef#1#2{%
    \begingroup \lccode`\~=#1\relax \lowercase{\endgroup\let#2~\def~}}
\def\lst@LetSaveDef#1#2{%
    \begingroup \lccode`\~=#1\relax \lowercase{\endgroup\let#2~\let~}}
%    \end{macrocode}
% \end{macro}
% \end{macro}
% \end{macro}
%
% Now we get to the more powerful definitions.
%
% \begin{macro}{\lst@CDef}
% Here we unfold the first parameter \meta{1st}\marg{2nd}\marg{rest} and say
% that this input string is `replaced' by \meta{save 1st}\marg{2nd}^^A
% \marg{rest}---plus \meta{execute}, \meta{pre}, and \meta{post}. This main
% work is done by |\lst@CDefIt|.
%    \begin{macrocode}
\def\lst@CDef#1{\lst@CDef@#1}
\def\lst@CDef@#1#2#3#4{\lst@CDefIt#1{#2}{#3}{#4#2#3}#4}
%    \end{macrocode}
% \end{macro}
%
% \begin{macro}{\lst@CDefX}
% drops the input string.
%    \begin{macrocode}
\def\lst@CDefX#1{\lst@CDefX@#1}
\def\lst@CDefX@#1#2#3{\lst@CDefIt#1{#2}{#3}{}}
%    \end{macrocode}
% \end{macro}
%
% \begin{macro}{\lst@CDefIt}
% is the main working procedure for the previous macros. It redefines the
% sequence |#1#2#3| of characters. At least |#1| must be active; the other two
% arguments might be empty, not equivalent to empty!
%    \begin{macrocode}
\def\lst@CDefIt#1#2#3#4#5#6#7#8{%
    \ifx\@empty#2\@empty
%    \end{macrocode}
% For a single character we just execute the arguments in the correct order.
% You might want to go back to section \ref{dCharacterTablesManipulated} to
% look them up.
%    \begin{macrocode}
        \def#1{#6\def\lst@next{#7#4#8}\lst@next}%
    \else \ifx\@empty#3\@empty
%    \end{macrocode}
% For a two character sequence we test whether \meta{pre} and \meta{post}
% must be executed.
%    \begin{macrocode}
        \def#1##1{%
            #6%
            \ifx##1#2\def\lst@next{#7#4#8}\else
                     \def\lst@next{#5##1}\fi
            \lst@next}%
    \else
%    \end{macrocode}
% We do the same for an arbitrary character sequence---except that we have to
% use |\lst@IfNextCharsArg| instead of |\ifx|\ldots|\fi|.
%    \begin{macrocode}
        \def#1{%
            #6%
            \lst@IfNextCharsArg{#2#3}{#7#4#8}%
                                     {\expandafter#5\lst@eaten}}%
    \fi \fi}
%    \end{macrocode}
% \end{macro}
%
% \begin{macro}{\lst@CArgX}
% We make |#1#2| active and call |\lst@CArg|.
%    \begin{macrocode}
\def\lst@CArgX#1#2\relax{%
    \lst@DefActive\lst@arg{#1#2}%
    \expandafter\lst@CArg\lst@arg\relax}
%    \end{macrocode}
% \end{macro}
%
% \begin{macro}{\lst@CArg}
% arranges the first two arguments for |\lst@CDef|[|X|]. We get an undefined
% macro and use |\@empty\@empty\relax| as delimiter for the submacro.
%    \begin{macrocode}
\def\lst@CArg#1#2\relax{%
    \lccode`\/=`#1\lowercase{\def\lst@temp{/}}%
    \lst@GetFreeMacro{lst@c\lst@temp}%
    \expandafter\lst@CArg@\lst@freemacro#1#2\@empty\@empty\relax}
%    \end{macrocode}
% Save meaning of \meta{1st}=|#2| in \meta{save 1st}=|#1| and call the macro
% |#6| with correct arguments. From version 1.0 on, |#2|, |#3| and |#4|
% (respectively empty arguments) are tied together with group braces.
% This allows us to save two arguments in other definitions, for example in
% |\lst@DefDelimB|.
%    \begin{macrocode}
\def\lst@CArg@#1#2#3#4\@empty#5\relax#6{%
    \let#1#2%
    \ifx\@empty#3\@empty
        \def\lst@next{#6{#2{}{}}}%
    \else
        \def\lst@next{#6{#2#3{#4}}}%
    \fi
    \lst@next #1}
%    \end{macrocode}
% \end{macro}
%
% \begin{macro}{\lst@CArgEmpty}
% `executes' an |\@empty|-delimited argument. We will use it for the delimiters.
%    \begin{macrocode}
\def\lst@CArgEmpty#1\@empty{#1}
%    \end{macrocode}
% \end{macro}
%
%
% \subsection{Delimiters}
%
% Here we start with general definitions common to all delimiters.
%
% \begin{lstkey}{excludedelims}
% controls which delimiters are not printed in \meta{whatever}style. We just
% define |\lst@ifex|\meta{whatever} to be true. Such switches are set false
% in the \hookname{ExcludeDelims} hook and are handled by the individual
% delimiters.
%    \begin{macrocode}
\lst@Key{excludedelims}\relax
    {\lsthk@ExcludeDelims \lst@NormedDef\lst@temp{#1}%
     \expandafter\lst@for\lst@temp\do
     {\expandafter\let\csname\@lst @ifex##1\endcsname\iftrue}}
%    \end{macrocode}
% \end{lstkey}
%
% \begin{macro}{\lst@DelimPrint}
% And this macro might help in doing so. |#1| is |\lst@ifex|\meta{whatever}
% (plus |\else|) or just |\iffalse|, and |#2| will be the delimiter. The
% temporary mode change ensures that the characters can't end the current
% delimiter or start a new one.
%    \begin{macrocode}
\def\lst@DelimPrint#1#2{%
    #1%
      \begingroup
        \lst@mode\lst@nomode \lst@modetrue
        #2\lst@XPrintToken
      \endgroup
      \lst@ResetToken
    \fi}
%    \end{macrocode}
% \end{macro}
%
% \begin{macro}{\lst@DelimOpen}
% We print preceding characters and the delimiter, enter the appropriate mode,
% print the delimiter again, and execute |#3|. In fact, the arguments |#1| and
% |#2| will ensure that the delimiter is printed only once.
%    \begin{macrocode}
\def\lst@DelimOpen#1#2#3#4#5#6\@empty{%
    \lst@TrackNewLines \lst@XPrintToken
    \lst@DelimPrint#1{#6}%
    \lst@EnterMode{#4}{\def\lst@currstyle#5}%
    \lst@DelimPrint{#1#2}{#6}%
    #3}
%    \end{macrocode}
% \end{macro}
%
% \begin{macro}{\lst@DelimClose}
% is the same in reverse order.
%    \begin{macrocode}
\def\lst@DelimClose#1#2#3\@empty{%
    \lst@TrackNewLines \lst@XPrintToken
    \lst@DelimPrint{#1#2}{#3}%
    \lst@LeaveMode
    \lst@DelimPrint{#1}{#3}}
%    \end{macrocode}
% \end{macro}
%
% \begin{macro}{\lst@BeginDelim}
% \begin{macro}{\lst@EndDelim}
% These definitions are applications of |\lst@DelimOpen| and |\lst@DelimClose|:
% the delimiters have the same style as the delimited text.
%    \begin{macrocode}
\def\lst@BeginDelim{\lst@DelimOpen\iffalse\else{}}
\def\lst@EndDelim{\lst@DelimClose\iffalse\else}
%    \end{macrocode}
% \end{macro}
% \end{macro}
%
% \begin{macro}{\lst@BeginIDelim}
% \begin{macro}{\lst@EndIDelim}
% Another application: no delimiter is printed.
%    \begin{macrocode}
\def\lst@BeginIDelim{\lst@DelimOpen\iffalse{}{}}
\def\lst@EndIDelim{\lst@DelimClose\iffalse{}}
%    \end{macrocode}
% \end{macro}
% \end{macro}
%
% \begin{macro}{\lst@DefDelims}
% This macro defines all delimiters and is therefore reset every language
% selection.
%    \begin{macrocode}
\lst@AddToHook{SelectCharTable}{\lst@DefDelims}
\lst@AddToHookExe{SetLanguage}{\let\lst@DefDelims\@empty}
%    \end{macrocode}
% \end{macro}
%
% \begin{macro}{\lst@Delim}
% First we set default values: no |\lst@modetrue|, cumulative style, and no
% argument to |\lst@Delim|[|DM|]|@|\meta{type}.
%    \begin{macrocode}
\def\lst@Delim#1{%
    \lst@false \let\lst@cumulative\@empty \let\lst@arg\@empty
%    \end{macrocode}
% These are the correct settings for the double-star-form, so we immediately
% call the submacro in this case. Otherwise we either just suppress cumulative
% style, or even indicate the usage of |\lst@modetrue| with |\lst@true|.
%    \begin{macrocode}
    \@ifstar{\@ifstar{\lst@Delim@{#1}}%
                     {\let\lst@cumulative\relax
                      \lst@Delim@{#1}}}%
            {\lst@true\lst@Delim@{#1}}}
%    \end{macrocode}
% The type argument is saved for later use. We check against the optional
% \meta{style} argument using |#1| as default, define |\lst@delimstyle| and
% look for the optional \meta{type option}, which is just saved in |\lst@arg|.
%    \begin{macrocode}
\def\lst@Delim@#1[#2]{%
    \gdef\lst@delimtype{#2}%
    \@ifnextchar[\lst@Delim@sty
                 {\lst@Delim@sty[#1]}}
\def\lst@Delim@sty[#1]{%
    \def\lst@delimstyle{#1}%
    \ifx\@empty#1\@empty\else
        \lst@Delim@sty@ #1\@nil
    \fi
    \@ifnextchar[\lst@Delim@option
                 \lst@Delim@delim}
\def\lst@Delim@option[#1]{\def\lst@arg{[#1]}\lst@Delim@delim}
%    \end{macrocode}
% |[| and |]| in the replacement text above have been added after a bug report
% by \lsthelper{Stephen~Reindl}{2002/05/28}{\inaccessible using Cobol}.
%
% The definition of |\lst@delimstyle| depends on whether the first token is a
% control sequence. Here we possibly build |\lst@|\meta{style}.
%    \begin{macrocode}
\def\lst@Delim@sty@#1#2\@nil{%
    \if\relax\noexpand#1\else
        \edef\lst@delimstyle{\expandafter\noexpand
                             \csname\@lst @\lst@delimstyle\endcsname}%
    \fi}
%    \end{macrocode}
% \end{macro}
%
% \begin{macro}{\lst@Delim@delim}
% Eventually this macro is called. First we might need to delete a bunch of
% delimiters. If there is no delimiter, we might delete a subclass.
%    \begin{macrocode}
\def\lst@Delim@delim#1\relax#2#3#4#5#6#7#8{%
    \ifx #4\@empty \lst@Delim@delall{#2}\fi
    \ifx\@empty#1\@empty
        \ifx #4\@nil
            \@ifundefined{\@lst @#2DM@\lst@delimtype}%
                {\lst@Delim@delall{#2@\lst@delimtype}}%
                {\lst@Delim@delall{#2DM@\lst@delimtype}}%
        \fi
    \else
%    \end{macrocode}
% If the delimiter is not empty, we convert the delimiter and append it to
% |\lst@arg|. Ditto |\lst@Begin|\ldots, |\lst@End|\ldots, and the style and
% mode selection.
%    \begin{macrocode}
        \expandafter\lst@Delim@args\expandafter
            {\lst@delimtype}{#1}{#5}#6{#7}{#8}#4%
%    \end{macrocode}
% If the type is known, we either choose dynamic or static mode and use the
% contents of |\lst@arg| as arguments. All this is put into |\lst@delim|.
%    \begin{macrocode}
        \let\lst@delim\@empty
        \expandafter\lst@IfOneOf\lst@delimtype\relax#3%
        {\@ifundefined{\@lst @#2DM@\lst@delimtype}%
             {\lst@lExtend\lst@delim{\csname\@lst @#2@\lst@delimtype
                                     \expandafter\endcsname\lst@arg}}%
             {\lst@lExtend\lst@delim{\expandafter\lst@UseDynamicMode
                                     \csname\@lst @#2DM@\lst@delimtype
                                     \expandafter\endcsname\lst@arg}}%
%    \end{macrocode}
% Now, depending on the mode |#4| we either remove this particular delimiter or
% append it to all current ones.
%    \begin{macrocode}
         \ifx #4\@nil
             \let\lst@temp\lst@DefDelims \let\lst@DefDelims\@empty
             \expandafter\lst@Delim@del\lst@temp\@empty\@nil\@nil\@nil
         \else
             \lst@lExtend\lst@DefDelims\lst@delim
         \fi}%
%    \end{macrocode}
% An unknown type issues an error.
%    \begin{macrocode}
        {\PackageError{Listings}{Illegal type `\lst@delimtype'}%
                                {#2 types are #3.}}%
     \fi}
%    \end{macrocode}
% \end{macro}
%
% \begin{macro}{\lst@Delim@args}
% Now let's look how we add the arguments to |\lst@arg|. First we initialize
% the conversion just to make all characters active. But if the first character
% of the type equals |#4|, \ldots
%    \begin{macrocode}
\def\lst@Delim@args#1#2#3#4#5#6#7{%
    \begingroup
    \lst@false \let\lst@next\lst@XConvert
%    \end{macrocode}
% \ldots\ we remove that character from |\lst@delimtype|, and |#5| might select
% a different conversion setting or macro.
%    \begin{macrocode}
    \@ifnextchar #4{\xdef\lst@delimtype{\expandafter\@gobble
                                        \lst@delimtype}%
                    #5\lst@next#2\@nil
                    \lst@lAddTo\lst@arg{\@empty#6}%
                    \lst@GobbleNil}%
%    \end{macrocode}
% Since we are in the `special' case above, we've also added the special
% |\lst@Begin|\ldots\space and |\lst@End|\ldots\space macros to |\lst@arg|
% (and |\@empty| as a brake for the delimiter). No special task must be done
% if the characters are not equal.
%    \begin{macrocode}
                   {\lst@next#2\@nil
                    \lst@lAddTo\lst@arg{\@empty#3}%
                    \lst@GobbleNil}%
                 #1\@nil
%    \end{macrocode}
% We always transfer the arguments to the outside of the group and append the
% style and mode selection if and only if we're not deleting a delimiter.
% Therefor we expand the delimiter style.
%    \begin{macrocode}
    \global\let\@gtempa\lst@arg
    \endgroup
    \let\lst@arg\@gtempa
    \ifx #7\@nil\else
        \expandafter\lst@Delim@args@\expandafter{\lst@delimstyle}%
    \fi}
%    \end{macrocode}
% Recall that the style is `selected' by |\def\lst@currstyle#5|, and this
% `argument' |#5| is to be added now. Depending on the settings at the very
% beginning, we use either |{\meta{style}}\lst@modetrue|---which selects the
% style and deactivates keyword detection---, or |{}\meta{style}|---which
% defines an empty style macro and executes the style for cumulative styles---,
% or |{\meta{style}|---which just defines the style macro. Note that we have to
% use two extra group levels below: one is discarded directly by |\lst@lAddTo|
% and the other by |\lst@Delim|[|DM|]|@|\meta{type}.
%    \begin{macrocode}
\def\lst@Delim@args@#1{%
    \lst@if
        \lst@lAddTo\lst@arg{{{#1}\lst@modetrue}}%
    \else
        \ifx\lst@cumulative\@empty
            \lst@lAddTo\lst@arg{{{}#1}}%
        \else
            \lst@lAddTo\lst@arg{{{#1}}}%
        \fi
    \fi}
%    \end{macrocode}
% \end{macro}
%
% \begin{macro}{\lst@Delim@del}
% To delete a particular delimiter, we iterate down the list of delimiters and
% compare the current item with the user supplied.
%    \begin{macrocode}
\def\lst@Delim@del#1\@empty#2#3#4{%
    \ifx #2\@nil\else
        \def\lst@temp{#1\@empty#2#3}%
        \ifx\lst@temp\lst@delim\else
            \lst@lAddTo\lst@DefDelims{#1\@empty#2#3{#4}}%
        \fi
        \expandafter\lst@Delim@del
    \fi}
%    \end{macrocode}
% \end{macro}
%
% \begin{macro}{\lst@Delim@delall}
% To delete a whole class of delimiters, we first expand the control sequence
% name, init some other data, and call a submacro to do the work.
%    \begin{macrocode}
\def\lst@Delim@delall#1{%
    \begingroup
    \edef\lst@delim{\expandafter\string\csname\@lst @#1\endcsname}%
    \lst@false \global\let\@gtempa\@empty
    \expandafter\lst@Delim@delall@\lst@DefDelims\@empty
    \endgroup
    \let\lst@DefDelims\@gtempa}
%    \end{macrocode}
% We first discard a preceding |\lst@UseDynamicMode|.
%    \begin{macrocode}
\def\lst@Delim@delall@#1{%
    \ifx #1\@empty\else
        \ifx #1\lst@UseDynamicMode
            \lst@true
            \let\lst@next\lst@Delim@delall@do
        \else
            \def\lst@next{\lst@Delim@delall@do#1}%
        \fi
        \expandafter\lst@next
    \fi}
%    \end{macrocode}
% Then we can check whether (the following) |\lst@|\meta{delimiter name}\ldots\
% matches the delimiter class given by |\lst@delim|.
%    \begin{macrocode}
\def\lst@Delim@delall@do#1#2\@empty#3#4#5{%
    \expandafter\lst@IfSubstring\expandafter{\lst@delim}{\string#1}%
      {}%
      {\lst@if \lst@AddTo\@gtempa\lst@UseDynamicMode \fi
       \lst@AddTo\@gtempa{#1#2\@empty#3#4{#5}}}%
    \lst@false \lst@Delim@delall@}
%    \end{macrocode}
% \end{macro}
%
% \begin{macro}{\lst@DefDelimB}
% Here we put the arguments together to fit |\lst@CDef|. Note that the very
% last argument |\@empty| to |\lst@CDef| is a brake for |\lst@CArgEmpty|
% and |\lst@DelimOpen|.
%    \begin{macrocode}
\gdef\lst@DefDelimB#1#2#3#4#5#6#7#8{%
    \lst@CDef{#1}#2%
        {#3}%
        {\let\lst@bnext\lst@CArgEmpty
         \lst@ifmode #4\else
             #5%
             \def\lst@bnext{#6{#7}{#8}}%
         \fi
         \lst@bnext}%
        \@empty}
%    \end{macrocode}
% After a bug report from \lsthelper{Vespe~Savikko}{2000/11/06}{bad output of
% doc-strings if HTML and Python are loaded} I added braces around |#7|.
% \end{macro}
%
% \begin{macro}{\lst@DefDelimE}
% The  |\ifnum #7=\lst@mode| in the 5th line ensures that the delimiters
% match each other.
%    \begin{macrocode}
\gdef\lst@DefDelimE#1#2#3#4#5#6#7{%
    \lst@CDef{#1}#2%
        {#3}%
        {\let\lst@enext\lst@CArgEmpty
         \ifnum #7=\lst@mode%
             #4%
             \let\lst@enext#6%
         \else
             #5%
         \fi
         \lst@enext}%
        \@empty}
%    \end{macrocode}
%    \begin{macrocode}
\lst@AddToHook{Init}{\let\lst@bnext\relax \let\lst@enext\relax}
%    \end{macrocode}
% \end{macro}
%
% \begin{macro}{\lst@DefDelimBE}
% This service macro will actually define all string delimiters.
%    \begin{macrocode}
\gdef\lst@DefDelimBE#1#2#3#4#5#6#7#8#9{%
    \lst@CDef{#1}#2%
        {#3}%
        {\let\lst@bnext\lst@CArgEmpty
         \ifnum #7=\lst@mode
             #4%
             \let\lst@bnext#9%
         \else
             \lst@ifmode\else
                 #5%
                 \def\lst@bnext{#6{#7}{#8}}%
             \fi
         \fi
         \lst@bnext}%
        \@empty}
%    \end{macrocode}
% \end{macro}
%
% \begin{macro}{\lst@delimtypes}
% is the list of general delimiter types.
%    \begin{macrocode}
\gdef\lst@delimtypes{s,l}
%    \end{macrocode}
% \end{macro}
%
% \begin{macro}{\lst@DelimKey}
% We just put together the arguments for |\lst@Delim|.
%    \begin{macrocode}
\gdef\lst@DelimKey#1#2{%
    \lst@Delim{}#2\relax
        {Delim}\lst@delimtypes #1%
                {\lst@BeginDelim\lst@EndDelim}
        i\@empty{\lst@BeginIDelim\lst@EndIDelim}}
%    \end{macrocode}
% \end{macro}
%
% \begin{lstkey}{delim}
% \begin{lstkey}{moredelim}
% \begin{lstkey}{deletedelim}
% all use |\lst@DelimKey|.
%    \begin{macrocode}
\lst@Key{delim}\relax{\lst@DelimKey\@empty{#1}}
\lst@Key{moredelim}\relax{\lst@DelimKey\relax{#1}}
\lst@Key{deletedelim}\relax{\lst@DelimKey\@nil{#1}}
%    \end{macrocode}
% \end{lstkey}
% \end{lstkey}
% \end{lstkey}
%
% \begin{macro}{\lst@DelimDM@l}
% \begin{macro}{\lst@DelimDM@s}
% Nohting special here.
%    \begin{macrocode}
\gdef\lst@DelimDM@l#1#2\@empty#3#4#5{%
    \lst@CArg #2\relax\lst@DefDelimB{}{}{}#3{#1}{#5\lst@Lmodetrue}}
%    \end{macrocode}
%    \begin{macrocode}
\gdef\lst@DelimDM@s#1#2#3\@empty#4#5#6{%
    \lst@CArg #2\relax\lst@DefDelimB{}{}{}#4{#1}{#6}%
    \lst@CArg #3\relax\lst@DefDelimE{}{}{}#5{#1}}
%    \end{macrocode}
%    \begin{macrocode}
%</kernel>
%    \end{macrocode}
% \end{macro}
% \end{macro}
%
%
% \subsubsection{Strings}
%
% \begin{aspect}{strings}
% Just starting a new aspect.
%    \begin{macrocode}
%<*misc>
\lst@BeginAspect{strings}
%    \end{macrocode}
%
% \begin{macro}{\lst@stringtypes}
% is the list of \ldots\space string types?
%    \begin{macrocode}
\gdef\lst@stringtypes{d,b,m,bd,db,s}
%    \end{macrocode}
% \end{macro}
%
% \begin{macro}{\lst@StringKey}
% We just put together the arguments for |\lst@Delim|.
%    \begin{macrocode}
\gdef\lst@StringKey#1#2{%
    \lst@Delim\lst@stringstyle #2\relax
        {String}\lst@stringtypes #1%
                     {\lst@BeginString\lst@EndString}%
        \@@end\@empty{}}
%    \end{macrocode}
% \end{macro}
%
% \begin{lstkey}{string}
% \begin{lstkey}{morestring}
% \begin{lstkey}{deletestring}
% all use |\lst@StringKey|.
%    \begin{macrocode}
\lst@Key{string}\relax{\lst@StringKey\@empty{#1}}
\lst@Key{morestring}\relax{\lst@StringKey\relax{#1}}
\lst@Key{deletestring}\relax{\lst@StringKey\@nil{#1}}
%    \end{macrocode}
% \end{lstkey}
% \end{lstkey}
% \end{lstkey}
%
% \begin{lstkey}{stringstyle}
% You shouldn't need comments on the following two lines, do you?
%    \begin{macrocode}
\lst@Key{stringstyle}{}{\def\lst@stringstyle{#1}}
\lst@AddToHook{EmptyStyle}{\let\lst@stringstyle\@empty}
%    \end{macrocode}
% \end{lstkey}
%
% \begin{lstkey}{showstringspaces}
% Thanks to \lsthelper{Knut~M\"uller}{1997/04/28}{\blankstringtrue} for
% reporting problems with |\blankstringtrue| (now |showstringspaces=false|).
% The problem has gone.
%    \begin{macrocode}
\lst@Key{showstringspaces}t[t]{\lstKV@SetIf{#1}\lst@ifshowstringspaces}
%    \end{macrocode}
% \end{lstkey}
%
% \begin{macro}{\lst@BeginString}
% Note that the tokens after |\lst@DelimOpen| are arguments! The only special
% here is that we switch to `keepspaces' after starting a string, if necessary.
% A bug reported by \lsthelper{Vespe~Savikko}{2000/09/27}{stringstyle used also
% on previous other characters} has gone due to the use of |\lst@DelimOpen|.
%    \begin{macrocode}
\gdef\lst@BeginString{%
    \lst@DelimOpen
        \lst@ifexstrings\else
        {\lst@ifshowstringspaces
             \lst@keepspacestrue
             \let\lst@outputspace\lst@visiblespace
         \fi}}
%    \end{macrocode}
%    \begin{macrocode}
\lst@AddToHookExe{ExcludeDelims}{\let\lst@ifexstrings\iffalse}
%    \end{macrocode}
% \end{macro}
%
% \begin{macro}{\lst@EndString}
% Again the two tokens following |\lst@DelimClose| are arguments.
%    \begin{macrocode}
\gdef\lst@EndString{\lst@DelimClose\lst@ifexstrings\else}
%    \end{macrocode}
% \end{macro}
%
% And now all the |\lst@StringDM@|\meta{type} definitions.
%
% \begin{macro}{\lst@StringDM@d}
% `d' means no extra work.; the first three arguments after |\lst@DefDelimBE|
% are left empty. The others are used to start and end the string.
%    \begin{macrocode}
\gdef\lst@StringDM@d#1#2\@empty#3#4#5{%
    \lst@CArg #2\relax\lst@DefDelimBE{}{}{}#3{#1}{#5}#4}
%    \end{macrocode}
% \end{macro}
%
% \begin{macro}{\lst@StringDM@b}
% The |\lst@ifletter|\ldots|\fi| has been inserted after bug reports by
% \lsthelper{Daniel~Gerigk}{2001/10/25}{improper strings in C++} and
% \lsthelper{Peter~Bartke}{2001/11/01}{improper strings in C++}. If the last
% other character is a backslash (4th line), we gobble the `end string' token
% sequence.
%    \begin{macrocode}
\gdef\lst@StringDM@b#1#2\@empty#3#4#5{%
    \let\lst@ifbstring\iftrue
    \lst@CArg #2\relax\lst@DefDelimBE
       {\lst@ifletter \lst@Output \lst@letterfalse \fi}%
       {\ifx\lst@lastother\lstum@backslash
            \expandafter\@gobblethree
        \fi}{}#3{#1}{#5}#4}
%    \end{macrocode}
%    \begin{macrocode}
\global\let\lst@ifbstring\iffalse % init
%    \end{macrocode}
% \lsthelper{Heiko~Heil}{2002/02/08}{string '\\' does not finish after the
% delimiter} reported problems with double backslashes. So:
%    \begin{macrocode}
\lst@AddToHook{SelectCharTable}{%
    \lst@ifbstring
        \lst@CArgX \\\\\relax \lst@CDefX{}%
           {\lst@ProcessOther\lstum@backslash
            \lst@ProcessOther\lstum@backslash
            \let\lst@lastother\relax}%
           {}%
    \fi}
%    \end{macrocode}
% The reset of |\lst@lastother| has been added after a bug reports by
% \lsthelper{Hermann~H\"uttler}{2002/10/05}{C++-string "... \\" does not
% end with second double quote} and \lsthelper{Dan~Luecking}{2003/01/15}
% {string "\\" doesn't end after the second quote}.
% \end{macro}
%
% \begin{macro}{\lst@StringDM@bd}
% \begin{macro}{\lst@StringDM@db}
% are just the same and the same as |\lst@StringDM@b|.
%    \begin{macrocode}
\global\let\lst@StringDM@bd\lst@StringDM@b
\global\let\lst@StringDM@db\lst@StringDM@bd
%    \end{macrocode}
% \end{macro}\end{macro}
%
% \begin{macro}{\lst@StringDM@m}
% is for Matlab. We enter string mode only if the last character is not in
% the following list of exceptional characters: letters, digits, period,
% quote, right parenthesis, right bracket, and right brace. The first list
% has been extended after bug reports from \lsthelper{Christian~Kindinger}
% {2002/03/??}{]' starts a string in Matlab}, \lsthelper{Benjamin~Schubert}
% {2003/02/05}{.' starts a string in Matlab}, and \lsthelper{Stefan~Stoll}
% {2003/02/18}{any of 0123456789\}' plus quote start a string in Matlab}.
%    \begin{macrocode}
\gdef\lst@StringDM@m#1#2\@empty#3#4#5{%
    \lst@CArg #2\relax\lst@DefDelimBE{}{}%
        {\let\lst@next\@gobblethree
         \lst@ifletter\else
             \lst@IfLastOtherOneOf{)].0123456789\lstum@rbrace'}%
                 {}%
                 {\let\lst@next\@empty}%
         \fi
         \lst@next}#3{#1}{#5}#4}
%    \end{macrocode}
% \end{macro}
%
% \begin{macro}{\lst@StringDM@s}
% is for string-delimited strings, just as for comments.  This is needed
% for Ruby, and possibly other languages.
%    \begin{macrocode}
\gdef\lst@StringDM@s#1#2#3\@empty#4#5#6{%
    \lst@CArg #2\relax\lst@DefDelimB{}{}{}#4{#1}{#6}%
    \lst@CArg #3\relax\lst@DefDelimE{}{}{}#5{#1}}
%    \end{macrocode}
% \end{macro}
%
% \begin{macro}{\lstum@rbrace}
% This has been used above.
%    \begin{macrocode}
\lst@SaveOutputDef{"7D}\lstum@rbrace
%    \end{macrocode}
% \end{macro}
%
%    \begin{macrocode}
\lst@EndAspect
%</misc>
%    \end{macrocode}
% \end{aspect}
%
%
% \begin{aspect}{mf}
% For MetaFont and MetaPost we now define macros to print the input-filenames
% in stringstyle.
%    \begin{macrocode}
%<*misc>
\lst@BeginAspect{mf}
%    \end{macrocode}
%
% \begin{macro}{\lst@mfinputmode}
% \begin{macro}{\lst@String@mf}
%    \begin{macrocode}
\lst@AddTo\lst@stringtypes{,mf}
\lst@NewMode\lst@mfinputmode
%    \end{macrocode}
%    \begin{macrocode}
\gdef\lst@String@mf#1\@empty#2#3#4{%
  \lst@CArg #1\relax\lst@DefDelimB
       {}{}{\lst@ifletter \expandafter\@gobblethree \fi}%
       \lst@BeginStringMFinput\lst@mfinputmode{#4\lst@Lmodetrue}%
  \@ifundefined{lsts@semicolon}%
  {\lst@DefSaveDef{`\;}\lsts@semicolon{% ; and space end the filename
      \ifnum\lst@mode=\lst@mfinputmode
          \lst@XPrintToken
          \expandafter\lst@LeaveMode
      \fi
      \lsts@semicolon}%
   \lst@DefSaveDef{`\ }\lsts@space{%
      \ifnum\lst@mode=\lst@mfinputmode
          \lst@XPrintToken
          \expandafter\lst@LeaveMode
      \fi
      \lsts@space}%
  }{}}
%    \end{macrocode}
% \end{macro}
% \end{macro}
%
% \begin{macro}{\lst@BeginStringMFinput}
% It remains to define this macro. In contrast to |\lst@PrintDelim|, we don't
% use |\lst@modetrue| to allow keyword detection here.
%    \begin{macrocode}
\gdef\lst@BeginStringMFinput#1#2#3\@empty{%
    \lst@TrackNewLines \lst@XPrintToken
      \begingroup
        \lst@mode\lst@nomode
        #3\lst@XPrintToken
      \endgroup
      \lst@ResetToken
    \lst@EnterMode{#1}{\def\lst@currstyle#2}%
    \lst@ifshowstringspaces
         \lst@keepspacestrue
         \let\lst@outputspace\lst@visiblespace
    \fi}
%    \end{macrocode}
% \end{macro}
%
%    \begin{macrocode}
\lst@EndAspect
%</misc>
%    \end{macrocode}
% \end{aspect}
%
%
% \subsubsection{Comments}
%
% \begin{aspect}{comments}
% That's what we are working on.
%    \begin{macrocode}
%<*misc>
\lst@BeginAspect{comments}
%    \end{macrocode}
%
% \begin{macro}{\lst@commentmode}
% is a general purpose mode for comments.
%    \begin{macrocode}
\lst@NewMode\lst@commentmode
%    \end{macrocode}
% \end{macro}
%
% \begin{macro}{\lst@commenttypes}
% Via \keyname{comment} available comment types: \textbf line, \textbf fixed
% column, \textbf single, and \textbf nested and all with
% preceding \textbf i for invisible comments.
%    \begin{macrocode}
\gdef\lst@commenttypes{l,f,s,n}
%    \end{macrocode}
% \end{macro}
%
% \begin{macro}{\lst@CommentKey}
% We just put together the arguments for |\lst@Delim|.
%    \begin{macrocode}
\gdef\lst@CommentKey#1#2{%
    \lst@Delim\lst@commentstyle #2\relax
        {Comment}\lst@commenttypes #1%
                {\lst@BeginComment\lst@EndComment}%
        i\@empty{\lst@BeginInvisible\lst@EndInvisible}}
%    \end{macrocode}
% \end{macro}
%
% \begin{lstkey}{comment}
% \begin{lstkey}{morecomment}
% \begin{lstkey}{deletecomment}
% The keys are easy since defined in terms of |\lst@CommentKey|.
%    \begin{macrocode}
\lst@Key{comment}\relax{\lst@CommentKey\@empty{#1}}
\lst@Key{morecomment}\relax{\lst@CommentKey\relax{#1}}
\lst@Key{deletecomment}\relax{\lst@CommentKey\@nil{#1}}
%    \end{macrocode}
% \end{lstkey}
% \end{lstkey}
% \end{lstkey}
%
% \begin{lstkey}{commentstyle}
% Any hints necessary?
%    \begin{macrocode}
\lst@Key{commentstyle}{}{\def\lst@commentstyle{#1}}
\lst@AddToHook{EmptyStyle}{\let\lst@commentstyle\itshape}
%    \end{macrocode}
% \end{lstkey}
%
% \begin{macro}{\lst@BeginComment}
% \begin{macro}{\lst@EndComment}
% Once more the three tokens following |\lst@DelimOpen| are arguments.
%    \begin{macrocode}
\gdef\lst@BeginComment{%
    \lst@DelimOpen
        \lst@ifexcomments\else
        \lsthk@AfterBeginComment}
%    \end{macrocode}
% Ditto.
%    \begin{macrocode}
\gdef\lst@EndComment{\lst@DelimClose\lst@ifexcomments\else}
%    \end{macrocode}
%    \begin{macrocode}
\lst@AddToHook{AfterBeginComment}{}
\lst@AddToHookExe{ExcludeDelims}{\let\lst@ifexcomments\iffalse}
%    \end{macrocode}
% \end{macro}
% \end{macro}
%
% \begin{macro}{\lst@BeginInvisible}
% \begin{macro}{\lst@EndInvisible}
% Print preceding characters and begin dropping the output.
%    \begin{macrocode}
\gdef\lst@BeginInvisible#1#2#3\@empty{%
    \lst@TrackNewLines \lst@XPrintToken
    \lst@BeginDropOutput{#1}}
%    \end{macrocode}
% Don't print the delimiter and end dropping the output.
%    \begin{macrocode}
\gdef\lst@EndInvisible#1\@empty{\lst@EndDropOutput}
%    \end{macrocode}
% \end{macro}
% \end{macro}
%
% Now we provide all |\lst@Comment|[|DM|]|@|\meta{type} macros.
%
% \begin{macro}{\lst@CommentDM@l}
% is easy---thanks to |\lst@CArg| and |\lst@DefDelimB|. Note that the
% `end comment' argument |#4| is not used here.
%    \begin{macrocode}
\gdef\lst@CommentDM@l#1#2\@empty#3#4#5{%
    \lst@CArg #2\relax\lst@DefDelimB{}{}{}#3{#1}{#5\lst@Lmodetrue}}
%    \end{macrocode}
% \end{macro}
%
% \begin{macro}{\lst@CommentDM@f}
% is slightly more work. First we provide the number of preceding columns.
%    \begin{macrocode}
\gdef\lst@CommentDM@f#1{%
    \@ifnextchar[{\lst@Comment@@f{#1}}%
                 {\lst@Comment@@f{#1}[0]}}
%    \end{macrocode}
% We define the comment in the same way as above, but we enter comment mode
% if and only if the character is in column |#2| (counting from zero).
%    \begin{macrocode}
\gdef\lst@Comment@@f#1[#2]#3\@empty#4#5#6{%
    \lst@CArg #3\relax\lst@DefDelimB{}{}%
        {\lst@CalcColumn
         \ifnum #2=\@tempcnta\else
             \expandafter\@gobblethree
         \fi}%
        #4{#1}{#6\lst@Lmodetrue}}
%    \end{macrocode}
% \end{macro}
%
% \begin{macro}{\lst@CommentDM@s}
% Nothing special here.
%    \begin{macrocode}
\gdef\lst@CommentDM@s#1#2#3\@empty#4#5#6{%
    \lst@CArg #2\relax\lst@DefDelimB{}{}{}#4{#1}{#6}%
    \lst@CArg #3\relax\lst@DefDelimE{}{}{}#5{#1}}
%    \end{macrocode}
% \end{macro}
%
% \begin{macro}{\lst@CommentDM@n}
% We either give an error message or define the nested comment.
%    \begin{macrocode}
\gdef\lst@CommentDM@n#1#2#3\@empty#4#5#6{%
    \ifx\@empty#3\@empty\else
        \def\@tempa{#2}\def\@tempb{#3}%
        \ifx\@tempa\@tempb
            \PackageError{Listings}{Identical delimiters}%
            {These delimiters make no sense with nested comments.}%
        \else
            \lst@CArg #2\relax\lst@DefDelimB
                {}%
%    \end{macrocode}
% Note that the following |\@gobble| eats an |\else| from |\lst@DefDelimB|.
%    \begin{macrocode}
                {\ifnum\lst@mode=#1\relax \expandafter\@gobble \fi}%
                {}#4{#1}{#6}%
            \lst@CArg #3\relax\lst@DefDelimE{}{}{}#5{#1}%
        \fi
    \fi}
%    \end{macrocode}
% \end{macro}
%
%    \begin{macrocode}
\lst@EndAspect
%</misc>
%    \end{macrocode}
% \end{aspect}
%
%
% \subsubsection{PODs}
%
% \begin{aspect}{pod}
% PODs are defined as a separate aspect.
%    \begin{macrocode}
%<*misc>
\lst@BeginAspect{pod}
%    \end{macrocode}
%
% \begin{lstkey}{printpod}
% \begin{lstkey}{podcomment}
% We begin with the user keys, which I introduced after communication with
% \lsthelper{Michael~Piotrowski}{1997/11/11}{printpod}.
%    \begin{macrocode}
\lst@Key{printpod}{false}[t]{\lstKV@SetIf{#1}\lst@ifprintpod}
\lst@Key{podcomment}{false}[t]{\lstKV@SetIf{#1}\lst@ifpodcomment}
\lst@AddToHookExe{SetLanguage}{\let\lst@ifpodcomment\iffalse}
%    \end{macrocode}
% \end{lstkey}
% \end{lstkey}
%
% \begin{macro}{\lst@PODmode}
% is the static mode for PODs.
%    \begin{macrocode}
\lst@NewMode\lst@PODmode
%    \end{macrocode}
% \end{macro}
%
% We adjust some characters if the user has selected |podcomment=true|.
%    \begin{macrocode}
\lst@AddToHook{SelectCharTable}
    {\lst@ifpodcomment
         \lst@CArgX =\relax\lst@DefDelimB{}{}%
%    \end{macrocode}
% The following code is executed if we've found an equality sign and haven't
% entered a mode (in fact if mode changes are allowed): We `begin drop output'
% and gobble the usual begin of comment sequence (via |\@gobblethree|) if PODs
% aren't be printed. Moreover we gobble it if the current column number is not
% zero---|\@tempcnta| is valued below.
%    \begin{macrocode}
           {\ifnum\@tempcnta=\z@
                \lst@ifprintpod\else
                    \def\lst@bnext{\lst@BeginDropOutput\lst@PODmode}%
                    \expandafter\expandafter\expandafter\@gobblethree
                \fi
            \else
               \expandafter\@gobblethree
            \fi}%
           \lst@BeginComment\lst@PODmode{{\lst@commentstyle}}%
%    \end{macrocode}
% If we come to |=|, we calculate the current column number (zero based).
%    \begin{macrocode}
         \lst@CArgX =cut\^^M\relax\lst@DefDelimE
           {\lst@CalcColumn}%
%    \end{macrocode}
% If there is additionally |cut|+EOL and if we are in |\lst@PODmode| but not in
% column one, we must gobble the `end comment sequence'.
%    \begin{macrocode}
           {\ifnum\@tempcnta=\z@\else
                \expandafter\@gobblethree
            \fi}%
           {}%
           \lst@EndComment\lst@PODmode
     \fi}
%    \end{macrocode}
%
%    \begin{macrocode}
\lst@EndAspect
%</misc>
%    \end{macrocode}
% \end{aspect}
%
%
% \subsubsection{Tags}
%
% \begin{aspect}{html}
% Support for HTML and other `markup languages'.
%    \begin{macrocode}
%<*misc>
\lst@BeginAspect[keywords]{html}
%    \end{macrocode}
%
% \begin{macro}{\lst@tagtypes}
% Again we begin with the list of tag types. It's rather short.
%    \begin{macrocode}
\gdef\lst@tagtypes{s}
%    \end{macrocode}
% \end{macro}
%
% \begin{macro}{\lst@TagKey}
% Again we just put together the arguments for |\lst@Delim| and \ldots
%    \begin{macrocode}
\gdef\lst@TagKey#1#2{%
    \lst@Delim\lst@tagstyle #2\relax
        {Tag}\lst@tagtypes #1%
                     {\lst@BeginTag\lst@EndTag}%
        \@@end\@empty{}}
%    \end{macrocode}
% \end{macro}
%
% \begin{lstkey}{tag}
% \ldots\ we use the definition here.
%    \begin{macrocode}
\lst@Key{tag}\relax{\lst@TagKey\@empty{#1}}
%    \end{macrocode}
% \end{lstkey}
%
% \begin{lstkey}{tagstyle}
% You shouldn't need comments on the following two lines, do you?
%    \begin{macrocode}
\lst@Key{tagstyle}{}{\def\lst@tagstyle{#1}}
\lst@AddToHook{EmptyStyle}{\let\lst@tagstyle\@empty}
%    \end{macrocode}
% \end{lstkey}
%
% \begin{macro}{\lst@BeginTag}
% The special things here are: (1) We activate keyword detection inside tags
% and (2) we initialize the switch |\lst@iffirstintag| if necessary.
%    \begin{macrocode}
\gdef\lst@BeginTag{%
    \lst@DelimOpen
        \lst@ifextags\else
        {\let\lst@ifkeywords\iftrue
         \lst@ifmarkfirstintag \lst@firstintagtrue \fi}}
%    \end{macrocode}
%    \begin{macrocode}
\lst@AddToHookExe{ExcludeDelims}{\let\lst@ifextags\iffalse}
%    \end{macrocode}
% \end{macro}
%
% \begin{macro}{\lst@EndTag}
% is just like the other |\lst@End|\meta{whatever} definitions.
%    \begin{macrocode}
\gdef\lst@EndTag{\lst@DelimClose\lst@ifextags\else}
%    \end{macrocode}
% \end{macro}
%
% \begin{lstkey}{usekeywordsintag}
% \begin{lstkey}{markfirstintag}
% The second key has already been `used'.
%    \begin{macrocode}
\lst@Key{usekeywordsintag}t[t]{\lstKV@SetIf{#1}\lst@ifusekeysintag}
\lst@Key{markfirstintag}f[t]{\lstKV@SetIf{#1}\lst@ifmarkfirstintag}
%    \end{macrocode}
% For this, we install a (global) switch, \ldots
%    \begin{macrocode}
\gdef\lst@firstintagtrue{\global\let\lst@iffirstintag\iftrue}
\global\let\lst@iffirstintag\iffalse
%    \end{macrocode}
% \ldots\ which is reset by the output of an identifier but not by other
% output.
%    \begin{macrocode}
\lst@AddToHook{PostOutput}{\lst@tagresetfirst}
\lst@AddToHook{Output}
    {\gdef\lst@tagresetfirst{\global\let\lst@iffirstintag\iffalse}}
\lst@AddToHook{OutputOther}{\gdef\lst@tagresetfirst{}}
%    \end{macrocode}
% Now we only need to test against this switch in the \hookname{Output} hook.
%    \begin{macrocode}
\lst@AddToHook{Output}
    {\ifnum\lst@mode=\lst@tagmode
         \lst@iffirstintag \let\lst@thestyle\lst@gkeywords@sty \fi
%    \end{macrocode}
% Moreover we check here, whether the keyword style is always to be used.
%    \begin{macrocode}
         \lst@ifusekeysintag\else \let\lst@thestyle\lst@gkeywords@sty\fi
     \fi}
%    \end{macrocode}
% \end{lstkey}
% \end{lstkey}
%
% \begin{macro}{\lst@tagmode}
% We allocate the mode and \ldots
%    \begin{macrocode}
\lst@NewMode\lst@tagmode
%    \end{macrocode}
% deactivate keyword detection if any tag delimiter is defined (see below).
%    \begin{macrocode}
\lst@AddToHook{Init}{\global\let\lst@ifnotag\iftrue}
\lst@AddToHook{SelectCharTable}{\let\lst@ifkeywords\lst@ifnotag}
%    \end{macrocode}
% \end{macro}
%
% \begin{macro}{\lst@Tag@s}
% The definition of the one and only delimiter type is not that interesting.
% Compared with the others we set |\lst@ifnotag| and enter tag mode only if
% we aren't in tag mode.
%    \begin{macrocode}
\gdef\lst@Tag@s#1#2\@empty#3#4#5{%
    \global\let\lst@ifnotag\iffalse
    \lst@CArg #1\relax\lst@DefDelimB {}{}%
        {\ifnum\lst@mode=\lst@tagmode \expandafter\@gobblethree \fi}%
        #3\lst@tagmode{#5}%
    \lst@CArg #2\relax\lst@DefDelimE {}{}{}#4\lst@tagmode}%
%    \end{macrocode}
% \end{macro}
%
% \begin{macro}{\lst@BeginCDATA}
% This macro is used by the XML language definition.
%    \begin{macrocode}
\gdef\lst@BeginCDATA#1\@empty{%
    \lst@TrackNewLines \lst@PrintToken
    \lst@EnterMode\lst@GPmode{}\let\lst@ifmode\iffalse
    \lst@mode\lst@tagmode #1\lst@mode\lst@GPmode\relax\lst@modetrue}
%    \end{macrocode}
% \end{macro}
%
%    \begin{macrocode}
\lst@EndAspect
%</misc>
%    \end{macrocode}
% \end{aspect}
%
%
% \subsection{Replacing input}
%
% \begingroup
%    \begin{macrocode}
%<*kernel>
%    \end{macrocode}
% \endgroup
%
% \begin{macro}{\lst@ReplaceInput}
% is defined in terms of |\lst@CArgX| and |\lst@CDefX|.
%    \begin{macrocode}
\def\lst@ReplaceInput#1{\lst@CArgX #1\relax\lst@CDefX{}{}}
%    \end{macrocode}
% \end{macro}
%
% \begin{lstkey}{literate}
% \lsthelper{Jason~Alexander}{1999/03/10}{literate programming} asked for
% something like that. The key looks for a star and saves the argument.
%    \begin{macrocode}
\def\lst@Literatekey#1\@nil@{\let\lst@ifxliterate\lst@if
                             \def\lst@literate{#1}}
\lst@Key{literate}{}{\@ifstar{\lst@true \lst@Literatekey}
                             {\lst@false\lst@Literatekey}#1\@nil@}
\lst@AddToHook{SelectCharTable}
    {\ifx\lst@literate\@empty\else
         \expandafter\lst@Literate\lst@literate{}\relax\z@
     \fi}
%    \end{macrocode}
% Internally we don't make use of the `replace input' feature any more.
%^^A We print the preceding text, assign token and length, and output it.
%    \begin{macrocode}
\def\lst@Literate#1#2#3{%
    \ifx\relax#2\@empty\else
        \lst@CArgX #1\relax\lst@CDef
            {}
            {\let\lst@next\@empty
             \lst@ifxliterate
                \lst@ifmode \let\lst@next\lst@CArgEmpty \fi
             \fi
             \ifx\lst@next\@empty
                 \ifx\lst@OutputBox\@gobble\else
                   \lst@XPrintToken \let\lst@scanmode\lst@scan@m
                   \lst@token{#2}\lst@length#3\relax
                   \lst@XPrintToken
                 \fi
                 \let\lst@next\lst@CArgEmptyGobble
             \fi
             \lst@next}%
            \@empty
        \expandafter\lst@Literate
    \fi}
\def\lst@CArgEmptyGobble#1\@empty{}
%    \end{macrocode}
% Note that we check |\lst@OutputBox| for being |\@gobble|. This is due to
% a bug report by \lsthelper{Jared~Warren}{2003/07/10}{literate replacement
% produces "ghosts"}.
% \end{lstkey}
%
% \begin{macro}{\lst@BeginDropInput}
% We deactivate all `process' macros. |\lst@modetrue| does this for all
% up-coming string delimiters, comments, and so on.
%    \begin{macrocode}
\def\lst@BeginDropInput#1{%
    \lst@EnterMode{#1}%
    {\lst@modetrue
     \let\lst@OutputBox\@gobble
     \let\lst@ifdropinput\iftrue
     \let\lst@ProcessLetter\@gobble
     \let\lst@ProcessDigit\@gobble
     \let\lst@ProcessOther\@gobble
     \let\lst@ProcessSpace\@empty
     \let\lst@ProcessTabulator\@empty
     \let\lst@ProcessFormFeed\@empty}}
\let\lst@ifdropinput\iffalse % init
%    \end{macrocode}
% \end{macro}
%
% \begingroup
%    \begin{macrocode}
%</kernel>
%    \end{macrocode}
% \endgroup
%
%
% \subsection{Escaping to \LaTeX}
%
% \begin{aspect}{escape}
% We now define the \ldots\ damned \ldots\ the aspect has escaped!
%    \begin{macrocode}
%<*misc>
\lst@BeginAspect{escape}
%    \end{macrocode}
%
% \begin{lstkey}{texcl}
% Communication with \lsthelper{J\"orn~Wilms}{1997/07/07}{\TeX\ comments} is
% responsible for this key. The definition and the first hooks are easy.
%    \begin{macrocode}
\lst@Key{texcl}{false}[t]{\lstKV@SetIf{#1}\lst@iftexcl}
\lst@AddToHook{TextStyle}{\let\lst@iftexcl\iffalse}
\lst@AddToHook{EOL}
    {\ifnum\lst@mode=\lst@TeXLmode
         \expandafter\lst@escapeend
         \expandafter\lst@LeaveAllModes
         \expandafter\lst@ReenterModes
     \fi}
%    \end{macrocode}
% If the user wants \TeX\ comment lines, we print the comment separator and
% interrupt the normal processing.
%    \begin{macrocode}
\lst@AddToHook{AfterBeginComment}
    {\lst@iftexcl \lst@ifLmode \lst@ifdropinput\else
         \lst@PrintToken
         \lst@LeaveMode \lst@InterruptModes
         \lst@EnterMode{\lst@TeXLmode}{\lst@modetrue\lst@commentstyle}%
         \expandafter\expandafter\expandafter\lst@escapebegin
     \fi \fi \fi}
%    \end{macrocode}
%    \begin{macrocode}
\lst@NewMode\lst@TeXLmode
%    \end{macrocode}
% \end{lstkey}
%
% \begin{macro}{\lst@ActiveCDefX}
% Same as |\lst@CDefX| but we both make |#1| active and assign a new catcode.
%    \begin{macrocode}
\gdef\lst@ActiveCDefX#1{\lst@ActiveCDefX@#1}
\gdef\lst@ActiveCDefX@#1#2#3{
    \catcode`#1\active\lccode`\~=`#1%
    \lowercase{\lst@CDefIt~}{#2}{#3}{}}
%    \end{macrocode}
% \end{macro}
%
% \begin{macro}{\lst@Escape}
% gets four arguments all in all. The first and second are the `begin' and
% `end' escape sequences, the third is executed when the escape starts, and the
% fourth right before ending it. We use the same mechanism as for \TeX\ comment
% lines. The |\lst@ifdropinput| test has been added after a bug report by
% \lsthelper{Michael~Weber}{2002/03/26}{escape on lines < firstline corrupts
% output}.  The |\lst@newlines\z@| was added due to a bug report by
% \lsthelper{Frank~Atanassow}{2004/10/07}{space after mathescape is not
% preserved}.
%    \begin{macrocode}
\gdef\lst@Escape#1#2#3#4{%
    \lst@CArgX #1\relax\lst@CDefX
        {}%
        {\lst@ifdropinput\else
         \lst@TrackNewLines\lst@OutputLostSpace \lst@XPrintToken
         \lst@InterruptModes
         \lst@EnterMode{\lst@TeXmode}{\lst@modetrue}%
%    \end{macrocode}
% Now we must define the character sequence to end the escape.
%    \begin{macrocode}
         \ifx\^^M#2%
             \lst@CArg #2\relax\lst@ActiveCDefX
                 {}%
                 {\lst@escapeend #4\lst@LeaveAllModes\lst@ReenterModes}%
                 {\lst@MProcessListing}%
         \else
             \lst@CArg #2\relax\lst@ActiveCDefX
                 {}%
                 {\lst@escapeend #4\lst@LeaveAllModes\lst@ReenterModes
                  \lst@newlines\z@ \lst@whitespacefalse}%
                 {}%
         \fi
         #3\lst@escapebegin
         \fi}%
        {}}
%    \end{macrocode}
% The |\lst@whitespacefalse| above was added after a bug report from
% \lsthelper{Martin~Steffen}{2001/04/07}{mathescape drops subsequent space}.
%    \begin{macrocode}
\lst@NewMode\lst@TeXmode
%    \end{macrocode}
% \end{macro}
%
% \begin{lstkey}{escapebegin}
% \begin{lstkey}{escapeend}
% The keys simply store the arguments.
%    \begin{macrocode}
\lst@Key{escapebegin}{}{\def\lst@escapebegin{#1}}
\lst@Key{escapeend}{}{\def\lst@escapeend{#1}}
%    \end{macrocode}
% \end{lstkey}
% \end{lstkey}
%
% \begin{lstkey}{escapechar}
% The introduction of this key is due to a communication with \lsthelper
% {Rui~Oliveira}{1998/06/05}{escape characters}. We define |\lst@DefEsc| and
% execute it after selecting the standard character table.
%    \begin{macrocode}
\lst@Key{escapechar}{}
    {\ifx\@empty#1\@empty
         \let\lst@DefEsc\relax
     \else
         \def\lst@DefEsc{\lst@Escape{#1}{#1}{}{}}%
     \fi}
\lst@AddToHook{TextStyle}{\let\lst@DefEsc\@empty}
\lst@AddToHook{SelectCharTable}{\lst@DefEsc}
%    \end{macrocode}
% \end{lstkey}
%
% \begin{lstkey}{escapeinside}
% Nearly the same.
%    \begin{macrocode}
\lst@Key{escapeinside}{}{\lstKV@TwoArg{#1}%
    {\let\lst@DefEsc\@empty
     \ifx\@empty##1@empty\else \ifx\@empty##2\@empty\else
         \def\lst@DefEsc{\lst@Escape{##1}{##2}{}{}}%
     \fi\fi}}
%    \end{macrocode}
% \end{lstkey}
%
% \begin{lstkey}{mathescape}
% This is a switch and checked after character table selection. We use
% |\lst@Escape| with math shifts as arguments, but all inside |\hbox|
% to determine the correct width.
%    \begin{macrocode}
\lst@Key{mathescape}{false}[t]{\lstKV@SetIf{#1}\lst@ifmathescape}
\lst@AddToHook{SelectCharTable}
    {\lst@ifmathescape \lst@Escape{\$}{\$}%
        {\setbox\@tempboxa=\hbox\bgroup$}%
        {$\egroup \lst@CalcLostSpaceAndOutput}\fi}
%    \end{macrocode}
% \end{lstkey}
%
%    \begin{macrocode}
\lst@EndAspect
%</misc>
%    \end{macrocode}
% \end{aspect}
%
%
% \section{Keywords}
%
%
% \subsection{Making tests}\label{iMakingTests}
%
% \begin{aspect}{keywords}
% We begin a new and very important aspect.
% First of all we need to initialize some variables in order to work around a
% bug reported by \lsthelper{Beat~Birkhofer}{2001/06/15}{savemem doesn't work}.
%    \begin{macrocode}
%<*misc>
\lst@BeginAspect{keywords}
%    \end{macrocode}
%    \begin{macrocode}
\global\let\lst@ifsensitive\iftrue % init
\global\let\lst@ifsensitivedefed\iffalse % init % \global
%    \end{macrocode}
% All keyword tests take the following three arguments.
% \begin{macroargs}
% \item \meta{prefix}
% \item |\lst@|\meta{name}|@list| (a list of macros which contain the keywords)
% \item |\lst@g|\meta{name}|@sty| (global style macro)
% \end{macroargs}
% We begin with non memory-saving tests.
% \begingroup
%    \begin{macrocode}
\lst@ifsavemem\else
%    \end{macrocode}
% \endgroup
%
% \begin{macro}{\lst@KeywordTest}
% Fast keyword tests take advance of the |\lst@UM| construction in section
% \ref{iCharacterTables}. If |\lst@UM| is empty, all `use macro' characters
% expand to their original characters. Since |\lst|\meta{prefix}|@|\meta{keyword}
% will be equivalent to the appropriate style, we only need to build the control
% sequence |\lst|\meta{prefix}|@|\meta{current token} and assign it to
% |\lst@thestyle|.
%    \begin{macrocode}
\gdef\lst@KeywordTest#1#2#3{%
    \begingroup \let\lst@UM\@empty
    \global\expandafter\let\expandafter\@gtempa
        \csname\@lst#1@\the\lst@token\endcsname
    \endgroup
    \ifx\@gtempa\relax\else
        \let\lst@thestyle\@gtempa
    \fi}
%    \end{macrocode}
% Note that we need neither |#2| nor |#3| here.
% \end{macro}
%
% \begin{macro}{\lst@KEYWORDTEST}
% Case insensitive tests make the current character string upper case and give
% it to a submacro similar to |\lst@KeywordTest|.
%    \begin{macrocode}
\gdef\lst@KEYWORDTEST{%
    \uppercase\expandafter{\expandafter
        \lst@KEYWORDTEST@\the\lst@token}\relax}
\gdef\lst@KEYWORDTEST@#1\relax#2#3#4{%
    \begingroup \let\lst@UM\@empty
    \global\expandafter\let\expandafter\@gtempa
        \csname\@lst#2@#1\endcsname
    \endgroup
    \ifx\@gtempa\relax\else
        \let\lst@thestyle\@gtempa
    \fi}
%    \end{macrocode}
% \end{macro}
%
% \begin{macro}{\lst@WorkingTest}
% \begin{macro}{\lst@WORKINGTEST}
% The same except that |\lst|\meta{prefix}|@|\meta{current token} might be
% a working procedure; it is executed.
%    \begin{macrocode}
\gdef\lst@WorkingTest#1#2#3{%
    \begingroup \let\lst@UM\@empty
    \global\expandafter\let\expandafter\@gtempa
        \csname\@lst#1@\the\lst@token\endcsname
    \endgroup
    \@gtempa}
%    \end{macrocode}
%    \begin{macrocode}
\gdef\lst@WORKINGTEST{%
    \uppercase\expandafter{\expandafter
        \lst@WORKINGTEST@\the\lst@token}\relax}
\gdef\lst@WORKINGTEST@#1\relax#2#3#4{%
    \begingroup \let\lst@UM\@empty
    \global\expandafter\let\expandafter\@gtempa
        \csname\@lst#2@#1\endcsname
    \endgroup
    \@gtempa}
%    \end{macrocode}
% \end{macro}
% \end{macro}
%
% \begin{macro}{\lst@DefineKeywords}
% Eventually we need macros which define and undefine
% |\lst|\meta{prefix}|@|\meta{keyword}. Here the arguments are
% \begin{macroargs}
% \item \meta{prefix}
% \item |\lst@|\meta{name} (a keyword list)
% \item |\lst@g|\meta{name}|@sty|
% \end{macroargs}
% We make the keywords upper case if necessary, \ldots
%    \begin{macrocode}
\gdef\lst@DefineKeywords#1#2#3{%
    \lst@ifsensitive
        \def\lst@next{\lst@for#2}%
    \else
        \def\lst@next{\uppercase\expandafter{\expandafter\lst@for#2}}%
    \fi
    \lst@next\do
%    \end{macrocode}
% \ldots\space iterate through the list, and make
% |\lst|\meta{prefix}|@|\meta{keyword} (if undefined) equivalent to
% |\lst@g|\meta{name}|@sty| which is possibly a working macro.
%    \begin{macrocode}
    {\expandafter\ifx\csname\@lst#1@##1\endcsname\relax
        \global\expandafter\let\csname\@lst#1@##1\endcsname#3%
     \fi}}
%    \end{macrocode}
% \end{macro}
%
% \begin{macro}{\lst@UndefineKeywords}
% We make the keywords upper case if necessary, \ldots
%    \begin{macrocode}
\gdef\lst@UndefineKeywords#1#2#3{%
    \lst@ifsensitivedefed
        \def\lst@next{\lst@for#2}%
    \else
        \def\lst@next{\uppercase\expandafter{\expandafter\lst@for#2}}%
    \fi
    \lst@next\do
%    \end{macrocode}
% \ldots\space iterate through the list, and `undefine'
% |\lst|\meta{prefix}|@|\meta{keyword} if it's equivalent to
% |\lst@g|\meta{name}|@sty|.
%    \begin{macrocode}
    {\expandafter\ifx\csname\@lst#1@##1\endcsname#3%
        \global\expandafter\let\csname\@lst#1@##1\endcsname\relax
     \fi}}
%    \end{macrocode}
% Thanks to \lsthelper{Magnus~Lewis-Smith}{1999/09/08}{keywords do not
% undefine} a wrong |#2| in the replacement text could be changed to |#3|.
% \end{macro}
%
% \begingroup
% And now memory-saving tests.
%    \begin{macrocode}
\fi
\lst@ifsavemem
%    \end{macrocode}
% \endgroup
%
% \begin{macro}{\lst@IfOneOutOf}
% The definition here is similar to |\lst@IfOneOf|, but its second argument
% is a |\lst@|\meta{name}|@list|. Therefore we test a list of macros here.
%    \begin{macrocode}
\gdef\lst@IfOneOutOf#1\relax#2{%
    \def\lst@temp##1,#1,##2##3\relax{%
        \ifx\@empty##2\else \expandafter\lst@IOOOfirst \fi}%
    \def\lst@next{\lst@IfOneOutOf@#1\relax}%
    \expandafter\lst@next#2\relax\relax}
%    \end{macrocode}
% We either execute the \meta{else} part or make the next test.
%    \begin{macrocode}
\gdef\lst@IfOneOutOf@#1\relax#2#3{%
    \ifx#2\relax
        \expandafter\@secondoftwo
    \else
        \expandafter\lst@temp\expandafter,#2,#1,\@empty\relax
        \expandafter\lst@next
    \fi}
\ifx\iffalse\else\fi
\gdef\lst@IOOOfirst#1\relax#2#3{\fi#2}
%    \end{macrocode}
% The line |\ifx\iffalse\else\fi| balances the |\fi| inside |\lst@IOOOfirst|.
% \end{macro}
%
% \begin{macro}{\lst@IFONEOUTOF}
% As in |\lst@IFONEOF| we need two |\uppercase|s here.
%    \begin{macrocode}
\gdef\lst@IFONEOUTOF#1\relax#2{%
    \uppercase{\def\lst@temp##1,#1},##2##3\relax{%
        \ifx\@empty##2\else \expandafter\lst@IOOOfirst \fi}%
    \def\lst@next{\lst@IFONEOUTOF@#1\relax}%
    \expandafter\lst@next#2\relax}
\gdef\lst@IFONEOUTOF@#1\relax#2#3{%
    \ifx#2\relax
        \expandafter\@secondoftwo
    \else
        \uppercase
            {\expandafter\lst@temp\expandafter,#2,#1,\@empty\relax}%
        \expandafter\lst@next
    \fi}
%    \end{macrocode}
% Note: The third last line uses the fact that keyword lists (not the list
% of keyword lists) are already made upper case if keywords are insensitive.
% \end{macro}
%
% \begin{macro}{\lst@KWTest}
% is a helper for the keyword and working identifier tests. We expand the
% token and call |\lst@IfOneOf|. The tests below will append appropriate
% \meta{then} and \meta{else} arguments.
%    \begin{macrocode}
\gdef\lst@KWTest{%
    \begingroup \let\lst@UM\@empty
    \expandafter\xdef\expandafter\@gtempa\expandafter{\the\lst@token}%
    \endgroup
    \expandafter\lst@IfOneOutOf\@gtempa\relax}
%    \end{macrocode}
% \end{macro}
%
% \begin{macro}{\lst@KeywordTest}
% \begin{macro}{\lst@KEYWORDTEST}
% are fairly easy now. Note that we don't need |#1|=\meta{prefix} here.
%    \begin{macrocode}
\gdef\lst@KeywordTest#1#2#3{\lst@KWTest #2{\let\lst@thestyle#3}{}}
\global\let\lst@KEYWORDTEST\lst@KeywordTest
%    \end{macrocode}
% For case insensitive tests we assign the insensitive version to
% |\lst@IfOneOutOf|. Thus we need no extra definition here.
% \end{macro}
% \end{macro}
%
% \begin{macro}{\lst@WorkingTest}
% \begin{macro}{\lst@WORKINGTEST}
% Ditto.
%    \begin{macrocode}
\gdef\lst@WorkingTest#1#2#3{\lst@KWTest #2#3{}}
\global\let\lst@WORKINGTEST\lst@WorkingTest
%    \end{macrocode}
% \end{macro}
% \end{macro}
%
% \begingroup
%    \begin{macrocode}
\fi
%    \end{macrocode}
% \endgroup
%
% \begin{lstkey}{sensitive}
% is a switch, preset \texttt{true} every language selection.
%    \begin{macrocode}
\lst@Key{sensitive}\relax[t]{\lstKV@SetIf{#1}\lst@ifsensitive}
\lst@AddToHook{SetLanguage}{\let\lst@ifsensitive\iftrue}
%    \end{macrocode}
% We select case insensitive definitions if necessary.
%    \begin{macrocode}
\lst@AddToHook{Init}
    {\lst@ifsensitive\else
         \let\lst@KeywordTest\lst@KEYWORDTEST
         \let\lst@WorkingTest\lst@WORKINGTEST
         \let\lst@IfOneOutOf\lst@IFONEOUTOF
     \fi}
%    \end{macrocode}
% \end{lstkey}
%
% \begin{macro}{\lst@MakeMacroUppercase}
% makes the contents of |#1| (if defined) upper case.
%    \begin{macrocode}
\gdef\lst@MakeMacroUppercase#1{%
    \ifx\@undefined#1\else \uppercase\expandafter
        {\expandafter\def\expandafter#1\expandafter{#1}}%
    \fi}
%    \end{macrocode}
% \end{macro}
%
%
% \subsection{Installing tests}
%
% \begin{macro}{\lst@InstallTest}
% The arguments are
% \begin{macroargs}
% \item \meta{prefix}
% \item |\lst@|\meta{name}|@list|
% \item |\lst@|\meta{name}
% \item |\lst@g|\meta{name}|@list|
% \item |\lst@g|\meta{name}
% \item |\lst@g|\meta{name}|@sty|
% \item \alternative{w,s} (working procedure or style)
% \item \alternative{d,o} (\hookname{DetectKeywords} or \hookname{Output} hook)
% \end{macroargs}
% We just insert hook material. The tests will be inserted on demand.
%    \begin{macrocode}
\gdef\lst@InstallTest#1#2#3#4#5#6#7#8{%
    \lst@AddToHook{TrackKeywords}{\lst@TrackKeywords{#1}#2#4#6#7#8}%
    \lst@AddToHook{PostTrackKeywords}{\lst@PostTrackKeywords#2#3#4#5}}
%    \end{macrocode}
%    \begin{macrocode}
\lst@AddToHook{Init}{\lsthk@TrackKeywords\lsthk@PostTrackKeywords}
\lst@AddToHook{TrackKeywords}
    {\global\let\lst@DoDefineKeywords\@empty}% init
\lst@AddToHook{PostTrackKeywords}
    {\lst@DoDefineKeywords
     \global\let\lst@DoDefineKeywords\@empty}% init
%    \end{macrocode}
% We have to detect the keywords somewhere.
%    \begin{macrocode}
\lst@AddToHook{Output}{\lst@ifkeywords \lsthk@DetectKeywords \fi}
\lst@AddToHook{DetectKeywords}{}% init
\lst@AddToHook{ModeTrue}{\let\lst@ifkeywords\iffalse}
\lst@AddToHookExe{Init}{\let\lst@ifkeywords\iftrue}
%    \end{macrocode}
% \end{macro}
%
% \begin{macro}{\lst@InstallTestNow}
% actually inserts a test.
% \begin{macroargs}
% \item \meta{prefix}
% \item |\lst@|\meta{name}|@list|
% \item |\lst@g|\meta{name}|@sty|
% \item \alternative{w,s} (working procedure or style)
% \item \alternative{d,o} (\hookname{DetectKeywords} or \hookname{Output} hook)
% \end{macroargs}
% For example, |#4#5|=|sd| will add
%    |\lst@KeywordTest{|\meta{prefix}|}|
%       |\lst@|\meta{name}|@list| |\lst@g|\meta{name}|@sty|
% to the \hookname{DetectKeywords} hook.
%    \begin{macrocode}
\gdef\lst@InstallTestNow#1#2#3#4#5{%
    \@ifundefined{\string#2#1}%
    {\global\@namedef{\string#2#1}{}%
     \edef\@tempa{%
         \noexpand\lst@AddToHook{\ifx#5dDetectKeywords\else Output\fi}%
         {\ifx #4w\noexpand\lst@WorkingTest
             \else\noexpand\lst@KeywordTest \fi
          {#1}\noexpand#2\noexpand#3}}%
%    \end{macrocode}
% If we are advised to save memory, we insert a test for each \meta{name}.
% Otherwise we install the tests according to \meta{prefix}.
%    \begin{macrocode}
     \lst@ifsavemem
         \@tempa
     \else
         \@ifundefined{\@lst#1@if@ins}%
             {\@tempa \global\@namedef{\@lst#1@if@ins}{}}%
             {}%
     \fi}
    {}}
%    \end{macrocode}
% \end{macro}
%
% \begin{macro}{\lst@TrackKeywords}
% Now it gets a bit tricky. We expand the class list |\lst@|\meta{name}|@list|
% behind |\lst@TK@{|\meta{prefix}|}||\lst@g|\meta{name}|@sty| and use two
% |\relax|es as terminators. This will define the keywords of all the classes
% as keywords of type \meta{prefix}. More details come soon.
%    \begin{macrocode}
\gdef\lst@TrackKeywords#1#2#3#4#5#6{%
    \lst@false
    \def\lst@arg{{#1}#4}%
    \expandafter\expandafter\expandafter\lst@TK@
        \expandafter\lst@arg#2\relax\relax
%    \end{macrocode}
% And nearly the same to undefine all out-dated keywords, which is necessary
% only if we don't save memory.
%    \begin{macrocode}
    \lst@ifsavemem\else
        \def\lst@arg{{#1}#4#2}%
        \expandafter\expandafter\expandafter\lst@TK@@
            \expandafter\lst@arg#3\relax\relax
    \fi
%    \end{macrocode}
% Finally we install the keyword test if keywords changed, in particular if
% they are defined the first time. Note that |\lst@InstallTestNow| inserts a
% test only once.
%    \begin{macrocode}
    \lst@if \lst@InstallTestNow{#1}#2#4#5#6\fi}
%    \end{macrocode}
% Back to the current keywords. Global macros |\lst@g|\meta{id} contain
% globally defined keywords, whereas |\lst@|\meta{id} conatin the true
% keywords. This way we can keep track of the keywords: If keywords or
% \keyname{sensitive} changed, we undefine the old (= globally defined)
% keywords and define the true ones. The arguments of |\lst@TK@| are
% \begin{macroargs}
% \item \meta{prefix}
% \item |\lst@g|\meta{name}|@sty|
% \item |\lst@|\meta{id}
% \item |\lst@g|\meta{id}
% \end{macroargs}
% Thanks to \lsthelper{Holger~Arndt}{2004/05/27}{bad \lst@UndefineKeywords
% \lst@DefineKeywords sequence if keyword crosses orders in two languages}
% the definition of keywords is now delayed via |\lst@DoDefineKeywords|.
%    \begin{macrocode}
\gdef\lst@TK@#1#2#3#4{%
  \ifx\lst@ifsensitive\lst@ifsensitivedefed
    \ifx#3#4\else
      \lst@true
      \lst@ifsavemem\else
          \lst@UndefineKeywords{#1}#4#2%
          \lst@AddTo\lst@DoDefineKeywords{\lst@DefineKeywords{#1}#3#2}%
      \fi
    \fi
  \else
    \ifx#3\relax\else
      \lst@true
      \lst@ifsavemem\else
          \lst@UndefineKeywords{#1}#4#2%
          \lst@AddTo\lst@DoDefineKeywords{\lst@DefineKeywords{#1}#3#2}%
      \fi
    \fi
  \fi
%    \end{macrocode}
% We don't define and undefine keywords if we try to save memory. But we
% possibly need to make them upper case, which again wastes some memory.
%    \begin{macrocode}
  \lst@ifsavemem \ifx#3\relax\else
      \lst@ifsensitive\else \lst@MakeMacroUppercase#3\fi
  \fi \fi
%    \end{macrocode}
% Reaching the end of the class list, we end the loop.
%    \begin{macrocode}
  \ifx#3\relax
      \expandafter\@gobblethree
  \fi
  \lst@TK@{#1}#2}
%    \end{macrocode}
% Here now we undefine the out-dated keywords. While not reaching the end of
% the global list, we look whether the keyword class |#4#5| is still in use or
% needs to be undefined. Our arguments are
% \begin{macroargs}
% \item \meta{prefix}
% \item |\lst@g|\meta{name}|@sty|
% \item |\lst@|\meta{name}|@list|
% \item |\lst@|\meta{id}
% \item |\lst@g|\meta{id}
% \end{macroargs}
%    \begin{macrocode}
\gdef\lst@TK@@#1#2#3#4#5{%
    \ifx#4\relax
        \expandafter\@gobblefour
    \else
        \lst@IfSubstring{#4#5}#3{}{\lst@UndefineKeywords{#1}#5#2}%
    \fi
    \lst@TK@@{#1}#2#3}
%    \end{macrocode}
% Keywords are up-to-date after \hookname{InitVars}.
%    \begin{macrocode}
\lst@AddToHook{InitVars}
    {\global\let\lst@ifsensitivedefed\lst@ifsensitive}
%    \end{macrocode}
% \end{macro}
%
% \begin{macro}{\lst@PostTrackKeywords}
% After updating all the keywords, the global keywords and the global list
% become equivalent to the local ones.
%    \begin{macrocode}
\gdef\lst@PostTrackKeywords#1#2#3#4{%
    \lst@ifsavemem\else
        \global\let#3#1%
        \global\let#4#2%
    \fi}
%    \end{macrocode}
% \end{macro}
%
%
% \subsection{Classes and families}
%
% \begin{lstkey}{classoffset}
% just stores the argument in a macro.
%    \begin{macrocode}
\lst@Key{classoffset}\z@{\def\lst@classoffset{#1}}
%    \end{macrocode}
% \end{lstkey}
%
% \begin{macro}{\lst@InstallFamily}
% Recall the parameters
% \begin{macroargs}
% \item \meta{prefix}
% \item \meta{name}
% \item \meta{style name}
% \item \meta{style init}
% \item \meta{default style name}
% \item \meta{working procedure}
% \item \alternative{l,o} (language or other key)
% \item \alternative{d,o} (\hookname{DetectKeywords} or \hookname{Output} hook)
% \end{macroargs}
% First we define the keys and the style key \meta{style name} if and only if
% the name is not empty.
%    \begin{macrocode}
\gdef\lst@InstallFamily#1#2#3#4#5{%
    \lst@Key{#2}\relax{\lst@UseFamily{#2}##1\relax\lst@MakeKeywords}%
    \lst@Key{more#2}\relax
        {\lst@UseFamily{#2}##1\relax\lst@MakeMoreKeywords}%
    \lst@Key{delete#2}\relax
        {\lst@UseFamily{#2}##1\relax\lst@DeleteKeywords}%
    \ifx\@empty#3\@empty\else
        \lst@Key{#3}{#4}{\lstKV@OptArg[\@ne]{##1}%
            {\@tempcnta\lst@classoffset \advance\@tempcnta####1\relax
             \@namedef{lst@#3\ifnum\@tempcnta=\@ne\else \the\@tempcnta
                             \fi}{####2}}}%
    \fi
    \expandafter\lst@InstallFamily@
        \csname\@lst @#2@data\expandafter\endcsname
        \csname\@lst @#5\endcsname {#1}{#2}{#3}}
%    \end{macrocode}
% Now we check whether \meta{working procedure} is empty. Accordingly we use
% \texttt working procedure or \texttt style in the `data' definition.
% The working procedure is defined right here if necessary.
%    \begin{macrocode}
\gdef\lst@InstallFamily@#1#2#3#4#5#6#7#8{%
    \gdef#1{{#3}{#4}{#5}#2#7}%
    \long\def\lst@temp##1{#6}%
    \ifx\lst@temp\@gobble
        \lst@AddTo#1{s#8}%
    \else
        \lst@AddTo#1{w#8}%
        \global\@namedef{lst@g#4@wp}##1{#6}%
    \fi}
%    \end{macrocode}
% Nothing else is defined here, all the rest is done on demand.
% \end{macro}
%
% \begin{macro}{\lst@UseFamily}
% We look for the optional class number, provide this member, \ldots
%    \begin{macrocode}
\gdef\lst@UseFamily#1{%
    \def\lst@family{#1}%
    \@ifnextchar[\lst@UseFamily@{\lst@UseFamily@[\@ne]}}
\gdef\lst@UseFamily@[#1]{%
    \@tempcnta\lst@classoffset \advance\@tempcnta#1\relax
    \lst@ProvideFamily\lst@family
%    \end{macrocode}
% \ldots\space and build the control sequences \ldots
%    \begin{macrocode}
    \lst@UseFamily@a
        {\lst@family\ifnum\@tempcnta=\@ne\else \the\@tempcnta \fi}}
\gdef\lst@UseFamily@a#1{%
    \expandafter\lst@UseFamily@b
       \csname\@lst @#1@list\expandafter\endcsname
       \csname\@lst @#1\expandafter\endcsname
       \csname\@lst @#1@also\expandafter\endcsname
       \csname\@lst @g#1\endcsname}
%    \end{macrocode}
% \ldots\space required for |\lst@MakeKeywords| and |#6|.
%    \begin{macrocode}
\gdef\lst@UseFamily@b#1#2#3#4#5\relax#6{\lstKV@XOptArg[]{#5}#6#1#2#3#4}
%    \end{macrocode}
% \end{macro}
%
% \begin{macro}{\lst@ProvideFamily}
% provides the member `|\the\@tempcnta|' of the family |#1|. We do nothing if
% the member already exists. Otherwise we expand the data macro defined above.
% Note that we don't use the counter if it equals one. Since a bug report by
% \lsthelper{Kris~Luyten}{2002/08/03}{Undefined control sequence \lst@thestyle}
% keyword families use the prefix |lstfam| instead of |lst|. The marker
% |\lstfam@#1|\oarg{number} is defined globally since a bug report by
% \lsthelper{Edsko~de~Vries}{2003/07/20}{bad keywords with language selections
% only in optional arguments}.
%    \begin{macrocode}
\gdef\lst@ProvideFamily#1{%
    \@ifundefined{lstfam@#1\ifnum\@tempcnta=\@ne\else\the\@tempcnta\fi}%
    {\global\@namedef{lstfam@#1\ifnum\@tempcnta=\@ne\else
                                        \the\@tempcnta\fi}{}%
     \expandafter\expandafter\expandafter\lst@ProvideFamily@
         \csname\@lst @#1@data\endcsname
         {\ifnum\@tempcnta=\@ne\else \the\@tempcnta \fi}}%
    {}}%
%    \end{macrocode}
% Now we have the following arguments
% \begin{macroargs}
% \item \meta{prefix}
% \item \meta{name}
% \item \meta{style name}
% \item \meta{default style name}
% \item \alternative{l,o} (language or other key)
% \item \alternative{w,s} (working procedure or style)
% \item \alternative{d,o} (\hookname{DetectKeywords} or \hookname{Output} hook)
% \item |\ifnum\@tempcnta=\@ne\else \the\@tempcnta \fi|
% \end{macroargs}
% We define |\lst@g|\meta{name}\meta{number}|@sty| to call either
% |\lst@g|\meta{name}|@wp| with the number as argument or
% |\lst@|\meta{style name}\meta{number} where the number belongs to the control
% sequence.
%    \begin{macrocode}
\gdef\lst@ProvideFamily@#1#2#3#4#5#6#7#8{%
    \expandafter\xdef\csname\@lst @g#2#8@sty\endcsname
    {\if #6w%
         \expandafter\noexpand\csname\@lst @g#2@wp\endcsname{#8}%
     \else
         \expandafter\noexpand\csname\@lst @#3#8\endcsname
     \fi}%
%    \end{macrocode}
% We ensure the existence of the style macro. This is done in the
% \hookname{Init} hook by assigning the default style if necessary.
%    \begin{macrocode}
    \ifx\@empty#3\@empty\else
        \edef\lst@temp{\noexpand\lst@AddToHook{Init}{%
            \noexpand\lst@ProvideStyle\expandafter\noexpand
                \csname\@lst @#3#8\endcsname\noexpand#4}}%
        \lst@temp
    \fi
%    \end{macrocode}
% We call a submacro to do the rest. It requires some control sequences.
%    \begin{macrocode}
    \expandafter\lst@ProvideFamily@@
         \csname\@lst @#2#8@list\expandafter\endcsname
         \csname\@lst @#2#8\expandafter\endcsname
         \csname\@lst @#2#8@also\expandafter\endcsname
         \csname\@lst @g#2#8@list\expandafter\endcsname
         \csname\@lst @g#2#8\expandafter\endcsname
         \csname\@lst @g#2#8@sty\expandafter\endcsname
         {#1}#5#6#7}
%    \end{macrocode}
% Now we have (except that \meta{number} is possibly always missing)
% \begin{macroargs}
% \item |\lst@|\meta{name}\meta{number}|@list|
% \item |\lst@|\meta{name}\meta{number}
% \item |\lst@|\meta{name}\meta{number}|@also|
% \item |\lst@g|\meta{name}\meta{number}|@list|
% \item |\lst@g|\meta{name}\meta{number}
% \item |\lst@g|\meta{name}\meta{number}|@sty|
% \item \meta{prefix}
% \item \alternative{l,o} (language or other key)
% \item \alternative{w,s} (working procedure or style)
% \item \alternative{d,o} (\hookname{DetectKeywords} or \hookname{Output} hook)
% \end{macroargs}
% Note that |#9| and `|#10|' are read by |\lst@InstallTest|. We initialize all
% required `variables' (at \hookname{SetLanguage}) and install the test (which
% definition is in fact also delayed).
%    \begin{macrocode}
\gdef\lst@ProvideFamily@@#1#2#3#4#5#6#7#8{%
    \gdef#1{#2#5}\global\let#2\@empty \global\let#3\@empty % init
    \gdef#4{#2#5}\global\let#5\@empty % init
    \if #8l\relax
        \lst@AddToHook{SetLanguage}{\def#1{#2#5}\let#2\@empty}%
    \fi
    \lst@InstallTest{#7}#1#2#4#5#6}
%    \end{macrocode}
% \end{macro}
%
% \begin{macro}{\lst@InstallKeywords}
% Now we take advance of the optional argument construction above. Thus, we
% just insert |[\@ne]| as \meta{number} in the definitions of the keys.
%    \begin{macrocode}
\gdef\lst@InstallKeywords#1#2#3#4#5{%
    \lst@Key{#2}\relax
        {\lst@UseFamily{#2}[\@ne]##1\relax\lst@MakeKeywords}%
    \lst@Key{more#2}\relax
        {\lst@UseFamily{#2}[\@ne]##1\relax\lst@MakeMoreKeywords}%
    \lst@Key{delete#2}\relax
        {\lst@UseFamily{#2}[\@ne]##1\relax\lst@DeleteKeywords}%
    \ifx\@empty#3\@empty\else
        \lst@Key{#3}{#4}{\@namedef{lst@#3}{##1}}%
    \fi
    \expandafter\lst@InstallFamily@
        \csname\@lst @#2@data\expandafter\endcsname
        \csname\@lst @#5\endcsname {#1}{#2}{#3}}
%    \end{macrocode}
% \end{macro}
%
% \begin{macro}{\lst@ProvideStyle}
% If the style macro |#1| is not defined, it becomes equivalent to |#2|.
%    \begin{macrocode}
\gdef\lst@ProvideStyle#1#2{%
    \ifx#1\@undefined \let#1#2%
    \else\ifx#1\relax \let#1#2\fi\fi}
%    \end{macrocode}
% \end{macro}
%
% Finally we define |\lst@MakeKeywords|, \ldots, |\lst@DeleteKeywords|.
% We begin with two helper.
%
% \begin{macro}{\lst@BuildClassList}
% After |#1| follows a comma separated list of keyword classes terminated by
% |,\relax,|, e.g.~|keywords2,emph1,\relax,|. For each \meta{item} in this
% list we \emph{append} the two macros |\lst@|\meta{item}|\lst@g|\meta{item}
% to |#1|.
%    \begin{macrocode}
\gdef\lst@BuildClassList#1#2,{%
    \ifx\relax#2\@empty\else
        \ifx\@empty#2\@empty\else
            \lst@lExtend#1{\csname\@lst @#2\expandafter\endcsname
                           \csname\@lst @g#2\endcsname}%
        \fi
        \expandafter\lst@BuildClassList\expandafter#1
    \fi}
%    \end{macrocode}
% \end{macro}
%
% \begin{macro}{\lst@DeleteClassesIn}
% deletes pairs of tokens, namely the arguments |#2#3| to the submacro.
%    \begin{macrocode}
\gdef\lst@DeleteClassesIn#1#2{%
    \expandafter\lst@DCI@\expandafter#1#2\relax\relax}
\gdef\lst@DCI@#1#2#3{%
    \ifx#2\relax
        \expandafter\@gobbletwo
    \else
%    \end{macrocode}
% If we haven't reached the end of the class list, we define a temporary macro
% which removes all appearances.
%    \begin{macrocode}
        \def\lst@temp##1#2#3##2{%
            \lst@lAddTo#1{##1}%
            \ifx ##2\relax\else
                \expandafter\lst@temp
            \fi ##2}%
        \let\@tempa#1\let#1\@empty
        \expandafter\lst@temp\@tempa#2#3\relax
    \fi
    \lst@DCI@#1}
%    \end{macrocode}
% \end{macro}
%
% \begin{macro}{\lst@MakeKeywords}
% We empty some macros and make use of |\lst@MakeMoreKeywords|.
% Note that this and the next two definitions have the following arguments:
% \begin{macroargs}
% \item class list (in brackets)
% \item keyword list
% \item |\lst@|\meta{name}|@list|
% \item |\lst@|\meta{name}
% \item |\lst@|\meta{name}|@also|
% \item |\lst@g|\meta{name}
% \end{macroargs}
%    \begin{macrocode}
\gdef\lst@MakeKeywords[#1]#2#3#4#5#6{%
    \def#3{#4#6}\let#4\@empty \let#5\@empty
    \lst@MakeMoreKeywords[#1]{#2}#3#4#5#6}
%    \end{macrocode}
% \end{macro}
%
% \begin{macro}{\lst@MakeMoreKeywords}
% We append classes and keywords.
%    \begin{macrocode}
\gdef\lst@MakeMoreKeywords[#1]#2#3#4#5#6{%
    \lst@BuildClassList#3#1,\relax,%
    \lst@DefOther\lst@temp{,#2}\lst@lExtend#4\lst@temp}
%    \end{macrocode}
% \end{macro}
%
% \begin{macro}{\lst@DeleteKeywords}
% We convert the keyword arguments via |\lst@MakeKeywords| and remove the
% classes and keywords.
%    \begin{macrocode}
\gdef\lst@DeleteKeywords[#1]#2#3#4#5#6{%
    \lst@MakeKeywords[#1]{#2}\@tempa\@tempb#5#6%
    \lst@DeleteClassesIn#3\@tempa
    \lst@DeleteKeysIn#4\@tempb}
%    \end{macrocode}
% \end{macro}
%
%
% \subsection{Main families and classes}
%
%
% \paragraph{Keywords}
%
% \begin{lstkey}{keywords}
% Defining the keyword family gets very, very easy.
%    \begin{macrocode}
\lst@InstallFamily k{keywords}{keywordstyle}\bfseries{keywordstyle}{}ld
%    \end{macrocode}
% The following macro sets a keywordstyle, which \ldots
%    \begin{macrocode}
\gdef\lst@DefKeywordstyle#1#2\@nil@{%
   \@namedef{lst@keywordstyle\ifnum\@tempcnta=\@ne\else\the\@tempcnta
                             \fi}{#1#2}}%
%    \end{macrocode}
% \ldots\space is put together here. If we detect a star after the class
% number, we insert code to make the keyword uppercase.
%    \begin{macrocode}
\lst@Key{keywordstyle}{\bfseries}{\lstKV@OptArg[\@ne]{#1}%
  {\@tempcnta\lst@classoffset \advance\@tempcnta##1\relax
   \@ifstar{\lst@DefKeywordstyle{\uppercase\expandafter{%
                                 \expandafter\lst@token
                                 \expandafter{\the\lst@token}}}}%
           {\lst@DefKeywordstyle{}}##2\@nil@}}
%    \end{macrocode}
% \end{lstkey}
%
% \begin{lstkey}{ndkeywords}
% Second order keywords use the same trick as |\lst@InstallKeywords|.
%    \begin{macrocode}
\lst@Key{ndkeywords}\relax
    {\lst@UseFamily{keywords}[\tw@]#1\relax\lst@MakeKeywords}%
\lst@Key{morendkeywords}\relax
    {\lst@UseFamily{keywords}[\tw@]#1\relax\lst@MakeMoreKeywords}%
\lst@Key{deletendkeywords}\relax
    {\lst@UseFamily{keywords}[\tw@]#1\relax\lst@DeleteKeywords}%
\lst@Key{ndkeywordstyle}\relax{\@namedef{lst@keywordstyle2}{#1}}%
%    \end{macrocode}
% \lsthelper{Dr.~Peter~Leibner}{1999/11/05}{undefined \lst@UseKeywords,
% Illegal parameter number (##1)} reported two bugs: |\lst@UseKeywords| and
% |##1| became |\lst@UseFamily| and |#1|.
% \end{lstkey}
%
% \begin{lstkey}{keywordsprefix}
% is implemented experimentally. The one and only prefix indicates its
% presence by making |\lst@prefixkeyword| empty. We can catch this information
% in the \keyname{Output} hook.
%    \begin{macrocode}
\lst@Key{keywordsprefix}\relax{\lst@DefActive\lst@keywordsprefix{#1}}
\global\let\lst@keywordsprefix\@empty
\lst@AddToHook{SelectCharTable}
    {\ifx\lst@keywordsprefix\@empty\else
         \expandafter\lst@CArg\lst@keywordsprefix\relax
             \lst@CDef{}%
                      {\lst@ifletter\else
                           \global\let\lst@prefixkeyword\@empty
                       \fi}%
                      {}%
     \fi}
\lst@AddToHook{Init}{\global\let\lst@prefixkeyword\relax}
\lst@AddToHook{Output}
    {\ifx\lst@prefixkeyword\@empty
         \let\lst@thestyle\lst@gkeywords@sty
         \global\let\lst@prefixkeyword\relax
     \fi}%
%    \end{macrocode}
% \end{lstkey}
%
% \begin{lstkey}{otherkeywords}
% Thanks to \lsthelper{Bradford~Chamberlain}{2001/07/07}{otherkeywords={@,@^}
% does not work} we now iterate down the list of `other keywords' and make each
% active---instead of making the whole argument active. We append the active
% token sequence to |\lst@otherkeywords| to define each `other' keyword.
%    \begin{macrocode}
\lst@Key{otherkeywords}{}{%
    \let\lst@otherkeywords\@empty
    \lst@for{#1}\do{%
      \lst@MakeActive{##1}%
      \lst@lExtend\lst@otherkeywords{%
          \expandafter\lst@CArg\lst@temp\relax\lst@CDef
              {}\lst@PrintOtherKeyword\@empty}}}
\lst@AddToHook{SelectCharTable}{\lst@otherkeywords}
%    \end{macrocode}
% |\lst@PrintOtherkeyword| has been changed to |\lst@PrintOtherKeyword| after a
% bug report by \lsthelper{Peter~Bartke}{2001/11/06}{undefined control sequence
% \lst@PrintOtherkeyword}.
% \end{lstkey}
%
% \begin{macro}{\lst@PrintOtherKeyword}
% print preceding characters, prepare the output and typeset the argument in
% keyword style. \lsthelper{James~Willans}{2004/07/23}{problem: otherkeywords}
% reported problems when the output routine is invoked within |\begingroup| and
% |\endgroup|. Now the definition is restructured.
%    \begin{macrocode}
\gdef\lst@PrintOtherKeyword#1\@empty{%
    \lst@XPrintToken
    \begingroup
      \lst@modetrue \lsthk@TextStyle
      \let\lst@ProcessDigit\lst@ProcessLetter
      \let\lst@ProcessOther\lst@ProcessLetter
      \lst@lettertrue
      #1%
	  \lst@SaveToken
    \endgroup
	\lst@RestoreToken
	\global\let\lst@savedcurrstyle\lst@currstyle
	\let\lst@currstyle\lst@gkeywords@sty
    \lst@Output
	\let\lst@currstyle\lst@savedcurrstyle}
%    \end{macrocode}
% \begin{TODO}
% Which part of \hookname{TextStyle} hook is required? Is it required anymore,
% i.e.after the restruction? Need to move it elsewhere?
% \end{TODO}
% \end{macro}
%
%    \begin{macrocode}
\lst@EndAspect
%</misc>
%    \end{macrocode}
% \end{aspect}
%
%
% \paragraph{The emphasize family}
%
% \begin{aspect}{emph}
% is just one macro call here.
%    \begin{macrocode}
%<*misc>
\lst@BeginAspect[keywords]{emph}
\lst@InstallFamily e{emph}{emphstyle}{}{emphstyle}{}od
\lst@EndAspect
%</misc>
%    \end{macrocode}
% \end{aspect}
%
%
% \paragraph{\TeX\ control sequences}
%
% \begin{aspect}{tex}
% Here we check the last `other' processed token.
%    \begin{macrocode}
%<*misc>
\lst@BeginAspect[keywords]{tex}
%    \end{macrocode}
%    \begin{macrocode}
\lst@InstallFamily {cs}{texcs}{texcsstyle}\relax{keywordstyle}
    {\ifx\lst@lastother\lstum@backslash
         \expandafter\let\expandafter\lst@thestyle
                         \csname lst@texcsstyle#1\endcsname
     \fi}
    ld
%    \end{macrocode}
% The style-key checks for the optional star (which must be in front of
% the optional class argument).
%    \begin{macrocode}
\lst@Key{texcsstyle}\relax
  {\@ifstar{\lst@true\lst@DefTexcsstyle}%
           {\lst@false\lst@DefTexcsstyle}#1\@nil@}
\gdef\lst@DefTexcsstyle#1\@nil@{%
    \let\lst@iftexcsincludebs\lst@if
    \lstKV@OptArg[\@ne]{#1}%
    {\@tempcnta\lst@classoffset \advance\@tempcnta##1\relax
     \@namedef{lst@texcsstyle\ifnum\@tempcnta=\@ne\else
                                   \the\@tempcnta \fi}{##2}}}%
\global\let\lst@iftexcsincludebs\iffalse
%    \end{macrocode}
% To make the backslash belong to the control sequence, it is merged with
% the following token. This option was suggested by \lsthelper{Morten~H\o gholm}
% {2004/07/16}{defining new (colored) texcs}.
% \lsthelper{Christian~Schneider}{-}{2006/09/08} pointed out that the original
% implementation was broken when the identifier was preceded by an ``other''
% character.  To fix this (and other bugs), we first output whatever is in the
% current token before merging.
%    \begin{macrocode}
\let\lst@iftexcsincludebs\iffalse
\lst@AddToHook{SelectCharTable}
{\lst@iftexcsincludebs \ifx\@empty\lst@texcs\else
     \lst@DefSaveDef{`\\}\lsts@texcsbs
      {\lst@ifletter
           \lst@Output
       \else
           \lst@OutputOther
       \fi
       \lst@Merge\lsts@texcsbs}%
 \fi \fi}
%    \end{macrocode}
%    \begin{macrocode}
\lst@EndAspect
%</misc>
%    \end{macrocode}
% \end{aspect}
%
%
% \paragraph{Compiler directives}
%
% \begin{aspect}{directives}
% \begin{lstkey}{directives}
% First some usual stuff.
%    \begin{macrocode}
%<*misc>
\lst@BeginAspect[keywords]{directives}
%    \end{macrocode}
% The initialization of |\lst@directives| has been added after a bug report
% from \lsthelper{Kris~Luyten}{2002/07/30}{Undefined control sequence
% \lst@thestyle caused by undefined \lst@directives after loading C}.
%    \begin{macrocode}
\lst@NewMode\lst@CDmode
\lst@AddToHook{EOL}{\ifnum\lst@mode=\lst@CDmode \lst@LeaveMode \fi}
\lst@InstallKeywords{d}{directives}{directivestyle}\relax{keywordstyle}
    {\ifnum\lst@mode=\lst@CDmode
         \let\lst@thestyle\lst@directivestyle
     \fi}
    ld
\global\let\lst@directives\@empty % init
%    \end{macrocode}
% Now we define a new delimiter for directives: We enter `directive mode'
% only in the first column.
%    \begin{macrocode}
\lst@AddTo\lst@delimtypes{,directive}
\gdef\lst@Delim@directive#1\@empty#2#3#4{%
    \lst@CArg #1\relax\lst@DefDelimB
        {\lst@CalcColumn}%
        {}%
        {\ifnum\@tempcnta=\z@
             \def\lst@bnext{#2\lst@CDmode{#4\lst@Lmodetrue}%
			                \let\lst@currstyle\lst@directivestyle}%
		 \fi
		 \@gobblethree}%
        #2\lst@CDmode{#4\lst@Lmodetrue}}
%    \end{macrocode}
% We introduce a new string type (thanks to \lsthelper{R.~Isernhagen}
% {1999/11/12}{float isn't keyword in #include <float>}), which \ldots
%    \begin{macrocode}
\lst@AddTo\lst@stringtypes{,directive}
\gdef\lst@StringDM@directive#1#2#3\@empty{%
    \lst@CArg #2\relax\lst@CDef
        {}%
%    \end{macrocode}
% \ldots\space is active only in |\lst@CDmode|:
%    \begin{macrocode}
        {\let\lst@bnext\lst@CArgEmpty
         \ifnum\lst@mode=\lst@CDmode
             \def\lst@bnext{\lst@BeginString{#1}}%
         \fi
         \lst@bnext}%
        \@empty
    \lst@CArg #3\relax\lst@CDef
        {}%
        {\let\lst@enext\lst@CArgEmpty
         \ifnum #1=\lst@mode
             \let\lst@bnext\lst@EndString
         \fi
         \lst@bnext}%
        \@empty}
%    \end{macrocode}
% \end{lstkey}
%
%    \begin{macrocode}
\lst@EndAspect
%</misc>
%    \end{macrocode}
% \end{aspect}
%
%
% \subsection{Keyword comments}
%
% \begin{aspect}{keywordcomments}
% includes both comment types and is possibly split into this and |dkcs|.
%    \begin{macrocode}
%<*misc>
\lst@BeginAspect[keywords,comments]{keywordcomments}
%    \end{macrocode}
%
% \begin{macro}{\lst@BeginKC}
% \begin{macro}{\lst@BeginKCS}
% Starting a keyword comment is easy, but: (1) The submacros are called
% outside of two group levels, and \ldots
%    \begin{macrocode}
\lst@NewMode\lst@KCmode \lst@NewMode\lst@KCSmode
\gdef\lst@BeginKC{\aftergroup\aftergroup\aftergroup\lst@BeginKC@}%
\gdef\lst@BeginKC@{%
    \lst@ResetToken
    \lst@BeginComment\lst@KCmode{{\lst@commentstyle}\lst@modetrue}%
                     \@empty}%
\gdef\lst@BeginKCS{\aftergroup\aftergroup\aftergroup\lst@BeginKCS@}%
\gdef\lst@BeginKCS@{%
    \lst@ResetToken
    \lst@BeginComment\lst@KCSmode{{\lst@commentstyle}\lst@modetrue}%
                     \@empty}%
%    \end{macrocode}
% (2) we must ensure that the comment starts after printing the comment
% delimiter since it could be a keyword. We assign |\lst@BeginKC|[|S|] to
% |\lst@KCpost|, which is executed and reset in \hookname{PostOutput}.
%    \begin{macrocode}
\lst@AddToHook{PostOutput}{\lst@KCpost \global\let\lst@KCpost\@empty}
\global\let\lst@KCpost\@empty % init
%    \end{macrocode}
% \end{macro}
% \end{macro}
%
% \begin{macro}{\lst@EndKC}
% leaves the comment mode before the (temporaryly saved) comment delimiter is
% printed.
%    \begin{macrocode}
\gdef\lst@EndKC{\lst@SaveToken \lst@LeaveMode \lst@RestoreToken
    \let\lst@thestyle\lst@identifierstyle \lsthk@Output}
%    \end{macrocode}
% \end{macro}
%
% \begin{lstkey}{keywordcomment}
% The delimiters must be identical here, thus we use |\lst@KCmatch|. Note the
% last argument |o| to |\lst@InstallKeywords|: The working test is installed
% in the \hookname{Output} hook and not in \hookname{DetectKeywords}.
% Otherwise we couldn't detect the ending delimiter since keyword detection is
% done if and only if mode changes are allowed.
%    \begin{macrocode}
\lst@InstallKeywords{kc}{keywordcomment}{}\relax{}
    {\ifnum\lst@mode=\lst@KCmode
         \edef\lst@temp{\the\lst@token}%
         \ifx\lst@temp\lst@KCmatch
             \lst@EndKC
         \fi
     \else
         \lst@ifmode\else
             \xdef\lst@KCmatch{\the\lst@token}%
             \global\let\lst@KCpost\lst@BeginKC
         \fi
     \fi}
    lo
%    \end{macrocode}
% \end{lstkey}
%
% \begin{lstkey}{keywordcommentsemicolon}
% The key simply stores the keywords. After a bug report by \lsthelper
% {Norbert~Eisinger}{2002/11/26}{keywordcommentsemicolon active after
% language change} the initialization in \hookname{SetLanguage} has been
% added.
%    \begin{macrocode}
\lst@Key{keywordcommentsemicolon}{}{\lstKV@ThreeArg{#1}%
    {\def\lst@KCAkeywordsB{##1}%
     \def\lst@KCAkeywordsE{##2}%
     \def\lst@KCBkeywordsB{##3}%
     \def\lst@KCkeywords{##1##2##3}}}
\lst@AddToHook{SetLanguage}{%
    \let\lst@KCAkeywordsB\@empty \let\lst@KCAkeywordsE\@empty
    \let\lst@KCBkeywordsB\@empty \let\lst@KCkeywords\@empty}
%    \end{macrocode}
% We define an appropriate semicolon if this keyword comment type is defined.
% Appropriate means that we leave any keyword comment mode if active.
% \lsthelper{Oldrich~Jedlicka}{2001/12/12}{keywordcomment(semicolon) fails}
% reported a bug and provided the fix, the two |\@empty|s.
%    \begin{macrocode}
\lst@AddToHook{SelectCharTable}
    {\ifx\lst@KCkeywords\@empty\else
        \lst@DefSaveDef{`\;}\lsts@EKC
            {\lst@XPrintToken
             \ifnum\lst@mode=\lst@KCmode \lst@EndComment\@empty \else
             \ifnum\lst@mode=\lst@KCSmode \lst@EndComment\@empty
             \fi \fi
             \lsts@EKC}%
     \fi}
%    \end{macrocode}
% The `working identifier' macros enter respectively leave comment mode.
%    \begin{macrocode}
\gdef\lst@KCAWorkB{%
    \lst@ifmode\else \global\let\lst@KCpost\lst@BeginKC \fi}
\gdef\lst@KCBWorkB{%
    \lst@ifmode\else \global\let\lst@KCpost\lst@BeginKCS \fi}
\gdef\lst@KCAWorkE{\ifnum\lst@mode=\lst@KCmode \lst@EndKC \fi}
%    \end{macrocode}
% Now we install the tests and initialize the given macros.
%    \begin{macrocode}
\lst@ProvideFamily@@
    \lst@KCAkeywordsB@list\lst@KCAkeywordsB \lst@KC@also
    \lst@gKCAkeywordsB@list\lst@gKCAkeywordsB \lst@KCAWorkB
    {kcb}owo % prefix, other key, working procedure, Output hook
\lst@ProvideFamily@@
    \lst@KCAkeywordsE@list\lst@KCAkeywordsE \lst@KC@also
    \lst@gKCAkeywordsE@list\lst@gKCAkeywordsE \lst@KCAWorkE
    {kce}owo
\lst@ProvideFamily@@
    \lst@KCBkeywordsB@list\lst@KCBkeywordsB \lst@KC@also
    \lst@gKCBkeywordsB@list\lst@gKCBkeywordsB \lst@KCBWorkB
    {kcs}owo
%    \end{macrocode}
% \end{lstkey}
%
%    \begin{macrocode}
\lst@EndAspect
%</misc>
%    \end{macrocode}
% \end{aspect}
%
%
% \subsection{Export of identifiers}
%
% \begin{aspect}{index}
% \begin{macro}{\lstindexmacro}
% One more `keyword' class.
%    \begin{macrocode}
%<*misc>
\lst@BeginAspect[keywords]{index}
\lst@InstallFamily w{index}{indexstyle}\lstindexmacro{indexstyle}
    {\csname\@lst @indexstyle#1\expandafter\endcsname
         \expandafter{\the\lst@token}}
    od
\lst@UserCommand\lstindexmacro#1{\index{{\ttfamily#1}}}
\lst@EndAspect
%</misc>
%    \end{macrocode}
% \end{macro}
% \end{aspect}
%
% \begin{aspect}{procnames}
% \begin{lstkey}{procnamestyle}
% \begin{lstkey}{procnamekeys}
% \begin{lstkey}{indexprocnames}
% The `idea' here is the usage of a global |\lst@ifprocname|, indicating a
% preceding `procedure keyword'. All the other is known stuff.
%    \begin{macrocode}
%<*misc>
\lst@BeginAspect[keywords]{procnames}
\gdef\lst@procnametrue{\global\let\lst@ifprocname\iftrue}
\gdef\lst@procnamefalse{\global\let\lst@ifprocname\iffalse}
\lst@AddToHook{Init}{\lst@procnamefalse}
\lst@AddToHook{DetectKeywords}
    {\lst@ifprocname
         \let\lst@thestyle\lst@procnamestyle
         \lst@ifindexproc \csname\@lst @gindex@sty\endcsname \fi
         \lst@procnamefalse
     \fi}
%    \end{macrocode}
%    \begin{macrocode}
\lst@Key{procnamestyle}{}{\def\lst@procnamestyle{#1}}
\lst@Key{indexprocnames}{false}[t]{\lstKV@SetIf{#1}\lst@ifindexproc}
\lst@AddToHook{Init}{\lst@ifindexproc \lst@indexproc \fi}
\gdef\lst@indexproc{%
    \@ifundefined{lst@indexstyle1}%
        {\@namedef{lst@indexstyle1}##1{}}%
        {}}
%    \end{macrocode}
% The default definition of |\lst@indexstyle| above has been moved outside the
% hook after a bug report from \lsthelper{Ulrich~G.~Wortmann}{2002/01/22}
% {procnames doesn't work}.
%    \begin{macrocode}
\lst@InstallKeywords w{procnamekeys}{}\relax{}
    {\global\let\lst@PNpost\lst@procnametrue}
    od
\lst@AddToHook{PostOutput}{\lst@PNpost\global\let\lst@PNpost\@empty}
\global\let\lst@PNpost\@empty % init
\lst@EndAspect
%</misc>
%    \end{macrocode}
% \end{lstkey}
% \end{lstkey}
% \end{lstkey}
% \end{aspect}
%
%
% \section{More aspects and keys}
%
% \begin{lstkey}{basicstyle}
% \begin{lstkey}{inputencoding}
% There is no better place to define these keys, I think.
%    \begin{macrocode}
%<*kernel>
\lst@Key{basicstyle}\relax{\def\lst@basicstyle{#1}}
\lst@Key{inputencoding}\relax{\def\lst@inputenc{#1}}
\lst@AddToHook{Init}
    {\lst@basicstyle
     \ifx\lst@inputenc\@empty\else
         \@ifundefined{inputencoding}{}%
            {\inputencoding\lst@inputenc}%
     \fi}
\lst@AddToHookExe{EmptyStyle}
    {\let\lst@basicstyle\@empty
     \let\lst@inputenc\@empty}
\lst@Key{multicols}{}{\@tempcnta=0#1\relax\def\lst@multicols{#1}}
%</kernel>
%    \end{macrocode}
% Michael Niedermair asked for a key like \keyname{inputencoding}.
% \end{lstkey}
% \end{lstkey}
%
%
% \subsection{Styles and languages}
%
% \begin{aspect}{style}
% We begin with style definition and selection.
%    \begin{macrocode}
%<*misc>
\lst@BeginAspect{style}
%    \end{macrocode}
%
% \begin{macro}{\lststylefiles}
% This macro is defined if and only if it's undefined yet.
%    \begin{macrocode}
\@ifundefined{lststylefiles}
    {\lst@UserCommand\lststylefiles{lststy0.sty}}{}
%    \end{macrocode}
% \end{macro}
%
% \begin{macro}{\lstdefinestyle}
% \begin{macro}{\lst@definestyle}
% \begin{macro}{\lst@DefStyle}
% are defined in terms of |\lst@DefStyle|, which is defined via
% |\lst@DefDriver|.
%    \begin{macrocode}
\lst@UserCommand\lstdefinestyle{\lst@DefStyle\iftrue}
\lst@UserCommand\lst@definestyle{\lst@DefStyle\iffalse}
\gdef\lst@DefStyle{\lst@DefDriver{style}{sty}\lstset}
%    \end{macrocode}
% The `empty' style calls the initial empty hook \hookname{EmptyStyle}.
%    \begin{macrocode}
\global\@namedef{lststy@$}{\lsthk@EmptyStyle}
\lst@AddToHook{EmptyStyle}{}% init
%    \end{macrocode}
% \end{macro}
% \end{macro}
% \end{macro}
%
% \begin{lstkey}{style}
% is an application of |\lst@LAS|. We just specify the hook and an empty
% argument as `pre' and `post' code.
%    \begin{macrocode}
\lst@Key{style}\relax{%
    \lst@LAS{style}{sty}{[]{#1}}\lst@NoAlias\lststylefiles
        \lsthk@SetStyle
        {}}
%    \end{macrocode}
%    \begin{macrocode}
\lst@AddToHook{SetStyle}{}% init
%    \end{macrocode}
% \end{lstkey}
%
%    \begin{macrocode}
\lst@EndAspect
%</misc>
%    \end{macrocode}
% \end{aspect}
%
% \begin{aspect}{language}
% Now we deal with commands used in defining and selecting programming
% languages, in particular with aliases.
%    \begin{macrocode}
%<*misc>
\lst@BeginAspect{language}
%    \end{macrocode}
%
% \begin{macro}{\lstlanguagefiles}
% This macro is defined if and only if it's undefined yet.
%    \begin{macrocode}
\@ifundefined{lstdriverfiles}
    {\lst@UserCommand\lstlanguagefiles{lstlang0.sty}}{}
%    \end{macrocode}
% \end{macro}
%
% \begin{macro}{\lstdefinelanguage}
% \begin{macro}{\lst@definelanguage}
% \begin{macro}{\lst@DefLang}
% are defined in terms of |\lst@DefLang|, which is defined via
% |\lst@DefDriver|.
%    \begin{macrocode}
\lst@UserCommand\lstdefinelanguage{\lst@DefLang\iftrue}
\lst@UserCommand\lst@definelanguage{\lst@DefLang\iffalse}
\gdef\lst@DefLang{\lst@DefDriver{language}{lang}\lstset}
%    \end{macrocode}
% Now we can provide the `empty' language.
%    \begin{macrocode}
\lstdefinelanguage{}{}
%    \end{macrocode}
% \end{macro}
% \end{macro}
% \end{macro}
%
% \begin{lstkey}{language}
% \begin{lstkey}{alsolanguage}
% is mainly an application of |\lst@LAS|.
%    \begin{macrocode}
\lst@Key{language}\relax{\lstKV@OptArg[]{#1}%
    {\lst@LAS{language}{lang}{[##1]{##2}}\lst@FindAlias\lstlanguagefiles
         \lsthk@SetLanguage
         {\lst@FindAlias[##1]{##2}%
          \let\lst@language\lst@malias
          \let\lst@dialect\lst@oalias}}}
%    \end{macrocode}
% Ditto, we simply don't execute |\lsthk@SetLanguage|.
%    \begin{macrocode}
\lst@Key{alsolanguage}\relax{\lstKV@OptArg[]{#1}%
    {\lst@LAS{language}{lang}{[##1]{##2}}\lst@FindAlias\lstlanguagefiles
         {}%
         {\lst@FindAlias[##1]{##2}%
          \let\lst@language\lst@malias
          \let\lst@dialect\lst@oalias}}}
%    \end{macrocode}
%    \begin{macrocode}
\lst@AddToHook{SetLanguage}{}% init
%    \end{macrocode}
% \end{lstkey}
% \end{lstkey}
%
% \begin{macro}{\lstalias}
% Now we concentrate on aliases and default dialects.
% |\lsta@|\meta{language}|$|\meta{dialect} and |\lsta@|\meta{language} contain
% the aliases of a particular dialect respectively a complete language.
% We'll use a |$|-character to separate a language name from its dialect.
% Thanks to \lsthelper{Walter~E.~Brown}{2004/02/25}{\lstalias
% (+\lstdefinelanguage) fails} for reporting a problem with the argument
% delimiter `[' in a previous definition of |\lstalias@|.
%    \begin{macrocode}
\lst@UserCommand\lstalias{\@ifnextchar[\lstalias@\lstalias@@}
\gdef\lstalias@[#1]#2{\lstalias@b #2$#1}
\gdef\lstalias@b#1[#2]#3{\lst@NormedNameDef{lsta@#1}{#3$#2}}
\gdef\lstalias@@#1#2{\lst@NormedNameDef{lsta@#1}{#2}}
%    \end{macrocode}
% \end{macro}
%
% \begin{lstkey}{defaultdialect}
% We simply store the dialect.
%    \begin{macrocode}
\lst@Key{defaultdialect}\relax
    {\lstKV@OptArg[]{#1}{\lst@NormedNameDef{lstdd@##2}{##1}}}
%    \end{macrocode}
% \end{lstkey}
%
% \begin{macro}{\lst@FindAlias}
% Now we have to find a language. First we test for a complete language alias,
% then we set the default dialect if necessary.
%    \begin{macrocode}
\gdef\lst@FindAlias[#1]#2{%
    \lst@NormedDef\lst@oalias{#1}%
    \lst@NormedDef\lst@malias{#2}%
    \@ifundefined{lsta@\lst@malias}{}%
        {\edef\lst@malias{\csname\@lst a@\lst@malias\endcsname}}%
%    \end{macrocode}
%    \begin{macrocode}
    \ifx\@empty\lst@oalias \@ifundefined{lstdd@\lst@malias}{}%
        {\edef\lst@oalias{\csname\@lst dd@\lst@malias\endcsname}}%
    \fi
%    \end{macrocode}
% Now we are ready for an alias of a single dialect.
%    \begin{macrocode}
    \edef\lst@temp{\lst@malias $\lst@oalias}%
    \@ifundefined{lsta@\lst@temp}{}%
        {\edef\lst@temp{\csname\@lst a@\lst@temp\endcsname}}%
%    \end{macrocode}
% Finally we again set the default dialect---for the case of a dialect alias.
%    \begin{macrocode}
    \expandafter\lst@FindAlias@\lst@temp $}
\gdef\lst@FindAlias@#1$#2${%
    \def\lst@malias{#1}\def\lst@oalias{#2}%
    \ifx\@empty\lst@oalias \@ifundefined{lstdd@\lst@malias}{}%
        {\edef\lst@oalias{\csname\@lst dd@\lst@malias\endcsname}}%
    \fi}
%    \end{macrocode}
% \end{macro}
%
% \begin{macro}{\lst@RequireLanguages}
% This definition will be equivalent to |\lstloadlanguages|. We requested the
% given list of languages and load additionally required aspects.
%    \begin{macrocode}
\gdef\lst@RequireLanguages#1{%
    \lst@Require{language}{lang}{#1}\lst@FindAlias\lstlanguagefiles
    \ifx\lst@loadaspects\@empty\else
        \lst@RequireAspects\lst@loadaspects
    \fi}
%    \end{macrocode}
% \end{macro}
%
% \begin{macro}{\lstloadlanguages}
% is the same as |\lst@RequireLanguages|.
%    \begin{macrocode}
\global\let\lstloadlanguages\lst@RequireLanguages
%    \end{macrocode}
% \end{macro}
%
%    \begin{macrocode}
\lst@EndAspect
%</misc>
%    \end{macrocode}
% \end{aspect}
%
%
% \subsection{Format definitions*}
%
% \begin{aspect}{formats}
%    \begin{macrocode}
%<*misc>
\lst@BeginAspect{formats}
%    \end{macrocode}
%
% \begin{macro}{\lstformatfiles}
% This macro is defined if and only if it's undefined yet.
%    \begin{macrocode}
\@ifundefined{lstformatfiles}
    {\lst@UserCommand\lstformatfiles{lstfmt0.sty}}{}
%    \end{macrocode}
% \end{macro}
%
% \begin{macro}{\lstdefineformat}
% \begin{macro}{\lst@defineformat}
% \begin{macro}{\lst@DefFormat}
% are defined in terms of |\lst@DefFormat|, which is defined via
% |\lst@DefDriver|.
%    \begin{macrocode}
\lst@UserCommand\lstdefineformat{\lst@DefFormat\iftrue}
\lst@UserCommand\lst@defineformat{\lst@DefFormat\iffalse}
\gdef\lst@DefFormat{\lst@DefDriver{format}{fmt}\lst@UseFormat}
%    \end{macrocode}
% We provide the `empty' format.
%    \begin{macrocode}
\lstdefineformat{}{}
%    \end{macrocode}
% \end{macro}
% \end{macro}
% \end{macro}
%
% \begin{lstkey}{format}
% is an application of |\lst@LAS|. We just specify the hook as `pre' and an
% empty argument as  `post' code.
%    \begin{macrocode}
\lst@Key{format}\relax{%
    \lst@LAS{format}{fmt}{[]{#1}}\lst@NoAlias\lstformatfiles
        \lsthk@SetFormat
        {}}
%    \end{macrocode}
%    \begin{macrocode}
\lst@AddToHook{SetFormat}{\let\lst@fmtformat\@empty}% init
%    \end{macrocode}
% \end{lstkey}
%
%
% \paragraph{Helpers}
% Our goal is to define the yet unkown |\lst@UseFormat|. This definition
% will parse the user supplied format. We start with some general macros.
%
% \begin{macro}{\lst@fmtSplit}
% splits the content of the macro |#1| at |#2| in the preceding characters
% |\lst@fmta| and the following ones |\lst@fmtb|. |\lst@if| is false if and
% only if |#1| doesn't contain |#2|.
%    \begin{macrocode}
\gdef\lst@fmtSplit#1#2{%
    \def\lst@temp##1#2##2\relax##3{%
        \ifnum##3=\z@
            \ifx\@empty##2\@empty
                \lst@false
                \let\lst@fmta#1%
                \let\lst@fmtb\@empty
            \else
                \expandafter\lst@temp#1\relax\@ne
            \fi
        \else
            \def\lst@fmta{##1}\def\lst@fmtb{##2}%
        \fi}%
    \lst@true
    \expandafter\lst@temp#1#2\relax\z@}
%    \end{macrocode}
% \end{macro}
%
% \begin{macro}{\lst@IfNextCharWhitespace}
% is defined in terms of |\lst@IfSubstring|.
%    \begin{macrocode}
\gdef\lst@IfNextCharWhitespace#1#2#3{%
    \lst@IfSubstring#3\lst@whitespaces{#1}{#2}#3}
%    \end{macrocode}
% And here come all white space characters.
%    \begin{macrocode}
\begingroup
\catcode`\^^I=12\catcode`\^^J=12\catcode`\^^M=12\catcode`\^^L=12\relax%
\lst@DefActive\lst@whitespaces{\ ^^I^^J^^M}% add ^^L
\global\let\lst@whitespaces\lst@whitespaces%
\endgroup
%    \end{macrocode}
% \end{macro}
%
% \begin{macro}{\lst@fmtIfIdentifier}
% tests the first character of |#1|
%    \begin{macrocode}
\gdef\lst@fmtIfIdentifier#1{%
    \ifx\relax#1\@empty
        \expandafter\@secondoftwo
    \else
        \expandafter\lst@fmtIfIdentifier@\expandafter#1%
    \fi}
%    \end{macrocode}
% against the `letters' |_|, |@|, |A|,\ldots,|Z| and |a|,\ldots,|z|.
%    \begin{macrocode}
\gdef\lst@fmtIfIdentifier@#1#2\relax{%
    \let\lst@next\@secondoftwo
    \ifnum`#1=`_\else
    \ifnum`#1<64\else
    \ifnum`#1<91\let\lst@next\@firstoftwo\else
    \ifnum`#1<97\else
    \ifnum`#1<123\let\lst@next\@firstoftwo\else
    \fi \fi \fi \fi \fi
    \lst@next}
%    \end{macrocode}
% \end{macro}
%
% \begin{macro}{\lst@fmtIfNextCharIn}
% is required for the optional \meta{exceptional characters}.
% The implementation is easy---refer section \ref{iSubstringTests}.
%    \begin{macrocode}
\gdef\lst@fmtIfNextCharIn#1{%
    \ifx\@empty#1\@empty \expandafter\@secondoftwo \else
                         \def\lst@next{\lst@fmtIfNextCharIn@{#1}}%
                         \expandafter\lst@next\fi}
\gdef\lst@fmtIfNextCharIn@#1#2#3#4{%
    \def\lst@temp##1#4##2##3\relax{%
        \ifx \@empty##2\expandafter\@secondoftwo
                 \else \expandafter\@firstoftwo \fi}%
    \lst@temp#1#4\@empty\relax{#2}{#3}#4}
%    \end{macrocode}
% \end{macro}
%
% \begin{macro}{\lst@fmtCDef}
% We need derivations of |\lst@CDef| and |\lst@CDefX|: we have to test the
% next character against the sequence |#5| of exceptional characters.
% These tests are inserted here.
%    \begin{macrocode}
\gdef\lst@fmtCDef#1{\lst@fmtCDef@#1}
\gdef\lst@fmtCDef@#1#2#3#4#5#6#7{%
    \lst@CDefIt#1{#2}{#3}%
               {\lst@fmtIfNextCharIn{#5}{#4#2#3}{#6#4#2#3#7}}%
               #4%
               {}{}{}}
%    \end{macrocode}
% \end{macro}
%
% \begin{macro}{\lst@fmtCDefX}
% The same but `drop input'.
%    \begin{macrocode}
\gdef\lst@fmtCDefX#1{\lst@fmtCDefX@#1}
\gdef\lst@fmtCDefX@#1#2#3#4#5#6#7{%
    \let#4#1%
    \ifx\@empty#2\@empty
        \def#1{\lst@fmtIfNextCharIn{#5}{#4}{#6#7}}%
    \else \ifx\@empty#3\@empty
        \def#1##1{%
            \ifx##1#2%
                \def\lst@next{\lst@fmtIfNextCharIn{#5}{#4##1}%
                                                      {#6#7}}%
            \else
                 \def\lst@next{#4##1}%
            \fi
            \lst@next}%
    \else
        \def#1{%
            \lst@IfNextCharsArg{#2#3}%
                {\lst@fmtIfNextCharIn{#5}{\expandafter#4\lst@eaten}%
                                         {#6#7}}%
                {\expandafter#4\lst@eaten}}%
    \fi \fi}
%    \end{macrocode}
% \end{macro}
%
%
% \paragraph{The parser}
% applies |\lst@fmtSplit| to cut a format definition into items, items into
% `input' and `output', and `output' into `pre' and 'post'. This should be
% clear if you are in touch with format definitions.
%
% \begin{macro}{\lst@UseFormat}
% Now we can start with the parser.
%    \begin{macrocode}
\gdef\lst@UseFormat#1{%
    \def\lst@fmtwhole{#1}%
    \lst@UseFormat@}
\gdef\lst@UseFormat@{%
    \lst@fmtSplit\lst@fmtwhole,%
%    \end{macrocode}
% We assign the rest of the format definition, \ldots
%    \begin{macrocode}
    \let\lst@fmtwhole\lst@fmtb
    \ifx\lst@fmta\@empty\else
%    \end{macrocode}
% \ldots\space split the item at the equal sign, and work on the item.
%    \begin{macrocode}
        \lst@fmtSplit\lst@fmta=%
        \ifx\@empty\lst@fmta\else
%    \end{macrocode}
% \begin{TODO}
% Insert |\let\lst@arg\@empty| |\expandafter\lst@XConvert\lst@fmtb\@nil|
% |\let\lst@fmtb\lst@arg|.
% \end{TODO}
%    \begin{macrocode}
            \expandafter\lstKV@XOptArg\expandafter[\expandafter]%
                \expandafter{\lst@fmtb}\lst@UseFormat@b
        \fi
    \fi
%    \end{macrocode}
% Finally we process the next item if the rest is not empty.
%    \begin{macrocode}
    \ifx\lst@fmtwhole\@empty\else
        \expandafter\lst@UseFormat@
    \fi}
%    \end{macrocode}
% We make |\lst@fmtc| contain the preceding characters as a braced argument.
% To add more arguments, we first split the replacement tokens at the control
% sequence |\string|.
%    \begin{macrocode}
\gdef\lst@UseFormat@b[#1]#2{%
    \def\lst@fmtc{{#1}}\lst@lExtend\lst@fmtc{\expandafter{\lst@fmta}}%
    \def\lst@fmtb{#2}%
    \lst@fmtSplit\lst@fmtb\string
%    \end{macrocode}
% We append an empty argument or |\lst@fmtPre| with `|\string|-preceding'
% tokens as argument. We do the same for the tokens after |\string|.
%    \begin{macrocode}
    \ifx\@empty\lst@fmta
        \lst@lAddTo\lst@fmtc{{}}%
    \else
        \lst@lExtend\lst@fmtc{\expandafter
            {\expandafter\lst@fmtPre\expandafter{\lst@fmta}}}%
    \fi
    \ifx\@empty\lst@fmtb
        \lst@lAddTo\lst@fmtc{{}}%
    \else
        \lst@lExtend\lst@fmtc{\expandafter
            {\expandafter\lst@fmtPost\expandafter{\lst@fmtb}}}%
    \fi
%    \end{macrocode}
% Eventually we extend |\lst@fmtformat| appropriately. Note that |\lst@if|
% still indicates whether the replacement tokens contain |\string|.
%    \begin{macrocode}
    \expandafter\lst@UseFormat@c\lst@fmtc}
%    \end{macrocode}
%    \begin{macrocode}
\gdef\lst@UseFormat@c#1#2#3#4{%
    \lst@fmtIfIdentifier#2\relax
    {\lst@fmtIdentifier{#2}%
     \lst@if\else \PackageWarning{Listings}%
         {Cannot drop identifier in format definition}%
     \fi}%
    {\lst@if
         \lst@lAddTo\lst@fmtformat{\lst@CArgX#2\relax\lst@fmtCDef}%
     \else
         \lst@lAddTo\lst@fmtformat{\lst@CArgX#2\relax\lst@fmtCDefX}%
     \fi
     \lst@DefActive\lst@fmtc{#1}%
     \lst@lExtend\lst@fmtformat{\expandafter{\lst@fmtc}{#3}{#4}}}}
%    \end{macrocode}
%    \begin{macrocode}
\lst@AddToHook{SelectCharTable}{\lst@fmtformat}
\global\let\lst@fmtformat\@empty
%    \end{macrocode}
% \end{macro}
%
%
% \paragraph{The formatting}
%
% \begin{macro}{\lst@fmtPre}
%    \begin{macrocode}
\gdef\lst@fmtPre#1{%
    \lst@PrintToken
    \begingroup
    \let\newline\lst@fmtEnsureNewLine
    \let\space\lst@fmtEnsureSpace
    \let\indent\lst@fmtIndent
    \let\noindent\lst@fmtNoindent
    #1%
    \endgroup}
%    \end{macrocode}
% \end{macro}
%
% \begin{macro}{\lst@fmtPost}
%    \begin{macrocode}
\gdef\lst@fmtPost#1{%
    \global\let\lst@fmtPostOutput\@empty
    \begingroup
    \def\newline{\lst@AddTo\lst@fmtPostOutput\lst@fmtEnsureNewLine}%
    \def\space{\aftergroup\lst@fmtEnsurePostSpace}%
    \def\indent{\lst@AddTo\lst@fmtPostOutput\lst@fmtIndent}%
    \def\noindent{\lst@AddTo\lst@fmtPostOutput\lst@fmtNoindent}%
    \aftergroup\lst@PrintToken
    #1%
    \endgroup}
%    \end{macrocode}
%    \begin{macrocode}
\lst@AddToHook{Init}{\global\let\lst@fmtPostOutput\@empty}
\lst@AddToHook{PostOutput}
    {\lst@fmtPostOutput \global\let\lst@fmtPostOutput\@empty}
%    \end{macrocode}
% \end{macro}
%
% \begin{macro}{\lst@fmtEnsureSpace}
% \begin{macro}{\lst@fmtEnsurePostSpace}
%    \begin{macrocode}
\gdef\lst@fmtEnsureSpace{%
    \lst@ifwhitespace\else \expandafter\lst@ProcessSpace \fi}
\gdef\lst@fmtEnsurePostSpace{%
    \lst@IfNextCharWhitespace{}{\lst@ProcessSpace}}
%    \end{macrocode}
% \end{macro}
% \end{macro}
%
% \begin{lstkey}{fmtindent}
% \begin{macro}{\lst@fmtIndent}
% \begin{macro}{\lst@fmtNoindent}
%    \begin{macrocode}
\lst@Key{fmtindent}{20pt}{\def\lst@fmtindent{#1}}
\newdimen\lst@fmtcurrindent
\lst@AddToHook{InitVars}{\global\lst@fmtcurrindent\z@}
\gdef\lst@fmtIndent{\global\advance\lst@fmtcurrindent\lst@fmtindent}
\gdef\lst@fmtNoindent{\global\advance\lst@fmtcurrindent-\lst@fmtindent}
%    \end{macrocode}
% \end{macro}
% \end{macro}
% \end{lstkey}
%
% \begin{macro}{\lst@fmtEnsureNewLine}
%    \begin{macrocode}
\gdef\lst@fmtEnsureNewLine{%
    \global\advance\lst@newlines\@ne
    \global\advance\lst@newlinesensured\@ne
    \lst@fmtignoretrue}
%    \end{macrocode}
%    \begin{macrocode}
\lst@AddToAtTop\lst@DoNewLines{%
    \ifnum\lst@newlines>\lst@newlinesensured
        \global\advance\lst@newlines-\lst@newlinesensured
    \fi
    \global\lst@newlinesensured\z@}
\newcount\lst@newlinesensured % global
\lst@AddToHook{Init}{\global\lst@newlinesensured\z@}
%    \end{macrocode}
%    \begin{macrocode}
\gdef\lst@fmtignoretrue{\let\lst@fmtifignore\iftrue}
\gdef\lst@fmtignorefalse{\let\lst@fmtifignore\iffalse}
\lst@AddToHook{InitVars}{\lst@fmtignorefalse}
\lst@AddToHook{Output}{\lst@fmtignorefalse}
%    \end{macrocode}
% \end{macro}
%
% \begin{macro}{\lst@fmtUseLostSpace}
%    \begin{macrocode}
\gdef\lst@fmtUseLostSpace{%
    \lst@ifnewline \kern\lst@fmtcurrindent \global\lst@lostspace\z@
    \else
        \lst@OldOLS
    \fi}
\lst@AddToHook{Init}
    {\lst@true
     \ifx\lst@fmtformat\@empty \ifx\lst@fmt\@empty \lst@false \fi\fi
     \lst@if
        \let\lst@OldOLS\lst@OutputLostSpace
        \let\lst@OutputLostSpace\lst@fmtUseLostSpace
        \let\lst@ProcessSpace\lst@fmtProcessSpace
     \fi}
%    \end{macrocode}
% \begin{TODO}
% This `lost space' doesn't use |\lst@alloverstyle| yet!
% \end{TODO}
% \end{macro}
%
% \begin{macro}{\lst@fmtProcessSpace}
%    \begin{macrocode}
\gdef\lst@fmtProcessSpace{%
    \lst@ifletter
        \lst@Output
        \lst@fmtifignore\else
            \lst@AppendOther\lst@outputspace
        \fi
    \else \lst@ifkeepspaces
        \lst@AppendOther\lst@outputspace
    \else \ifnum\lst@newlines=\z@
        \lst@AppendSpecialSpace
    \else \ifnum\lst@length=\z@
            \global\advance\lst@lostspace\lst@width
            \global\advance\lst@pos\m@ne
        \else
            \lst@AppendSpecialSpace
        \fi
    \fi \fi \fi
    \lst@whitespacetrue}
%    \end{macrocode}
% \end{macro}
%
%
% \paragraph{Formatting identifiers}
%
% \begin{macro}{\lst@fmtIdentifier}
% We install a (keyword) test for the `format identifiers'.
%    \begin{macrocode}
\lst@InstallTest{f}
    \lst@fmt@list\lst@fmt \lst@gfmt@list\lst@gfmt
    \lst@gfmt@wp
    wd
\gdef\lst@fmt@list{\lst@fmt\lst@gfmt}\global\let\lst@fmt\@empty
\gdef\lst@gfmt@list{\lst@fmt\lst@gfmt}\global\let\lst@gfmt\@empty
%    \end{macrocode}
% The working procedure expands |\lst@fmt$|\meta{string} (and defines
% |\lst@PrintToken| to do nothing).
%    \begin{macrocode}
\gdef\lst@gfmt@wp{%
    \begingroup \let\lst@UM\@empty
    \let\lst@PrintToken\@empty
    \csname\@lst @fmt$\the\lst@token\endcsname
    \endgroup}
%    \end{macrocode}
% This control sequence is probably defined as `working identifier'.
%    \begin{macrocode}
\gdef\lst@fmtIdentifier#1#2#3#4{%
    \lst@DefOther\lst@fmta{#2}\edef\lst@fmt{\lst@fmt,\lst@fmta}%
    \@namedef{\@lst @fmt$\lst@fmta}{#3#4}}
%    \end{macrocode}
% |\lst@fmt$|\meta{identifier} expands to a |\lst@fmtPre|/|\lst@fmtPost|
% sequence defined by |#2| and |#3|.
% \end{macro}
%
%    \begin{macrocode}
\lst@EndAspect
%</misc>
%    \end{macrocode}
% \end{aspect}
%
%
%
% \subsection{Line numbers}
%
% \begin{aspect}{labels}
% \lsthelper{Rolf~Niepraschk}{1997/04/24}{line numbers} asked for line numbers.
%    \begin{macrocode}
%<*misc>
\lst@BeginAspect{labels}
%    \end{macrocode}
%
% \begin{lstkey}{numbers}
% Depending on the argument we define |\lst@PlaceNumber| to print the line
% number.
%    \begin{macrocode}
\lst@Key{numbers}{none}{%
    \let\lst@PlaceNumber\@empty
    \lstKV@SwitchCases{#1}%
    {none&\\%
     left&\def\lst@PlaceNumber{\llap{\normalfont
                \lst@numberstyle{\thelstnumber}\kern\lst@numbersep}}\\%
     right&\def\lst@PlaceNumber{\rlap{\normalfont
                \kern\linewidth \kern\lst@numbersep
                \lst@numberstyle{\thelstnumber}}}%
    }{\PackageError{Listings}{Numbers #1 unknown}\@ehc}}
%    \end{macrocode}
% \end{lstkey}
%
% \begin{lstkey}{numberstyle}
% \begin{lstkey}{numbersep}
% \begin{lstkey}{stepnumber}
% \begin{lstkey}{numberblanklines}
% \begin{lstkey}{numberfirstline}
% Definition of the keys.
%    \begin{macrocode}
\lst@Key{numberstyle}{}{\def\lst@numberstyle{#1}}
\lst@Key{numbersep}{10pt}{\def\lst@numbersep{#1}}
\lst@Key{stepnumber}{1}{\def\lst@stepnumber{#1\relax}}
\lst@AddToHook{EmptyStyle}{\let\lst@stepnumber\@ne}
%    \end{macrocode}
%    \begin{macrocode}
\lst@Key{numberblanklines}{true}[t]
    {\lstKV@SetIf{#1}\lst@ifnumberblanklines}
\lst@Key{numberfirstline}{f}[t]{\lstKV@SetIf{#1}\lst@ifnumberfirstline}
\gdef\lst@numberfirstlinefalse{\let\lst@ifnumberfirstline\iffalse}
%    \end{macrocode}
% \end{lstkey}
% \end{lstkey}
% \end{lstkey}
% \end{lstkey}
% \end{lstkey}
%
% \begin{lstkey}{firstnumber}
% We select the first number according to the argument.
%    \begin{macrocode}
\lst@Key{firstnumber}{auto}{%
    \lstKV@SwitchCases{#1}%
    {auto&\let\lst@firstnumber\@undefined\\%
     last&\let\lst@firstnumber\c@lstnumber
    }{\def\lst@firstnumber{#1\relax}}}
\lst@AddToHook{PreSet}{\let\lst@advancenumber\z@}
%    \end{macrocode}
% |\lst@firstnumber| now set to |\lst@lineno| instead of |\lst@firstline|,
% as per changes in |lstpatch.sty| from 1.3b pertaining to linerange markers.
%    \begin{macrocode}
\lst@AddToHook{PreInit}
    {\ifx\lst@firstnumber\@undefined
         \def\lst@firstnumber{\lst@lineno}%
     \fi}
%    \end{macrocode}
% \end{lstkey}
%
% \begin{macro}{\lst@SetFirstNumber}
% \begin{macro}{\lst@SaveFirstNumber}
% \lsthelper{Boris~Veytsman}{1998/03/25}{continue line numbering: a.c b.c a.c}
% proposed to continue line numbers according to listing names. We define the
% label number of the first printing line here. A bug reported by
% \lsthelper{Jens~Schwarzer}{2001/05/29}{wrong line numbering of lstlisting
% with first>1} has been removed by replacing |\@ne| by |\lst@firstline|.
%    \begin{macrocode}
\gdef\lst@SetFirstNumber{%
    \ifx\lst@firstnumber\@undefined
        \@tempcnta 0\csname\@lst no@\lst@intname\endcsname\relax
        \ifnum\@tempcnta=\z@ \@tempcnta\lst@firstline
                       \else \lst@nololtrue \fi
        \advance\@tempcnta\lst@advancenumber
        \edef\lst@firstnumber{\the\@tempcnta\relax}%
    \fi}
%    \end{macrocode}
% The current label is stored in|\lstno@|\meta{name}. If the name is empty,
% we use a space instead, which leaves |\lstno@| undefined.
%    \begin{macrocode}
\gdef\lst@SaveFirstNumber{%
    \expandafter\xdef
        \csname\@lst no\ifx\lst@intname\@empty @ \else @\lst@intname\fi
        \endcsname{\the\c@lstnumber}}
%    \end{macrocode}
% \end{macro}
% \end{macro}
%
% \begin{macro}{\c@lstnumber}
% This counter keeps the current label number. We use it as current label to
% make line numbers referenced by |\ref|. This was proposed by
% \lsthelper{Boris~Veytsman}{1998/03/25}{make line numbers referenced via
% \label and \ref}. We now use |\refstepcounter| to do the job---thanks to a
% bug report from \lsthelper{Christian~Gudrian}{2000/11/13}{\ref{lst:line}
% jumps to top of listing and not to the line}.
%    \begin{macrocode}
\newcounter{lstnumber}% \global
\global\c@lstnumber\@ne % init
\renewcommand*\thelstnumber{\@arabic\c@lstnumber}
\lst@AddToHook{EveryPar}
    {\global\advance\c@lstnumber\lst@advancelstnum
     \global\advance\c@lstnumber\m@ne \refstepcounter{lstnumber}%
     \lst@SkipOrPrintLabel}%
\global\let\lst@advancelstnum\@ne
%    \end{macrocode}
% Note that the counter advances \emph{before} the label is printed and not
% afterwards. Otherwise we have wrong references---reported by
% \lsthelper{Gregory~Van~Vooren}{1999/06/04}{reference one unit too large}.
%    \begin{macrocode}
\lst@AddToHook{Init}{\def\@currentlabel{\thelstnumber}}
%    \end{macrocode}
% The label number is initialized and we ensure correct line numbers for
% continued listings.  An apparently-extraneous advancement of the line
% number by \verb|-\lst@advancelstnum| when \texttt{firstnumber=last} is
% specified was removed, following a bug report by \lsthelper{Joachim~Breitner}%
% {2006/05/14}{failure to continue counting correctly}.
%    \begin{macrocode}
\lst@AddToHook{InitVars}
    {\global\c@lstnumber\lst@firstnumber
     \global\advance\c@lstnumber\lst@advancenumber
     \global\advance\c@lstnumber-\lst@advancelstnum}
\lst@AddToHook{ExitVars}
    {\global\advance\c@lstnumber\lst@advancelstnum}
%    \end{macrocode}
% \lsthelper{Walter~E.~Brown}{2001/05/22}{pdftex 3.14159-14f warning:
% destination with the same identifier} reported problems with pdftex and
% \packagename{hyperref}. A bad default of |\theHlstlabel| was the reason.
% \lsthelper{Heiko~Oberdiek}{2001/11/08}{pdftex warning: destination with
% the same identifier} found another bug which was due to the localization
% of |\lst@neglisting|. He also provided the following fix, replacing
% |\thelstlisting| with the |\ifx| \ldots\ |\fi| construction.
%    \begin{macrocode}
\AtBeginDocument{%
    \def\theHlstnumber{\ifx\lst@@caption\@empty \lst@neglisting
                                          \else \thelstlisting \fi
                       .\thelstnumber}}
%    \end{macrocode}
% \end{macro}
%
% \begin{macro}{\lst@skipnumbers}
% There are more things to do. We calculate how many lines must skip their
% label. The formula is
%	$$|\lst@skipnumbers|=
%		\textrm{\emph{first printing line}}\bmod|\lst@stepnumber|.$$
% Note that we use a nonpositive representative for |\lst@skipnumbers|.
%    \begin{macrocode}
\newcount\lst@skipnumbers % \global
\lst@AddToHook{Init}
    {\ifnum \z@>\lst@stepnumber
         \let\lst@advancelstnum\m@ne
         \edef\lst@stepnumber{-\lst@stepnumber}%
     \fi
     \ifnum \z@<\lst@stepnumber
         \global\lst@skipnumbers\lst@firstnumber
         \global\divide\lst@skipnumbers\lst@stepnumber
         \global\multiply\lst@skipnumbers-\lst@stepnumber
         \global\advance\lst@skipnumbers\lst@firstnumber
         \ifnum\lst@skipnumbers>\z@
             \global\advance\lst@skipnumbers -\lst@stepnumber
         \fi
%    \end{macrocode}
% If |\lst@stepnumber| is zero, no line numbers are printed:
%    \begin{macrocode}
     \else
         \let\lst@SkipOrPrintLabel\relax
     \fi}
%    \end{macrocode}
% \end{macro}
%
% \begin{macro}{\lst@SkipOrPrintLabel}
% But default is this. We use the fact that |\lst@skipnumbers| is nonpositive.
% The counter advances every line and if that counter is zero, we print a line
% number and decrement the counter by |\lst@stepnumber|.
%    \begin{macrocode}
\gdef\lst@SkipOrPrintLabel{%
    \ifnum\lst@skipnumbers=\z@
        \global\advance\lst@skipnumbers-\lst@stepnumber\relax
        \lst@PlaceNumber
        \lst@numberfirstlinefalse
    \else
%    \end{macrocode}
% If the first line of a listing should get a number, it gets it here.
%    \begin{macrocode}
        \lst@ifnumberfirstline
            \lst@PlaceNumber
            \lst@numberfirstlinefalse
        \fi
    \fi
    \global\advance\lst@skipnumbers\@ne}%
%    \end{macrocode}
%    \begin{macrocode}
\lst@AddToHook{OnEmptyLine}{%
    \lst@ifnumberblanklines\else \ifnum\lst@skipnumbers=\z@
        \global\advance\lst@skipnumbers-\lst@stepnumber\relax
    \fi\fi}
%    \end{macrocode}
% \end{macro}
%
%    \begin{macrocode}
\lst@EndAspect
%</misc>
%    \end{macrocode}
% \end{aspect}
%
%
% \subsection{Line shape and line breaking}
%
% \begin{macro}{\lst@parshape}
% We define a default version of |\lst@parshape| for the case that the
% \aspectname{lineshape} aspect is not loaded. We use this parshape every line
% (in fact every paragraph). Furthermore we must repeat the parshape if we
% close a group level---or the shape is forgotten.
%    \begin{macrocode}
%<*kernel>
\def\lst@parshape{\parshape\@ne \z@ \linewidth}
\lst@AddToHookAtTop{EveryLine}{\lst@parshape}
\lst@AddToHookAtTop{EndGroup}{\lst@parshape}
%</kernel>
%    \end{macrocode}
% \end{macro}
%
% \begin{aspect}{lineshape}
% Our first aspect in this section.
%    \begin{macrocode}
%<*misc>
\lst@BeginAspect{lineshape}
%    \end{macrocode}
%
% \begin{lstkey}{xleftmargin}
% \begin{lstkey}{xrightmargin}
% \begin{lstkey}{resetmargins}
% \begin{lstkey}{linewidth}
% Usual stuff.
%    \begin{macrocode}
\lst@Key{xleftmargin}{\z@}{\def\lst@xleftmargin{#1}}
\lst@Key{xrightmargin}{\z@}{\def\lst@xrightmargin{#1}}
\lst@Key{resetmargins}{false}[t]{\lstKV@SetIf{#1}\lst@ifresetmargins}
%    \end{macrocode}
% The margins become zero if we make an exact box around the listing.
%    \begin{macrocode}
\lst@AddToHook{BoxUnsafe}{\let\lst@xleftmargin\z@
                          \let\lst@xrightmargin\z@}
\lst@AddToHook{TextStyle}{%
    \let\lst@xleftmargin\z@ \let\lst@xrightmargin\z@
    \let\lst@ifresetmargins\iftrue}
%    \end{macrocode}
% Added above hook after bug report from \lsthelper{Magnus~Lewis-Smith}
%{1999/08/06}{|\lstinline| indented} and \lsthelper{Jos\'e~Romildo~Malaquias}
%{2000/08/22}{|\lstinline| indented (resetmargins)} respectively.
%    \begin{macrocode}
\lst@Key{linewidth}\linewidth{\def\lst@linewidth{#1}}
\lst@AddToHook{PreInit}{\linewidth\lst@linewidth\relax}
%    \end{macrocode}
% \end{lstkey}
% \end{lstkey}
% \end{lstkey}
% \end{lstkey}
%
% \begin{macro}{\lst@parshape}
% The definition itself is easy.
%    \begin{macrocode}
\gdef\lst@parshape{%
    \parshape\@ne \@totalleftmargin \linewidth}
%    \end{macrocode}
% We calculate the line width and (inner/outer) indent for a listing.
%    \begin{macrocode}
\lst@AddToHook{Init}
    {\lst@ifresetmargins
         \advance\linewidth\@totalleftmargin
         \advance\linewidth\rightmargin
         \@totalleftmargin\z@
     \fi
     \advance\linewidth-\lst@xleftmargin
     \advance\linewidth-\lst@xrightmargin
     \advance\@totalleftmargin\lst@xleftmargin\relax}
%    \end{macrocode}
% \end{macro}
%
% \begin{lstkey}{lineskip}
% The introduction of this key is due to communication with
% \lsthelper{Andreas~Bartelt}{1997/09/11}{problem with redefed \parskip;
% \lstlineskip introduced}. Version 1.0 implements this feature by
% redefining |\baselinestretch|.
%    \begin{macrocode}
\lst@Key{lineskip}{\z@}{\def\lst@lineskip{#1\relax}}
\lst@AddToHook{Init}
    {\parskip\z@
     \ifdim\z@=\lst@lineskip\else
         \@tempdima\baselineskip
         \advance\@tempdima\lst@lineskip
%    \end{macrocode}
% The following three lines simulate the `bad' |\divide| |\@tempdima|
% |\strip@pt| |\baselineskip| |\relax|. Thanks to \lsthelper{Peter~Bartke}
% {2002/04/10}{bad use of \strip@pt} for the bug report.
%    \begin{macrocode}
         \multiply\@tempdima\@cclvi
         \divide\@tempdima\baselineskip\relax
         \multiply\@tempdima\@cclvi
%    \end{macrocode}
%    \begin{macrocode}
         \edef\baselinestretch{\strip@pt\@tempdima}%
         \selectfont
     \fi}
%    \end{macrocode}
% \end{lstkey}
%
% \begin{lstkey}{breaklines}
% \begin{lstkey}{breakindent}
% \begin{lstkey}{breakautoindent}
% \begin{lstkey}{breakatwhitespace}
% \begin{lstkey}{prebreak}
% \begin{lstkey}{postbreak}
% As usual we have no problems in announcing more keys.
% \keyname{breakatwhitespace} is due to \lsthelper{Javier~Bezos}{2003/09/23}
% {breaklines breaks at odd places}. Unfortunately a previous definition of
% that key was wrong as \lsthelper{Franz~Rinnerthaler}{2004/03/12}
% {breakatwhitespace has no effect} and \lsthelper{Ulrike~Fischer}{2004/07/11}
% {breakatwhitespace has no effect} reported.
%    \begin{macrocode}
\lst@Key{breaklines}{false}[t]{\lstKV@SetIf{#1}\lst@ifbreaklines}
\lst@Key{breakindent}{20pt}{\def\lst@breakindent{#1}}
\lst@Key{breakautoindent}{t}[t]{\lstKV@SetIf{#1}\lst@ifbreakautoindent}
\lst@Key{breakatwhitespace}{false}[t]%
    {\lstKV@SetIf{#1}\lst@ifbreakatwhitespace}
\lst@Key{prebreak}{}{\def\lst@prebreak{#1}}
\lst@Key{postbreak}{}{\def\lst@postbreak{#1}}
%    \end{macrocode}
% We assign some different macros and (if necessary) suppress ``underfull
% |\hbox|'' messages (and use different pretolerance):
%    \begin{macrocode}
\lst@AddToHook{Init}
    {\lst@ifbreaklines
         \hbadness\@M \pretolerance\@M
         \@rightskip\@flushglue \rightskip\@rightskip % \raggedright
         \leftskip\z@skip \parindent\z@
%    \end{macrocode}
% A |\raggedright| above has been replaced by setting the values by hand after
% a bug report from \lsthelper{Morten~H\o gholm}{2004/09/06}{ltugboat.cls and
% listings}.
%
% We use the normal parshape and the calculated |\lst@breakshape| (see below).
%    \begin{macrocode}
         \def\lst@parshape{\parshape\tw@ \@totalleftmargin\linewidth
                           \lst@breakshape}%
     \else
         \let\lst@discretionary\@empty
     \fi}
\lst@AddToHook{OnNewLine}
    {\lst@ifbreaklines \lst@breakNewLine \fi}
%    \end{macrocode}
% \end{lstkey}\end{lstkey}\end{lstkey}\end{lstkey}
% \end{lstkey}\end{lstkey}
%
% \begin{macro}{\lst@discretionary}
% \begin{macro}{\lst@spacekern}
% Here comes the whole magic: We set a discretionary break after each `output
% unit'. However we redefine |\space| to be used inside |\discretionary| and
% use \hookname{EveryLine} hook. After a bug report by \lsthelper{Carsten~Hamm}
% {2002/04/19}{wrong frame rules with breaklines and xleftmargin>0pt} I've
% added |\kern-\lst@xleftmargin|, which became |\kern-\@totalleftmargin| after
% a bug report by \lsthelper{Christian~Kaiser}{2002/12/13}{wrong frame inside
% itemize with breaklines=true}.
%    \begin{macrocode}
\gdef\lst@discretionary{%
    \lst@ifbreakatwhitespace
        \lst@ifwhitespace \lst@@discretionary \fi
    \else
        \lst@@discretionary
    \fi}%
\gdef\lst@@discretionary{%
    \discretionary{\let\space\lst@spacekern\lst@prebreak}%
                  {\llap{\lsthk@EveryLine
                   \kern\lst@breakcurrindent \kern-\@totalleftmargin}%
                   \let\space\lst@spacekern\lst@postbreak}{}}
\lst@AddToHook{PostOutput}{\lst@discretionary}
\gdef\lst@spacekern{\kern\lst@width}
%    \end{macrocode}
% \begin{ALTERNATIVE}
% |\penalty\@M \hskip\z@ plus 1fil \penalty0\hskip\z@ plus-1fil| \emph{before}
% each `output unit' (i.e.~before |\hbox{...}| in the output macros) also break
% the lines as desired. But we wouldn't have |prebreak| and |postbreak|.
% \end{ALTERNATIVE}
% \end{macro}\end{macro}
%
% \begin{macro}{\lst@breakNewLine}
% We use \keyname{breakindent}, and additionally the current line indention
% (coming from white spaces at the beginning of the line) if `auto indent' is
% on.
%    \begin{macrocode}
\gdef\lst@breakNewLine{%
    \@tempdima\lst@breakindent\relax
    \lst@ifbreakautoindent \advance\@tempdima\lst@lostspace \fi
%    \end{macrocode}
% Now we calculate the margin and line width of the wrapped part \ldots
%    \begin{macrocode}
    \@tempdimc-\@tempdima \advance\@tempdimc\linewidth
                          \advance\@tempdima\@totalleftmargin
%    \end{macrocode}
% \ldots\space and store it in |\lst@breakshape|.
%    \begin{macrocode}
    \xdef\lst@breakshape{\noexpand\lst@breakcurrindent \the\@tempdimc}%
    \xdef\lst@breakcurrindent{\the\@tempdima}}
\global\let\lst@breakcurrindent\z@ % init
%    \end{macrocode}
% The initialization of |\lst@breakcurrindent| has been added after a bug
% report by \lsthelper{Alvaro~Herrera}{2002/12/09}{`undefined control
% sequence \lst@breakcurrindent' with fancyvrb and breaklines}.
% \begin{TODO}
% We could speed this up by allocating two global dimensions.
% \end{TODO}
% \end{macro}
%
% \begin{macro}{\lst@breakshape}
% \lsthelper{Andreas~Deininger}{2000/08/25}{`breaklines,first>1' leads to
% ``undefined control sequence'' error} reported a problem which is resolved
% by providing a default break shape.
%    \begin{macrocode}
\gdef\lst@breakshape{\@totalleftmargin \linewidth}
%    \end{macrocode}
% \end{macro}
%
% \begin{macro}{\lst@breakProcessOther}
% is the same as |\lst@ProcessOther| except that it also outputs the current
% token string. This inserts a potential linebreak point.
% Only the closing parenthesis uses this macro yet.
%    \begin{macrocode}
\gdef\lst@breakProcessOther#1{\lst@ProcessOther#1\lst@OutputOther}
\lst@AddToHook{SelectCharTable}
    {\lst@ifbreaklines \lst@Def{`)}{\lst@breakProcessOther)}\fi}
%    \end{macrocode}
% A bug reported by \lsthelper{Gabriel~Tauro}{2001/04/18}{unexpected `)' if
% the character appears before first printed line} has been removed by using
% |\lst@ProcessOther| instead of |\lst@AppendOther|.
% \end{macro}
%
%    \begin{macrocode}
\lst@EndAspect
%</misc>
%    \end{macrocode}
% \end{aspect}
%
%
% \subsection{Frames}
%
% \begin{aspect}{frames}
% Another aspect.
%    \begin{macrocode}
%<*misc>
\lst@BeginAspect[lineshape]{frames}
%    \end{macrocode}
%
% \begin{lstkey}{framexleftmargin}
% \begin{lstkey}{framexrightmargin}
% \begin{lstkey}{framextopmargin}
% \begin{lstkey}{framexbottommargin}
% These keys just save the argument.
%    \begin{macrocode}
\lst@Key{framexleftmargin}{\z@}{\def\lst@framexleftmargin{#1}}
\lst@Key{framexrightmargin}{\z@}{\def\lst@framexrightmargin{#1}}
\lst@Key{framextopmargin}{\z@}{\def\lst@framextopmargin{#1}}
\lst@Key{framexbottommargin}{\z@}{\def\lst@framexbottommargin{#1}}
%    \end{macrocode}
% \end{lstkey}
% \end{lstkey}
% \end{lstkey}
% \end{lstkey}
%
% \begin{lstkey}{backgroundcolor}
% \lsthelper{Ralf~Imh\"auser}{2000/01/08}{coloured background} inspired the
% key \keyname{backgroundcolor}. All keys save the argument, and \ldots
%    \begin{macrocode}
\lst@Key{backgroundcolor}{}{\def\lst@bkgcolor{#1}}
\lst@Key{fillcolor}{}{\def\lst@fillcolor{#1}}
\lst@Key{rulecolor}{}{\def\lst@rulecolor{#1}}
\lst@Key{rulesepcolor}{}{\def\lst@rulesepcolor{#1}}
%    \end{macrocode}
% \ldots\space some have default settings if they are empty.
%    \begin{macrocode}
\lst@AddToHook{Init}{%
    \ifx\lst@fillcolor\@empty
        \let\lst@fillcolor\lst@bkgcolor
    \fi
    \ifx\lst@rulesepcolor\@empty
        \let\lst@rulesepcolor\lst@fillcolor
    \fi}
%    \end{macrocode}
% \end{lstkey}
%
% \begin{lstkey}{rulesep}
% \begin{lstkey}{framerule}
% \begin{lstkey}{framesep}
% \begin{lstkey}{frameshape}
% Another set of keys, which mainly save their respective argument.
% \keyname{frameshape} capitalizes all letters, and checks whether at least one
% round corner is specified. Eventually we define |\lst@frame| to be empty if
% and only if there is no frameshape.
%    \begin{macrocode}
\lst@Key{rulesep}{2pt}{\def\lst@rulesep{#1}}
\lst@Key{framerule}{.4pt}{\def\lst@framerulewidth{#1}}
\lst@Key{framesep}{3pt}{\def\lst@frametextsep{#1}}
\lst@Key{frameshape}{}{%
    \let\lst@xrulecolor\@empty
    \lstKV@FourArg{#1}%
    {\uppercase{\def\lst@frametshape{##1}}%
     \uppercase{\def\lst@framelshape{##2}}%
     \uppercase{\def\lst@framershape{##3}}%
     \uppercase{\def\lst@framebshape{##4}}%
     \let\lst@ifframeround\iffalse
     \lst@IfSubstring R\lst@frametshape{\let\lst@ifframeround\iftrue}{}%
     \lst@IfSubstring R\lst@framebshape{\let\lst@ifframeround\iftrue}{}%
     \def\lst@frame{##1##2##3##4}}}
%    \end{macrocode}
% \end{lstkey}
% \end{lstkey}
% \end{lstkey}
% \end{lstkey}
%
% \begin{lstkey}{frameround}
% \begin{lstkey}{frame}
% We have to do some conversion here.
%    \begin{macrocode}
\lst@Key{frameround}\relax
    {\uppercase{\def\lst@frameround{#1}}%
     \expandafter\lstframe@\lst@frameround ffff\relax}
\global\let\lst@frameround\@empty
%    \end{macrocode}
% In case of an verbose argument, we use the |trbl|-subset replacement.
%    \begin{macrocode}
\lst@Key{frame}\relax{%
    \let\lst@xrulecolor\@empty
    \lstKV@SwitchCases{#1}%
    {none&\let\lst@frame\@empty\\%
     leftline&\def\lst@frame{l}\\%
     topline&\def\lst@frame{t}\\%
     bottomline&\def\lst@frame{b}\\%
     lines&\def\lst@frame{tb}\\%
     single&\def\lst@frame{trbl}\\%
     shadowbox&\def\lst@frame{tRBl}%
            \def\lst@xrulecolor{\lst@rulesepcolor}%
            \def\lst@rulesep{\lst@frametextsep}%
    }{\def\lst@frame{#1}}%
    \expandafter\lstframe@\lst@frameround ffff\relax}
%    \end{macrocode}
% Adding |t|, |r|, |b|, and |l| in case of their upper case versions makes
% later tests easier.
%    \begin{macrocode}
\gdef\lstframe@#1#2#3#4#5\relax{%
    \lst@IfSubstring T\lst@frame{\edef\lst@frame{t\lst@frame}}{}%
    \lst@IfSubstring R\lst@frame{\edef\lst@frame{r\lst@frame}}{}%
    \lst@IfSubstring B\lst@frame{\edef\lst@frame{b\lst@frame}}{}%
    \lst@IfSubstring L\lst@frame{\edef\lst@frame{l\lst@frame}}{}%
%    \end{macrocode}
% We now check top and bottom frame rules, \ldots
%    \begin{macrocode}
    \let\lst@frametshape\@empty \let\lst@framebshape\@empty
    \lst@frameCheck
        ltr\lst@framelshape\lst@frametshape\lst@framershape #4#1%
    \lst@frameCheck
        LTR\lst@framelshape\lst@frametshape\lst@framershape #4#1%
    \lst@frameCheck
        lbr\lst@framelshape\lst@framebshape\lst@framershape #3#2%
    \lst@frameCheck
        LBR\lst@framelshape\lst@framebshape\lst@framershape #3#2%
%    \end{macrocode}
% \ldots\space look for round corners \ldots
%    \begin{macrocode}
    \let\lst@ifframeround\iffalse
    \lst@IfSubstring R\lst@frametshape{\let\lst@ifframeround\iftrue}{}%
    \lst@IfSubstring R\lst@framebshape{\let\lst@ifframeround\iftrue}{}%
%    \end{macrocode}
% and define left and right frame shape.
%    \begin{macrocode}
    \let\lst@framelshape\@empty \let\lst@framershape\@empty
    \lst@IfSubstring L\lst@frame
        {\def\lst@framelshape{YY}}%
        {\lst@IfSubstring l\lst@frame{\def\lst@framelshape{Y}}{}}%
    \lst@IfSubstring R\lst@frame
        {\def\lst@framershape{YY}}%
        {\lst@IfSubstring r\lst@frame{\def\lst@framershape{Y}}{}}}
%    \end{macrocode}
% Now comes the macro used to define top and bottom frame shape.
% It extends the macro |#5|.
% The last two arguments show whether left and right corners are round.
% |#4| and |#6| are temporary macros.
% |#1#2#3| are the three characters we test for.
%    \begin{macrocode}
\gdef\lst@frameCheck#1#2#3#4#5#6#7#8{%
    \lst@IfSubstring #1\lst@frame
        {\if #7T\def#4{R}\else \def#4{Y}\fi}%
        {\def#4{N}}%
    \lst@IfSubstring #3\lst@frame
        {\if #8T\def#6{R}\else \def#6{Y}\fi}%
        {\def#6{N}}%
    \lst@IfSubstring #2\lst@frame{\edef#5{#5#4Y#6}}{}}
%    \end{macrocode}
% For text style listings all frames and the background color are
% deactivated -- added after bug reports by \lsthelper{Stephen~Reindl}%
% {2002/06/04}{frames not deactivated for text style listings} and
% \lsthelper{Thomas~ten~Cate}{2006/07/14}{inline listings get background
% color after a line break}
%    \begin{macrocode}
\lst@AddToHook{TextStyle}
   {\let\lst@frame\@empty
    \let\lst@frametshape\@empty
    \let\lst@framershape\@empty
    \let\lst@framebshape\@empty
    \let\lst@framelshape\@empty
    \let\lst@bkgcolor\@empty}
%    \end{macrocode}
% \end{lstkey}
% \end{lstkey}
%
% As per a bug report by \lsthelper{Ignacio~Fern\'andez~Galv\'an}{2006/07/26}%
% {Frame with background color has slight hole on left side}, the small section
% of background color to the left of the margin is now drawn before the left
% side of the frame is drawn, so that they overlap correctly in Acrobat.
%
% \begin{macro}{\lst@frameMakeVBox}
%    \begin{macrocode}
\gdef\lst@frameMakeBoxV#1#2#3{%
    \setbox#1\hbox{%
      \color@begingroup \lst@rulecolor
      \ifx\lst@framelshape\@empty
      \else
            \llap{%
                \lst@frameBlock\lst@fillcolor\lst@frametextsep{#2}{#3}%
                \kern\lst@framexleftmargin}%
      \fi
      \llap{\setbox\z@\hbox{\vrule\@width\z@\@height#2\@depth#3%
                            \lst@frameL}%
            \rlap{\lst@frameBlock\lst@rulesepcolor{\wd\z@}%
                                                  {\ht\z@}{\dp\z@}}%
            \box\z@
            \kern\lst@frametextsep\relax
            \kern\lst@framexleftmargin}%
      \rlap{\kern-\lst@framexleftmargin
                    \@tempdima\linewidth
            \advance\@tempdima\lst@framexleftmargin
            \advance\@tempdima\lst@framexrightmargin
            \lst@frameBlock\lst@bkgcolor\@tempdima{#2}{#3}%
            \ifx\lst@framershape\@empty
                \kern\lst@frametextsep\relax
            \else
                \lst@frameBlock\lst@fillcolor\lst@frametextsep{#2}{#3}%
            \fi
            \setbox\z@\hbox{\vrule\@width\z@\@height#2\@depth#3%
                            \lst@frameR}%
            \rlap{\lst@frameBlock\lst@rulesepcolor{\wd\z@}%
                                                  {\ht\z@}{\dp\z@}}%
            \box\z@}%
      \color@endgroup}}
%    \end{macrocode}
% \end{macro}
%
% \begin{macro}{\lst@frameBlock}
%    \begin{macrocode}
\gdef\lst@frameBlock#1#2#3#4{%
    \color@begingroup
      #1%
      \setbox\z@\hbox{\vrule\@height#3\@depth#4%
                      \ifx#1\@empty \@width\z@ \kern#2\relax
                              \else \@width#2\relax \fi}%
      \box\z@
    \color@endgroup}
%    \end{macrocode}
% \end{macro}
%
% \begin{macro}{\lst@frameR}
% typesets right rules.
% We only need to iterate through |\lst@framershape|.
%    \begin{macrocode}
\gdef\lst@frameR{%
    \expandafter\lst@frameR@\lst@framershape\relax
    \kern-\lst@rulesep}
\gdef\lst@frameR@#1{%
    \ifx\relax#1\@empty\else
        \if #1Y\lst@framevrule \else \kern\lst@framerulewidth \fi
        \kern\lst@rulesep
        \expandafter\lst@frameR@b
    \fi}
\gdef\lst@frameR@b#1{%
    \ifx\relax#1\@empty
    \else
        \if #1Y\color@begingroup
               \lst@xrulecolor
               \lst@framevrule
               \color@endgroup
        \else
               \kern\lst@framerulewidth
        \fi
        \kern\lst@rulesep
        \expandafter\lst@frameR@
    \fi}
%    \end{macrocode}
% \end{macro}
%
% \begin{macro}{\lst@frameL}
% Ditto left rules.
%    \begin{macrocode}
\gdef\lst@frameL{%
    \kern-\lst@rulesep
    \expandafter\lst@frameL@\lst@framelshape\relax}
\gdef\lst@frameL@#1{%
    \ifx\relax#1\@empty\else
        \kern\lst@rulesep
        \if#1Y\lst@framevrule \else \kern\lst@framerulewidth \fi
        \expandafter\lst@frameL@
    \fi}
%    \end{macrocode}
% \end{macro}
%
% \begin{macro}{\lst@frameH}
% This is the central macro used to draw top and bottom frame rules.
% The first argument is either |T| or |B| and the second contains the shape.
% We use |\@tempcntb| as size counter.
%    \begin{macrocode}
\gdef\lst@frameH#1#2{%
    \global\let\lst@framediml\z@ \global\let\lst@framedimr\z@
    \setbox\z@\hbox{}\@tempcntb\z@
    \expandafter\lst@frameH@\expandafter#1#2\relax\relax\relax
            \@tempdimb\lst@frametextsep\relax
    \advance\@tempdimb\lst@framerulewidth\relax
            \@tempdimc-\@tempdimb
    \advance\@tempdimc\ht\z@
    \advance\@tempdimc\dp\z@
    \setbox\z@=\hbox{%
      \lst@frameHBkg\lst@fillcolor\@tempdimb\@firstoftwo
      \if#1T\rlap{\raise\dp\@tempboxa\box\@tempboxa}%
       \else\rlap{\lower\ht\@tempboxa\box\@tempboxa}\fi
      \lst@frameHBkg\lst@rulesepcolor\@tempdimc\@secondoftwo
      \advance\@tempdimb\ht\@tempboxa
      \if#1T\rlap{\raise\lst@frametextsep\box\@tempboxa}%
       \else\rlap{\lower\@tempdimb\box\@tempboxa}\fi
      \rlap{\box\z@}%
    }}
\gdef\lst@frameH@#1#2#3#4{%
    \ifx\relax#4\@empty\else
        \lst@frameh \@tempcntb#1#2#3#4%
        \advance\@tempcntb\@ne
        \expandafter\lst@frameH@\expandafter#1%
    \fi}
\gdef\lst@frameHBkg#1#2#3{%
    \setbox\@tempboxa\hbox{%
        \kern-\lst@framexleftmargin
        #3{\kern-\lst@framediml\relax}{\@tempdima\z@}%
        \ifdim\lst@framediml>\@tempdimb
            #3{\@tempdima\lst@framediml \advance\@tempdima-\@tempdimb
               \lst@frameBlock\lst@rulesepcolor\@tempdima\@tempdimb\z@}%
              {\kern-\lst@framediml
               \advance\@tempdima\lst@framediml\relax}%
        \fi
        #3{\@tempdima\z@
           \ifx\lst@framelshape\@empty\else
               \advance\@tempdima\@tempdimb
           \fi
           \ifx\lst@framershape\@empty\else
               \advance\@tempdima\@tempdimb
           \fi}%
          {\ifdim\lst@framedimr>\@tempdimb
              \advance\@tempdima\lst@framedimr\relax
           \fi}%
        \advance\@tempdima\linewidth
        \advance\@tempdima\lst@framexleftmargin
        \advance\@tempdima\lst@framexrightmargin
        \lst@frameBlock#1\@tempdima#2\z@
        #3{\ifdim\lst@framedimr>\@tempdimb
               \@tempdima-\@tempdimb
               \advance\@tempdima\lst@framedimr\relax
               \lst@frameBlock\lst@rulesepcolor\@tempdima\@tempdimb\z@
           \fi}{}%
        }}
%    \end{macrocode}
% \end{macro}
%
% \begin{macro}{\lst@frameh}
% This is the low-level macro used to draw top and bottom frame rules.
% It \emph{adds} one rule plus corners to box 0.
% The first parameter gives the size of the corners and the second is either
% |T| or |B|.
% |#3#4#5| is a left-to-right description of the frame and is in
% $\{$\texttt{Y,N,R}$\}\times\{$\texttt{Y,N}$\}\times\{$\texttt{Y,N,R}$\}$.
% We move to the correct horizontal position, set the left corner, the
% horizontal line, and the right corner.
%    \begin{macrocode}
\gdef\lst@frameh#1#2#3#4#5{%
    \lst@frameCalcDimA#1%
    \lst@ifframeround \@getcirc\@tempdima \fi
%    \end{macrocode}
%    \begin{macrocode}
    \setbox\z@\hbox{%
      \begingroup
      \setbox\z@\hbox{%
        \kern-\lst@framexleftmargin
        \color@begingroup
        \ifnum#1=\z@ \lst@rulecolor \else \lst@xrulecolor \fi
%    \end{macrocode}
% |\lst@frameCorner| gets four arguments:
% |\llap|, |TL| or |BL|, the corner type $\in\{$\texttt{Y,N,R}$\}$, and the
% size |#1|.
%    \begin{macrocode}
        \lst@frameCornerX\llap{#2L}#3#1%
        \ifdim\lst@framediml<\@tempdimb
            \xdef\lst@framediml{\the\@tempdimb}%
        \fi
        \begingroup
        \if#4Y\else \let\lst@framerulewidth\z@ \fi
                \@tempdima\lst@framexleftmargin
        \advance\@tempdima\lst@framexrightmargin
        \advance\@tempdima\linewidth
        \vrule\@width\@tempdima\@height\lst@framerulewidth \@depth\z@
        \endgroup
        \lst@frameCornerX\rlap{#2R}#5#1%
        \ifdim\lst@framedimr<\@tempdimb
            \xdef\lst@framedimr{\the\@tempdimb}%
        \fi
        \color@endgroup}%
%    \end{macrocode}
%    \begin{macrocode}
      \if#2T\rlap{\raise\dp\z@\box\z@}%
       \else\rlap{\lower\ht\z@\box\z@}\fi
      \endgroup
      \box\z@}}
%    \end{macrocode}
% \end{macro}
%
% \begin{macro}{\lst@frameCornerX}
% typesets a single corner and returns |\@tempdimb|, the width of the corner.
%    \begin{macrocode}
\gdef\lst@frameCornerX#1#2#3#4{%
    \setbox\@tempboxa\hbox{\csname\@lst @frame\if#3RR\fi #2\endcsname}%
    \@tempdimb\wd\@tempboxa
    \if #3R%
        #1{\box\@tempboxa}%
    \else
        \if #3Y\expandafter#1\else
               \@tempdimb\z@ \expandafter\vphantom \fi
        {\box\@tempboxa}%
    \fi}
%    \end{macrocode}
% \end{macro}
%
% \begin{macro}{\lst@frameCalcDimA}
% calculates an all over width; used by |\lst@frameh| and |\lst@frameInit|.
%    \begin{macrocode}
\gdef\lst@frameCalcDimA#1{%
            \@tempdima\lst@rulesep
    \advance\@tempdima\lst@framerulewidth
    \multiply\@tempdima#1\relax
    \advance\@tempdima\lst@frametextsep
    \advance\@tempdima\lst@framerulewidth
    \multiply\@tempdima\tw@}
%    \end{macrocode}
% \end{macro}
%
% \begin{macro}{\lst@frameInit}
% First we look which frame types we have on the left and on the right.
% We speed up things if there are no vertical rules.
%    \begin{macrocode}
\lst@AddToHook{Init}{\lst@frameInit}
\newbox\lst@framebox
\gdef\lst@frameInit{%
    \ifx\lst@framelshape\@empty \let\lst@frameL\@empty \fi
    \ifx\lst@framershape\@empty \let\lst@frameR\@empty \fi
    \def\lst@framevrule{\vrule\@width\lst@framerulewidth\relax}%
%    \end{macrocode}
% We adjust values to round corners if necessary.
%    \begin{macrocode}
    \lst@ifframeround
        \lst@frameCalcDimA\z@ \@getcirc\@tempdima
        \@tempdimb\@tempdima \divide\@tempdimb\tw@
        \advance\@tempdimb -\@wholewidth
        \edef\lst@frametextsep{\the\@tempdimb}%
        \edef\lst@framerulewidth{\the\@wholewidth}%
%    \end{macrocode}
%    \begin{macrocode}
        \lst@frameCalcDimA\@ne \@getcirc\@tempdima
        \@tempdimb\@tempdima \divide\@tempdimb\tw@
        \advance\@tempdimb -\tw@\@wholewidth
        \advance\@tempdimb -\lst@frametextsep
        \edef\lst@rulesep{\the\@tempdimb}%
    \fi
%    \end{macrocode}
%    \begin{macrocode}
    \lst@frameMakeBoxV\lst@framebox{\ht\strutbox}{\dp\strutbox}%
    \def\lst@framelr{\copy\lst@framebox}%
%    \end{macrocode}
% Finally we typeset the rules (+ corners).
% We possibly need to insert negative |\vskip| to remove space between
% preceding text and top rule.
% \begin{TODO}
% Use |\vspace| instead of |\vskip|?
% \end{TODO}
%    \begin{macrocode}
    \ifx\lst@frametshape\@empty\else
        \lst@frameH T\lst@frametshape
        \ifvoid\z@\else
            \par\lst@parshape
            \@tempdima-\baselineskip \advance\@tempdima\ht\z@
            \ifdim\prevdepth<\@cclvi\p@\else
                \advance\@tempdima\prevdepth
            \fi
            \ifdim\@tempdima<\z@
                \vskip\@tempdima\vskip\lineskip
            \fi
            \noindent\box\z@\par
            \lineskiplimit\maxdimen \lineskip\z@
        \fi
        \lst@frameSpreadV\lst@framextopmargin
    \fi}
%    \end{macrocode}
% |\parshape\lst@parshape| ensures that the top rules correctly indented.
% The bug was reported by \lsthelper{Marcin~Kasperski}{1999/04/28}{top rules
% indented right inside itemize}.
%
% We typeset left and right rules every line.
%    \begin{macrocode}
\lst@AddToHook{EveryLine}{\lst@framelr}
\global\let\lst@framelr\@empty
%    \end{macrocode}
% \end{macro}
%
% \begin{macro}{\lst@frameExit}
% The rules at the bottom.
%    \begin{macrocode}
\lst@AddToHook{DeInit}
    {\ifx\lst@framebshape\@empty\else \lst@frameExit \fi}
\gdef\lst@frameExit{%
    \lst@frameSpreadV\lst@framexbottommargin
    \lst@frameH B\lst@framebshape
    \ifvoid\z@\else
        \everypar{}\par\lst@parshape\nointerlineskip\noindent\box\z@
    \fi}
%    \end{macrocode}
% \end{macro}
%
% \begin{macro}{\lst@frameSpreadV}
% sets rules for vertical spread.
%    \begin{macrocode}
\gdef\lst@frameSpreadV#1{%
    \ifdim\z@=#1\else
        \everypar{}\par\lst@parshape\nointerlineskip\noindent
        \lst@frameMakeBoxV\z@{#1}{\z@}%
        \box\z@
    \fi}
%    \end{macrocode}
% \end{macro}
%
% \begin{macro}{\lst@frameTR}
% \begin{macro}{\lst@frameBR}
% \begin{macro}{\lst@frameBL}
% \begin{macro}{\lst@frameTL}
% These macros make a vertical and horizontal rule.
% The implicit argument |\@tempdima| gives the size of two corners and is
% provided by |\lst@frameh|.
%    \begin{macrocode}
\gdef\lst@frameTR{%
    \vrule\@width.5\@tempdima\@height\lst@framerulewidth\@depth\z@
    \kern-\lst@framerulewidth
    \raise\lst@framerulewidth\hbox{%
        \vrule\@width\lst@framerulewidth\@height\z@\@depth.5\@tempdima}}
\gdef\lst@frameBR{%
    \vrule\@width.5\@tempdima\@height\lst@framerulewidth\@depth\z@
    \kern-\lst@framerulewidth
    \vrule\@width\lst@framerulewidth\@height.5\@tempdima\@depth\z@}
\gdef\lst@frameBL{%
    \vrule\@width\lst@framerulewidth\@height.5\@tempdima\@depth\z@
    \kern-\lst@framerulewidth
    \vrule\@width.5\@tempdima\@height\lst@framerulewidth\@depth\z@}
\gdef\lst@frameTL{%
    \raise\lst@framerulewidth\hbox{%
        \vrule\@width\lst@framerulewidth\@height\z@\@depth.5\@tempdima}%
    \kern-\lst@framerulewidth
    \vrule\@width.5\@tempdima\@height\lst@framerulewidth\@depth\z@}
%    \end{macrocode}
% \end{macro}\end{macro}\end{macro}\end{macro}
%
% \begin{macro}{\lst@frameRoundT}
% \begin{macro}{\lst@frameRoundB}
% are helper macros to typeset round corners. We set height and depth to
% the visible parts of the circle font.
%    \begin{macrocode}
\gdef\lst@frameRoundT{%
    \setbox\@tempboxa\hbox{\@circlefnt\char\@tempcnta}%
    \ht\@tempboxa\lst@framerulewidth
    \box\@tempboxa}
\gdef\lst@frameRoundB{%
    \setbox\@tempboxa\hbox{\@circlefnt\char\@tempcnta}%
    \dp\@tempboxa\z@
    \box\@tempboxa}
%    \end{macrocode}
% \end{macro}
% \end{macro}
%
% \begin{macro}{\lst@frameRTR}
% \begin{macro}{\lst@frameRBR}
% \begin{macro}{\lst@frameRBL}
% \begin{macro}{\lst@frameRTL}
% The round corners.
%    \begin{macrocode}
\gdef\lst@frameRTR{%
    \hb@xt@.5\@tempdima{\kern-\lst@framerulewidth
                           \kern.5\@tempdima \lst@frameRoundT \hss}}
\gdef\lst@frameRBR{%
    \hb@xt@.5\@tempdima{\kern-\lst@framerulewidth
    \advance\@tempcnta\@ne \kern.5\@tempdima \lst@frameRoundB \hss}}
\gdef\lst@frameRBL{%
    \advance\@tempcnta\tw@ \lst@frameRoundB
    \kern-.5\@tempdima}
\gdef\lst@frameRTL{%
    \advance\@tempcnta\thr@@\lst@frameRoundT
    \kern-.5\@tempdima}
%    \end{macrocode}
% \end{macro}\end{macro}\end{macro}\end{macro}
%
%    \begin{macrocode}
\lst@EndAspect
%</misc>
%    \end{macrocode}
% \end{aspect}
%
%
% \subsection{Macro use for make}
%
% \begin{aspect}{make}
% \begin{macro}{\lst@makemode}
% \begin{macro}{\lst@ifmakekey}
% If we've entered the special mode for Make, we save whether the last
% identifier has been a first order keyword.
%    \begin{macrocode}
%<*misc>
\lst@BeginAspect[keywords]{make}
%    \end{macrocode}
%    \begin{macrocode}
\lst@NewMode\lst@makemode
\lst@AddToHook{Output}{%
    \ifnum\lst@mode=\lst@makemode
        \ifx\lst@thestyle\lst@gkeywords@sty
            \lst@makekeytrue
        \fi
    \fi}
%    \end{macrocode}
%    \begin{macrocode}
\gdef\lst@makekeytrue{\let\lst@ifmakekey\iftrue}
\gdef\lst@makekeyfalse{\let\lst@ifmakekey\iffalse}
\global\lst@makekeyfalse % init
%    \end{macrocode}
% \end{macro}\end{macro}
%
% \begin{lstkey}{makemacrouse}
% adjusts the character table if necessary
%    \begin{macrocode}
\lst@Key{makemacrouse}f[t]{\lstKV@SetIf{#1}\lst@ifmakemacrouse}
%    \end{macrocode}
% \end{lstkey}
%
% \begin{macro}{\lst@MakeSCT}
% If `macro use' is on, the opening |$(| prints preceding characters, enters
% the special mode and merges the two characters with the following output.
%
%    \begin{macrocode}
\gdef\lst@MakeSCT{%
    \lst@ifmakemacrouse
        \lst@ReplaceInput{$(}{%
            \lst@PrintToken
            \lst@EnterMode\lst@makemode{\lst@makekeyfalse}%
            \lst@Merge{\lst@ProcessOther\$\lst@ProcessOther(}}%
%    \end{macrocode}
% The closing parenthesis tests for the mode and either processes |)| as usual
% or outputs it right here (in keyword style if a keyword was between |$(| and
% |)|).
%    \begin{macrocode}
        \lst@ReplaceInput{)}{%
            \ifnum\lst@mode=\lst@makemode
                \lst@PrintToken
                \begingroup
                    \lst@ProcessOther)%
                    \lst@ifmakekey
                        \let\lst@currstyle\lst@gkeywords@sty
                    \fi
                    \lst@OutputOther
                \endgroup
                \lst@LeaveMode
            \else
                \expandafter\lst@ProcessOther\expandafter)%
            \fi}%
%    \end{macrocode}
% If \keyname{makemacrouse} is off then both |$(| are just `others'.
%    \begin{macrocode}
    \else
        \lst@ReplaceInput{$(}{\lst@ProcessOther\$\lst@ProcessOther(}%
    \fi}
%    \end{macrocode}
% \end{macro}
%
%    \begin{macrocode}
\lst@EndAspect
%</misc>
%    \end{macrocode}
% \end{aspect}
%
%
% \section{Typesetting a listing}
%
% \begingroup
%    \begin{macrocode}
%<*kernel>
%    \end{macrocode}
% \endgroup
% \begin{macro}{\lst@lineno}
% \begin{lstkey}{print}
% \begin{lstkey}{firstline}
% \begin{lstkey}{lastline}
% \begin{lstkey}{linerange}
% The `current line' counter and three keys.
%    \begin{macrocode}
\newcount\lst@lineno % \global
\lst@AddToHook{InitVars}{\global\lst@lineno\@ne}
%    \end{macrocode}
%    \begin{macrocode}
\lst@Key{print}{true}[t]{\lstKV@SetIf{#1}\lst@ifprint}
\lst@Key{firstline}\relax{\def\lst@firstline{#1\relax}}
\lst@Key{lastline}\relax{\def\lst@lastline{#1\relax}}
%    \end{macrocode}
%    \begin{macrocode}
\lst@AddToHook{PreSet}
    {\let\lst@firstline\@ne \def\lst@lastline{9999999\relax}}
%    \end{macrocode}
% \end{lstkey}
% \end{lstkey}\end{lstkey}\end{lstkey}\end{macro}
% The following code is just copied from the current development version, and
% from the |lstpatch.sty| file that Carsten left in version 1.3b for doing
% line ranges with numbers and range markers.
%
% First, the options that control the line-range handling.
%    \begin{macrocode}
\lst@Key{linerange}\relax{\lstKV@OptArg[]{#1}{%
    \def\lst@interrange{##1}\def\lst@linerange{##2,}}}
\lst@Key{rangeprefix}\relax{\def\lst@rangebeginprefix{#1}%
                            \def\lst@rangeendprefix{#1}}
\lst@Key{rangesuffix}\relax{\def\lst@rangebeginsuffix{#1}%
                            \def\lst@rangeendsuffix{#1}}
\lst@Key{rangebeginprefix}{}{\def\lst@rangebeginprefix{#1}}
\lst@Key{rangebeginsuffix}{}{\def\lst@rangebeginsuffix{#1}}
\lst@Key{rangeendprefix}{}{\def\lst@rangeendprefix{#1}}
\lst@Key{rangeendsuffix}{}{\def\lst@rangeendsuffix{#1}}
\lst@Key{includerangemarker}{true}[t]{\lstKV@SetIf{#1}\lst@ifincluderangemarker}
\lst@AddToHook{PreSet}{\def\lst@firstline{1\relax}%
                       \let\lst@linerange\@empty}
\lst@AddToHook{Init}
{\ifx\lst@linerange\@empty
     \edef\lst@linerange{{\lst@firstline}-{\lst@lastline},}%
 \fi
 \lst@GetLineInterval}%
\def\lst@GetLineInterval{\expandafter\lst@GLI\lst@linerange\@nil}
\def\lst@GLI#1,#2\@nil{\def\lst@linerange{#2}\lst@GLI@#1--\@nil}
\def\lst@GLI@#1-#2-#3\@nil{%
    \lst@IfNumber{#1}%
    {\ifx\@empty#1\@empty
         \let\lst@firstline\@ne
     \else
         \def\lst@firstline{#1\relax}%
     \fi
     \ifx\@empty#3\@empty
         \def\lst@lastline{9999999\relax}%
     \else
         \ifx\@empty#2\@empty
             \let\lst@lastline\lst@firstline
         \else
             \def\lst@lastline{#2\relax}%
         \fi
     \fi}%
%    \end{macrocode}
%    If we've found a general marker, we set firstline and lastline to 9999999.
%    This prevents (almost) anything from being printed for now.
%    \begin{macrocode}
    {\def\lst@firstline{9999999\relax}%
     \let\lst@lastline\lst@firstline
%    \end{macrocode}
%    We add the prefixes and suffixes to the markers.
%    \begin{macrocode}
     \let\lst@rangebegin\lst@rangebeginprefix
     \lst@AddTo\lst@rangebegin{#1}\lst@Extend\lst@rangebegin\lst@rangebeginsuffix
     \ifx\@empty#3\@empty
         \let\lst@rangeend\lst@rangeendprefix
         \lst@AddTo\lst@rangeend{#1}\lst@Extend\lst@rangeend\lst@rangeendsuffix
     \else
         \ifx\@empty#2\@empty
             \let\lst@rangeend\@empty
         \else
             \let\lst@rangeend\lst@rangeendprefix
             \lst@AddTo\lst@rangeend{#2}\lst@Extend\lst@rangeend\lst@rangeendsuffix
         \fi
     \fi
%    \end{macrocode}
%    The following definition will be executed in the SelectCharTable hook
%    and here right now if we are already processing a listing.
%    \begin{macrocode}
     \global\def\lst@DefRange{\expandafter\lst@CArgX\lst@rangebegin\relax\lst@DefRangeB}%
     \ifnum\lst@mode=\lst@Pmode \expandafter\lst@DefRange \fi}}
%    \end{macrocode}
%    \lst@DefRange is not inserted via a hook anymore. Instead it is now called
%    directly from \lst@SelectCharTable. This was necessary to get rid of an
%    interference with the escape-to-LaTeX-feature. The bug was reported by
%    \lsthelper{Michael~Bachmann}{2004/07/21}{Keine label-Referenzierung
%    m\"oglich...}. Another chance is due to the same bug: \lst@DefRange is
%    redefined globally when the begin of code is found, see below. The bug was
%    reported by \lsthelper{Tobias~Rapp}{2004/04/06}{undetected end of range if
%    listing crosses page break} \lsthelper{Markus~Luisser}{2004/08/13}{Bug mit
%    'linerangemarker' in umgebrochenen listings}
%    \begin{macrocode}
\lst@AddToHookExe{DeInit}{\global\let\lst@DefRange\@empty}
%    \end{macrocode}
%
%    Actually defining the marker (via \lst@GLI@, \lst@DefRange, \lst@CArgX as
%    seen above) is similar to \lst@DefDelimB---except that we unfold the first
%    parameter and use different <execute>, <pre>, and <post> statements.
%    \begin{macrocode}
\def\lst@DefRangeB#1#2{\lst@DefRangeB@#1#2}
\def\lst@DefRangeB@#1#2#3#4{%
    \lst@CDef{#1{#2}{#3}}#4{}%
    {\lst@ifincluderangemarker
         \lst@LeaveMode
         \let#1#4%
         \lst@DefRangeEnd
         \lst@InitLstNumber
     \else
         \@tempcnta\lst@lineno \advance\@tempcnta\@ne
         \edef\lst@firstline{\the\@tempcnta\relax}%
         \gdef\lst@OnceAtEOL{\let#1#4\lst@DefRangeEnd}%
         \lst@InitLstNumber
     \fi
	 \global\let\lst@DefRange\lst@DefRangeEnd
     \lst@CArgEmpty}%
    \@empty}
%    \end{macrocode}
%
% Modify labels and define |\lst@InitLstNumber| used above.
% \lsthelper{Omair-Inam~Abdul-Matin}{2004/05/10}{experimental linerange
% feature does not work with firstnumber}
%    \begin{macrocode}
\def\lstpatch@labels{%
\gdef\lst@SetFirstNumber{%
    \ifx\lst@firstnumber\@undefined
        \@tempcnta 0\csname\@lst no@\lst@intname\endcsname\relax
        \ifnum\@tempcnta=\z@ \else
            \lst@nololtrue
            \advance\@tempcnta\lst@advancenumber
            \edef\lst@firstnumber{\the\@tempcnta\relax}%
        \fi
    \fi}%
}
\def\lst@InitLstNumber{%
     \global\c@lstnumber\lst@firstnumber
     \global\advance\c@lstnumber\lst@advancenumber
     \global\advance\c@lstnumber-\lst@advancelstnum
     \ifx \lst@firstnumber\c@lstnumber
         \global\advance\c@lstnumber-\lst@advancelstnum
     \fi%
%    \end{macrocode}
% \lstthanks{Byron~K.~Boulton}{bkboulton@berriehill.com}{2013/11/21}
% reported, that the line numbers are off by one, if the are displayed when
% a linerange is given by patterns and |includerangemarker=false| is
% set. Adding this test corrects this behaviour.
%    \begin{macrocode}
     \lst@ifincluderangemarker\else%
         \global\advance\c@lstnumber by 1%
     \fi%
     }
%    \end{macrocode}
%
%    The end-marker is defined if and only if it's not empty. The definition is
%    similar to \lst@DefDelimE---with the above exceptions and except that we
%    define the re-entry point \lst@DefRangeE@@ as it is defined in the new
%    version of \lst@MProcessListing above.
%    \begin{macrocode}
\def\lst@DefRangeEnd{%
    \ifx\lst@rangeend\@empty\else
        \expandafter\lst@CArgX\lst@rangeend\relax\lst@DefRangeE
    \fi}
\def\lst@DefRangeE#1#2{\lst@DefRangeE@#1#2}
\def\lst@DefRangeE@#1#2#3#4{%
    \lst@CDef{#1#2{#3}}#4{}%
    {\let#1#4%
     \edef\lst@lastline{\the\lst@lineno\relax}%
     \lst@DefRangeE@@}%
    \@empty}
\def\lst@DefRangeE@@#1\@empty{%
    \lst@ifincluderangemarker
        #1\lst@XPrintToken
    \fi
    \lst@LeaveModeToPmode
    \lst@BeginDropInput{\lst@Pmode}}
\def\lst@LeaveModeToPmode{%
    \ifnum\lst@mode=\lst@Pmode
        \expandafter\lsthk@EndGroup
    \else
        \expandafter\egroup\expandafter\lst@LeaveModeToPmode
    \fi}
%    \end{macrocode}
%
%    Eventually we shouldn't forget to install \lst@OnceAtEOL, which must
%    also be called in \lst@MSkipToFirst.
%    \begin{macrocode}
\lst@AddToHook{EOL}{\lst@OnceAtEOL\global\let\lst@OnceAtEOL\@empty}
\gdef\lst@OnceAtEOL{}% Init
\def\lst@MSkipToFirst{%
    \global\advance\lst@lineno\@ne
    \ifnum \lst@lineno=\lst@firstline
        \def\lst@next{\lst@LeaveMode \global\lst@newlines\z@
        \lst@OnceAtEOL \global\let\lst@OnceAtEOL\@empty
        \lst@InitLstNumber % Added to work with modified \lsthk@PreInit.
        \lsthk@InitVarsBOL
        \lst@BOLGobble}%
        \expandafter\lst@next
    \fi}
\def\lst@SkipToFirst{%
    \ifnum \lst@lineno<\lst@firstline
        \def\lst@next{\lst@BeginDropInput\lst@Pmode
        \lst@Let{13}\lst@MSkipToFirst
        \lst@Let{10}\lst@MSkipToFirst}%
        \expandafter\lst@next
    \else
        \expandafter\lst@BOLGobble
    \fi}
%    \end{macrocode}
%
%    Finally the service macro \lst@IfNumber:
%    \begin{macrocode}
\def\lst@IfNumber#1{%
    \ifx\@empty#1\@empty
        \let\lst@next\@firstoftwo
    \else
        \lst@IfNumber@#1\@nil
    \fi
    \lst@next}
\def\lst@IfNumber@#1#2\@nil{%
    \let\lst@next\@secondoftwo
    \ifnum`#1>47\relax \ifnum`#1>57\relax\else
        \let\lst@next\@firstoftwo
    \fi\fi}
%    \end{macrocode}
%
% \begin{lstkey}{nolol}
% is just a key here. We'll use it below, of course.
%    \begin{macrocode}
\lst@Key{nolol}{false}[t]{\lstKV@SetIf{#1}\lst@ifnolol}
\def\lst@nololtrue{\let\lst@ifnolol\iftrue}
\let\lst@ifnolol\iffalse % init
%    \end{macrocode}
% \end{lstkey}
%
%
% \subsection{Floats, boxes and captions}
%
% \begin{lstkey}{captionpos}
% \begin{lstkey}{abovecaptionskip}
% \begin{lstkey}{belowcaptionskip}
% \begin{lstkey}{label}
% \begin{lstkey}{title}
% \begin{lstkey}{caption}
% Some keys and \ldots
%    \begin{macrocode}
\lst@Key{captionpos}{t}{\def\lst@captionpos{#1}}
\lst@Key{abovecaptionskip}\smallskipamount{\def\lst@abovecaption{#1}}
\lst@Key{belowcaptionskip}\smallskipamount{\def\lst@belowcaption{#1}}
%    \end{macrocode}
% \lsthelper{Rolf~Niepraschk}{2000/01/10}{key: title} proposed \keyname{title}.
%    \begin{macrocode}
\lst@Key{label}\relax{\def\lst@label{#1}}
\lst@Key{title}\relax{\def\lst@title{#1}\let\lst@caption\relax}
\lst@Key{caption}\relax{\lstKV@OptArg[{#1}]{#1}%
    {\def\lst@caption{##2}\def\lst@@caption{##1}}%
     \let\lst@title\@empty}
\lst@AddToHookExe{TextStyle}
    {\let\lst@caption\@empty \let\lst@@caption\@empty
     \let\lst@title\@empty \let\lst@label\@empty}
%    \end{macrocode}
% \end{lstkey}
% \end{lstkey}
% \end{lstkey}
% \end{lstkey}
% \end{lstkey}
% \end{lstkey}
%
% \begin{macro}{\thelstlisting}
% \begin{macro}{\lstlistingname}
% \begin{lstkey}{numberbychapter}
% \ldots\space and how the caption numbers look like. I switched to
% |\@ifundefined| (instead of |\ifx| |\@undefined|) after an error report from
% \lsthelper{Denis~Girou}{1999/07/26}{incompatible if hyperref loaded before
% listings}.
%
% This is set |\AtBeginDocument| so that the user can specify whether or not
% the counter should be reset at each chapter before the counter is defined,
% using the |numberbychapter| key.
%    \begin{macrocode}
\AtBeginDocument{
  \@ifundefined{thechapter}{\let\lst@ifnumberbychapter\iffalse}{}
  \lst@ifnumberbychapter
      \newcounter{lstlisting}[chapter]
      \gdef\thelstlisting%
           {\ifnum \c@chapter>\z@ \thechapter.\fi \@arabic\c@lstlisting}
  \else
      \newcounter{lstlisting}
      \gdef\thelstlisting{\@arabic\c@lstlisting}
  \fi}
%    \end{macrocode}
%    \begin{macrocode}
\lst@UserCommand\lstlistingname{Listing}
%    \end{macrocode}
%    \begin{macrocode}
\lst@Key{numberbychapter}{true}[t]{\lstKV@SetIf{#1}\lst@ifnumberbychapter}
%    \end{macrocode}
% \end{lstkey}
% \end{macro}
% \end{macro}
%
% \begin{macro}{\lst@MakeCaption}
% Before defining this macro, we ensure that some other control sequences
% exist---\lsthelper{Adam~Prugel-Bennett}{2001/02/19}{\abovecaptionskip
% undefined in slides.cls} reported problems with the slides document class.
% In particular we allocate above- and belowcaption skip registers and define
% |\@makecaption|, which is an exact copy of the definition in the article
% class. To respect the LPPL: you should have a copy of this class on your
% \TeX\ system or you can obtain a copy from the CTAN, e.g.~from the ftp-server
% \texttt{ftp.dante.de}.
%
% Axel Sommerfeldt proposed a couple of improvements regarding captions and
% titles. The first is to separate the definitions of the skip registers and
% |\@makecaption|.
%    \begin{macrocode}
\@ifundefined{abovecaptionskip}
{\newskip\abovecaptionskip
 \newskip\belowcaptionskip}{}
\@ifundefined{@makecaption}
{\long\def\@makecaption#1#2{%
   \vskip\abovecaptionskip
   \sbox\@tempboxa{#1: #2}%
   \ifdim \wd\@tempboxa >\hsize
     #1: #2\par
   \else
     \global \@minipagefalse
     \hb@xt@\hsize{\hfil\box\@tempboxa\hfil}%
   \fi
   \vskip\belowcaptionskip}%
}{}
%    \end{macrocode}
% The introduction of |\fnum@lstlisting| is also due to Axel. Previously the
% replacement text was used directly in |\lst@MakeCaption|. A |\noindent| has
% been moved elsewhere and became |\@parboxrestore| after a bug report from
% \lsthelper{Frank~Mittelbach}{2004/02/13}{Re: Info: Inkompatibilit\"at
% zwischen caption und listings}.
%    \begin{macrocode}
\def\fnum@lstlisting{%
  \lstlistingname
  \ifx\lst@@caption\@empty\else~\thelstlisting\fi}%
%    \end{macrocode}
% Captions are set only for display style listings -- thanks to
% \lsthelper{Peter~L\"offler}{2004/04/24}{pdfTeX warning (dest): name{figure.1}
% has been referenced but does not exist} for reporting the bug and to
% \lsthelper{Axel~Sommerfeldt}{2004/02/27}{Re: caption + listings + hyperref}
% for analyzing the bug.
% We |\refstepcounter| the listing counter if and only if |\lst@@caption| is
% not empty. Otherwise we ensure correct hyper-references,
% see |\lst@HRefStepCounter| below. We do this once a listing, namely at the
% top.
%    \begin{macrocode}
\def\lst@MakeCaption#1{%
  \lst@ifdisplaystyle
    \ifx #1t%
        \ifx\lst@@caption\@empty\expandafter\lst@HRefStepCounter \else
                                \expandafter\refstepcounter
        \fi {lstlisting}%
        \ifx\lst@label\@empty\else \label{\lst@label}\fi
%    \end{macrocode}
% The following code has been moved here from the \hookname{Init} hook after
% a bug report from \lsthelper{Rolf~Niepraschk}{2003/06/11}{pagebreak between
% caption and listing}. Moreover the initialization of |\lst@name| et al have
% been inserted here after a bug report from \lsthelper{Werner~Struckmann}
% {2003/06/25}{undefined control sequence \lst@name}.
% We make a `lol' entry if the name is neither empty nor a single space. But
% we test |\lst@|(|@|)|caption| and |\lst@ifnolol| first.
%    \begin{macrocode}
        \let\lst@arg\lst@intname \lst@ReplaceIn\lst@arg\lst@filenamerpl
        \global\let\lst@name\lst@arg \global\let\lstname\lst@name
        \lst@ifnolol\else
            \ifx\lst@@caption\@empty
                \ifx\lst@caption\@empty
                    \ifx\lst@intname\@empty \else \def\lst@temp{ }%
                    \ifx\lst@intname\lst@temp \else
                        \addcontentsline{lol}{lstlisting}\lst@name
                    \fi\fi
                \fi
            \else
                \addcontentsline{lol}{lstlisting}%
                    {\protect\numberline{\thelstlisting}\lst@@caption}%
            \fi
         \fi
     \fi
%    \end{macrocode}
% We make a caption if and only if the caption is not empty and the user
% requested a caption at |#1| $\in\{\mathtt t,\mathtt b\}$. To disallow
% pagebreaks between caption (or title) and a listing, we redefine the
% primitive |\vskip| locally to insert |\nobreak|s. Note that we allow
% pagebreaks in front of a `top-caption' and after a `bottom-caption'.
% Also, the |\ignorespaces| in the |\@makecaption| call is added to match
% what \LaTeX\ does in |\@caption|; the AMSbook class (and perhaps others)
% assume this is present and attempt to strip it off when testing for an
% empty caption, causing a bug noted by \lsthelper{Xiaobo~Peng}{2006/06/29}%
% {captions not shown with amsbook class}.
% \begin{TODO}
% This redefinition is a brute force method. Is there a better one?
% \end{TODO}
%    \begin{macrocode}
    \ifx\lst@caption\@empty\else
        \lst@IfSubstring #1\lst@captionpos
            {\begingroup \let\@@vskip\vskip
             \def\vskip{\afterassignment\lst@vskip \@tempskipa}%
             \def\lst@vskip{\nobreak\@@vskip\@tempskipa\nobreak}%
             \par\@parboxrestore\normalsize\normalfont % \noindent (AS)
             \ifx #1t\allowbreak \fi
             \ifx\lst@title\@empty
                 \lst@makecaption\fnum@lstlisting{\ignorespaces \lst@caption}
             \else
                 \lst@maketitle\lst@title % (AS)
             \fi
             \ifx #1b\allowbreak \fi
             \endgroup}{}%
    \fi
  \fi}
%    \end{macrocode}
% I've inserted |\normalsize| after a bug report from
% \lsthelper{Andreas~Matthias}{2000/01/04}{caption affected by basicstyle}
% and moved it in front of |\@makecaption| after receiving another from
% \lsthelper{Sonja~Weidmann}{2000/02/01}{listings and caption packages
% not compatible}.
% \end{macro}
%
% \begin{macro}{\lst@makecaption}
% \begin{macro}{\lst@maketitle}
% Axel proposed the first definition. The other two are default definitions.
% They may be adjusted to make \packagename{listings} compatible with other
% packages and classes.
%    \begin{macrocode}
\def\lst@makecaption{\@makecaption}
\def\lst@maketitle{\@makecaption\lst@title@dropdelim}
\def\lst@title@dropdelim#1{\ignorespaces}
%    \end{macrocode}
% The following \packagename{caption}(\packagename{2}) support comes also from
% Axel.
%    \begin{macrocode}
\AtBeginDocument{%
\@ifundefined{captionlabelfalse}{}{%
  \def\lst@maketitle{\captionlabelfalse\@makecaption\@empty}}%
\@ifundefined{caption@startrue}{}{%
  \def\lst@maketitle{\caption@startrue\@makecaption\@empty}}%
}
%    \end{macrocode}
% \end{macro}\end{macro}
%
% \begin{macro}{\lst@HRefStepCounter}
% This macro sets the listing number to a negative value since the user
% shouldn't refer to such a listing. If the \packagename{hyperref} package
% is present, we use `lstlisting' (argument from above) to hyperref to.
% The groups have been added to prevent other packages (namely
% \packagename{tabularx}) from reading the locally changed counter
% and writing it back globally. Thanks to \lsthelper{Michael~Niedermair}
% {2001/09/18}{strange numbering of listings} for the report. Unfortunately
% this localization led to another bug, see |\theHlstnumber|.
%    \begin{macrocode}
\def\lst@HRefStepCounter#1{%
    \begingroup
    \c@lstlisting\lst@neglisting
    \advance\c@lstlisting\m@ne \xdef\lst@neglisting{\the\c@lstlisting}%
    \ifx\hyper@refstepcounter\@undefined\else
        \hyper@refstepcounter{#1}%
    \fi
    \endgroup}
\gdef\lst@neglisting{\z@}% init
%    \end{macrocode}
% \end{macro}
%
% \begin{lstkey}{boxpos}
% \begin{macro}{\lst@boxtrue}
% sets the vertical alignment of the (possibly) used box respectively indicates
% that a box is used.
%    \begin{macrocode}
\lst@Key{boxpos}{c}{\def\lst@boxpos{#1}}
%    \end{macrocode}
%    \begin{macrocode}
\def\lst@boxtrue{\let\lst@ifbox\iftrue}
\let\lst@ifbox\iffalse
%    \end{macrocode}
% \end{macro}\end{lstkey}
%
% \begin{lstkey}{float}
% \begin{lstkey}{floatplacement}
% Matthias Zenger asked for double-column floats, so I've inserted some code.
% We first check for a star \ldots
%    \begin{macrocode}
\lst@Key{float}\relax[\lst@floatplacement]{%
    \lstKV@SwitchCases{#1}%
    {true&\let\lst@floatdefault\lst@floatplacement
          \let\lst@float\lst@floatdefault\\%
     false&\let\lst@floatdefault\relax
           \let\lst@float\lst@floatdefault
    }{\def\lst@next{\@ifstar{\let\lst@beginfloat\@dblfloat
                             \let\lst@endfloat\end@dblfloat
                             \lst@KFloat}%
                            {\let\lst@beginfloat\@float
                             \let\lst@endfloat\end@float
                             \lst@KFloat}}
      \edef\lst@float{#1}%
      \expandafter\lst@next\lst@float\relax}}
%    \end{macrocode}
% \ldots\ and define |\lst@float|.
%    \begin{macrocode}
\def\lst@KFloat#1\relax{%
    \ifx\@empty#1\@empty
        \let\lst@float\lst@floatplacement
    \else
        \def\lst@float{#1}%
    \fi}
%    \end{macrocode}
% The setting |\lst@AddToHook{PreSet}{\let\lst@float\relax}| has been
% changed on request of \lsthelper{Tanguy~Fautr\'e}{2004/02/02}{listings
% not following float directive?}. This also led to some adjustments above.
%    \begin{macrocode}
\lst@Key{floatplacement}{tbp}{\def\lst@floatplacement{#1}}
\lst@AddToHook{PreSet}{\let\lst@float\lst@floatdefault}
\lst@AddToHook{TextStyle}{\let\lst@float\relax}
\let\lst@floatdefault\relax % init
%    \end{macrocode}
% |\lst@doendpe| is set according to |\lst@float| -- thanks to
% \lsthelper{Andreas~Schmidt}{2004/05/15}{wrong spacing when a floating listing
% follows \section} and \lsthelper{Heiko~Oberdiek}{2004/05/18}{dito}.
%    \begin{macrocode}
\lst@AddToHook{DeInit}{%
    \ifx\lst@float\relax
        \global\let\lst@doendpe\@doendpe
    \else
        \global\let\lst@doendpe\@empty
    \fi}
%    \end{macrocode}
% The float type |\ftype@lstlisting| is set according to whether the
% \packagename{float} package is loaded and whether \texttt{figure} and
% \texttt{table} floats are defined. This is done at |\begin{document}| to
% make the code independent of the order of package loading.
%    \begin{macrocode}
\AtBeginDocument{%
\@ifundefined{c@float@type}%
    {\edef\ftype@lstlisting{\ifx\c@figure\@undefined 1\else 4\fi}}
    {\edef\ftype@lstlisting{\the\c@float@type}%
     \addtocounter{float@type}{\value{float@type}}}%
}
%    \end{macrocode}
% \end{lstkey}
% \end{lstkey}
%
%
% \subsection{Init and EOL}
%
% \begin{lstkey}{aboveskip}
% \begin{lstkey}{belowskip}
% We define and initialize these keys and prevent extra spacing for `inline'
% listings (in particular if \packagename{fancyvrb} interface is active,
% problem reported by \lsthelper{Denis~Girou}{1999/08/03}{wrong spacing}).
%    \begin{macrocode}
\lst@Key{aboveskip}\medskipamount{\def\lst@aboveskip{#1}}
\lst@Key{belowskip}\medskipamount{\def\lst@belowskip{#1}}
\lst@AddToHook{TextStyle}
    {\let\lst@aboveskip\z@ \let\lst@belowskip\z@}
%    \end{macrocode}
% \end{lstkey}\end{lstkey}
%
% \begin{lstkey}{everydisplay}
% \begin{macro}{\lst@ifdisplaystyle}
% Some things depend on display-style listings.
%    \begin{macrocode}
\lst@Key{everydisplay}{}{\def\lst@EveryDisplay{#1}}
\lst@AddToHook{TextStyle}{\let\lst@ifdisplaystyle\iffalse}
\lst@AddToHook{DisplayStyle}{\let\lst@ifdisplaystyle\iftrue}
\let\lst@ifdisplaystyle\iffalse
%    \end{macrocode}
% \end{macro}
% \end{lstkey}
%
% \begin{macro}{\lst@Init}
% Begin a float or multicolumn environment if requested.
%    \begin{macrocode}
\def\lst@Init#1{%
    \begingroup
    \ifx\lst@float\relax\else
        \edef\@tempa{\noexpand\lst@beginfloat{lstlisting}[\lst@float]}%
        \expandafter\@tempa
    \fi
    \ifx\lst@multicols\@empty\else
        \edef\lst@next{\noexpand\multicols{\lst@multicols}}
        \expandafter\lst@next
    \fi
%    \end{macrocode}
% In restricted horizontal \TeX\ mode we switch to |\lst@boxtrue|.
% In that case we make appropriate box(es) around the listing.
%    \begin{macrocode}
    \ifhmode\ifinner \lst@boxtrue \fi\fi
    \lst@ifbox
        \lsthk@BoxUnsafe
        \hbox to\z@\bgroup
             $\if t\lst@boxpos \vtop
        \else \if b\lst@boxpos \vbox
        \else \vcenter \fi\fi
        \bgroup \par\noindent
    \else
        \lst@ifdisplaystyle
            \lst@EveryDisplay
            \par\penalty-50\relax
            \vspace\lst@aboveskip
        \fi
    \fi
%    \end{macrocode}
% Moved |\vspace| after |\par|---or we can get an empty line atop listings.
% Bug reported by \lsthelper{Jim~Hefferon}{1999/08/27}{empty line before
% listings with |\lstinputlisting|}.
%
% Now make the top caption.
%    \begin{macrocode}
    \normalbaselines
    \abovecaptionskip\lst@abovecaption\relax
    \belowcaptionskip\lst@belowcaption\relax
    \lst@MakeCaption t%
%    \end{macrocode}
% Some initialization.
% I removed |\par\nointerlineskip| |\normalbaselines| after bug report from
% \lsthelper{Jim~Hefferon}{1999/08/23}{bad vertical space after lstlisting}.
% He reported the same problem as Aidan Philip Heerdegen (see below), but I
% immediately saw the bug here since Jim used |\parskip|$\,\neq0$.
%    \begin{macrocode}
    \lsthk@PreInit \lsthk@Init
    \lst@ifdisplaystyle
        \global\let\lst@ltxlabel\@empty
        \if@inlabel
            \lst@ifresetmargins
                \leavevmode
            \else
                \xdef\lst@ltxlabel{\the\everypar}%
                \lst@AddTo\lst@ltxlabel{%
                    \global\let\lst@ltxlabel\@empty
                    \everypar{\lsthk@EveryLine\lsthk@EveryPar}}%
            \fi
        \fi
        \everypar\expandafter{\lst@ltxlabel
                              \lsthk@EveryLine\lsthk@EveryPar}%
    \else
        \everypar{}\let\lst@NewLine\@empty
    \fi
    \lsthk@InitVars \lsthk@InitVarsBOL
%    \end{macrocode}
% The end of line character chr(13)=|^^M| controls the processing, see the
% definition of |\lst@MProcessListing| below.
% The argument |#1| is either |\relax| or |\lstenv@backslash|.
%    \begin{macrocode}
    \lst@Let{13}\lst@MProcessListing
    \let\lst@Backslash#1%
    \lst@EnterMode{\lst@Pmode}{\lst@SelectCharTable}%
    \lst@InitFinalize}
%    \end{macrocode}
% Note: From version 0.19 on `listing processing' is implemented as an internal
% mode, namely a mode with special character table. Since a bug report from
% \lsthelper{Fermin~Reig}{2002/09/04}{bad top frame inside figure+centering}
% |\rightskip| and the others are reset via \hookname{PreInit} and not via
% \hookname{InitVars}.
%    \begin{macrocode}
\let\lst@InitFinalize\@empty % init
\lst@AddToHook{PreInit}
    {\rightskip\z@ \leftskip\z@ \parfillskip=\z@ plus 1fil
     \let\par\@@par}
\lst@AddToHook{EveryLine}{}% init
\lst@AddToHook{EveryPar}{}% init
%    \end{macrocode}
% \end{macro}
%
% \begin{lstkey}{showlines}
% lets the user control whether empty lines at the end of a listing are
% printed. But you know that if you've read the User's guide.
%    \begin{macrocode}
\lst@Key{showlines}f[t]{\lstKV@SetIf{#1}\lst@ifshowlines}
%    \end{macrocode}
% \end{lstkey}
%
% \begin{macro}{\lst@DeInit}
% Output the remaining characters and update all things. First I missed to
% to use |\lst@ifdisplaystyle| here, but then \lsthelper{KP~Gores}{2001/07/11}
% {\csname{par} after each \lstinline} reported a problem.
% The |\everypar| has been put behind |\lsthk@ExitVars| after a bug report by
% \lsthelper{Michael~Niedermair}{2002/05/14}{listings.sty und caption} and
% I've added |\normalbaselines| after a bug report by \lsthelper{Georg~Rehm}
% {2002/05/14}{listings.sty und lange captions} and |\normalcolor| after a
% report by \lsthelper{Walter~E.~Brown}{2004/03/01}{captions at bottom of
% listings inherit color from basicstyle}.
%    \begin{macrocode}
\def\lst@DeInit{%
    \lst@XPrintToken \lst@EOLUpdate
    \global\advance\lst@newlines\m@ne
    \lst@ifshowlines
        \lst@DoNewLines
    \else
        \setbox\@tempboxa\vbox{\lst@DoNewLines}%
    \fi
    \lst@ifdisplaystyle \par\removelastskip \fi
    \lsthk@ExitVars\everypar{}\lsthk@DeInit\normalbaselines\normalcolor
%    \end{macrocode}
% Place the bottom caption.
%    \begin{macrocode}
    \lst@MakeCaption b%
%    \end{macrocode}
% Close the boxes if necessary and make a rule to get the right width.
% I added the |\par\nointerlineskip| (and removed |\nointerlineskip| later
% again) after receiving a bug report from \lsthelper{Aidan~Philip~Heerdegen}
% {1999/07/23}{wrong vertical spacing}. |\everypar{}| is due to a bug report
% from \lsthelper{Sonja~Weidmann}{2000/02/01}{listings and caption packages
% not compatible}.
%    \begin{macrocode}
    \lst@ifbox
        \egroup $\hss \egroup
        \vrule\@width\lst@maxwidth\@height\z@\@depth\z@
    \else
        \lst@ifdisplaystyle
            \par\penalty-50\vspace\lst@belowskip
        \fi
    \fi
%    \end{macrocode}
% End the multicolumn environment and/or float if necessary.
%    \begin{macrocode}
    \ifx\lst@multicols\@empty\else
        \def\lst@next{\global\let\@checkend\@gobble
                      \endmulticols
                      \global\let\@checkend\lst@@checkend}
        \expandafter\lst@next
    \fi
    \ifx\lst@float\relax\else
        \expandafter\lst@endfloat
    \fi
    \endgroup}
\let\lst@@checkend\@checkend
%    \end{macrocode}
% \end{macro}
%
% \begin{macro}{\lst@maxwidth}
% is to be allocated, initialized and updated.
%    \begin{macrocode}
\newdimen\lst@maxwidth % \global
\lst@AddToHook{InitVars}{\global\lst@maxwidth\z@}
\lst@AddToHook{InitVarsEOL}
    {\ifdim\lst@currlwidth>\lst@maxwidth
         \global\lst@maxwidth\lst@currlwidth
     \fi}
%    \end{macrocode}
% \end{macro}
%
% \begin{macro}{\lst@EOLUpdate}
% What do you think this macro does?
%    \begin{macrocode}
\def\lst@EOLUpdate{\lsthk@EOL \lsthk@InitVarsEOL}
%    \end{macrocode}
% \end{macro}
%
% \begin{macro}{\lst@MProcessListing}
% This is what we have to do at EOL while processing a listing.
% We output all remaining characters and update the variables.
% If we've reached the last line, we check whether there is a next line
% interval to input or not.
%    \begin{macrocode}
\def\lst@MProcessListing{%
    \lst@XPrintToken \lst@EOLUpdate \lsthk@InitVarsBOL
    \global\advance\lst@lineno\@ne
    \ifnum \lst@lineno>\lst@lastline
        \lst@ifdropinput \lst@LeaveMode \fi
        \ifx\lst@linerange\@empty
            \expandafter\expandafter\expandafter\lst@EndProcessListing
        \else
            \lst@interrange
            \lst@GetLineInterval
            \expandafter\expandafter\expandafter\lst@SkipToFirst
        \fi
    \else
        \expandafter\lst@BOLGobble
    \fi}
%    \end{macrocode}
% \end{macro}
%
% \begin{macro}{\lst@EndProcessListing}
% Default definition is |\endinput|.
% This works for |\lstinputlisting|.
%    \begin{macrocode}
\let\lst@EndProcessListing\endinput
%    \end{macrocode}
% \end{macro}
%
% \begin{lstkey}{gobble}
% The key sets the number of characters to gobble each line.
%    \begin{macrocode}
\lst@Key{gobble}{0}{\def\lst@gobble{#1}}
%    \end{macrocode}
% \end{lstkey}
%
% \begin{macro}{\lst@BOLGobble}
% If the number is positive, we set a temporary counter and start a loop.
%    \begin{macrocode}
\def\lst@BOLGobble{%
    \ifnum\lst@gobble>\z@
        \@tempcnta\lst@gobble\relax
        \expandafter\lst@BOLGobble@
	\fi}
%    \end{macrocode}
% A nonpositive number terminates the loop (by not continuing).
% Note: This is not the macro just used in |\lst@BOLGobble|.
%    \begin{macrocode}
\def\lst@BOLGobble@@{%
    \ifnum\@tempcnta>\z@
        \expandafter\lst@BOLGobble@
    \fi}
%    \end{macrocode}
% If we gobble a backslash, we have to look whether this backslash ends an
% environment. Whether the coming characters equal e.g.~|end{lstlisting}|,
% we either end the environment or insert all just eaten characters after the
% `continue loop' macro.
%    \begin{macrocode}
\def\lstenv@BOLGobble@@{%
    \lst@IfNextChars\lstenv@endstring{\lstenv@End}%
    {\advance\@tempcnta\m@ne \expandafter\lst@BOLGobble@@\lst@eaten}}
%    \end{macrocode}
% Now comes the loop: if we read |\relax|, EOL or FF, the next operation is
% exactly the same token. Note that for FF (and tabs below) we test against
% a macro which contains |\lst@ProcessFormFeed|. This was a bug analyzed by
% \lsthelper{Heiko~Oberdiek}{2002/04/16}{Re: first experience ...}.
%    \begin{macrocode}
\def\lst@BOLGobble@#1{%
    \let\lst@next#1%
    \ifx \lst@next\relax\else
    \ifx \lst@next\lst@MProcessListing\else
    \ifx \lst@next\lst@processformfeed\else
%    \end{macrocode}
% Otherwise we use one of the two submacros.
%    \begin{macrocode}
    \ifx \lst@next\lstenv@backslash
        \let\lst@next\lstenv@BOLGobble@@
    \else
        \let\lst@next\lst@BOLGobble@@
%    \end{macrocode}
% Now we really gobble characters. A tabulator decreases the temporary counter
% by |\lst@tabsize| (and deals with remaining amounts, if necessary), \ldots
%    \begin{macrocode}
        \ifx #1\lst@processtabulator
            \advance\@tempcnta-\lst@tabsize\relax
            \ifnum\@tempcnta<\z@
                \lst@length-\@tempcnta \lst@PreGotoTabStop
            \fi
%    \end{macrocode}
% \ldots\space whereas any other character decreases the counter by one.
%    \begin{macrocode}
        \else
            \advance\@tempcnta\m@ne
        \fi
    \fi \fi \fi \fi
    \lst@next}
%    \end{macrocode}
%    \begin{macrocode}
\def\lst@processformfeed{\lst@ProcessFormFeed}
\def\lst@processtabulator{\lst@ProcessTabulator}
%    \end{macrocode}
% \end{macro}
%
%
% \subsection{List of listings}
%
% \begin{lstkey}{name}
% \begin{macro}{\lstname}
% \begin{macro}{\lst@name}
% \begin{macro}{\lst@intname}
% Each pretty-printing command values |\lst@intname| before setting any keys.
%    \begin{macrocode}
\lst@Key{name}\relax{\def\lst@intname{#1}}
\lst@AddToHookExe{PreSet}{\global\let\lst@intname\@empty}
\lst@AddToHook{PreInit}{%
    \let\lst@arg\lst@intname \lst@ReplaceIn\lst@arg\lst@filenamerpl
    \global\let\lst@name\lst@arg \global\let\lstname\lst@name}
%    \end{macrocode}
% Use of |\lst@ReplaceIn| removes a bug first reported by
% \lsthelper{Magne~Rudshaug}{1998/01/09}{_ and list of listings}.
% Here is the replacement list.
%    \begin{macrocode}
\def\lst@filenamerpl{_\textunderscore $\textdollar -\textendash}
%    \end{macrocode}
% \end{macro}
% \end{macro}
% \end{macro}
% \end{lstkey}
%
% \begin{macro}{\l@lstlisting}
% prints one `lol' line.
%    \begin{macrocode}
\def\l@lstlisting#1#2{\@dottedtocline{1}{1.5em}{2.3em}{#1}{#2}}
%    \end{macrocode}
% \end{macro}
%
% \begin{macro}{\lstlistlistingname}
% contains simply the header name.
%    \begin{macrocode}
\lst@UserCommand\lstlistlistingname{Listings}
%    \end{macrocode}
% \end{macro}
%
% \begin{macro}{\lstlistoflistings}
% We make local adjustments and call |\tableofcontents|. This way,
% redefinitions of that macro (e.g.~without any |\MakeUppercase| inside)
% also take effect on the list of listings.
%    \begin{macrocode}
\lst@UserCommand\lstlistoflistings{\bgroup
    \let\contentsname\lstlistlistingname
    \let\lst@temp\@starttoc \def\@starttoc##1{\lst@temp{lol}}%
    \tableofcontents \egroup}
%    \end{macrocode}
% For KOMA-script classes, we define it a la KOMA thanks to a bug report by
% \lsthelper{Tino~Langer}{2003/11/01}{koma-script's listsleft option does not
% affect lol}.  \lsthelper{Markus~Kohm}{2006/08/12}{koma-script support is
% broken} suggested a much-improved version of this, which also works with
% the \packagename{float} package.  The following few comments are from Markus.
%
% Make use of |\float@listhead| if defined (e.g. using float or KOMA-Script)
%    \begin{macrocode}
\@ifundefined{float@listhead}{}{%
  \renewcommand*{\lstlistoflistings}{%
    \begingroup
%    \end{macrocode}
% Switch to one-column mode if the switch for switching is available.
%    \begin{macrocode}
      \@ifundefined{@restonecoltrue}{}{%
        \if@twocolumn
          \@restonecoltrue\onecolumn
        \else
          \@restonecolfalse
        \fi
      }%
      \float@listhead{\lstlistlistingname}%
%    \end{macrocode}
% Set |\parskip| to 0pt (should be!), |\parindent| to 0pt (better but not always
% needed), |\parfillskip| to 0pt plus 1fil (should be!).
%    \begin{macrocode}
      \parskip\z@\parindent\z@\parfillskip \z@ \@plus 1fil%
      \@starttoc{lol}%
%    \end{macrocode}
% Switch back to twocolumn (see above).
%    \begin{macrocode}
      \@ifundefined{@restonecoltrue}{}{%
        \if@restonecol\twocolumn\fi
      }%
    \endgroup
  }%
}
%    \end{macrocode}
% \end{macro}
%
% \begin{macro}{\float@addtolists}
% The \packagename{float} package defines a generic way for packages to add
% things (such as chapter names) to all of the lists of floats other than the
% standard figure and table lists.  Each package that defines a list of
% floats adds a command to |\float@addtolists|, and then packages (such as
% the KOMA-script document classes) which wish to add things to all lists of
% floats can then use it, without needing to be aware of all of the possible
% lists that could exist.  Thanks to \lsthelper{Markus~Kohm}{-}{2007/02/25}
% for the suggestion.
%
% Unfortunately, \packagename{float} defines this with |\newcommand|; thus,
% to avoid conflict, we have to redefine it after \packagename{float} is
% loaded.  |\AtBeginDocument| is the easiest way to do this.  Again, thanks
% to Markus for the advice.
%    \begin{macrocode}
\AtBeginDocument{%
  \@ifundefined{float@addtolists}%
    {\gdef\float@addtolists#1{\addtocontents{lol}{#1}}}%
    {\let\orig@float@addtolists\float@addtolists
     \gdef\float@addtolists#1{%
       \addtocontents{lol}{#1}%
       \orig@float@addtolists{#1}}}%
}%
%    \end{macrocode}
% \end{macro}
%
%
% \subsection{Inline listings}\label{iInlineListings}
%
% \subsubsection{Processing inline listings}\label{uProcessingInline}
%
% \begin{macro}{\lstinline}
% In addition to |\lsthk@PreSet|, we use |boxpos=b| and flexiblecolumns.
% I've inserted |\leavevmode| after bug report from \lsthelper{Michael~Weber}
% {1999/12/16}{wrong spacing in list environments}. \lsthelper{Olivier~Lecarme}
% {2001/07/30}{inconsistent `break' when \lstinline is used inside caption}
% reported a problem which has gone after removing |\let| |\lst@newlines|
% |\@empty| (now |\lst@newlines| is a counter!). Unfortunately I don't know
% the reason for inserting this code some time ago! At the end of the macro we
% check the delimiter.
%    \begin{macrocode}
\newcommand\lstinline[1][]{%
    \leavevmode\bgroup % \hbox\bgroup --> \bgroup
      \def\lst@boxpos{b}%
      \lsthk@PreSet\lstset{flexiblecolumns,#1}%
      \lsthk@TextStyle
      \@ifnextchar\bgroup{%
%    \end{macrocode}
% \lstthanks{Luc~Van~Eycken}{Luc.VanEycken@esat.kuleuven.be}{2014/01/22}
% reported, that the experimental implementation of |\lstinline| with
% braces instead of characters surrounding the source code resulted in an
% error if used in a tabular environment. He found that this error comes
% from the master counter (cf. appendix D (Dirty Tricks), item 5. (Brace
% hacks), of the TeXbook (p. 385-386)). Adding the following line at this
% point
% \begin{verbatim}
%         \ifnum`{=0}\fi%
% \end{verbatim}
% remedies the wrong behaviour. But \lstthanks{Qing Lee}{2014/06/28} pointed out,
% that this breaks code like the one showed in \ref{uListingsArguments} on
% \pageref{uListingsArguments} and proposed another
% solution which in turn broke the code needed by Luc:
% \begin{verbatim}
% % \renewcommand\lstinline[1][]{%
% %   \leavevmode\bgroup % \hbox\bgroup --> \bgroup
% %   \def\lst@boxpos{b}%
% %   \lsthk@PreSet\lstset{flexiblecolumns,#1}%
% %   \lsthk@TextStyle
% %   \ifnum\iffalse{\fi`}=\z@\fi
% %   \@ifnextchar\bgroup{%
% %     \ifnum`{=\z@}\fi%
% %     \afterassignment\lst@InlineG \let\@let@token}{%
% %     \ifnum`{=\z@}\fi\lstinline@}}
% \end{verbatim}
% So finally the old code comes back and the people, who need a
% |\lstinline| with braces, should use the workaround from section
% \ref{uListingsArguments} on page \pageref{uListingsArguments}.
%    \begin{macrocode}
        \afterassignment\lst@InlineG \let\@let@token}%
                         \lstinline@}
\def\lstinline@#1{%
    \lst@Init\relax
    \lst@IfNextCharActive{\lst@InlineM#1}{\lst@InlineJ#1}}
\lst@AddToHook{TextStyle}{}% init
%    \end{macrocode}
%    \begin{macrocode}
\lst@AddToHook{SelectCharTable}{\lst@inlinechars}
\global\let\lst@inlinechars\@empty
%    \end{macrocode}
% \end{macro}
%
% \begin{macro}{\lst@InlineM}
% \begin{macro}{\lst@InlineJ}
% treat the cases of `normal' inlines and inline listings inside an argument.
% In the first case the given character ends the inline listing and EOL within
% such a listing immediately ends it and produces an error message.
%    \begin{macrocode}
\def\lst@InlineM#1{\gdef\lst@inlinechars{%
    \lst@Def{`#1}{\lst@DeInit\egroup\global\let\lst@inlinechars\@empty}%
    \lst@Def{13}{\lst@DeInit\egroup \global\let\lst@inlinechars\@empty
        \PackageError{Listings}{lstinline ended by EOL}\@ehc}}%
    \lst@inlinechars}
%    \end{macrocode}
% In the other case we get all characters up to |#1|, make these characters
% active, execute (typeset) them and end the listing (all via temporary macro).
% That's all about it.
%    \begin{macrocode}
\def\lst@InlineJ#1{%
    \def\lst@temp##1#1{%
        \let\lst@arg\@empty \lst@InsideConvert{##1}\lst@arg
        \lst@DeInit\egroup}%
    \lst@temp}
%    \end{macrocode}
% \end{macro}
% \end{macro}
%
% \begin{macro}{\lst@InlineG}
% is experimental.
%    \begin{macrocode}
\def\lst@InlineG{%
    \lst@Init\relax
    \lst@IfNextCharActive{\lst@InlineM\}}%
                         {\let\lst@arg\@empty \lst@InlineGJ}}
\def\lst@InlineGJ{\futurelet\@let@token\lst@InlineGJTest}
\def\lst@InlineGJTest{%
    \ifx\@let@token\egroup
        \afterassignment\lst@InlineGJEnd
        \expandafter\let\expandafter\@let@token
    \else
        \ifx\@let@token\@sptoken
            \let\lst@next\lst@InlineGJReadSp
        \else
            \let\lst@next\lst@InlineGJRead
        \fi
        \expandafter\lst@next
    \fi}
\def\lst@InlineGJEnd{\lst@arg\lst@DeInit\egroup}
\def\lst@InlineGJRead#1{%
    \lccode`\~=`#1\lowercase{\lst@lAddTo\lst@arg~}%
    \lst@InlineGJ}
\def\lst@InlineGJReadSp#1{%
    \lccode`\~=`\ \lowercase{\lst@lAddTo\lst@arg~}%
    \lst@InlineGJ#1}
%    \end{macrocode}
% \end{macro}
%
%
% \subsubsection{Short inline listing environments}
%
% The implementation in this section is based on the \packagename{shortvrb}
% package, which is part of |doc.dtx| from the Standard \LaTeX\ documentation
% package, version 2006/02/02 v2.1d.  Portions of it are thus copyright
% 1993--2006 by The \LaTeX3 Project and copyright 1989--1999 by Frank
% Mittelbach.
%
% \begin{macro}{\lstMakeShortInline}
% \begin{macro}{\lstMakeShortInline@}
% First, we supply an optional argument if it's omitted.
%    \begin{macrocode}
\newcommand\lstMakeShortInline[1][]{%
  \def\lst@shortinlinedef{\lstinline[#1]}%
  \lstMakeShortInline@}%
\def\lstMakeShortInline@#1{%
  \expandafter\ifx\csname lst@ShortInlineOldCatcode\string#1\endcsname\relax
    \lst@shortlstinlineinfo{Made }{#1}%
    \lst@add@special{#1}%
%    \end{macrocode}
% The character's current catcode is stored in
% |\lst@ShortInlineOldCatcode\|\meta{c}.
%    \begin{macrocode}
    \expandafter
    \xdef\csname lst@ShortInlineOldCatcode\string#1\endcsname{\the\catcode`#1}%
%    \end{macrocode}
% The character is spliced into the definition using the same trick as
% used in |\verb| (for instance), having activated |~| in a group.
%    \begin{macrocode}
    \begingroup
      \catcode`\~\active  \lccode`\~`#1%
      \lowercase{%
%    \end{macrocode}
% The character's old meaning is recorded
% in |\lst@ShortInlineOldMeaning\|\meta{c} prior to assigning it a new one.
%    \begin{macrocode}
        \global\expandafter\let
          \csname lst@ShortInlineOldMeaning\string#1\endcsname~%
          \expandafter\gdef\expandafter~\expandafter{\lst@shortinlinedef#1}}%
    \endgroup
%    \end{macrocode}
% Finally the character is made active.
%    \begin{macrocode}
    \global\catcode`#1\active
%    \end{macrocode}
% If we suspect that \meta{c} is already a short reference, we tell
% the user. Now he or she is responsible if anything goes wrong\,\dots
% (Change in \packagename{listings}: We give a proper error here.)
%    \begin{macrocode}
  \else
    \PackageError{Listings}%
    {\string\lstMakeShorterInline\ definitions cannot be nested}%
    {Use \string\lstDeleteShortInline first.}%
    {}%
  \fi}
%    \end{macrocode}
% \end{macro}
% \end{macro}
% \begin{macro}{\lstDeleteShortInline}
%    \begin{macrocode}
\def\lstDeleteShortInline#1{%
  \expandafter\ifx\csname lst@ShortInlineOldCatcode\string#1\endcsname\relax
    \PackageError{Listings}%
    {#1 is not a short reference for \string\lstinline}%
    {Use \string\lstMakeShortInline first.}%
    {}%
  \else
    \lst@shortlstinlineinfo{Deleted }{#1 as}%
    \lst@rem@special{#1}%
    \global\catcode`#1\csname lst@ShortInlineOldCatcode\string#1\endcsname
    \global \expandafter\let%
      \csname lst@ShortInlineOldCatcode\string#1\endcsname \relax
    \ifnum\catcode`#1=\active
      \begingroup
        \catcode`\~\active  \lccode`\~`#1%
        \lowercase{%
          \global\expandafter\let\expandafter~%
          \csname lst@ShortInlineOldMeaning\string#1\endcsname}%
      \endgroup
    \fi
  \fi}
%    \end{macrocode}
% \end{macro}
%
% \begin{macro}{\lst@shortlstinlineinfo}
%    \begin{macrocode}
\def\lst@shortlstinlineinfo#1#2{%
     \PackageInfo{Listings}{%
       #1\string#2 a short reference for \string\lstinline}}
%    \end{macrocode}
%  \end{macro}
%
% \begin{macro}{\lst@add@special}
% This helper macro adds its argument to the
% |\dospecials| macro which is conventionally used by verbatim macros
% to alter the catcodes of the currently active characters.  We need
% to add |\do\|\meta{c} to the expansion of |\dospecials| after
% removing the character if it was already there to avoid multiple
% copies building up should |\lstMakeShortInline| not be balanced by
% |\lstDeleteShortInline| (in case anything that uses |\dospecials|
% cares about repetitions).
%    \begin{macrocode}
\def\lst@add@special#1{%
  \lst@rem@special{#1}%
  \expandafter\gdef\expandafter\dospecials\expandafter
    {\dospecials \do #1}%
%    \end{macrocode}
% Similarly we have to add |\@makeother\|\meta{c} to |\@sanitize|
% (which is used in things like "\index" to re-catcode all special
% characters except braces).
%    \begin{macrocode}
  \expandafter\gdef\expandafter\@sanitize\expandafter
    {\@sanitize \@makeother #1}}
%    \end{macrocode}
% \end{macro}
% \begin{macro}{\lst@rem@special}
% The inverse of |\lst@add@special| is slightly trickier.  |\do| is
% re-defined to expand to nothing if its argument is the character of
% interest, otherwise to expand simply to the argument.  We can then
% re-define |\dospecials| to be the expansion of itself.  The space
% after |=`##1| prevents an expansion to |\relax|!
%    \begin{macrocode}
\def\lst@rem@special#1{%
  \def\do##1{%
    \ifnum`#1=`##1 \else \noexpand\do\noexpand##1\fi}%
  \xdef\dospecials{\dospecials}%
%    \end{macrocode}
% Fixing |\@sanitize| is the same except that we need to re-define
% |\@makeother| which obviously needs to be done in a group.
%    \begin{macrocode}
  \begingroup
    \def\@makeother##1{%
      \ifnum`#1=`##1 \else \noexpand\@makeother\noexpand##1\fi}%
    \xdef\@sanitize{\@sanitize}%
  \endgroup}
%    \end{macrocode}
% \end{macro}
%
%
% \subsection{The input command}\label{iTheInputCommand}
%
% \begin{macro}{\lst@MakePath}
% \begin{lstkey}{inputpath}
% The macro appends a slash to a path if necessary.
%    \begin{macrocode}
\def\lst@MakePath#1{\ifx\@empty#1\@empty\else\lst@MakePath@#1/\@nil/\fi}
\def\lst@MakePath@#1/{#1/\lst@MakePath@@}
\def\lst@MakePath@@#1/{%
    \ifx\@nil#1\expandafter\@gobble
         \else \ifx\@empty#1\else #1/\fi \fi
    \lst@MakePath@@}
%    \end{macrocode}
% Now we can empty the path or use |\lst@MakePath|.
%    \begin{macrocode}
\lst@Key{inputpath}{}{\edef\lst@inputpath{\lst@MakePath{#1}}}
%    \end{macrocode}
% \end{lstkey}
% \end{macro}
%
% \begin{macro}{\lstinputlisting}
% inputs the listing or asks the user for a new file name.
%    \begin{macrocode}
\def\lstinputlisting{%
    \begingroup \lst@setcatcodes \lst@inputlisting}
\newcommand\lst@inputlisting[2][]{%
    \endgroup
    \def\lst@set{#1}%
    \IfFileExists{\lst@inputpath#2}%
        {\expandafter\lst@InputListing\expandafter{\lst@inputpath#2}}%
        {\filename@parse{\lst@inputpath#2}%
         \edef\reserved@a{\noexpand\lst@MissingFileError
             {\filename@area\filename@base}%
             {\ifx\filename@ext\relax tex\else\filename@ext\fi}}%
         \reserved@a}%
    \lst@doendpe \@newlistfalse \ignorespaces}
%    \end{macrocode}
% We use |\lst@doendpe| to remove indention at the beginning of the next
% line---except there is an empty line after |\lstinputlisting|. Bug was
% reported by \lsthelper{David~John~Evans}{1999/06/08}{indention after
% listings} and \lsthelper{David~Carlisle}{1999/06/08}{LaTeX `display
% environment' code} pointed me to the solution.
% \end{macro}
%
% \begin{macro}{\lst@MissingFileError}
% is a derivation of \LaTeX's |\@missingfileerror|. The parenthesis have been
% added after \lsthelper{Heiko~Oberdiek}{2003/01/14}{File `Makefile.tex' not
% found} reported about a problem discussed on TEX-D-L.
%    \begin{macrocode}
\def\lst@MissingFileError#1#2{%
    \typeout{^^J! Package Listings Error: File `#1(.#2)' not found.^^J%
        ^^JType X to quit or <RETURN> to proceed,^^J%
        or enter new name. (Default extension: #2)^^J}%
    \message{Enter file name: }%
    {\endlinechar\m@ne \global\read\m@ne to\@gtempa}%
%    \end{macrocode}
% Typing |x| or |X| exits.
%    \begin{macrocode}
    \ifx\@gtempa\@empty \else
        \def\reserved@a{x}\ifx\reserved@a\@gtempa\batchmode\@@end\fi
        \def\reserved@a{X}\ifx\reserved@a\@gtempa\batchmode\@@end\fi
%    \end{macrocode}
% In all other cases we try the new file name.
%    \begin{macrocode}
        \filename@parse\@gtempa
        \edef\filename@ext{%
            \ifx\filename@ext\relax#2\else\filename@ext\fi}%
        \edef\reserved@a{\noexpand\IfFileExists %
                {\filename@area\filename@base.\filename@ext}%
            {\noexpand\lst@InputListing %
                {\filename@area\filename@base.\filename@ext}}%
            {\noexpand\lst@MissingFileError
                {\filename@area\filename@base}{\filename@ext}}}%
        \expandafter\reserved@a %
    \fi}
%    \end{macrocode}
% \end{macro}
%
% \begin{macro}{\lst@ifdraft}
% makes use of |\lst@ifprint|. \lsthelper{Enrico~Straube}{2002/02/12}
% {de.comp.text.tex: listings und draft Modus} requested the final option.
%    \begin{macrocode}
\let\lst@ifdraft\iffalse
\DeclareOption{draft}{\let\lst@ifdraft\iftrue}
\DeclareOption{final}{\let\lst@ifdraft\iffalse}
\lst@AddToHook{PreSet}
    {\lst@ifdraft
         \let\lst@ifprint\iffalse
         \@gobbletwo\fi\fi
     \fi}
%    \end{macrocode}
% \end{macro}
%
% \begin{macro}{\lst@InputListing}
% The one and only argument is the file name, but we have the `implicit'
% argument |\lst@set|. Note that |\lst@Init| takes |\relax| as argument.
%    \begin{macrocode}
\def\lst@InputListing#1{%
    \begingroup
      \lsthk@PreSet \gdef\lst@intname{#1}%
      \expandafter\lstset\expandafter{\lst@set}%
      \lsthk@DisplayStyle
      \catcode\active=\active
      \lst@Init\relax \let\lst@gobble\z@
      \lst@SkipToFirst
      \lst@ifprint \def\lst@next{\input{#1}}%
             \else \let\lst@next\@empty \fi
      \lst@next
      \lst@DeInit
    \endgroup}
%    \end{macrocode}
% The line |\catcode\active=\active|, which makes the CR-character active,
% has been added after a bug report by \lsthelper{Rene~H.~Larsen}{2002/04/15}
% {\lstinputlistings and texcl conflict}.
% \end{macro}
%
% \begin{macro}{\lst@SkipToFirst}
% The end of line character either processes the listing or is responsible for
% dropping lines up to first printing line.
%    \begin{macrocode}
\def\lst@SkipToFirst{%
    \ifnum \lst@lineno<\lst@firstline
%    \end{macrocode}
% We drop the input and redefine the end of line characters.
%    \begin{macrocode}
        \lst@BeginDropInput\lst@Pmode
        \lst@Let{13}\lst@MSkipToFirst
        \lst@Let{10}\lst@MSkipToFirst
    \else
        \expandafter\lst@BOLGobble
    \fi}
%    \end{macrocode}
% \end{macro}
%
% \begin{macro}{\lst@MSkipToFirst}
% We just look whether to drop more lines or to leave the mode which restores
% the definition of chr(13) and chr(10).
%    \begin{macrocode}
\def\lst@MSkipToFirst{%
    \global\advance\lst@lineno\@ne
    \ifnum \lst@lineno=\lst@firstline
        \lst@LeaveMode \global\lst@newlines\z@
        \lsthk@InitVarsBOL
        \expandafter\lst@BOLGobble
    \fi}
%    \end{macrocode}
% \end{macro}
%
%
% \subsection{The environment}
%
%
% \subsubsection{Low-level processing}
%
% \begin{macro}{\lstenv@DroppedWarning}
% gives a warning if characters have been dropped.
%    \begin{macrocode}
\def\lstenv@DroppedWarning{%
    \ifx\lst@dropped\@undefined\else
        \PackageWarning{Listings}{Text dropped after begin of listing}%
    \fi}
\let\lst@dropped\@undefined % init
%    \end{macrocode}
% \end{macro}
%
% \begin{macro}{\lstenv@Process}
% We execute `|\lstenv@ProcessM|' or |\lstenv@ProcessJ| according to whether we
% find an active EOL or a nonactive |^^J|.
%    \begin{macrocode}
\begingroup \lccode`\~=`\^^M\lowercase{%
\gdef\lstenv@Process#1{%
    \ifx~#1%
%    \end{macrocode}
% We make no extra |\lstenv@ProcessM| definition since there is nothing to do
% at all if we've found an active EOL.
%    \begin{macrocode}
        \lstenv@DroppedWarning \let\lst@next\lst@SkipToFirst
    \else\ifx^^J#1%
        \lstenv@DroppedWarning \let\lst@next\lstenv@ProcessJ
    \else
        \let\lst@dropped#1\let\lst@next\lstenv@Process
    \fi \fi
    \lst@next}
}\endgroup
%    \end{macrocode}
% \end{macro}
%
% \begin{macro}{\lstenv@ProcessJ}
% Now comes the horrible scenario: a listing inside an argument. We've
% already worked in section \ref{iApplicationsTo} for this. Here we must get
% all characters up to `end environment'. We distinguish the cases `command
% fashion' and `true environment'.
%    \begin{macrocode}
\def\lstenv@ProcessJ{%
    \let\lst@arg\@empty
    \ifx\@currenvir\lstenv@name
        \expandafter\lstenv@ProcessJEnv
    \else
%    \end{macrocode}
% The first case is pretty simple: The code is terminated by
% |\end|\meta{name of environment}. Thus we expand that control sequence
% before defining a temporary macro, which gets the listing and does all
% the rest. Back to the definition of |\lstenv@ProcessJ| we call the
% temporary macro after expanding |\fi|.
%    \begin{macrocode}
        \expandafter\def\expandafter\lst@temp\expandafter##1%
            \csname end\lstenv@name\endcsname
                {\lst@InsideConvert{##1}\lstenv@ProcessJ@}%
        \expandafter\lst@temp
    \fi}
%    \end{macrocode}
% We must append an active backslash and the `end string' to |\lst@arg|. So all
% (in fact most) other processing won't notice that the code has been inside
% an argument. But the EOL character is chr(10)=|^^J| now and not chr(13).
%    \begin{macrocode}
\begingroup \lccode`\~=`\\\lowercase{%
\gdef\lstenv@ProcessJ@{%
    \lst@lExtend\lst@arg
        {\expandafter\ \expandafter~\lstenv@endstring}%
    \catcode10=\active \lst@Let{10}\lst@MProcessListing
%    \end{macrocode}
% We execute |\lst@arg| to typeset the listing.
%    \begin{macrocode}
    \lst@SkipToFirst \lst@arg}
}\endgroup
%    \end{macrocode}
% \end{macro}
%
% \begin{macro}{\lstenv@ProcessJEnv}
% The `true environment' case is more complicated. We get all characters up to
% an |\end| and the following argument. If that equals |\lstenv@name|, we have
% found the end of environment and start typesetting.
%    \begin{macrocode}
\def\lstenv@ProcessJEnv#1\end#2{\def\lst@temp{#2}%
    \ifx\lstenv@name\lst@temp
        \lst@InsideConvert{#1}%
        \expandafter\lstenv@ProcessJ@
    \else
%    \end{macrocode}
% Otherwise we append the characters including the eaten |\end| and the eaten
% argument to current |\lst@arg|. And we look for the end of environment again.
%    \begin{macrocode}
        \lst@InsideConvert{#1\\end\{#2\}}%
        \expandafter\lstenv@ProcessJEnv
    \fi}
%    \end{macrocode}
% \end{macro}
%
% \begin{macro}{\lstenv@backslash}
% Coming to a backslash we either end the listing or process a backslash and
% insert the eaten characters again.
%    \begin{macrocode}
\def\lstenv@backslash{%
    \lst@IfNextChars\lstenv@endstring
        {\lstenv@End}%
        {\expandafter\lsts@backslash \lst@eaten}}%
%    \end{macrocode}
% \end{macro}
%
% \begin{macro}{\lstenv@End}
% This macro has just been used and terminates a listing environment:
% We call the `end environment' macro using |\end| or as a command.
%    \begin{macrocode}
\def\lstenv@End{%
    \ifx\@currenvir\lstenv@name
        \edef\lst@next{\noexpand\end{\lstenv@name}}%
    \else
        \def\lst@next{\csname end\lstenv@name\endcsname}%
    \fi
    \lst@next}
%    \end{macrocode}
% \end{macro}
%
%
% \subsubsection{Defining new environments}
%
% \begin{macro}{\lstnewenvironment}
% Now comes the main command. We define undefined environments only. On the
% parameter text |#1#2#| (in particular the last sharp) see the paragraph
% following example 20.5 on page 204 of `The \TeX book'.
%    \begin{macrocode}
\lst@UserCommand\lstnewenvironment#1#2#{%
    \@ifundefined{#1}%
        {\let\lst@arg\@empty
         \lst@XConvert{#1}\@nil
         \expandafter\lstnewenvironment@\lst@arg{#1}{#2}}%
        {\PackageError{Listings}{Environment `#1' already defined}\@eha
         \@gobbletwo}}
\def\@tempa#1#2#3{%
\gdef\lstnewenvironment@##1##2##3##4##5{%
    \begingroup
%    \end{macrocode}
% A lonely `end environment' produces an error.
%    \begin{macrocode}
    \global\@namedef{end##2}{\lstenv@Error{##2}}%
%    \end{macrocode}
% The `main' environment macro defines the environment name for later use and
% calls a submacro getting all arguments. We open a group and make EOL active.
% This ensures |\@ifnextchar[| not to read characters of the listing---it reads
% the active EOL instead.
%    \begin{macrocode}
    \global\@namedef{##2}{\def\lstenv@name{##2}%
        \begingroup \lst@setcatcodes \catcode\active=\active
        \csname##2@\endcsname}%
%    \end{macrocode}
% The submacro is defined via |\new@command|. We misuse |\l@ngrel@x| to make
% the definition |\global| and refine \LaTeX's |\@xargdef|.
%    \begin{macrocode}
    \let\l@ngrel@x\global
    \let\@xargdef\lstenv@xargdef
    \expandafter\new@command\csname##2@\endcsname##3%
%    \end{macrocode}
% First we execute |##4|=\meta{begin code}. Then follows the definition of
% the terminating string (|end{lstlisting}| or |endlstlisting|, for example):
%    \begin{macrocode}
        {\lsthk@PreSet ##4%
         \ifx\@currenvir\lstenv@name
             \def\lstenv@endstring{#1#2##1#3}%
         \else
             \def\lstenv@endstring{#1##1}%
         \fi
%    \end{macrocode}
% We redefine (locally) `end environment' since ending is legal now.
% Note that the redefinition also works inside a \TeX\ comment line.
%    \begin{macrocode}
         \@namedef{end##2}{\lst@DeInit ##5\endgroup
                          \lst@doendpe \@ignoretrue}%
%    \end{macrocode}
% |\lst@doendpe| again removes the indention problem.
%
% Finally we start the processing. The |\lst@EndProcessListing| assignment
% has been moved in front of |\lst@Init| after a bug report by
% \lsthelper{Andreas~Deininger}{2002/11/11}{Compiling just stops}.
%    \begin{macrocode}
         \lsthk@DisplayStyle
         \let\lst@EndProcessListing\lstenv@SkipToEnd
         \lst@Init\lstenv@backslash
         \lst@ifprint
             \expandafter\expandafter\expandafter\lstenv@Process
         \else
             \expandafter\lstenv@SkipToEnd
         \fi
         \lst@insertargs}%
    \endgroup}%
}
\let\lst@arg\@empty \lst@XConvert{end}\{\}\@nil
\expandafter\@tempa\lst@arg
\let\lst@insertargs\@empty
%    \end{macrocode}
% \end{macro}
%
% \begin{macro}{\lstenv@xargdef}
% This is a derivation of \LaTeX's |\@xargdef|. We expand the submacro's name,
% use |\gdef| instead of |\def|, and hard code a kind of |\@protected@testopt|.
%    \begin{macrocode}
\def\lstenv@xargdef#1{
    \expandafter\lstenv@xargdef@\csname\string#1\endcsname#1}
\def\lstenv@xargdef@#1#2[#3][#4]#5{%
  \@ifdefinable#2{%
       \gdef#2{%
          \ifx\protect\@typeset@protect
            \expandafter\lstenv@testopt
          \else
            \@x@protect#2%
          \fi
          #1%
          {#4}}%
       \@yargdef
          #1%
           \tw@
           {#3}%
           {#5}}}
%    \end{macrocode}
% \end{macro}
%
% \begin{macro}{\lstenv@testopt}
% The difference between this macro and |\@testopt| is that we temporaryly
% reset the catcode of the EOL character |^^M| to read the optional argument.
%    \begin{macrocode}
\long\def\lstenv@testopt#1#2{%
  \@ifnextchar[{\catcode\active5\relax \lstenv@testopt@#1}%
               {#1[{#2}]}}
\def\lstenv@testopt@#1[#2]{%
    \catcode\active\active
    #1[#2]}
%    \end{macrocode}
% \end{macro}
%
% \begin{macro}{\lstenv@SkipToEnd}
% We use the temporary definition
% \begin{itemize}\item[]
%    |\long\def\lst@temp##1\|\meta{content of \textup{\cs{lstenv@endstring}}}|{\lstenv@End}|
% \end{itemize}
% which gobbles all characters up to the end of environment and finishes it.
%    \begin{macrocode}
\begingroup \lccode`\~=`\\\lowercase{%
\gdef\lstenv@SkipToEnd{%
    \long\expandafter\def\expandafter\lst@temp\expandafter##\expandafter
        1\expandafter~\lstenv@endstring{\lstenv@End}%
    \lst@temp}
}\endgroup
%    \end{macrocode}
% \end{macro}
%
% \begin{macro}{\lstenv@Error}
% is called by a lonely `end environment'.
%    \begin{macrocode}
\def\lstenv@Error#1{\PackageError{Listings}{Extra \string\end#1}%
    {I'm ignoring this, since I wasn't doing a \csname#1\endcsname.}}
%    \end{macrocode}
% \end{macro}
%
% \begin{macro}{\lst@TestEOLChar}
% Here we test for the two possible EOL characters.
%    \begin{macrocode}
\begingroup \lccode`\~=`\^^M\lowercase{%
\gdef\lst@TestEOLChar#1{%
    \def\lst@insertargs{#1}%
    \ifx ~#1\@empty \else
    \ifx^^J#1\@empty \else
        \global\let\lst@intname\lst@insertargs
        \let\lst@insertargs\@empty
    \fi \fi}
}\endgroup
%    \end{macrocode}
% \end{macro}
%
% \begin{environment}{lstlisting}
% The awkward work is done, the definition is quite easy now. We test whether
% the user has given the name argument, set the keys, and deal with
% continued line numbering.
%    \begin{macrocode}
\lstnewenvironment{lstlisting}[2][]
    {\lst@TestEOLChar{#2}%
     \lstset{#1}%
     \csname\@lst @SetFirstNumber\endcsname}
    {\csname\@lst @SaveFirstNumber\endcsname}
%    \end{macrocode}
%    \begin{macrocode}
%</kernel>
%    \end{macrocode}
% \end{environment}
%
%
% \section{Documentation support}
%
% \begin{syntax}
% \item[0.19]
%   |\begin{lstsample}|\marg{point list}\marg{left}\marg{right}
%
%   \leavevmode\hspace*{-\leftmargini}|\end{lstsample}|
%
%       Roughly speaking all material in between this environment is executed
%       `on the left side' and typeset verbatim on the right. \meta{left} is
%       executed before the left side is typeset, and similarly \meta{right}
%       before the right-hand side.
%
%       \meta{point list} is used as argument to the \keyname{point} key.
%       This is a special key used to highlight the keys in the examples.
%
% \item[1.0]
%   |\begin{lstxsample}|\marg{point list}
%
%   \leavevmode\hspace*{-\leftmargini}|\end{lstxsample}|
%
%       The material in between is (a) added to the left side of the next
%       \texttt{lstsample} environment and (b) typeset verbatim using the
%       whole line width.
%
% \item[0.21] |\newdocenvironment|\marg{name}\marg{short name}\marg{begin code}\marg{end code}
%
%       The \meta{name} environment can be used in the same way as `macro'.
%       The provided(!) definitions
%           |\Print|\meta{short name}|Name|
%       and |\SpecialMain|\meta{short name}|Index|
%       control printing in the margin and indexing as the defaults
%       |\PrintMacroName| and |\SpecialMainIndex| do.
%
%       This command is used to define the `aspect' and `lstkey' environments.
%
%\item[0.21] \texttt{macroargs} environment
%
%       This `enumerate' environment uses as labels `|#1| =', `|#2| =',
%       and so on.
%
% \item \texttt{TODO} environment
% \item \texttt{ALTERNATIVE} environment
% \item \texttt{REMOVED} environment
% \item \texttt{OLDDEF} environment
%
%       These environments enclose comments on `to do's', alternatives and
%       removed or old definitions.
%
% \item[0.21] |\lstscanlanguages|\meta{list macro}\marg{input files}\marg{don't input}
%
%       scans \marg{input files}$\setminus$\marg{don't input} for language
%       definitions. The available languages are stored in \meta{list macro}
%       using the form \meta{language}|(|\meta{dialect}|),|.
%
% \item[0.21] |\lstprintlanguages|\meta{list macro}
%
%       prints the languages in two column format.
% \end{syntax}
% and a lot of more simple commands.
%
%
% \subsection{Required packages}
%
% Most of the `required' packages are optional.
% \lsthelper{Stephan~Hennig}{2006-09-25}{documentation incompatible with algorithmic}
% noted a bug where |\ifalgorithmic| conflicts with an update to |algorithmic.sty|, so
% this has been changed to |\ifalgorithmicpkg|.
%    \begin{macrocode}
%<*doc>
\let\lstdoc@currversion\fileversion
\RequirePackage[writefile]{listings}[2004/09/07]
\newif\iffancyvrb \IfFileExists{fancyvrb.sty}{\fancyvrbtrue}{}
\newif\ifcolor \IfFileExists{color.sty}{\colortrue}{}
\lst@false
\newif\ifhyper
\@ifundefined{pdfoutput}
    {}
    {\ifnum\pdfoutput>\z@ \lst@true \fi}
\@ifundefined{VTeXversion}
    {}
    {\ifnum\OpMode>\z@ \lst@true \fi}
\lst@if \IfFileExists{hyperref.sty}{\hypertrue}{}\fi
\newif\ifalgorithmicpkg \IfFileExists{algorithmic.sty}{\algorithmicpkgtrue}{}
\newif\iflgrind \IfFileExists{lgrind.sty}{\lgrindtrue}{}
\iffancyvrb \RequirePackage{fancyvrb}\fi
\ifhyper \RequirePackage[colorlinks]{hyperref}\else
    \def\href#1{\texttt}\fi
\ifcolor \RequirePackage{color}\fi
\ifalgorithmicpkg \RequirePackage{algorithmic}\fi
\iflgrind \RequirePackage{lgrind}\fi
\RequirePackage{nameref}
\RequirePackage{url}
\renewcommand\ref{\protect\T@ref}
\renewcommand\pageref{\protect\T@pageref}
%    \end{macrocode}
%
%
% \subsection{Environments for notes}
%
% \begin{macro}{\lst@BeginRemark}
% \begin{macro}{\lst@EndRemark}
% We begin with two simple definitions \ldots
%    \begin{macrocode}
\def\lst@BeginRemark#1{%
    \begin{quote}\topsep0pt\let\small\footnotesize\small#1:}
\def\lst@EndRemark{\end{quote}}
%    \end{macrocode}
% \end{macro}\end{macro}
%
% \begin{environment}{TODO}
% \begin{environment}{ALTERNATIVE}
% \begin{environment}{REMOVED}
% \begin{environment}{OLDDEF}
% \ldots\space used to define some environments.
%    \begin{macrocode}
\newenvironment{TODO}
    {\lst@BeginRemark{To do}}{\lst@EndRemark}
\newenvironment{ALTERNATIVE}
    {\lst@BeginRemark{Alternative}}{\lst@EndRemark}
\newenvironment{REMOVED}
    {\lst@BeginRemark{Removed}}{\lst@EndRemark}
\newenvironment{OLDDEF}
    {\lst@BeginRemark{Old definition}}{\lst@EndRemark}
%    \end{macrocode}
% \end{environment}\end{environment}\end{environment}\end{environment}
%
% \begin{environment}{advise}
% \begin{macro}{\advisespace}
% The environment uses |\@listi|.
%    \begin{macrocode}
\def\advise{\par\list\labeladvise
    {\advance\linewidth\@totalleftmargin
     \@totalleftmargin\z@
     \@listi
     \let\small\footnotesize \small\sffamily
     \parsep \z@ \@plus\z@ \@minus\z@
     \topsep6\p@ \@plus1\p@\@minus2\p@
     \def\makelabel##1{\hss\llap{##1}}}}
\let\endadvise\endlist
%    \end{macrocode}
%    \begin{macrocode}
\def\advisespace{\hbox{}\qquad}
\def\labeladvise{$\to$}
%    \end{macrocode}
% \end{macro}
% \end{environment}
%
% \begin{environment}{syntax}
% \begin{macro}{\syntaxbreak}
% \begin{macro}{\syntaxnewline}
% \begin{macro}{\syntaxor}
% This environment uses |\list| with a special |\makelabel|, \ldots
%    \begin{macrocode}
\newenvironment{syntax}
   {\list{}{\itemindent-\leftmargin
    \def\makelabel##1{\hss\lst@syntaxlabel##1,,,,\relax}}}
   {\endlist}
%    \end{macrocode}
% \ldots\ which is defined here. The comma separated items are placed as
% needed.
%    \begin{macrocode}
\def\lst@syntaxlabel#1,#2,#3,#4\relax{%
    \llap{\scriptsize\itshape#3}%
    \def\lst@temp{#2}%
    \expandafter\lst@syntaxlabel@\meaning\lst@temp\relax
    \rlap{\hskip-\itemindent\hskip\itemsep\hskip\linewidth
          \llap{\ttfamily\lst@temp}\hskip\labelwidth
          \def\lst@temp{#1}%
          \ifx\lst@temp\lstdoc@currversion#1\fi}}
\def\lst@syntaxlabel@#1>#2\relax
    {\edef\lst@temp{\zap@space#2 \@empty}}
%    \end{macrocode}
%    \begin{macrocode}
\newcommand*\syntaxnewline{\newline\hbox{}\kern\labelwidth}
\newcommand*\syntaxor{\qquad or\qquad}
\newcommand*\syntaxbreak
    {\hfill\kern0pt\discretionary{}{\kern\labelwidth}{}}
\let\syntaxfill\hfill
%    \end{macrocode}
% \end{macro}
% \end{macro}
% \end{macro}
% \end{environment}
%
% \begin{macro}{\alternative}
% iterates down the list and inserts vertical rule(s).
%    \begin{macrocode}
\def\alternative#1{\lst@true \alternative@#1,\relax,}
\def\alternative@#1,{%
    \ifx\relax#1\@empty
        \expandafter\@gobble
    \else
        \ifx\@empty#1\@empty\else
            \lst@if \lst@false \else $\vert$\fi
            \textup{\texttt{#1}}%
        \fi
    \fi
    \alternative@}
%    \end{macrocode}
% \end{macro}
%
%
% \subsection{Extensions to \textsf{doc}}
%
% \begin{macro}{\m@cro@}
% We need a slight modification of \packagename{doc}'s internal macro.
% The former argument |#2| has become |#3|. This change is not marked below.
% The second argument is now \meta{short name}.
%    \begin{macrocode}
\long\def\m@cro@#1#2#3{\endgroup \topsep\MacroTopsep \trivlist
  \edef\saved@macroname{\string#3}%
  \def\makelabel##1{\llap{##1}}%
  \if@inlabel
    \let\@tempa\@empty \count@\macro@cnt
    \loop \ifnum\count@>\z@
      \edef\@tempa{\@tempa\hbox{\strut}}\advance\count@\m@ne \repeat
    \edef\makelabel##1{\llap{\vtop to\baselineskip
                               {\@tempa\hbox{##1}\vss}}}%
    \advance \macro@cnt \@ne
  \else  \macro@cnt\@ne  \fi
  \edef\@tempa{\noexpand\item[%
     #1%
       \noexpand\PrintMacroName
     \else
%    \end{macrocode}
% The next line has been modified.
%    \begin{macrocode}
       \expandafter\noexpand\csname Print#2Name\endcsname % MODIFIED
     \fi
     {\string#3}]}%
  \@tempa
  \global\advance\c@CodelineNo\@ne
   #1%
      \SpecialMainIndex{#3}\nobreak
      \DoNotIndex{#3}%
   \else
%    \end{macrocode}
% Ditto.
%    \begin{macrocode}
      \csname SpecialMain#2Index\endcsname{#3}\nobreak % MODIFIED
   \fi
  \global\advance\c@CodelineNo\m@ne
  \ignorespaces}
%    \end{macrocode}
% \end{macro}
%
% \begin{macro}{\macro}
% \begin{macro}{\environment}
% These two definitions need small adjustments due to the modified |\m@cro@|.
%    \begin{macrocode}
\def\macro{\begingroup
   \catcode`\\12
   \MakePrivateLetters \m@cro@ \iftrue {Macro}}% MODIFIED
\def\environment{\begingroup
   \catcode`\\12
   \MakePrivateLetters \m@cro@ \iffalse {Env}}% MODIFIED
%    \end{macrocode}
% \end{macro}\end{macro}
%
% \begin{macro}{\newdocenvironment}
% This command simply makes definitions similar to `environment' and provides
% the printing and indexing commands.
%    \begin{macrocode}
\def\newdocenvironment#1#2#3#4{%
    \@namedef{#1}{#3\begingroup \catcode`\\12\relax
                  \MakePrivateLetters \m@cro@ \iffalse {#2}}%
    \@namedef{end#1}{#4\endmacro}%
    \@ifundefined{Print#2Name}{\expandafter
        \let\csname Print#2Name\endcsname\PrintMacroName}{}%
    \@ifundefined{SpecialMain#2Index}{\expandafter
        \let\csname SpecialMain#2Index\endcsname\SpecialMainIndex}{}}
%    \end{macrocode}
% \end{macro}
%
% \begin{environment}{aspect}
% \begin{macro}{\PrintAspectName}
% \begin{macro}{\SpecialMainAspectIndex}
% The environment and its `print' and `index' commands.
%    \begin{macrocode}
\newdocenvironment{aspect}{Aspect}{}{}
\def\PrintAspectName#1{}
\def\SpecialMainAspectIndex#1{%
    \@bsphack
    \index{aspects:\levelchar\protect\aspectname{#1}}%
    \@esphack}
%    \end{macrocode}
% \end{macro}\end{macro}\end{environment}
%
% \begin{environment}{lstkey}
% \begin{macro}{\PrintKeyName}
% \begin{macro}{\SpecialMainKeyIndex}
% One more environment with its `print' and `index' commands.
%    \begin{macrocode}
\newdocenvironment{lstkey}{Key}{}{}
\def\PrintKeyName#1{\strut\keyname{#1}\ }
\def\SpecialMainKeyIndex#1{%
    \@bsphack
    \index{keys\levelchar\protect\keyname{#1}}%
    \@esphack}
%    \end{macrocode}
% \end{macro}\end{macro}\end{environment}
%
% \begin{macro}{\labelargcount}
% \begin{environment}{macroargs}
% We just allocate a counter and use \LaTeX's |\list| to implement this
% environment.
%    \begin{macrocode}
\newcounter{argcount}
\def\labelargcount{\texttt{\#\arabic{argcount}}\hskip\labelsep$=$}
%    \end{macrocode}
%    \begin{macrocode}
\def\macroargs{\list\labelargcount
    {\usecounter{argcount}\leftmargin=2\leftmargin
     \parsep \z@ \@plus\z@ \@minus\z@
     \topsep4\p@ \@plus\p@ \@minus2\p@
     \itemsep\z@ \@plus\z@ \@minus\z@
     \def\makelabel##1{\hss\llap{##1}}}}
\def\endmacroargs{\endlist\@endparenv}
%    \end{macrocode}
% \end{environment}\end{macro}
%
%
% \subsection{The \texttt{lstsample} environment}
%
% \begin{environment}{lstsample}
% We store the verbatim part and write the source code also to file.
%    \begin{macrocode}
\lst@RequireAspects{writefile}
%    \end{macrocode}
%    \begin{macrocode}
\newbox\lst@samplebox
\lstnewenvironment{lstsample}[3][]
    {\global\let\lst@intname\@empty
     \gdef\lst@sample{#2}%
     \setbox\lst@samplebox=\hbox\bgroup
         \setkeys{lst}{language={},style={},tabsize=4,gobble=5,%
             basicstyle=\small\ttfamily,basewidth=0.51em,point={#1}}
         #3%
         \lst@BeginAlsoWriteFile{\jobname.tmp}}
    {\lst@EndWriteFile\egroup
%    \end{macrocode}
% Now |\lst@samplebox| contains the verbatim part.
% If it's too wide, we use atop and below instead of left and right.
%    \begin{macrocode}
     \ifdim \wd\lst@samplebox>.5\linewidth
         \begin{center}%
             \hbox to\linewidth{\box\lst@samplebox\hss}%
         \end{center}%
         \lst@sampleInput
     \else
         \begin{center}%
         \begin{minipage}{0.45\linewidth}\lst@sampleInput\end{minipage}%
         \qquad
         \begin{minipage}{0.45\linewidth}%
             \hbox to\linewidth{\box\lst@samplebox\hss}%
         \end{minipage}%
         \end{center}%
     \fi}
%    \end{macrocode}
% The new keyword class \keyname{point}.
%    \begin{macrocode}
\lst@InstallKeywords{p}{point}{pointstyle}\relax{keywordstyle}{}ld
%    \end{macrocode}
% \end{environment}
%
% \begin{environment}{lstxsample}
% Omitting |\lst@EndWriteFile| leaves the file open.
%    \begin{macrocode}
\lstnewenvironment{lstxsample}[1][]
    {\begingroup
         \setkeys{lst}{belowskip=-\medskipamount,language={},style={},%
             tabsize=4,gobble=5,basicstyle=\small\ttfamily,%
             basewidth=0.51em,point={#1}}
         \lst@BeginAlsoWriteFile{\jobname.tmp}}
    {\endgroup
     \endgroup}
%    \end{macrocode}
% \end{environment}
%
% \begin{macro}{\lst@sampleInput}
% inputs the `left-hand' side.
%    \begin{macrocode}
\def\lst@sampleInput{%
    \MakePercentComment\catcode`\^^M=10\relax
    \small\lst@sample
    {\setkeys{lst}{SelectCharTable=\lst@ReplaceInput{\^\^I}%
                                  {\lst@ProcessTabulator}}%
     \leavevmode \input{\jobname.tmp}}\MakePercentIgnore}
%    \end{macrocode}
% \end{macro}
%
%
% \subsection{Miscellaneous}
%
% \paragraph{Sectioning and cross referencing}
% We begin with a redefinition paragraph.
%    \begin{macrocode}
\renewcommand\paragraph{\@startsection{paragraph}{4}{\z@}%
                                      {1.25ex \@plus1ex \@minus.2ex}%
                                      {-1em}%
                                      {\normalfont\normalsize\bfseries}}
%    \end{macrocode}
% We introduce |\lstref| which prints section number together with its name.
%    \begin{macrocode}
\def\lstref#1{\emph{\ref{#1} \nameref{#1}}}
%    \end{macrocode}
% Moreover we adjust the table of contents.  The |\phantomsection| before
% adding the contents line provides \packagename{hyperref} with an appropriate
% destination for the contents line link, thereby ensuring that the contents
% line is at the right level in the PDF bookmark tree.
%    \begin{macrocode}
\def\@part[#1]#2{\ifhyper\phantomsection\fi
    \addcontentsline{toc}{part}{#1}%
    {\parindent\z@ \raggedright \interlinepenalty\@M
     \normalfont \huge \bfseries #2\markboth{}{}\par}%
    \nobreak\vskip 3ex\@afterheading}
\renewcommand*\l@section[2]{%
    \addpenalty\@secpenalty
    \addvspace{.25em \@plus\p@}%
    \setlength\@tempdima{1.5em}%
    \begingroup
      \parindent \z@ \rightskip \@pnumwidth
      \parfillskip -\@pnumwidth
      \leavevmode
      \advance\leftskip\@tempdima
      \hskip -\leftskip
      #1\nobreak\hfil \nobreak\hb@xt@\@pnumwidth{\hss #2}\par
    \endgroup}
\renewcommand*\l@subsection{\@dottedtocline{2}{0pt}{2.3em}}
\renewcommand*\l@subsubsection{\@dottedtocline{3}{0pt}{3.2em}}
%    \end{macrocode}
%
% \paragraph{Indexing}
% The `user' commands. |\rstyle| is defined below.
%    \begin{macrocode}
\newcommand\ikeyname[1]{%
    \lstkeyindex{#1}{}%
    \lstaspectindex{#1}{}%
    \keyname{#1}}
\newcommand\ekeyname[1]{%
    \@bsphack
    \lstkeyindex{#1}{}%
    \lstaspectindex{#1}{}%
    \@esphack}
\newcommand\rkeyname[1]{%
    \@bsphack
    \lstkeyindex{#1}{}%
    \lstaspectindex{#1}{}%
    \@esphack{\rstyle\keyname{#1}}}
%    \end{macrocode}
%    \begin{macrocode}
\newcommand\icmdname[1]{%
    \@bsphack
    \lstaspectindex{#1}{}%
    \@esphack\texttt{\string#1}}
\newcommand\rcmdname[1]{%
    \@bsphack
    \lstaspectindex{#1}{}%
    \@esphack\texttt{\rstyle\string#1}}
%    \end{macrocode}
% One of the two yet unknown `index'-macros is empty, the other looks up
% the aspect name for the given argument.
%    \begin{macrocode}
\def\lstaspectindex#1#2{%
    \global\@namedef{lstkandc@\string#1}{}%
    \@ifundefined{lstisaspect@\string#1}
        {\index{unknown\levelchar
                \protect\texttt{\protect\string\string#1}#2}}%
        {\index{\@nameuse{lstisaspect@\string#1}\levelchar
                \protect\texttt{\protect\string\string#1}#2}}%
}
\def\lstkeyindex#1#2{%
%    \index{key\levelchar\protect\keyname{#1}#2}%
}
%    \end{macrocode}
% The key/command to aspect relation is defined near the top of this file using
% the following command. In future the package should read this information
% from the aspect files.
%    \begin{macrocode}
\def\lstisaspect[#1]#2{%
    \global\@namedef{lstaspect@#1}{#2}%
    \lst@AddTo\lst@allkeysandcmds{,#2}%
    \@for\lst@temp:=#2\do
    {\ifx\@empty\lst@temp\else
         \global\@namedef{lstisaspect@\lst@temp}{#1}%
     \fi}}
\gdef\lst@allkeysandcmds{}
%    \end{macrocode}
% This relation is also good to print all keys and commands of a particular
% aspect \ldots
%    \begin{macrocode}
\def\lstprintaspectkeysandcmds#1{%
    \lst@true
    \expandafter\@for\expandafter\lst@temp
    \expandafter:\expandafter=\csname lstaspect@#1\endcsname\do
    {\lst@if\lst@false\else, \fi \texttt{\lst@temp}}}
%    \end{macrocode}
% \ldots\ or to check the reference. Note that we've defined
% |\lstkandc@|\meta{name} in |\lstaspectindex|.
%    \begin{macrocode}
\def\lstcheckreference{%
   \@for\lst@temp:=\lst@allkeysandcmds\do
   {\ifx\lst@temp\@empty\else
        \@ifundefined{lstkandc@\lst@temp}
        {\typeout{\lst@temp\space not in reference guide?}}{}%
    \fi}}
%    \end{macrocode}
%
% \paragraph{Unique styles}
%    \begin{macrocode}
\newcommand*\lst{\texttt{lst}}
\newcommand*\Cpp{C\texttt{++}}
\let\keyname\texttt
\let\keyvalue\texttt
\let\hookname\texttt
\newcommand*\aspectname[1]{{\normalfont\sffamily#1}}
%    \end{macrocode}
%    \begin{macrocode}
\DeclareRobustCommand\packagename[1]{%
    {\leavevmode\text@command{#1}%
     \switchfontfamily\sfdefault\rmdefault
     \check@icl #1\check@icr
     \expandafter}}%
\renewcommand\packagename[1]{{\normalfont\sffamily#1}}
\def\switchfontfamily#1#2{%
    \begingroup\xdef\@gtempa{#1}\endgroup
    \ifx\f@family\@gtempa\fontfamily#2%
                    \else\fontfamily#1\fi
    \selectfont}
%    \end{macrocode}
% The color mainly for keys and commands in the reference guide.
%    \begin{macrocode}
\ifcolor
    \definecolor{darkgreen}{rgb}{0,0.5,0}
    \def\rstyle{\color{darkgreen}}
\else
    \let\rstyle\empty
\fi
%    \end{macrocode}
%
% \paragraph{Commands for credits and helpers}
%    \begin{macrocode}
\gdef\lst@emails{}
\newcommand*\lstthanks[2]
    {#1\lst@AddTo\lst@emails{,#1,<#2>}%
     \ifx\@empty#2\@empty\typeout{Missing email for #1}\fi}
\newcommand*\lsthelper[3]
    {{\let~\ #1}%
     \lst@IfOneOf#1\relax\lst@emails
     {}{\typeout{^^JWarning: Unknown helper #1.^^J}}}
%    \end{macrocode}
%
% \paragraph{Languages and styles}
%    \begin{macrocode}
\lstdefinelanguage[doc]{Pascal}{%
  morekeywords={alfa,and,array,begin,boolean,byte,case,char,const,div,%
     do,downto,else,end,false,file,for,function,get,goto,if,in,%
     integer,label,maxint,mod,new,not,of,or,pack,packed,page,program,%
     procedure,put,read,readln,real,record,repeat,reset,rewrite,set,%
     text,then,to,true,type,unpack,until,var,while,with,write,writeln},%
  sensitive=false,%
  morecomment=[s]{(*}{*)},%
  morecomment=[s]{\{}{\}},%
  morestring=[d]{'}}
%    \end{macrocode}
%    \begin{macrocode}
\lstdefinestyle{}
    {basicstyle={},%
     keywordstyle=\bfseries,identifierstyle={},%
     commentstyle=\itshape,stringstyle={},%
     numberstyle={},stepnumber=1,%
     pointstyle=\pointstyle}
\def\pointstyle{%
    {\let\lst@um\@empty \xdef\@gtempa{\the\lst@token}}%
    \expandafter\lstkeyindex\expandafter{\@gtempa}{}%
    \expandafter\lstaspectindex\expandafter{\@gtempa}{}%
    \rstyle}
\lstset{defaultdialect=[doc]Pascal,language=Pascal,style={}}
%    \end{macrocode}
%
%
% \subsection{Scanning languages}
%
% \begin{macro}{\lstscanlanguages}
% We modify some internal definitions and input the files.
%    \begin{macrocode}
\def\lstscanlanguages#1#2#3{%
    \begingroup
        \def\lst@DefDriver@##1##2##3##4[##5]##6{%
           \lst@false
           \lst@lAddTo\lst@scan{##6(##5),}%
           \begingroup
           \@ifnextchar[{\lst@XDefDriver{##1}##3}{\lst@DefDriver@@##3}}%
        \def\lst@XXDefDriver[##1]{}%
        \lst@InputCatcodes
        \def\lst@dontinput{#3}%
        \let\lst@scan\@empty
        \lst@for{#2}\do{%
            \lst@IfOneOf##1\relax\lst@dontinput
                {}%
                {\InputIfFileExists{##1}{}{}}}%
        \global\let\@gtempa\lst@scan
    \endgroup
    \let#1\@gtempa}
%    \end{macrocode}
% \end{macro}
%
% \begin{macro}{\lstprintlanguages}
% |\do| creates a box of width 0.5|\linewidth| or |\linewidth| depending
% on how wide the argument is. This leads to `two column' output.
% The other main thing is sorting the list and begin with the output.
%    \begin{macrocode}
\def\lstprintlanguages#1{%
    \def\do##1{\setbox\@tempboxa\hbox{##1\space\space}%
        \ifdim\wd\@tempboxa<.5\linewidth \wd\@tempboxa.5\linewidth
                                   \else \wd\@tempboxa\linewidth \fi
        \box\@tempboxa\allowbreak}%
    \begin{quote}
      \par\noindent
      \hyphenpenalty=\@M \rightskip=\z@\@plus\linewidth\relax
      \lst@BubbleSort#1%
      \expandafter\lst@NextLanguage#1\relax(\relax),%
    \end{quote}}
%    \end{macrocode}
% We get and define the current language and \ldots
%    \begin{macrocode}
\def\lst@NextLanguage#1(#2),{%
    \ifx\relax#1\else
        \def\lst@language{#1}\def\lst@dialects{(#2),}%
        \expandafter\lst@NextLanguage@
    \fi}
%    \end{macrocode}
% \ldots\space gather all available dialect of this language (note that the
% list has been sorted)
%    \begin{macrocode}
\def\lst@NextLanguage@#1(#2),{%
    \def\lst@temp{#1}%
    \ifx\lst@temp\lst@language
        \lst@lAddTo\lst@dialects{(#2),}%
        \expandafter\lst@NextLanguage@
    \else
%    \end{macrocode}
% or begin to print this language with all its dialects. Therefor we sort the
% dialects
%    \begin{macrocode}
        \do{\lst@language
        \ifx\lst@dialects\lst@emptydialect\else
            \expandafter\lst@NormedDef\expandafter\lst@language
                \expandafter{\lst@language}%
            \space(%
            \lst@BubbleSort\lst@dialects
            \expandafter\lst@PrintDialects\lst@dialects(\relax),%
            )%
        \fi}%
        \def\lst@next{\lst@NextLanguage#1(#2),}%
        \expandafter\lst@next
    \fi}
\def\lst@emptydialect{(),}
%    \end{macrocode}
% and print the dialect with appropriate commas in between.
%    \begin{macrocode}
\def\lst@PrintDialects(#1),{%
    \ifx\@empty#1\@empty empty\else
        \lst@PrintDialect{#1}%
    \fi
    \lst@PrintDialects@}
\def\lst@PrintDialects@(#1),{%
    \ifx\relax#1\else
        , \lst@PrintDialect{#1}%
        \expandafter\lst@PrintDialects@
    \fi}
%    \end{macrocode}
% Here we take care of default dialects.
%    \begin{macrocode}
\def\lst@PrintDialect#1{%
    \lst@NormedDef\lst@temp{#1}%
    \expandafter\ifx\csname\@lst dd@\lst@language\endcsname\lst@temp
        \texttt{\underbar{#1}}%
    \else
        \texttt{#1}%
    \fi}
%    \end{macrocode}
% \end{macro}
%
%
% \subsection{Bubble sort}
%
% \begin{macro}{\lst@IfLE}
% \meta{string 1}|\relax\@empty|\meta{string 2}|\relax\@empty|\marg{then}\meta{else}.
% If \meta{string 1} $\leq$ \meta{string 2}, we execute \meta{then} and
% \meta{else} otherwise.
% Note that this comparision is case insensitive.
%    \begin{macrocode}
\def\lst@IfLE#1#2\@empty#3#4\@empty{%
    \ifx #1\relax
        \let\lst@next\@firstoftwo
    \else \ifx #3\relax
        \let\lst@next\@secondoftwo
    \else
        \lowercase{\ifx#1#3}%
            \def\lst@next{\lst@IfLE#2\@empty#4\@empty}%
        \else
            \lowercase{\ifnum`#1<`#3}\relax
                \let\lst@next\@firstoftwo
            \else
                \let\lst@next\@secondoftwo
            \fi
        \fi
    \fi \fi
    \lst@next}
%    \end{macrocode}
% \end{macro}
%
% \begin{macro}{\lst@BubbleSort}
% is in fact a derivation of bubble sort.
%    \begin{macrocode}
\def\lst@BubbleSort#1{%
    \ifx\@empty#1\else
        \lst@false
%    \end{macrocode}
% We `bubble sort' the first, second, \ldots\ elements and \ldots
%    \begin{macrocode}
        \expandafter\lst@BubbleSort@#1\relax,\relax,%
%    \end{macrocode}
% \ldots\space then the second, third, \ldots\ elements until no elemets have
% been swapped.
%    \begin{macrocode}
        \expandafter\lst@BubbleSort@\expandafter,\lst@sorted
                                      \relax,\relax,%
        \let#1\lst@sorted
        \lst@if
            \def\lst@next{\lst@BubbleSort#1}%
            \expandafter\expandafter\expandafter\lst@next
        \fi
    \fi}
\def\lst@BubbleSort@#1,#2,{%
    \ifx\@empty#1\@empty
        \def\lst@sorted{#2,}%
        \def\lst@next{\lst@BubbleSort@@}%
    \else
        \let\lst@sorted\@empty
        \def\lst@next{\lst@BubbleSort@@#1,#2,}%
    \fi
    \lst@next}
%    \end{macrocode}
% But the bubbles rise only one step per call. Putting the elements at their
% top most place would be inefficient (since \TeX\ had to read much more
% parameters in this case).
%    \begin{macrocode}
\def\lst@BubbleSort@@#1,#2,{%
    \ifx\relax#1\else
        \ifx\relax#2%
            \lst@lAddTo\lst@sorted{#1,}%
            \expandafter\expandafter\expandafter\lst@BubbleSort@@@
        \else
            \lst@IfLE #1\relax\@empty #2\relax\@empty
                          {\lst@lAddTo\lst@sorted{#1,#2,}}%
                {\lst@true \lst@lAddTo\lst@sorted{#2,#1,}}%
            \expandafter\expandafter\expandafter\lst@BubbleSort@@
        \fi
    \fi}
\def\lst@BubbleSort@@@#1\relax,{}
%    \end{macrocode}
%    \begin{macrocode}
%</doc>
%    \end{macrocode}
% \end{macro}
%
%
% \section{Interfaces to other programs}
%
%
% \subsection{0.21 compatibility}
%
% \begin{aspect}{0.21}
% Some keys have just been renamed.
%    \begin{macrocode}
%<*0.21>
\lst@BeginAspect{0.21}
%    \end{macrocode}
%
%    \begin{macrocode}
\lst@Key{labelstyle}{}{\def\lst@numberstyle{#1}}
\lst@Key{labelsep}{10pt}{\def\lst@numbersep{#1}}
\lst@Key{labelstep}{0}{%
    \ifnum #1=\z@ \KV@lst@numbers{none}%
            \else \KV@lst@numbers{left}\fi
    \def\lst@stepnumber{#1\relax}}
\lst@Key{firstlabel}\relax{\def\lst@firstnumber{#1\relax}}
\lst@Key{advancelabel}\relax{\def\lst@advancenumber{#1\relax}}
\let\c@lstlabel\c@lstnumber
\lst@AddToHook{Init}{\def\thelstnumber{\thelstlabel}}
\newcommand*\thelstlabel{\@arabic\c@lstlabel}
%    \end{macrocode}
% A |\let| in the second last line has been changed to |\def| after a bug
% report by \lsthelper{Venkatesh~Prasad~Ranganath}{2002/08/31}{Undefined
% control sequence \thelstnumber with 0.21-option}.
%    \begin{macrocode}
\lst@Key{first}\relax{\def\lst@firstline{#1\relax}}
\lst@Key{last}\relax{\def\lst@lastline{#1\relax}}
%    \end{macrocode}
%    \begin{macrocode}
\lst@Key{framerulewidth}{.4pt}{\def\lst@framerulewidth{#1}}
\lst@Key{framerulesep}{2pt}{\def\lst@rulesep{#1}}
\lst@Key{frametextsep}{3pt}{\def\lst@frametextsep{#1}}
\lst@Key{framerulecolor}{}{\lstKV@OptArg[]{#1}%
    {\ifx\@empty##2\@empty
         \let\lst@rulecolor\@empty
     \else
         \ifx\@empty##1\@empty
             \def\lst@rulecolor{\color{##2}}%
         \else
             \def\lst@rulecolor{\color[##1]{##2}}%
         \fi
     \fi}}
\lst@Key{backgroundcolor}{}{\lstKV@OptArg[]{#1}%
    {\ifx\@empty##2\@empty
         \let\lst@bkgcolor\@empty
     \else
         \ifx\@empty##1\@empty
             \def\lst@bkgcolor{\color{##2}}%
         \else
             \def\lst@bkgcolor{\color[##1]{##2}}%
         \fi
     \fi}}
\lst@Key{framespread}{\z@}{\def\lst@framespread{#1}}
\lst@AddToHook{PreInit}
    {\@tempdima\lst@framespread\relax \divide\@tempdima\tw@
     \edef\lst@framextopmargin{\the\@tempdima}%
     \let\lst@framexrightmargin\lst@framextopmargin
     \let\lst@framexbottommargin\lst@framextopmargin
     \advance\@tempdima\lst@xleftmargin\relax
     \edef\lst@framexleftmargin{\the\@tempdima}}
%    \end{macrocode}
% \lsthelper{Harald~Harders}{1998/03/30}{inner- and outerspread} had the idea
% of two spreads (inner and outer). We either divide the dimension by two or
% assign the two dimensions to inner- and outerspread.
%    \begin{macrocode}
\newdimen\lst@innerspread \newdimen\lst@outerspread
\lst@Key{spread}{\z@,\z@}{\lstKV@CSTwoArg{#1}%
    {\lst@innerspread##1\relax
     \ifx\@empty##2\@empty
         \divide\lst@innerspread\tw@\relax
         \lst@outerspread\lst@innerspread
     \else
         \lst@outerspread##2\relax
     \fi}}
\lst@AddToHook{BoxUnsafe}{\lst@outerspread\z@ \lst@innerspread\z@}
\lst@Key{wholeline}{false}[t]{\lstKV@SetIf{#1}\lst@ifresetmargins}
\lst@Key{indent}{\z@}{\def\lst@xleftmargin{#1}}
\lst@AddToHook{PreInit}
    {\lst@innerspread=-\lst@innerspread
     \lst@outerspread=-\lst@outerspread
     \ifodd\c@page \advance\lst@innerspread\lst@xleftmargin
             \else \advance\lst@outerspread\lst@xleftmargin \fi
     \ifodd\c@page
         \edef\lst@xleftmargin{\the\lst@innerspread}%
         \edef\lst@xrightmargin{\the\lst@outerspread}%
     \else
         \edef\lst@xleftmargin{\the\lst@outerspread}%
         \edef\lst@xrightmargin{\the\lst@innerspread}%
     \fi}
%    \end{macrocode}
%    \begin{macrocode}
\lst@Key{defaultclass}\relax{\def\lst@classoffset{#1}}
\lst@Key{stringtest}\relax{}% dummy
\lst@Key{outputpos}\relax{\lst@outputpos#1\relax\relax}
%    \end{macrocode}
%    \begin{macrocode}
\lst@Key{stringspaces}\relax[t]{\lstKV@SetIf{#1}\lst@ifshowstringspaces}
\lst@Key{visiblespaces}\relax[t]{\lstKV@SetIf{#1}\lst@ifshowspaces}
\lst@Key{visibletabs}\relax[t]{\lstKV@SetIf{#1}\lst@ifshowtabs}
%    \end{macrocode}
%
%    \begin{macrocode}
\lst@EndAspect
%</0.21>
%    \end{macrocode}
% \end{aspect}
%
%
% \subsection{\textsf{fancyvrb}}
%
% \lsthelper{Denis~Girou}{1998/07/26}{fancyvrb} asked whether
% \packagename{fancyvrb} and \packagename{listings} could work together.
%
% \begin{lstkey}{fancyvrb}
% We set the boolean and call a submacro.
%    \begin{macrocode}
%<*kernel>
\lst@Key{fancyvrb}\relax[t]{%
    \lstKV@SetIf{#1}\lst@iffancyvrb
    \lstFV@fancyvrb}
\ifx\lstFV@fancyvrb\@undefined
    \gdef\lstFV@fancyvrb{\lst@RequireAspects{fancyvrb}\lstFV@fancyvrb}
\fi
%</kernel>
%    \end{macrocode}
% \end{lstkey}
%
% \begin{aspect}{fancyvrb}
% We end the job if \packagename{fancyvrb} is not present.
%    \begin{macrocode}
%<*misc>
\lst@BeginAspect{fancyvrb}
%    \end{macrocode}
%    \begin{macrocode}
\@ifundefined{FancyVerbFormatLine}
    {\typeout{^^J%
     ***^^J%
     *** `listings.sty' needs `fancyvrb.sty' right now.^^J%
     *** Please ensure its availability and try again.^^J%
     ***^^J}%
     \batchmode \@@end}{}
%    \end{macrocode}
%
% \begin{macro}{\lstFV@fancyvrb}
% We assign the correct |\FancyVerbFormatLine| macro.
%    \begin{macrocode}
\gdef\lstFV@fancyvrb{%
    \lst@iffancyvrb
        \ifx\FancyVerbFormatLine\lstFV@FancyVerbFormatLine\else
            \let\lstFV@FVFL\FancyVerbFormatLine
            \let\FancyVerbFormatLine\lstFV@FancyVerbFormatLine
        \fi
    \else
        \ifx\lstFV@FVFL\@undefined\else
            \let\FancyVerbFormatLine\lstFV@FVFL
            \let\lstFV@FVFL\@undefined
        \fi
    \fi}
%    \end{macrocode}
% \end{macro}
%
% \begin{macro}{\lstFV@VerbatimBegin}
% We initialize things if necessary.
%    \begin{macrocode}
\gdef\lstFV@VerbatimBegin{%
    \ifx\FancyVerbFormatLine\lstFV@FancyVerbFormatLine
        \lsthk@TextStyle \lsthk@BoxUnsafe
        \lsthk@PreSet
        \lst@activecharsfalse
        \let\normalbaselines\relax
%    \end{macrocode}
% \begin{TODO}
% Is this |\let| bad?
% \end{TODO}
% I inserted |\lst@ifresetmargins|\ldots|\fi| after a bug report from
% \lsthelper{Peter~Bartke}{1999/11/18}{wrong fancyvrb frame}. The linewidth
% is saved and restored since a bug report by \lsthelper{Denis~Girou}
% {2003/07/04}{problem in list environments with fancyvrb=true}.
%    \begin{macrocode}
\xdef\lstFV@RestoreData{\noexpand\linewidth\the\linewidth\relax}%
        \lst@Init\relax
        \lst@ifresetmargins \advance\linewidth-\@totalleftmargin \fi
\lstFV@RestoreData
        \everypar{}\global\lst@newlines\z@
        \lst@mode\lst@nomode \let\lst@entermodes\@empty
        \lst@InterruptModes
%    \end{macrocode}
% \lsthelper{Rolf~Niepraschk}{1998/11/25}{ligatures problem} reported a bug
% concerning ligatures to \lsthelper{Denis~Girou}{1998/11/27}{use |\@noligs|}.
%    \begin{macrocode}
%% D.G. modification begin - Nov. 25, 1998
        \let\@noligs\relax
%% D.G. modification end
    \fi}
%    \end{macrocode}
% \end{macro}
%
% \begin{macro}{\lstFV@VerbatimEnd}
% A box and macro must exist after |\lst@DeInit|.
% We store them globally.
%    \begin{macrocode}
\gdef\lstFV@VerbatimEnd{%
    \ifx\FancyVerbFormatLine\lstFV@FancyVerbFormatLine
        \global\setbox\lstFV@gtempboxa\box\@tempboxa
        \global\let\@gtempa\FV@ProcessLine
        \lst@mode\lst@Pmode
        \lst@DeInit
        \let\FV@ProcessLine\@gtempa
        \setbox\@tempboxa\box\lstFV@gtempboxa
        \par
    \fi}
%    \end{macrocode}
% The |\par| has been added after a bug report by \lsthelper{Peter~Bartke}
% {2002/04/10}{TeX is not in vertical mode when leaving "Verbatim"}.
%    \begin{macrocode}
\newbox\lstFV@gtempboxa
%    \end{macrocode}
% \end{macro}
%
% \noindent
% We insert |\lstFV@VerbatimBegin| and |\lstFV@VerbatimEnd| where necessary.
%    \begin{macrocode}
\lst@AddTo\FV@VerbatimBegin\lstFV@VerbatimBegin
\lst@AddToAtTop\FV@VerbatimEnd\lstFV@VerbatimEnd
\lst@AddTo\FV@LVerbatimBegin\lstFV@VerbatimBegin
\lst@AddToAtTop\FV@LVerbatimEnd\lstFV@VerbatimEnd
\lst@AddTo\FV@BVerbatimBegin\lstFV@VerbatimBegin
\lst@AddToAtTop\FV@BVerbatimEnd\lstFV@VerbatimEnd
%    \end{macrocode}
%
% \begin{macro}{\lstFV@FancyVerbFormatLine}
% `@' terminates the argument of |\lst@FVConvert|.
% Moreover |\lst@ReenterModes| and |\lst@InterruptModes| encloses some code.
% This ensures that we have same group level at the beginning and at the end of
% the macro---even if the user begins but doesn't end a comment, which means
% one open group.
% Furthermore we use |\vtop| and reset |\lst@newlines| to allow line breaking.
%    \begin{macrocode}
\gdef\lstFV@FancyVerbFormatLine#1{%
    \let\lst@arg\@empty \lst@FVConvert#1\@nil
    \global\lst@newlines\z@
    \vtop{\noindent\lst@parshape
          \lst@ReenterModes
          \lst@arg \lst@PrintToken\lst@EOLUpdate\lsthk@InitVarsBOL
          \lst@InterruptModes}}
%    \end{macrocode}
% The |\lst@parshape| inside |\vtop| is due to a bug report from
% \lsthelper{Peter~Bartke}{1999/11/18}{wrong par indention with fancyvrb}.
% A |\leavevmode| became |\noindent|.
% \end{macro}
%
% \begin{lstkey}{fvcmdparams}
% \begin{lstkey}{morefvcmdparams}
% These keys adjust \lst@FVcmdparams, which will be used by the following
% conversion macro. The base set of commands and parameter numbers was
% provided by \lsthelper{Denis~Girou}{2002/05/31}{init of fvcmdparams}.
%    \begin{macrocode}
\lst@Key{fvcmdparams}%
    {\overlay\@ne}%
    {\def\lst@FVcmdparams{,#1}}
\lst@Key{morefvcmdparams}\relax{\lst@lAddTo\lst@FVcmdparams{,#1}}
%    \end{macrocode}
% \end{lstkey}
% \end{lstkey}
%
% \begin{macro}{\lst@FVConvert}
% We do conversion or \ldots
%    \begin{macrocode}
\gdef\lst@FVConvert{\@tempcnta\z@ \lst@FVConvertO@}%
\gdef\lst@FVConvertO@{%
    \ifcase\@tempcnta
        \expandafter\futurelet\expandafter\@let@token
        \expandafter\lst@FVConvert@@
    \else
%    \end{macrocode}
% \ldots\ we append arguments without conversion, argument by argument,
% |\@tempcnta| times.
%    \begin{macrocode}
        \expandafter\lst@FVConvertO@a
    \fi}
\gdef\lst@FVConvertO@a#1{%
    \lst@lAddTo\lst@arg{{#1}}\advance\@tempcnta\m@ne
    \lst@FVConvertO@}%
%    \end{macrocode}
% Since |\@ifnextchar\bgroup| might fail, we have to use |\ifcat| here.
% Bug reported by \lsthelper{Denis~Girou}{1999/07/26}{fancyvrb=true + `second
% commandchar' other than \{ doesn't work}.
% However we don't gobble space tokens as |\@ifnextchar| does.
%    \begin{macrocode}
\gdef\lst@FVConvert@@{%
    \ifcat\noexpand\@let@token\bgroup \expandafter\lst@FVConvertArg
                                \else \expandafter\lst@FVConvert@ \fi}
%    \end{macrocode}
% Coming to such a catcode${}={}$1 character we convert the argument and add
% it together with group delimiters to |\lst@arg|.
% We also add |\lst@PrintToken|, which prints all collected characters before
% we forget them.
% Finally we continue the conversion.
%    \begin{macrocode}
\gdef\lst@FVConvertArg#1{%
    {\let\lst@arg\@empty
     \lst@FVConvert#1\@nil
     \global\let\@gtempa\lst@arg}%
     \lst@lExtend\lst@arg{\expandafter{\@gtempa\lst@PrintToken}}%
     \lst@FVConvert}
%    \end{macrocode}
%    \begin{macrocode}
\gdef\lst@FVConvert@#1{%
    \ifx \@nil#1\else
       \if\relax\noexpand#1%
          \lst@lAddTo\lst@arg{\lst@OutputLostSpace\lst@PrintToken#1}%
       \else
          \lccode`\~=`#1\lowercase{\lst@lAddTo\lst@arg~}%
       \fi
       \expandafter\lst@FVConvert
    \fi}
%    \end{macrocode}
% Having no |\bgroup|, we look whether we've found the end of the input, and
% convert one token ((non)active character or control sequence).
%    \begin{macrocode}
\gdef\lst@FVConvert@#1{%
    \ifx \@nil#1\else
       \if\relax\noexpand#1%
          \lst@lAddTo\lst@arg{\lst@OutputLostSpace\lst@PrintToken#1}%
%    \end{macrocode}
% Here we check for registered commands with arguments and set the value of
% |\@tempcnta| as required.
%    \begin{macrocode}
          \def\lst@temp##1,#1##2,##3##4\relax{%
              \ifx##3\@empty \else \@tempcnta##2\relax \fi}%
          \expandafter\lst@temp\lst@FVcmdparams,#1\z@,\@empty\relax
       \else
          \lccode`\~=`#1\lowercase{\lst@lAddTo\lst@arg~}%
       \fi
       \expandafter\lst@FVConvertO@
    \fi}
%    \end{macrocode}
% \end{macro}
%
%    \begin{macrocode}
\lst@EndAspect
%</misc>
%    \end{macrocode}
% \end{aspect}
%
%
% \subsection{Omega support}
%
% \begingroup
% $\Omega$ support looks easy---I hope it works at least in some cases.
%    \begin{macrocode}
%<*kernel>
%    \end{macrocode}
%    \begin{macrocode}
\@ifundefined{ocp}{}
    {\lst@AddToHook{OutputBox}%
         {\let\lst@ProcessLetter\@firstofone
          \let\lst@ProcessDigit\@firstofone
          \let\lst@ProcessOther\@firstofone}}
%    \end{macrocode}
%    \begin{macrocode}
%</kernel>
%    \end{macrocode}
% \endgroup
%
%
% \subsection{\textsf{LGrind}}
%
% \begin{aspect}{lgrind}
% \begin{macro}{\lst@LGGetNames}
% is used to extract the language names from |\lst@arg| (the
% \packagename{LGrind} definition).
%    \begin{macrocode}
%<*misc>
\lst@BeginAspect[keywords,comments,strings,language]{lgrind}
%    \end{macrocode}
%    \begin{macrocode}
\gdef\lst@LGGetNames#1:#2\relax{%
    \lst@NormedDef\lstlang@{#1}\lst@ReplaceInArg\lstlang@{|,}%
    \def\lst@arg{:#2}}
%    \end{macrocode}
% \end{macro}
%
% \begin{macro}{\lst@LGGetValue}
% returns in |\lst@LGvalue| the value of capability |#1| given by the list
% |\lst@arg|. If |#1| is not found, we have |\lst@if|=|\iffalse|.
% Otherwise it is true and the ``cap=value'' pair is removed from the list.
% First we test for |#1| and
%    \begin{macrocode}
\gdef\lst@LGGetValue#1{%
    \lst@false
    \def\lst@temp##1:#1##2##3\relax{%
        \ifx\@empty##2\else \lst@LGGetValue@{#1}\fi}
    \expandafter\lst@temp\lst@arg:#1\@empty\relax}
%    \end{macrocode}
% remove the pair if necessary.
%    \begin{macrocode}
\gdef\lst@LGGetValue@#1{%
    \lst@true
    \def\lst@temp##1:#1##2:##3\relax{%
        \@ifnextchar=\lst@LGGetValue@@{\lst@LGGetValue@@=}##2\relax
        \def\lst@arg{##1:##3}}%
    \expandafter\lst@temp\lst@arg\relax}
\gdef\lst@LGGetValue@@=#1\relax{\def\lst@LGvalue{#1}}
%    \end{macrocode}
% \end{macro}
%
% \begin{macro}{\lst@LGGetComment}
% stores the comment delimiters (enclosed in braces) in |#2| if comment of type
% |#1| is present and not a comment line. Otherwise |#2| is empty.
%    \begin{macrocode}
\gdef\lst@LGGetComment#1#2{%
    \let#2\@empty
    \lst@LGGetValue{#1b}%
    \lst@if
        \let#2\lst@LGvalue
        \lst@LGGetValue{#1e}%
        \ifx\lst@LGvalue\lst@LGEOL
            \edef\lstlang@{\lstlang@,commentline={#2}}%
            \let#2\@empty
        \else
            \edef#2{{#2}{\lst@LGvalue}}%
        \fi
    \fi}
%    \end{macrocode}
% \end{macro}
%
% \begin{macro}{\lst@LGGetString}
% does the same for string delimiters, but it doesn't `return' any value.
%    \begin{macrocode}
\gdef\lst@LGGetString#1#2{%
    \lst@LGGetValue{#1b}%
    \lst@if
        \let#2\lst@LGvalue
        \lst@LGGetValue{#1e}%
        \ifx\lst@LGvalue\lst@LGEOL
            \edef\lstlang@{\lstlang@,morestringizer=[l]{#2}}%
        \else
%    \end{macrocode}
% we must check for |\e|, i.e.~whether we have to use \texttt doubled or
% \texttt backslashed stringizer.
%    \begin{macrocode}
            \ifx #2\lst@LGvalue
                \edef\lstlang@{\lstlang@,morestringizer=[d]{#2}}%
            \else
                \edef\lst@temp{\lst@LGe#2}%
                \ifx \lst@temp\lst@LGvalue
                    \edef\lstlang@{\lstlang@,morestringizer=[b]{#2}}%
                \else
                    \PackageWarning{Listings}%
                    {String #2...\lst@LGvalue\space not supported}%
                \fi
            \fi
        \fi
    \fi}
%    \end{macrocode}
% \end{macro}
%
% \begin{macro}{\lst@LGDefLang}
% defines the language given by |\lst@arg|, the definition part, and
% |\lst@language@|, the language name. First we remove unwanted stuff from
% |\lst@arg|, e.g.~we replace |:\ :| by |:|.
%    \begin{macrocode}
\gdef\lst@LGDefLang{%
    \lst@LGReplace
    \let\lstlang@\empty
%    \end{macrocode}
% Get the keywords and values of friends.
%    \begin{macrocode}
    \lst@LGGetValue{kw}%
    \lst@if
        \lst@ReplaceInArg\lst@LGvalue{{ },}%
        \edef\lstlang@{\lstlang@,keywords={\lst@LGvalue}}%
    \fi
%    \end{macrocode}
%    \begin{macrocode}
    \lst@LGGetValue{oc}%
    \lst@if
        \edef\lstlang@{\lstlang@,sensitive=f}%
    \fi
%    \end{macrocode}
%    \begin{macrocode}
    \lst@LGGetValue{id}%
    \lst@if
        \edef\lstlang@{\lstlang@,alsoletter=\lst@LGvalue}%
    \fi
%    \end{macrocode}
% Now we get the comment delimiters and use them as single or double comments
% according to whether there are two or four delimiters.
% Note that |\lst@LGGetComment| takes care of comment lines.
%    \begin{macrocode}
    \lst@LGGetComment a\lst@LGa
    \lst@LGGetComment c\lst@LGc
    \ifx\lst@LGa\@empty
        \ifx\lst@LGc\@empty\else
            \edef\lstlang@{\lstlang@,singlecomment=\lst@LGc}%
        \fi
    \else
        \ifx\lst@LGc\@empty
            \edef\lstlang@{\lstlang@,singlecomment=\lst@LGa}%
        \else
            \edef\lstlang@{\lstlang@,doublecomment=\lst@LGc\lst@LGa}%
        \fi
    \fi
%    \end{macrocode}
% Now we parse the stringizers.
%    \begin{macrocode}
    \lst@LGGetString s\lst@LGa
    \lst@LGGetString l\lst@LGa
%    \end{macrocode}
% We test for the continuation capability and
%    \begin{macrocode}
    \lst@LGGetValue{tc}%
    \lst@if
        \edef\lstlang@{\lstlang@,lgrindef=\lst@LGvalue}%
    \fi
%    \end{macrocode}
% define the language.
%    \begin{macrocode}
    \expandafter\xdef\csname\@lst LGlang@\lst@language@\endcsname
        {\noexpand\lstset{\lstlang@}}%
%    \end{macrocode}
% Finally we inform the user of all ignored capabilities.
%    \begin{macrocode}
    \lst@ReplaceInArg\lst@arg{{: :}:}\let\lst@LGvalue\@empty
    \expandafter\lst@LGDroppedCaps\lst@arg\relax\relax
    \ifx\lst@LGvalue\@empty\else
        \PackageWarningNoLine{Listings}{Ignored capabilities for
            \space `\lst@language@' are\MessageBreak\lst@LGvalue}%
    \fi}
%    \end{macrocode}
% \end{macro}
%
% \begin{macro}{\lst@LGDroppedCaps}
% just drops a previous value and appends the next capabilty name to
% |\lst@LGvalue|.
%    \begin{macrocode}
\gdef\lst@LGDroppedCaps#1:#2#3{%
    \ifx#2\relax
        \lst@RemoveCommas\lst@LGvalue
    \else
        \edef\lst@LGvalue{\lst@LGvalue,#2#3}%
        \expandafter\lst@LGDroppedCaps
    \fi}
%    \end{macrocode}
% \end{macro}
%
% \begin{macro}{\lst@LGReplace}
% \begin{macro}{\lst@LGe}
% We replace `escaped \verb!:^$|!' by catcode 11 versions, and other strings
% by some kind of short versions (which is necessary to get the above
% definitions work).
%    \begin{macrocode}
\begingroup
\catcode`\/=0
\lccode`\z=`\:\lccode`\y=`\^\lccode`\x=`\$\lccode`\v=`\|
\catcode`\\=12\relax
/lowercase{%
/gdef/lst@LGReplace{/lst@ReplaceInArg/lst@arg
    {{\:}{z }{\^}{y}{\$}{x}{\|}{v}{ \ }{ }{:\ :}{:}{\ }{ }{\(}({\)})}}
/gdef/lst@LGe{\e}
}
/endgroup
%    \end{macrocode}
% \end{macro}\end{macro}
%
% \begin{macro}{\lst@LGRead}
% reads one language definition and defines the language if the correct one
% is found.
%    \begin{macrocode}
\gdef\lst@LGRead#1\par{%
    \lst@LGGetNames#1:\relax
    \def\lst@temp{endoflanguagedefinitions}%
    \ifx\lstlang@\lst@temp
        \let\lst@next\endinput
    \else
        \expandafter\lst@IfOneOf\lst@language@\relax\lstlang@
            {\lst@LGDefLang \let\lst@next\endinput}%
            {\let\lst@next\lst@LGRead}%
    \fi
    \lst@next}
%    \end{macrocode}
% \end{macro}
%
% \begin{lstkey}{lgrindef}
% We only have to request the language and
%    \begin{macrocode}
\lst@Key{lgrindef}\relax{%
    \lst@NormedDef\lst@language@{#1}%
    \begingroup
    \@ifundefined{lstLGlang@\lst@language@}%
        {\everypar{\lst@LGRead}%
         \catcode`\\=12\catcode`\{=12\catcode`\}=12\catcode`\%=12%
         \catcode`\#=14\catcode`\$=12\catcode`\^=12\catcode`\_=12\relax
         \input{\lstlgrindeffile}%
        }{}%
    \endgroup
%    \end{macrocode}
% select it or issue an error message.
%    \begin{macrocode}
    \@ifundefined{lstLGlang@\lst@language@}%
        {\PackageError{Listings}%
         {LGrind language \lst@language@\space undefined}%
         {The language is not loadable. \@ehc}}%
        {\lsthk@SetLanguage
         \csname\@lst LGlang@\lst@language@\endcsname}}
%    \end{macrocode}
% \end{lstkey}
%
% \begin{macro}{\lstlgrindeffile}
% contains just the file name.
%    \begin{macrocode}
\@ifundefined{lstlgrindeffile}
    {\lst@UserCommand\lstlgrindeffile{lgrindef.}}{}
%    \end{macrocode}
% \end{macro}
%
%    \begin{macrocode}
\lst@EndAspect
%</misc>
%    \end{macrocode}
% \end{aspect}
%
%
% \subsection{\textsf{hyperref}}
%
% \begin{aspect}{hyper}
%    \begin{macrocode}
%<*misc>
\lst@BeginAspect[keywords]{hyper}
%    \end{macrocode}
%
% \begin{lstkey}{hyperanchor}
% \begin{lstkey}{hyperlink}
% determine the macro to set an anchor and a link, respectively.
%    \begin{macrocode}
\lst@Key{hyperanchor}\hyper@@anchor{\let\lst@hyperanchor#1}
\lst@Key{hyperlink}\hyperlink{\let\lst@hyperlink#1}
%    \end{macrocode}
% \end{lstkey}\end{lstkey}
% Again, the main thing is a special working procedure. First we extract the
% contents of |\lst@token| and get a free macro name for this current character
% string (using prefix |lstHR@| and a number as suffix). Then we make this
% free macro equivalent to |\@empty|, so it is not used the next time.
%    \begin{macrocode}
\lst@InstallKeywords{h}{hyperref}{}\relax{}
    {\begingroup
         \let\lst@UM\@empty \xdef\@gtempa{\the\lst@token}%
     \endgroup
     \lst@GetFreeMacro{lstHR@\@gtempa}%
     \global\expandafter\let\lst@freemacro\@empty
%    \end{macrocode}
% |\@tempcnta| is the suffix of the free macro. We use it here to refer to
% the last occurence of the same string. To do this, we redefine the output
% macro |\lst@alloverstyle| to set an anchor \ldots
%    \begin{macrocode}
     \@tempcntb\@tempcnta \advance\@tempcntb\m@ne
     \edef\lst@alloverstyle##1{%
         \let\noexpand\lst@alloverstyle\noexpand\@empty
         \noexpand\smash{\raise\baselineskip\hbox
             {\noexpand\lst@hyperanchor{lst.\@gtempa\the\@tempcnta}%
                                       {\relax}}}%
%    \end{macrocode}
% \ldots\space and a link to the last occurence (if there is any).
%    \begin{macrocode}
         \ifnum\@tempcnta=\z@ ##1\else
             \noexpand\lst@hyperlink{lst.\@gtempa\the\@tempcntb}{##1}%
         \fi}%
    }
    od
%    \end{macrocode}
%
%    \begin{macrocode}
\lst@EndAspect
%</misc>
%    \end{macrocode}
% \end{aspect}
%
%
% \section{Epilogue}
%
% \begingroup
%    \begin{macrocode}
%<*kernel>
%    \end{macrocode}
% Each option adds the aspect name to |\lst@loadaspects| or removes it from that data macro.
%    \begin{macrocode}
\DeclareOption*{\expandafter\lst@ProcessOption\CurrentOption\relax}
\def\lst@ProcessOption#1#2\relax{%
    \ifx #1!%
        \lst@DeleteKeysIn\lst@loadaspects{#2}%
    \else
        \lst@lAddTo\lst@loadaspects{,#1#2}%
    \fi}
%    \end{macrocode}
% The following aspects are loaded by default.
%    \begin{macrocode}
\@ifundefined{lst@loadaspects}
  {\def\lst@loadaspects{strings,comments,escape,style,language,%
      keywords,labels,lineshape,frames,emph,index}%
  }{}
%    \end{macrocode}
% We load the patch file, \ldots
%    \begin{macrocode}
\InputIfFileExists{lstpatch.sty}{}{}
%    \end{macrocode}
% \ldots\ process the options, \ldots
%    \begin{macrocode}
\let\lst@ifsavemem\iffalse
\DeclareOption{savemem}{\let\lst@ifsavemem\iftrue}
\DeclareOption{noaspects}{\let\lst@loadaspects\@empty}
\ProcessOptions
%    \end{macrocode}
% \ldots\ and load the aspects.
%    \begin{macrocode}
\lst@RequireAspects\lst@loadaspects
\let\lst@loadaspects\@empty
%    \end{macrocode}
% If present we select the empty style and language.
%    \begin{macrocode}
\lst@UseHook{SetStyle}\lst@UseHook{EmptyStyle}
\lst@UseHook{SetLanguage}\lst@UseHook{EmptyLanguage}
%    \end{macrocode}
% Finally we load the configuration files.
%    \begin{macrocode}
\InputIfFileExists{listings.cfg}{}{}
\InputIfFileExists{lstlocal.cfg}{}{}
%<info>\lst@ReportAllocs
%    \end{macrocode}
%    \begin{macrocode}
%</kernel>
%    \end{macrocode}
% \endgroup
%
%
% \section{History}
% \begingroup\small
% Only major changes are listed here. Introductory version numbers of commands
% and keys are in the sources of the guides, which makes this history fairly
% short.
% \renewcommand\labelitemi{--}
% \begin{itemize}
% \item[0.1] from 1996/03/09
%   \item test version to look whether package is possible or not
% \item[0.11] from 1996/08/19
%\iffalse
%   \item additional blank option (= language)
%\fi
%   \item improved alignment
% \item[0.12] from 1997/01/16
%   \item nearly `perfect' alignment
% \item[0.13] from 1997/02/11
%\iffalse
%   \item additional languages: Eiffel, Fortran 90, Modula-2, Pascal XSC
%\fi
%   \item load on demand: language specific macros moved to driver files
%   \item comments are declared now and not implemented for each language again
%         (this makes the \TeX\ sources easier to read)
% \item[0.14] from 1997/02/18
%   \item User's guide rewritten, Implementation guide uses macro environment
%   \item (non) case sensitivity implemented and multiple string types,
%         i.e.~Modula-2 handles both string types: quotes and double quotes
%\iffalse
%   \item comment declaration is user-accessible
%   \item package compatible to \verb!german.sty!
%\fi
% \item[0.15] from 1997/04/18
%\iffalse
%   \item additional languages: Java, Turbo Pascal
%\fi
%   \item package renamed from \packagename{listing} to \packagename{listings}
%         since the first already exists
% \item[0.16] from 1997/06/01
%\iffalse
%   \item changed `$<$' to `$>$' in |\lst@SkipToFirst|
%   \item bug removed: |\lst@Init| must be placed before |\lst@SkipToFirst|
%\fi
%   \item listing environment rewritten
% \item[0.17] from 1997/09/29
%\iffalse
%   \item |\spreadlisting| works correct now (e.g.~page numbers don't move right)
%\fi
%   \item speed up things (quick `if parameter empty', all |\long| except one
%         removed, faster \verb!\lst@GotoNextTabStop!, etc.)
%   \item improved alignment of wide other characters (e.g.~$==$)
%\iffalse
%   \item many new languages: Ada, Algol, Cobol, Comal 80, Elan, Fortran 77,
%         Lisp, Logo, Matlab, Oberon, Perl, PL/I, Simula, SQL, \TeX
%\fi
% \item[pre-0.18] from 1998/03/24 (unpublished)
%\iffalse
%   \item bug concerning |\labelstyle| (becomes \keyname{numberstyle}) removed
%         (now oldstylenum example works)
%\fi
%   \item experimental implementation of character classes
% \item[0.19] from 1998/11/09
%   \item character classes and new \lst-aspects seem to be good
%   \item user interface uses \packagename{keyval} package
%   \item \packagename{fancyvrb} support
% \item[0.20] from 1999/07/12
%   \item new keyword detection mechanism
%   \item new aspects: \aspectname{writefile}, \aspectname{breaklines},
%         captions, \aspectname{html}
%\iffalse
%   \item improved \packagename{fancyvrb} support
%\fi
%   \item all aspects reside in a single file and the language drivers in
%         currently two files
% \item[0.21] 2000/08/23
%   \item completely new User's guide
%   \item experimental format definitions
%   \item keyword classes replaced by families
%   \item dynamic modes
% \item[1.0$\beta$] 2001/09/21
%   \item key names synchronized with \packagename{fancyvrb}
%   \item \aspectname{frames} aspect extended
%   \item new output concept (delaying and merging)
% \item[1.0] 2002/04/01
%   \item update of all documentation sections including Developer's guide
%   \item delimiters unified
% \item[1.1] 2003/06/21
%   \item bugfix-release with some new keys
% \item[1.2] 2004/02/13
%   \item bugfix-release with two new keys and new section \ref{rArbitraryLinerangeMarkers}
% \item[1.3] 2004/09/07
%   \item another bugfix-release with LPPL-1.3-compliance
% \item[1.4] 2007/02/26
%   \item many bugfixes, and new maintainership
%   \item several new and updated language definitions
%   \item many small documentation improvements
%   \item new keys, multicharacter string delimiters, short inline listings, and more.
% \item[1.5] 2013/06/27
%   \item new maintainership
% \end{itemize}
% \endgroup
%
%
% \Finale
%
\endinput
