\documentclass[10pt,letterpaper,onecolumn,draftclsnofoot]{IEEEtran}
\usepackage[margin=0.75in]{geometry}
\usepackage{listings}
\usepackage{color}
\usepackage{longtable}
\usepackage{tabu}
\usepackage{hyperref}
%Fixes descritpion label and body font overlap issues
\usepackage{enumitem}
\definecolor{dkgreen}{rgb}{0,0.6,0}
\definecolor{gray}{rgb}{0.5,0.5,0.5}
\definecolor{mauve}{rgb}{0.58,0,0.82}

\lstset{frame=tb,
  language=C,
  columns=flexible,
  numberstyle=\tiny\color{gray},
  keywordstyle=\color{blue},
  commentstyle=\color{dkgreen},
  stringstyle=\color{mauve},
  breaklines=true,
  breakatwhitespace=true,
  tabsize=4
}



\begin{document}
\begin{titlepage}
  \title{CS 461 - Fall 2016 - Client Requirements Document}
  \author{Matthew Johnson, Garrett Smith, Cody Malick\\Cloud Orchestra}
  \date{October 28, 2016}
  \maketitle
  \vspace{4cm}
  \begin{abstract}
  	\noindent This document outlines the requirements for the Cloud Orchestration
  	project sponsored by Intel Corporation. It formally defines the purpose, scope,
  	description, function, use, constraints, and specific requirements of the project.
  	Although there are no specific design decisions made, it will be used as a building
  	block for the rest of the design, implementation, and testing process.

  \end{abstract}

\end{titlepage}
\tableofcontents
\clearpage
\section{Introduction}
\subsection{Purpose}
Intel, while developing its Software Defined Network implementation into Ciao, has found
a need for a more advanced form of network bride than the standard Linux bridge. The initial
implementation of the Linux bridge GRE has worked well, but as further development was done on
Ciao, the need for modern packet encapsulation and other innovative protocols were found to
be needed. Implementing an Open vSwitch GRE would allow increased performance, access
to contemporary Software Defined Network innovations, and modern packet encapsulation methods.
This addition would be used by those implementing Ciao in their own organizations or businesses
to further increase speed and availability of features in their cloud. 
\subsection{Scope}
The scope of the Cloud Orchestration project encapsulates the following two main goals, followed
by one additional stretch goal:
\subsubsection{Open vSwitch GRE Tunnel}
The first goal of the project is to switch the GRE tunnel implementation with the Open vSwitch
created GRE tunnel. This will allow for newer packet encapsulation techniques to be used, as
well as provide the option to test packet acceleration.
\subsubsection{Test and Implement Best Performing Tunnel Implementation}
Switch the tunneling implementation to VxLAN/nvGRE based on performance measurements of VxLAN and nvGRE
on data center network cards.
\subsubsection{Stretch Goal, Replace Linux Bridge}
The final objective of the project, and stretch goal, is to replace the Linux bridges with Open
vSwitch instances

\subsection{Definitions, acronyms, and abbreviations}
\begin{description}[leftmargin=12em,style=nextline]
    \item[Bridge]
	    Software or hardware that connects two or more network segments
    \item [Cloud] 
	    A collection of networked servers
	\item[Cloud Orchestration]
		Workload scheduling and deployment in a cloud environment
	\item[GRE] 
		Generic Routing Encapsulation
    \item[Linux Bridge]
	    The built in Linux software bridge
    \item[nvGRE] 
	    Network Virtualization using Generic Routing Encapsulation
    \item [Open vSwitch] 
	    An open source multilayer software switch with support for distribution across multiple physical devices. 
	    The project can be found at \url{https://github.com/openvswitch/ovs}.
	\item[OVS] 
		Open vSwitch 
    \item[Packet Acceleration] 
	    Increasing the speed of the processing and transfer of network packets
    \item[Packet Encapsulation]
	    The wrapping on a network packet in the headers for a protocol
    \item [Tunnel]
	    Point to point network connection that encapsulates traffic between the points
    \item[VxLAN] 
	    Virtual Extensible Local Area Network
\end{description}

\subsection{References}
This subsection should
a) Provide a complete list of all documents referenced elsewhere in the SRS;
b) Identify each document by title, report number (if applicable), date, and publishing organization;
c) Specify the sources from which the references can be obtained.
This information may be provided by reference to an appendix or to another document.
\subsection{Overview}
This subsection should
a) Describe what the rest of the SRS contains;
b) Explain how the SRS is organized.
\clearpage
\section{Overall Description}
\subsection{Product perspective}
\subsection{Product functions}
\subsection{User characteristics}
\subsection{Constraints}
\subsection{Assumptions and dependencies}
\section{Specific Requirements}
\clearpage
\section{Appendixes}
\section{Index}

\end{document}
