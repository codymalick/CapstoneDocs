\documentclass[10pt,letterpaper,onecolumn,draftclsnofoot]{IEEEtran}
\usepackage[margin=0.75in]{geometry}
\usepackage{listings}
\usepackage{color}
\usepackage{float}
\usepackage{longtable}
\usepackage{tabu}
\usepackage{hyperref}
\usepackage{graphicx}
\usepackage{pgfgantt}
%Fixes descritpion label and body font overlap issues
\usepackage{enumitem}
\definecolor{dkgreen}{rgb}{0,0.6,0}
\definecolor{gray}{rgb}{0.5,0.5,0.5}
\definecolor{mauve}{rgb}{0.58,0,0.82}
\graphicspath{{../images/}}

\renewcommand\thesection{\arabic{section}}
\renewcommand\thesubsection{\thesection.\arabic{subsection}}
\renewcommand\thesubsubsection{\thesubsection.\arabic{subsubsection}}

\renewcommand\thesectiondis{\arabic{section}}
\renewcommand\thesubsectiondis{\thesectiondis.\arabic{subsection}}
\renewcommand\thesubsubsectiondis{\thesubsectiondis.\arabic{subsubsection}}

\setlength{\parindent}{0cm}

\lstset{frame=tb,
	language=C,
	columns=flexible,
	numberstyle=\tiny\color{gray},
	keywordstyle=\color{blue},
	commentstyle=\color{dkgreen},
	stringstyle=\color{mauve},
	breaklines=true,
	breakatwhitespace=true,
	tabsize=4
}
\usepackage{etoolbox}
\patchcmd{\thebibliography}{\section*}{\subsection}{}{}
\patchcmd{\thebibliography}{\addcontentsline{toc}{section}{\refname}}{}{}{}

\begin{document}
\begin{titlepage}
	\title{CS 461 - Fall 2016 - Client Requirements Document}
	\author{Matthew Johnson, Garrett Smith, Cody Malick\\Cloud Orchestra}
	\date{October 28, 2016}
	\maketitle
	\vspace{4cm}
	\begin{abstract}
		\noindent This document outlines the requirements for the Cloud Orchestration
		Networking project sponsored by Intel Corporation. It formally defines the
		purpose, scope, description, function, use, constraints, and specific
		requirements of the project. Although there are no specific design decisions
		made, it will be used as a building block for the rest of the design,
		implementation, and testing process.

	\end{abstract}

\end{titlepage}
\tableofcontents
\clearpage

\section{Introduction}

\subsection{Purpose}

Ciao is Intel's Cloud Integrated Advanced Orchestrator (Ciao). Ciao schedules and balances workload
throughout a large set of servers. The job of distributing workload over a large number of servers is a difficult
task. Ciao does this well, but has room for improvement in the area of network security and speed. Our
project is to add additional functionality to Ciao's network systems, and enable technology that will
increase the speed that these systems operate at.


While developing its Software Defined Network implementation into Ciao, Intel has found a need for a
more advanced form of network bridge than the standard Linux bridge.
The initial implementation using Linux bridges and GRE tunnels has worked well, but as further
development was done on Ciao, the need for modern packet encapsulation and other innovative protocols
was found to be needed. Implementing a network mode in Ciao utilizing Open vSwitch-created GRE
tunnels would allow the network to utilize advanced networking techniques to increase performance,
such as modern packet encapsulation methods. This addition would be used by those implementing Ciao in
their own organizations or businesses to further increase speed and availability of features in
their cloud.

\subsection{Scope}

The scope of the Cloud Orchestration project encapsulates the following two main goals, followed
by one additional stretch goal:

\subsubsection{Open vSwitch GRE Tunnel}

The first goal of the project is to switch the current GRE tunnel implementation with the Open vSwitch
created GRE tunnel. This will allow for newer packet encapsulation techniques to be used, as
well as provide the option to test packet acceleration.

\subsubsection{Test and Implement Best Performing Tunnel Implementation}

Switch the tunneling implementation to VxLAN/nvGRE based on performance measurements of VxLAN and nvGRE
on data center network cards.

\subsubsection{Stretch Goal, Replace Linux Bridge}

The final objective and stretch goal of the project is to replace the Linux bridges with Open
vSwitch instances.


\subsection{Definitions, acronyms, and abbreviations}

\begin{description}[leftmargin=12em,style=nextline]
	\item[Bridge]
		Software or hardware that connects two or more network segments.
	\item [Cloud]
		A huge, amorphous network of servers somewhere~\cite{xkcd908}.
	\item[Cloud Orchestration]
		An easy to deploy, secure, scalable cloud orchestration system
		which handles virtual machines, containers, and bare metal apps
		agnostically as generic workloads~\cite{ciaoReadme}.
	\item[CNCI]
		Virtual Machines automatically configured by the ciao-controller,
		scheduled by the ciao-scheduler on a need basis, when tenant
		workloads are created~\cite{ciaoNetworking}.
	\item[Generic Routing Encapsulation (GRE)]
		Encapsulation of an arbitrary network layer protocol so it can
		be sent over another arbitrary network layer protocol~\cite{rfc1701}.
	\item[Linux Bridge]
		Configurable software bridge built into the Linux
		kernel\cite{linuxBridge}. 
	\item[Network Node (NN)]
		A Network Node is used to aggregate network traffic for all
		tenants while still keeping individual tenant traffic isolated
		from all other the tenants using special virtual machines called
		Compute Node Concentrators (or CNCIs)~\cite{ciaoNetworking}.
	\item[nvGRE]
		Network Virtualization using Generic Routing Encapsulation~\cite{rfc7637}.
	\item[Open vSwitch]
		Open source multilayer software switch with support for
		distribution across multiple physical devices~\cite{ovs}.
	\item[OVS]
		Open vSwitch~\cite{ovs}.
	\item[Packet Acceleration]
		Increasing the speed of the processing and transfer of network packets.
	\item[Packet Encapsulation]
		Attaching the headers for a network protocol to a packet so it 
		can be transmitted using that protocol~\cite{networkingTextbook}.
	\item[SSNTP]
		The Simple and Secure Node Transfer Protocol (SSNTP) is a
		custom, fully asynchronous and TLS based application layer
		protocol. All Cloud Integrated Advanced Orchestrator (CIAO)
		components communicate with each others over SSNTP~\cite{ciaoSSNTP}.
	\item[Tunnel]
		Point to point network connection that encapsulates traffic
		between points~\cite{networkingTextbook}.
	\item[VxLAN]
		Virtual Extensible Local Area Network~\cite{rfc7348}.
\end{description}

\bibliographystyle{IEEEtran}
\bibliography{client_req}

\subsection{Overview}
The following section describes more details about the product, including 
product perspective, specific requirements, functionality requirements, and any
assumptions or dependencies used. The section is organized in the following
fashion:
\begin{enumerate}
	\item Overall Description
	\item Product Perspective
	\item Product Functions
	\item User Characteristics
	\item Constraints
	\item Reliability
	\item Security
	\item Specific Requirements
\end{enumerate}

\section{Overall Description}

\subsection{Product Perspective}

The Software Defined Network (SDN) we implement will be utilized by Ciao to transfer packets between
compute nodes and control nodes on a cloud cluster. For this purpose the mode
must be fully integrated into the Ciao infrastructure and must behave similarly
to what is already in place.

Because this software will be a component of a larger system, it must follow the
design of that larger system and be fully integrated. As Ciao is
implemented in the Go programming language, so this SDN implementation must
be written in Go. The networking mode must route packets between ports on
different nodes using networking protocols.

The SDN that is implemented must support the following structure of the cloud in
Ciao:

\begin{figure}[H]
	\caption{Ciao Network Topology~\cite{ciaoNetTopology}}
	\begin{center}
		\makebox[\textwidth]{\includegraphics[width=\textwidth]{ciao-networking.eps}}
	\end{center}
\end{figure}


Ciao has several functional components that must be worked with:

\begin{description}[leftmargin=12em,style=nextline]
	\item[libsnnet]
		Provides networking APIs to the ciao-launcher to create tenant
		specific network interfaces on compute nodes and CNCI specific
		network interfaces on a network node.
	\item[ciao-cnci-agent]
		A SSNTP client which connects to the ciao-scheduler and runs
		within a CNCI VM and configures tenant network connectivity by
		interacting with the ciao-controller and ciao-launchers using
		the ciao-scheduler. The ciao-cnci-agent can also be run on
		physical nodes if desired.
	\item[docker-plugin]
		Provides unified networking between VM and Docker workloads.
\end{description}

\subsection{Product Functions}

Ciao networking must support the following functionality.

\begin{enumerate}
	\item Secure isolated overlay network - The networking implementation
		must provide each tenant with a secure isolated overlay network
		without the need for any configuration by the tenant and minimal
		configuration by the data center operator.
	\item Auto discovery of compute/network nodes - Auto discover roles of
		nodes when they are attached to the network.
	\item Diverse and scalable - Support large number of tenants with large
		or small number of workloads.
	\item Work on different platforms - Operate on any Linux distribution
		by limiting the number of dependencies on user-space tools and
		leveraging Linux kernel interfaces whenever possible.
	\item Migrate workloads - Provide the ability to migrate workloads from
		a Compute Node on demand or when a CN crashes without tenant
		intervention. Provide the ability to migrate CNCIs on demand or
		when a Network Node crashes.
	\item Encrypt traffic - Provide the ability to transparently encrypt all
		tenant traffic even within the data center network.
	\item Create GRE tunnels with Open VSwitch
	\item Support for tenant and workload level security rules
	\item Support for tenant and workload level NAT rules (inbound and outbound)
\end{enumerate}

\subsection{User characteristics}

The users of this mode in Ciao will be educated and technically-minded data
center administrators. They will be familiar with cloud technologies and
networks, both software-defined and hardware-defined.

Due to Ciao's goal of minimum configuration, users are not required to
have the knowledge base of a normal Openstack network administrator.

\subsection{Constraints}

\subsubsection{High-order language requirements}

As mentioned above, it is necessary to write this mode in the Go programming
language in order to integrate with the rest of Ciao.

\subsubsection{Reliability}

The networking mode must be completely reliable and include the ability to
migrate workloads when a compute or network node crashes with little-to-no
interruption to the network capabilities for the tenants and must do so without
tenant intervention.

\subsubsection{Security}

Traffic must be fully and transparently encrypted even within the data center
network. It must utilize Simple and Secure Node Transfer Protocol (SSNTP) to
ensure the security of traffic between the nodes.

\section{Specific Requirements}

The main requirement is that Open VSwitch is used to create the GRE tunnels in
the SDN implementation. A further goal is to switch the tunneling implementation
to VxLAN or nvGRE based on performance measurements of each. The result of those
performance metrics will dictate which is used.

A stretch goal is to replace the currently-implemented Linux bridges with
Open VSwitch switch instances.

The specific requirements for this project are as follows:

\begin{enumerate}
	\item Documentation
		\begin{enumerate}
			\item Problem statement
			\item Requirements document
			\item Design document
		\end{enumerate}
	\item Implementation
		\begin{enumerate}
			\item Open vSwitch GRE tunnels - Open VSwitch must be
				used to create the GRE tunnels in the SDN
				implementation. Currently, Ciao uses
				Linux-created GRE tunnels.
			\item VxLAN/nvGRE implementation - The second part is
				to switch the tunneling implementation to a
				VxLAN or nvGRE implementation based on speed
				performance. The faster implementation will be
				kept.
			\item Optional: replace Linux bridges with OVS
				switches - if enough time remains, replace the
				Linux bridges with switches created by Open
				VSwitch.
		\end{enumerate}
	\item Presentation
		\begin{enumerate}
			\item Build presentation board
			\item Present at Engineering Expo
		\end{enumerate}
\end{enumerate}

\section{Gantt Chart}
Following is the distribution of work scheduled for the duration of the project:\\\\
% Gantt chart location
\begin{ganttchart}[
		hgrid=true,
		vgrid={*{10}{blue, dashed}},
		y unit chart=0.75cm,
		x unit=1.5cm
	]{1}{9}
\gantttitle{2016-2017}{9} \\
\gantttitlelist{9,10,11,12,1,2,3,4,5}{1} \\
\ganttgroup{Project Documentation}{1}{4} \\
\ganttbar[progress=100]{Problem Statement}{1}{2} \\
\ganttbar[progress=50]{Requirements Document}{2}{3} \\
\ganttbar[progress=0]{Design Document}{3}{4}  \ganttnewline
\ganttgroup{Implementation}{4}{7} \ganttnewline
\ganttbar[progress=0,progress label text={}]{Open vSwitch GRE Tunnels}{4}{6} \\
\ganttbar[progress=0,progress label text={}]{VxLan/nvGRE Implementation}{6}{7} \\
\ganttbar[progress=0,progress label text={}]{Linux Bridges to OVS Switches}{7}{7} \\
\ganttgroup{Presentation}{7}{9} \\
\ganttbar[progress=0,progress label text={}]{Build Presentation Board}{7}{9}\\
\ganttbar[progress=0,progress label text={}]{Engineering Expo}{9}{9}
\end{ganttchart}


\clearpage

\end{document}
