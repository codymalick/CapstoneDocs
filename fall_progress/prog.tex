\documentclass[10pt,onecolumn,journal,draftclsnofoot]{IEEEtran}
\usepackage[margin=0.75in]{geometry}
\usepackage{listings}
\usepackage{color}
\usepackage{longtable}
\usepackage{graphicx}
\usepackage{float}
\usepackage{tabu}
\usepackage{enumitem}
\usepackage{courier}
\usepackage{hyperref}
\usepackage{parskip}
\definecolor{dkgreen}{rgb}{0,0.6,0}
\definecolor{gray}{rgb}{0.5,0.5,0.5}
\definecolor{mauve}{rgb}{0.58,0,0.82}

%Subsection headers to Arabic numerals
%\renewcommand\thesection{\arabic{section}}
%\renewcommand\thesubsection{\thesection.\arabic{subsection}}
%\renewcommand\thesubsubsection{\thesubsection.\arabic{subsubsection}}

%Section headers to Arabic numerals
%\renewcommand\thesectiondis{\arabic{section}}
%\renewcommand\thesubsectiondis{\thesectiondis.\arabic{subsection}}
%\renewcommand\thesubsubsectiondis{\thesubsectiondis.\arabic{subsubsection}}

%Remove numbering from the bibliography section

\lstset{frame=none,
language=C,
columns=flexible,
numberstyle=\tiny\color{gray},
keywordstyle=\color{blue},
commentstyle=\color{dkgreen},
stringstyle=\color{mauve},
breaklines=true,
breakatwhitespace=true,
tabsize=4,
showstringspaces=false,
basicstyle=\ttfamily
}

\setlength{\parindent}{0cm}

\begin{document}

\begin{titlepage}
	\title{Intel Cloud Orchestration Networking\\ Fall Progress Report}
	%\author{Matthew~Johnson,~Cody~Malick,~and~Garrett~Smith\\
	%	Team 51, Cloud Orchestra}
	\date{\today}
	\markboth{Senior Design, CS 461, Fall 2016}{}
	\maketitle
	\vspace{4cm}
	\begin{abstract}
		\noindent This document outlines the progress of the Cloud
		Orchestration Networking project over the entirety of the fall
		term. It contains a short description of the project's purposes
		and goals, current progress, current issues, and any solutions
		to those issues. It also contains a week by week retrospective
		for all ten weeks of fall term. \end{abstract}

\end{titlepage}
\tableofcontents
\clearpage

\section{Project Goals}

Our project is to first switch the Linux-created GRE tunnel implementation in
Ciao to use GRE tunnels created by Open vSwitch. From that point we will switch
the actual tunneling implementation from GRE to VxLAN/nvGRE based on performance
measurements of each on data center networking cards. After this is completed, a
stretch goal is to replace Linux bridges with Open vSwitch switch instances.

\section{Purpose}

The current implementation of Ciao tightly integrates software defined
networking principles to leverage a limited local awareness of just enough of
the global cloud's state. Tenant overlay networks are used to overcome
traditional hardware networking challenges by using a distributed, stateless,
self-configuring network topology running over dedicated network software
appliances. This design is achieved using Linux-native Global Routing
Encapsulation (GRE) tunnels and Linux bridges, and scales well in an environment
of a few hundred nodes.

While this initial network implementation in Ciao satisfies current simple
networking needs in Ciao, all innovation around software defined networks has
shifted to the Open vSwitch (OVS) framework. Moving Ciao to OVS will allow
leverage of packet acceleration frameworks like the Data Plane Development Kit
(DPDK) as well as provide support for multiple tunneling protocols such as VxLAN
and nvGRE. VxLAN and nvGRE are equal cost multipath routing (ECMP) friendly,
which could increase network performance overall.

\section{Current Progress}
At present, the project is moving along smoothly. Our testing environment has
been set up and is networked appropriately. Each Intel NUC (Next Unit of
Computing) has Clear Linux installed. Come Winter term, we will get Ciao set up
on each machine and begin development on Ciao. Software development on the
project has yet to begin as we have just wrapped up the design phase.

Designing has been quite helpful in developing our understanding of the project,
its goals, and purpose. Because this is a small component of a very complicated
system, taking the time to investigate what a software defined network is, why
it's being used, and why we are implementing the piece that we are has been
quite beneficial.

While in the design phase, we found an extremely useful library for interfacing
with Open vSwitch, libovsdb. Libovsdb is a library written in the Go programming
language that allows for simple and efficient calls to the OVS Database
Management Protocol. Interfacing with OVS is going to be a very large
portion of the project for us, so finding the library is quite the boon. Here
is a quick example of how this library functions:\\

\begin{lstlisting}[caption=Example insert operation using libovsdb]
	// simple insert operation
	insertOp := libovsdb.Operation{
		Op:	  "insert",
		Table:	  "Bridge",
		Row:	  bridge,
		UUIDName: namedUUID,
	}
\end{lstlisting}

The above example can be reused for all major operations in the OVS Database
Management Protocol. Other example operations include select, delete, and
update. Using just the operations listed here, we can accomplish most of the
needed configuration changes within Open vSwitch.

\section{Week by Week Reports}
%just paste in weekly blogs?

\subsection{Background}

Over the Summer, 2016, Matthew worked as a Software Engineering Intern for the
Advanced Systems Engineering (ASE) group within the Open Source Technology
Center (OTC) that is in turn within the Software Services Group (SSG) at Intel.
Matthew's coworkers had a need to integrate Open vSwitch in their cloud
orchestration software (Cloud Integrated Advanced Orchestrator, or Ciao) but did
not have the man hours to contribute time to it. Matthew worked with the team to
propose the project as a Senior Capstone project at Oregon State University.

Matthew identified two other students, Cody Malick and Garrett Smith, as
intelligent hard workers who would benefit the project. Because of this Robert
Nesius, the Intel Engineering Manager serving as our client, requested Matthew,
Cody, and Garrett specifically for this project.

\subsection{Weeks Zero Through Two}

During the first week of class, we all visited Intel in Hillsboro. The principal
engineer in charge of Ciao networking, Manohar Castelino, gave us all an
explanation of Ciao, how the networking works, and what he expects us to
accomplish.

It was also during this time that Rob provided us with five Intel NUCs that
would serve as our local cluster. We found out that we needed to register the
MAC addresses with the university, and communicated this need to Todd Shechter,
the Oregon State University Director of Information Technology.

\subsection{Week Three}

During week three we attempted to install Clear Linux OS for Intel
Architecture~\cite{clearlinux} on all five Intel NUCs. We were unsuccessful
because the installer requires a network connection to download the packaging.
At this point, network access had yet to be approved by OSU IT.

Network access is required to install Clear Linux, Ciao, and access the machines
remotely. Since Clear Linux is a datacenter OS, not a desktop OS, it does not
support wireless internet connections. Because ethernet is required our hardware
must be registered with the university. If we were unable to obtain network
access for the hardware on OSU's network we would have needed to find somewhere
else to house it.

We also wrote our problem statement in week three, earlier than most groups were
able to, since we were ahead of schedule with regard to meeting our team and
choosing a project.

\subsection{Week Four}

During week four Matthew installed Clear Linux on the Intel NUCs. Since there
the networking issues had still not been solved he brought them to his house to
use the wired connection there. At this point our hardware had been registered
with the university, but for unknown reasons our NUCs were not connecting to the
network. Todd Shechter was devoting a lot of his time to help us debug, but we
were not yet successful. He set us up with two five-port switches in Kevin
McGrath's lab, but they were not receiving IP addresses from the network.

We had our problem statement reviewed and were waiting for feedback by the end
of the week. We also started working on the client requirements document, though
much of our contributions were simple outline and templating work.

\subsection{Week Five}

This week we spent most of our time writing the rough draft for the requirements
document. We turned it in by the end of the week and were satisfied with our
progress. The final draft of the requirements document was due the next week,
week six. This week we also turned in our signed copy of the problem statement.

This week our networking issues were resolved. We had another email conversation
with Todd Shechter, who was at a loss as to why we could not access the network
from the NUCs. He granted us access to an HP switch we had successfully
connected via in the past. On Thursday, we went to move all our NUCs to the new
switch, but tried out the network on the mini switches one last time. This time
they all worked. Our hardware was now set up and ready to go.

\subsection{Week Six}

This week we finished writing the requirements document that was due at the end
of the week. After the rough draft we turned in the previous week we continued
working on it ourselves until Tuesday. On Wednesday Frank emailed us some
suggestions and we addressed those right away. We got the document signed that
afternoon by Rob Nesius, our client at Intel.

\subsection{Week Seven}

This week we started working on the technical review document due the following
Monday. This document outlines nine different components of our system. For each
component we explored three different technologies that could be used to
implement the component. Since our project is implementing a component of a
larger system, it was difficult for us to come up with nine components and three
technologies each (twenty-seven different options). We spent much of our week
working together to figure out how to split the project up.

\subsection{Week Eight}

This week we submitted the tech review document. It was a lot of work, but we
got it in on time unlike many other capstone groups. Garrett researched software
switch options, network latency tools, and network throughput tools. Matthew
dealt with high-level language, testing, and logging tools. Cody handled the
network-specific implementation pieces, such as packet protocols, network
virtualization implementation, and bridge implementation.

\subsection{Week Nine}

This was a short week with Thursday and Friday given over to the Thanksgiving
holiday. We started working on the design document but did not make much headway
before breaking for the holiday weekend.

Our team also got together and talked about how we were going to execute the
final presentation video for a few minutes this week.

\subsection{Week Ten}

This week we focused on the design document due Friday. We spent time
researching design strategies and writing up our plan to execute. During this
research we found a very useful Go library that interfaces with Open vSwitch.
This library will simplify our implementation, allowing us more time to do
network performance testing, which the client is very interested in. Our client
signed the document with a half-hour to spare before the deadline, and the
technical advisors for the project, Manohar Castelino and Tim Pepper, indicated
they were impressed with the design we had outlined.

\section{Fall Term Retrospective}

\begin{center}
	\begin{tabular}{| p{0.05\linewidth} | p{0.3\linewidth} | p{0.3\linewidth} | p{0.3\linewidth} |}\hline
		Week & Positives & Deltas & Actions \\ \hline
		3 & this is a test  & beep boop bap & Hack the world \\ \hline
	\end{tabular}
\end{center}

\end{document}
