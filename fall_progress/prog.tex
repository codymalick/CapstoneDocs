\documentclass[10pt,onecolumn,journal,draftclsnofoot]{IEEEtran}
\usepackage[margin=0.75in]{geometry}
\usepackage{listings}
\usepackage{color}
\usepackage{longtable}
\usepackage{graphicx}
\usepackage{float}
\usepackage{tabu}
\usepackage{enumitem}
\usepackage{courier}
\usepackage{hyperref}
\usepackage{parskip}
\definecolor{dkgreen}{rgb}{0,0.6,0}
\definecolor{gray}{rgb}{0.5,0.5,0.5}
\definecolor{mauve}{rgb}{0.58,0,0.82}

%Subsection headers to Arabic numerals
%\renewcommand\thesection{\arabic{section}}
%\renewcommand\thesubsection{\thesection.\arabic{subsection}}
%\renewcommand\thesubsubsection{\thesubsection.\arabic{subsubsection}}

%Section headers to Arabic numerals
%\renewcommand\thesectiondis{\arabic{section}}
%\renewcommand\thesubsectiondis{\thesectiondis.\arabic{subsection}}
%\renewcommand\thesubsubsectiondis{\thesubsectiondis.\arabic{subsubsection}}

%Remove numbering from the bibliography section

\lstset{frame=none,
language=C,
columns=flexible,
numberstyle=\tiny\color{gray},
keywordstyle=\color{blue},
commentstyle=\color{dkgreen},
stringstyle=\color{mauve},
breaklines=true,
breakatwhitespace=true,
tabsize=4,
showstringspaces=false,
basicstyle=\ttfamily
}

\setlength{\parindent}{0cm}

\begin{document}

\begin{titlepage}
	\title{Intel Cloud Orchestration Networking\\ Fall Progress Report}
	%\author{Matthew~Johnson,~Cody~Malick,~and~Garrett~Smith\\
	%	Team 51, Cloud Orchestra}
	\date{\today}
	\markboth{Senior Design, CS 461, Fall 2016}{}
	\maketitle
	\vspace{4cm}
	\begin{abstract}
		\noindent This document outlines the progress of the Cloud
		Orchestration Networking project over the entirety of the fall
		term. It contains a short description of the project's purposes
		and goals, current progress, current issues, and any solutions
		to those issues. It also contains a week by week retrospective
		for all ten weeks of fall term. \end{abstract}

\end{titlepage}
\tableofcontents
\clearpage

\section{Project Goals}

Our project is to first switch the Linux-created GRE tunnel implementation in
Ciao to use GRE tunnels created by Open vSwitch. From that point we will switch
the actual tunneling implementation from GRE to VxLAN/nvGRE based on performance
measurements of each on data center networking cards. After this is completed, a
stretch goal is to replace Linux bridges with Open vSwitch switch instances.

\section{Purpose}

The current implementation of Ciao tightly integrates software defined
networking principles to leverage a limited local awareness of just enough of
the global cloud's state. Tenant overlay networks are used to overcome
traditional hardware networking challenges by using a distributed, stateless,
self-configuring network topology running over dedicated network software
appliances. This design is achieved using Linux-native Global Routing
Encapsulation (GRE) tunnels and Linux bridges, and scales well in an environment
of a few hundred nodes.

While this initial network implementation in Ciao satisfies current simple
networking needs in Ciao, all innovation around software defined networks has
shifted to the Open vSwitch (OVS) framework. Moving Ciao to OVS will allow
leverage of packet acceleration frameworks like the Data Plane Development Kit
(DPDK) as well as provide support for multiple tunneling protocols such as VxLAN
and nvGRE. VxLAN and nvGRE are equal cost multipath routing (ECMP) friendly,
which could increase network performance overall.

\section{Current Progress}
At present, the project is moving along smoothly. Our testing environment has
been set up and is networked appropriately. Each Intel NUC (Next Unit of
Computing) has Clear Linux installed. Come Winter term, we will get Ciao set up
on each machine and begin development on Ciao. Software development on the
project has yet to begin as we have just wrapped up the design phase.

Designing has been quite helpful in developing our understanding of the project,
its goals, and purpose. Because this is a small component of a very complicated
system, taking the time to investigate what a software defined network is, why
it's being used, and why we are implementing the piece that we are has been
quite beneficial.

While in the design phase, we found an extremely useful library for interfacing
with Open vSwitch, libovsdb. Libovsdb is a library written in the Go programming
language that allows for simple and efficient calls to the OVS Database
Management Protocol. Interfacing with OVS is going to be a very large
portion of the project for us, so finding the library is quite the boon. Here
is a quick example of how this library functions:\\

\begin{lstlisting}[caption=Example insert operation using libovsdb]
	// simple insert operation
	insertOp := libovsdb.Operation{
		Op:	  "insert",
		Table:	  "Bridge",
		Row:	  bridge,
		UUIDName: namedUUID,
	}
\end{lstlisting}

The above example can be reused for all major operations in the OVS Database
Management Protocol. Other example operations include select, delete, and
update. Using just the operations listed here, we can accomplish most of the
needed configuration changes within Open vSwitch.

\section{Week by Week Reports}
%just paste in weekly blogs?

\subsection{Background}

Over the Summer, 2016, Matthew worked as a Software Engineering Intern for the
Advanced Systems Engineering (ASE) group within the Open Source Technology
Center (OTC) that is in turn within the Software Services Group (SSG) at Intel.
Matthew's coworkers had a need to integrate Open vSwitch in their cloud
orchestration software (Cloud Integrated Advanced Orchestrator, or Ciao) but did
not have the man hours to contribute time to it. Matthew worked with the team to
propose the project as a Senior Capstone project at Oregon State University.

Matthew identified two other students, Cody Malick and Garrett Smith, as
intelligent hard workers who would benefit the project. Because of this Robert
Nesius, the Intel Engineering Manager serving as our client, requested Matthew,
Cody, and Garrett specifically for this project.

\subsection{Weeks Zero Through Two}

During the first week of class, we all visited Intel in Hillsboro. The principal
engineer in charge of Ciao networking, Manohar Castelino, gave us all an
explanation of Ciao, how the networking works, and what he expects us to
accomplish.

It was also during this time we were provided with five Intel NUCs that would
serve as our local cluster. We found out that we needed to register the MAC
addresses with the university, and communicated this need to Todd Shechter, the
Oregon State University Director of Information Technology.


\subsection{Week Three}

Garrett:

Progress

A couple of weeks ago our team drove up to Intel's campus in Hillsboro to meet
with our clients.
They went over Ciao's network stack with us, explained the project in depth, and
answered questions. In addition to going over Ciao with us, they helped us write
an abstract and problem statement for the project. We submitted the abstract and
problem statement for the project on Thursday afternoon (2016/10/13).

Intel is lending us 5 NUCs (small form factor computers), and a managed Cisco
switch to use as our datacenter for the project. Kevin McGrath is giving us
access to his lab so we have a secure place to set up the hardware. We have the
hardware physically in place in Kevin's lab, but we need wired network access to
install Clear Linux on the NUCs and set up Ciao. We requested network access but
the request has not yet been approved.

Issues

We are still waiting on access to OSU's wired network for the 5 NUCs and the
switch that Intel are loaning us for the project. Network access is required to
install Clear Linux, Ciao, and access the machines remotely. If we are unable to
obtain network access for the hardware on OSU's network we will need to find
somewhere else to house it.

Next Steps

We would like to get the network issues sorted out as soon as possible so we can
finish setting up the NUCs and switch. Once we have network access we will
install Clear Linux, set up Ciao, examine how Ciao's network stack currently
works, and gather performance metrics. The performance metrics we gather will be
used to compare Ciao's current network stack implementation against ours.

Matthew:

Progress

We have made some progress since being put on this project. A couple weeks ago
we all traveled up to Hillsboro to meet with our clients, which turns out to
include a couple Principal Engineers (PEs) who are miles more technical than us.
During this meeting, one of the PEs was kind enough to spend an hour of his time
explaining the more technical aspects of Ciao (I'm assuming I don't have to go
into an explanation of our project here) and what they wanted us to do. Also
during this meeting, we collaborated to write our problem statement due in this
class. The next week we set up the hardware required to run Ciao in a cluster
and began the process to register our hardware with the university so we could
be granted IP addresses.

Issues

Since then we have been trying to jump through various hoops to get ethernet
access on the hardware. This has been a continuing battle that we expect to be
resolved soon. We need internet to use git, remote updates of Ciao and Clear
Linux, and remote access to the cluster. We need wired internet because Clear
Linux is a datacenter OS, and therefore does not even package the common wifi
setup tools, such as iw, wifi-menu, wpa-supplicant, etc.

Next Steps

The next step is to install the Clear Linux OS on the five Intel NUCs in the
cluster. Once that has been done we can set up Ciao and take initial
measurements as a baseline. This baseline will be recorded so we can take
further measurements once we implement our new networking mode using OVS.

Cody:

Progress

We've set up the five NUCs, small Intel computers, that will act as our "cloud."
Once the project is all setup, we will be simulating a production environment
that we can test and deploy our project on. I would like to thank Intel for
providing such a wonderful and high performance testing environment.

While visiting Intel, we also got our problem statement and abstract put
together and approved by the customer. The team has been incredibly helpful, and
efficient at getting us any information we need to get moving on the project.

Issues

The only wall that we've hit is getting the networking capability set up on
campus. Because of the set up on campus, Network Services prevents deployment of
random Ethernet devices. A pragmatic policy to be sure. Knowing all the mischief
that college students get up to, I completely understand why this policy is in
place. Because of this, we need to get approval from Network Services before our
initial deployment of the Ciao infrastructure.

Next Steps

Once the network issues are out of the way, we can move forward with our initial
testing setup with Ciao. Once that's set up, we can define what the requirements
of the project are, and begin designing what exactly needs to be implemented.

Thanks for taking the time to read this, and look for future updates!

\subsection{Week Four}

Cody:

Progress

We had our problem statement reviewed, and we're waiting on the feedback from
Kevin. That should arrive later tonight. We also spent some time getting the
five Intel NUCs (Next Unit of Computing) set up in the lab. Matthew spent a good
amount of time Wednesday getting them imaged with Clear Linux, Intel's flavor of
Linux. We also got the base latex files for the requirements document pushed to
a new branch of the docs repo, "client\_req." We'll be updating that and merge it
once the first version is complete.

We also looked through the basic requirements of the requirements document from
IEEE.

Issues

Our issues with networking continue through this week. Todd, from the
engineering network team, has been incredibly helpful and prompt in his
responses. We plan on getting this issue resolved early next week.

Next Steps

After the network issues are ironed out, we'll get our test environment set up
with Ciao, and get our requirements document written up.

Matthew:

Progress

This week was almost entirely devoted to figuring out networking for our
hardware, 5 Intel NUCs. As mentioned previously, these NUCs will make up the
control node and compute nodes for the cluster we will be running and testing
Ciao on. After several days of debugging and a 22-email chain with the IT
director at OSU, we are still blocked.


Other than debugging the networking with the IT director, we have successfully
installed Clear Linux on all NUCs. This was done by bringing them all home and
installing Clear on my home network. Tonight we will also receive feedback for
our first draft of the Problem Statement. Hopefully we will have the feedback
addressed by tomorrow morning (Friday, October 21st), since I will be driving up
to Intel and will be able to get any changes approved and signed.

Issues

As I mentioned above, the issue with the networking is a big one. Hopefully we
can get this figured out when the IT director returns on Monday.

Next Steps

The next step is to get networking figured out. This is a necessary step because
we need to use the network to ssh into the NUCs, as well as communicate with the
internet for github access and Clear Linux updates.

Garrett:

Progress

Paperwork

Our problem statement was reviewed and returned to us without any feedback. Cody
emailed Kevin asking if we still need to resubmit the problem statement. We do.

Hardware and Software

In the land of NUCs and networks we are still having problems connecting the
NUCs to OSU's network. Todd set us up with two 5 port switches in Kevin's lab,
but we are having problems using them to connect to the network. Clear Linux
requires network access to install, but network access requires us to set a
hostname. We worked around the problem for NUC number 1 by bridging my laptop's
wireless connection to the NUC over ethernet. This was a temporary workaround
that let us install Clear Linux and set a hostname for the NUC. Unfortunately
once the NUC was configured, we were still unable to obtain an IP address from
the OSU network using the switches Todd provided. Several emails with Todd later
we had NUC number 1 plugged into port 25 of the big 25 port switch in the middle
of the lab, and connected to the network. This means the NUCs are configured
correctly, and the issue likes somewhere in the network. At this point Todd
emailed us that he was out of the office until next week, putting our network
setup on pause. In the meantime, Matthew took the other 4 NUCs home and
installed Clear Linux on them there.

Issues

As I said above, we are having problems getting the NUCs connected to OSU's
network. Todd has been very helpful, but between email latency, and school and
work schedules consuming chunks of the day the networking issues are taking a
long time to resolve.

Next Steps

We need to work with Todd next week to get network access for the NUCs in
Kevin's lab. If that goes through we will set up Ciao on the NUCs. Our problem
statement does not require any edits so we will resubmit it. The requirements
document is due on the 28th so we will be working on that.

\subsection{Week Five}

Matthew:

Progress

This week we spent most of our time writing the rough draft for the requirements
document. We have turned it in and it is definitely a good start. We will polish
it over the next week to turn in next Friday.

Another big update for this week is that we have networking! I had another email
conversation with Todd Shechter, who was at a loss as to why we could not access
the network on our NUCs. He granted us access to the HP switch we had
successfully connected via in the past. On Thursday, I went to move all our NUCs
to the new switch. On a hunch I decided to try out the network one last time.
They all worked! Our hardware is now set up and hopefully we will be able to get
to work soon.

Issues

No issues currently, except for an increased workload from other classes. This
being week 5, we all had midterms to contend with as well. Hopefully things calm
down for a couple weeks.

Next Steps

The next step is to finish our requirements document and get it signed next
week. We should also get Ciao installed on our NUCs and our cluster set up.

Cody:

Progress

This week we submitted our finalized and signed copy of the problem statement.
With that out of the way, we switched gears into the requirements document.
We've worked through a rough draft, and will be fleshing out the details over
the weekend. The requirements document has been good to read through, even in
it's current form. It's given me a little more insight into what exactly the
project is, and how we're going to go about design and implementation. Overall,
I'm glad we're doing one.

This week we also got our networking issues resolved! This is great news as it
will now allow us to completely configure and update our NUCs as needed without
having to be in the lab.

We met with Frank on Tuesday and gave him the update on where we're at
currently.

Issues

No notable issues this week. We're moving forward at a good clip.

Next Steps

We'll be completing the requirements document this coming week, as well as
working on configuring the current implementation of Ciao on our five NUCs.
We're also expecting some books from Intel that should give us more insight and
information into software defined networks.

Garrett:

Progress

The network issues have been solved. Matthew was going to move the NUCs from the
small Cisco switches Todd provided us to the large HP switch in the middle of
the room when Matthew discovered that the NUCs were able to connect to the
network using the small switches. Many thanks to Todd Shechter for his
assistance as we worked through the network issues. All five NUCs have access to
OSU's wired network. Now we can access them remotely and configure Ciao.

In documentation land we submitted the rough draft of our requirements document.

Issues

We didn't have any issues with the Capstone project its self. Between midterms,
homework for other classes, and attending the career fair I was very busy which
left less time to work on capstone.

Next Steps

The requirements document is due next Friday (2016/11/4) so we will be working
on it next week. We will also set up Ciao in its current form on the NUCs.

\subsection{Week Six}

Matthew:

Progress

This week we finished writing the requirements document that was due today.
After the rough draft we turned in last week we continued working on it
ourselves until Tuesday. On Wednesday Frank emailed us some suggestions (thanks
Frank) and we addressed those right away. We got the document signed that
afternoon by Rob Nesius, our client at Intel. We will soon be receiving a book
from Intel on SDN basics that will help us get started with implementation. We
would have liked to be starting on implementation by now but the
writing-intensive course load has tied up our time a bit.

Issues

Our workloads continue to be heavy, but that won't change. We are doing fine,
completing our assignments on time, and have no immediate blockers.

Cody:

Progress

This week we submitted our finalized and signed copy of the requirements
document. It's been a rather busy week, but our team has been working well
together! We laid out a tentative plan for our next steps, which involve the
design document and getting the books we need to being working with Software
Defined Networks.

We met with Frank on Tuesday and gave him the update on where we're at
currently.

Issues

No notable issues this week. We're moving forward at a good clip.

Next Steps

The next major steps are the tech review documents due next week, and begin work
on the design document. We're supposed to be receiving a book from Intel soon as
well which should help us get started.

Garrett:

Progress

The final version of our requirements document has been signed and submitted. I
think we would like to be furhter along in setting up the project and starting
the implementation, but the writing requirements of the class have kept us busy.

Issues

There haven't been any issues with the class or eachother. We are making
progress.

\subsection{Week Seven}

Matthew:

Progress

This week we started working on the technical review document due next Monday.
This document outlines nine different components of our system. For each
component we are exploring three different technologies that could be used to
implement the component. Since our project is implementing a component of a
larger system, it was difficult for us to come up with nine components and three
technologies each (twenty-seven different options). We spent much of our week
working together to figure out how to split the project up. We will all be
working on our portions of the document over the weekend.

Issues

The biggest issue with the technical review document is that our system is a
single component of a larger system, and is therefore difficult to split up into
nine different parts.

Cody:

Progress

We've moved forward on our Tech Review paper. We have the outline detailed, and
which person is working on each component. It was a little difficult for us to
break up the project into the individual components. The main reason for the
difficulty was that we are implementing a small subcomponent of a very specific
system. We ended up figuring it out, and we'll all be writing our pieces this
weekend.

Issues

No notable issues this week.

Garrett:

Progress

This week we started work on the technology review. This involved splitting our
project into 9 parts and assigning 3 to each member of the team. The goal of our
project is to implement part of a larger system, so finding 9 smaller parts was
difficult. After meeting two separate times we were able to figure out how to
split the project into 9 parts. Each of us has 3 parts to work on now, and we
will be completing those over the weekend.

Issues

There were no issues this week.

\subsection{Week Eight}

Matthew:

Progress

Over the last weekend we completed our technology review document, coming in at
just over 16 pages total. Unlike several other groups we managed to complete and
turn in our technology review on time. We are all satisfied with the result.
Since then we have shifted to working on the design document that is due
December 4th. This will be the largest document we produce this term and mostly
marks the end of the documentation phase. The one document due after the design
document is a term status report, which is mostly a combination of all the
weekly updates we have written to give a coherent picture of our progress so far
this year.

Issues

No issues currently.

Cody:

Progress

It was a busy week! The team managed to get the Tech Review done on time. It
ended up being quite a bit of work for each of the components the team was
working on. My favorite section of the tech review was covering the network
protocols section. I ended up choosing SSNTP, Simple Secure Network Transfer
Protocol. The protocol was quite interesting as it was custom tailored to our
project's needs. It was created by one of the principal engineers at Intel, who
was also the creator of the project. It builds off of the security and handshake
protocols that SSL and TLS established, but reduces the protocol overhead by
only needing to do the initial handshake once. There were other sections, but
they weren't nearly as interesting.

Issues

No issues this week, we're moving along very nicely.

Garrett:

Progress

This past Monday (2016/11/14) we submitted the tech review document. It was a
lot of work, but we got it in on time unlike many other capstone groups. I
researched software switch options, network latency tools, and network
throughput tools. For the software switch, Open vSwitch, the software switch
recommended by Intel, was the best option. Nothing else I found came close in
terms of features and portability. There were a lot of options for tools that
measure network throughput and latency. I had a lot of choice when it came to
choosing tools so I spent a lot of time reading recommendations, looking at all
of the options, and comparing them to determine what would work best for us.
Even though using the tools isn't part of the assignment, I was curious about
how the networking tools I was researching worked so I set up and used some of
them to get more of an idea of what I would be doing.

Issues

We didn't have any issues this week.

\subsection{Week Nine}

Matthew:

Progress

This was a short week with Thursday and Friday given over to the Thanksgiving
holiday. We have transitioned to using the template requested by Kirsten for the
design document and are ready to start with writing the document and preparing
for our final presentation.

Issues

No issues currently.

Cody:

Progress

We had a short week this week. We've started on the design document. We will be
making the transition over to the template provided by Kirsten for this
assignment. In the last few documents, we had not used the exact template she
provided. We will be doing so for the final documents this term.

Our team also got together and talked about how we were going to execute the
final presentation video for a few minutes this week.

Issues

No issues this week. One thing I could use more of, however, is time!

Garrett:

Progress

This week was short due to the Thanksgiving holiday. We met and discussed how we
will be completing the design document and progress report. We have the design
document laid out using the template Kirsten wants us to use. We received
feedback from Kirsten on the tech review. Some of the feedback was helpful
generally like watching out for spacing issues.

In more of a personal progress report, I have been reading through the book on
Software Defined Networking. It is very insightful, and I wish I had it at the
start of the term.

Issues

No issues this week. The holiday is cutting into project time, but that is
unavoidable.

\subsection{Week Ten}

Matthew:

Progress

This week we focused on the design document due Friday. We spent time
researching design strategies and writing up our plan to execute. During this
research we found a very useful Go library that interfaces with Open vSwitch.
This library will simplify our implementation, allowing us more time to do
network performance testing, which the client is very interested in.

Issues

No issues this week, we all worked hard and got the document done.

Cody:

Progress

This week we wrote the design document up, and got it signed just in time for
submission on Friday. Writing the design document was helpful, as it forced the
team to read through what libaries and interfaces in those libraries we need to
use. As well as getting a somewhat formal design plan in place, it gave us a
more informed overview of what peices of the project need to get done.

Issues

No issues this week! Everything is moving smoothly.

Garrett:

Progress

This week we finished the design document. We each wrote up part of the design,
then spent time editing and revising the document. We turned in the signed copy
by the due date and did not need to turn in an unsigned copy. It feels good to
have the first iteration of the design document complete because we are now in a
position where we can work on implementing the project over winter break.

Issues

No major issues this week.

\section{Fall Term Retrospective}

\begin{center}
	\begin{tabular}{| p{0.05\linewidth} | p{0.3\linewidth} | p{0.3\linewidth} | p{0.3\linewidth} |}\hline
		Week & Positives & Deltas & Actions \\ \hline
		3 & this is a test  & beep boop bap & Hack the world \\ \hline
	\end{tabular}
\end{center}

\end{document}
