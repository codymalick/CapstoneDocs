\documentclass[10pt,onecolumn,journal,draftclsnofoot]{IEEEtran}
\usepackage[margin=0.75in]{geometry}
\usepackage{listings}
\usepackage{color}
\usepackage{longtable}
\usepackage{graphicx}
\usepackage{float}
\usepackage{tabu}
\usepackage{enumitem}
\usepackage{courier}
\usepackage{hyperref}
\usepackage{parskip}
\definecolor{dkgreen}{rgb}{0,0.6,0}
\definecolor{gray}{rgb}{0.5,0.5,0.5}
\definecolor{mauve}{rgb}{0.58,0,0.82}

%Subsection headers to Arabic numerals
%\renewcommand\thesection{\arabic{section}}
%\renewcommand\thesubsection{\thesection.\arabic{subsection}}
%\renewcommand\thesubsubsection{\thesubsection.\arabic{subsubsection}}

%Section headers to Arabic numerals
%\renewcommand\thesectiondis{\arabic{section}}
%\renewcommand\thesubsectiondis{\thesectiondis.\arabic{subsection}}
%\renewcommand\thesubsubsectiondis{\thesubsectiondis.\arabic{subsubsection}}

%Remove numbering from the bibliography section

\lstset{frame=none,
language=C,
columns=flexible,
numberstyle=\tiny\color{gray},
keywordstyle=\color{blue},
commentstyle=\color{dkgreen},
stringstyle=\color{mauve},
breaklines=true,
breakatwhitespace=true,
tabsize=4,
showstringspaces=false,
basicstyle=\ttfamily
}

\setlength{\parindent}{0cm}

\begin{document}

\begin{titlepage}
	\title{Intel Cloud Orchestration Networking\\ Fall Progress Report}
	%\author{Matthew~Johnson,~Cody~Malick,~and~Garrett~Smith\\
	%	Team 51, Cloud Orchestra}
	\date{\today}
	\markboth{Senior Design, CS 461, Fall 2016}{}
	\maketitle
	\vspace{4cm}
	\begin{abstract}
		\noindent This document outlines the progress of the Cloud
		Orchestration Networking project over the entirety of the fall
		term. It contains a short description of the project's purposes
		and goals, current progress, current issues, and any solutions
		to those issues. It also contains a week by week retrospective
		for all ten weeks of fall term. \end{abstract}

\end{titlepage}
\tableofcontents
\clearpage

\section{Project Goals}

Our project is to first switch the Linux-created GRE tunnel implementation in
Ciao to use GRE tunnels created by Open vSwitch. From that point we will switch
the actual tunneling implementation from GRE to VxLAN/nvGRE based on performance
measurements of each on data center networking cards. After this is completed, a
stretch goal is to replace Linux bridges with Open vSwitch switch instances.

\section{Purpose}

The current implementation of Ciao tightly integrates software defined
networking principles to leverage a limited local awareness of just enough of
the global cloud's state. Tenant overlay networks are used to overcome
traditional hardware networking challenges by using a distributed, stateless,
self-configuring network topology running over dedicated network software
appliances. This design is achieved using Linux-native Global Routing
Encapsulation (GRE) tunnels and Linux bridges, and scales well in an environment
of a few hundred nodes.

While this initial network implementation in Ciao satisfies current simple
networking needs in Ciao, all innovation around software defined networks has
shifted to the Open vSwitch (OVS) framework. Moving Ciao to OVS will allow
leverage of packet acceleration frameworks like the Data Plane Development Kit
(DPDK) as well as provide support for multiple tunneling protocols such as VxLAN
and nvGRE. VxLAN and nvGRE are equal cost multipath routing (ECMP) friendly,
which could increase network performance overall.

\section{Current Progress}

\section{Week by Week Reports}
%just paste in weekly blogs?

\subsection{Weeks Zero Through Two}

\subsection{Week Three}

\subsection{Week Four}

\subsection{Week Five}

\subsection{Week Six}

\subsection{Week Seven}

\subsection{Week Eight}

\subsection{Week Nine}

\subsection{Week Ten}

\section{Fall Term Retrospective}

\begin{center}
	\begin{tabular}{| p{0.05\linewidth} | p{0.3\linewidth} | p{0.3\linewidth} | p{0.3\linewidth} |}\hline
		Week & Positives & Deltas & Actions \\ \hline
		3 & this is a test  & beep boop bap & Hack the world \\ \hline
	\end{tabular}
\end{center}

\end{document}
