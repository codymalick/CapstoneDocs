\documentclass[10pt,letterpaper,onecolumn,draftclsnofoot]{IEEEtran}
\usepackage[margin=0.75in]{geometry}
\usepackage{listings}
\usepackage{color}
\usepackage{longtable}
\usepackage{tabu}
\definecolor{dkgreen}{rgb}{0,0.6,0}
\definecolor{gray}{rgb}{0.5,0.5,0.5}
\definecolor{mauve}{rgb}{0.58,0,0.82}

\lstset{frame=tb,
  language=C,
  columns=flexible,
  numberstyle=\tiny\color{gray},
  keywordstyle=\color{blue},
  commentstyle=\color{dkgreen},
  stringstyle=\color{mauve},
  breaklines=true,
  breakatwhitespace=true,
  tabsize=4
}

\begin{document}
\begin{titlepage}
  \title{CS 461 - Fall 2016 - Project Abstract}
  \author{Matthew Johnson, Garrett Smith, Cody Malick}
  \date{October 4, 2016}
  \maketitle
  \vspace{4cm}
  \begin{abstract}
  	\noindent Intel's Cloud Integrated Advanced Orchestrator (ciao) cloud
	orchestration platform currently uses Linux bridges and Generic Routing
	Encapsulation (GRE) tunnels to implement a software defined network
	(SDN). While the current implementation allows for scalability above
	typical designs, it lacks compatibility with modern tunneling protocols
	such as VxLAN and nvGRE. The proposed solution is to switch the
	implementation from Linux GRE tunnels to Open vSwitch (OVS) Tunnels and
	Linux bridges to Open vSwitch. This will allow Ciao to leverage packet
	acceleration frameworks such as the Data Plane Development Kit (DPDK)
	and provide needed compatibility.
  \end{abstract}
\end{titlepage}
\end{document}
