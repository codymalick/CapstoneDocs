\documentclass[pdf]{beamer}
\usepackage{listings}
\usepackage{color}

\definecolor{dkgreen}{rgb}{0,0.6,0}
\definecolor{gray}{rgb}{0.5,0.5,0.5}
\definecolor{mauve}{rgb}{0.58,0,0.82}

\lstset{frame=none,
language=C,
columns=flexible,
numberstyle=\tiny\color{gray},
keywordstyle=\color{blue},
commentstyle=\color{dkgreen},
stringstyle=\color{mauve},
breaklines=true,
breakatwhitespace=true,
tabsize=4,
showstringspaces=false,
basicstyle=\ttfamily
}

\mode<presentation>
{
	\usetheme{default}      % or try Darmstadt, Madrid, Warsaw, ...
	\usecolortheme{default} % or try albatross, beaver, crane, ...
	\usefonttheme{default}  % or try serif, structurebold, ...
	\setbeamertemplate{navigation symbols}{}
	\setbeamertemplate{caption}[numbered]
}

\begin{document}
\title{Intel Cloud Orchestration Networking Design and Goals}
\author{Matthew Johnson, Cody Malick, and Garrett Smith}
\date{7 December, 2016}

\maketitle
\begin{frame}
	\frametitle{Table of Contents}
	\tableofcontents
\end{frame}

\begin{frame}
	\frametitle{Context}
	\section{Context}
	Context goes here
\end{frame}
\begin{frame}
	\frametitle{Project Goals}
	Goals go here
\end{frame}

\begin{frame}
	\frametitle{Project Design}
	\section{Open vSwitch Database Management Protocol}
	Open vSwitch uses a database to manage configuration while running.
	The configuration can be updated on the fly by accessing its management
	protocol using the Open vSwitch Database Management Protocol, defined
	in RFC 7047\cite{rfc7047}
\end{frame}

\begin{frame}[fragile]
	\frametitle{Libovsdb}
	Libovsdb is an open source library that provides a Go programming
	language wrapper around the OVS Database Management Protocol. Here is
	an example:\cite{gosample} \\

\begin{lstlisting}[caption=Example insert operation using libovsdb]
	// simple insert operation
	insertOp := libovsdb.Operation{
	    Op:	  "insert",
	    Table:	  "Bridge",
	    Row:	  bridge,
	    UUIDName: namedUUID,
	}

\end{lstlisting}
\end{frame}
\begin{frame}
	\frametitle{nvGRE and VxLAN}
	Two alternative tunneling protocols to replace GRE, the current
	protocol used by Ciao.\\
	nvGRE: Network virtualization standard created in tandem by HP, Dell,
	and Intel
	VxLAN: Network virtualization standard created in tandem by Cisco,
	VMware, Citrix, and Redhat

	Overall performance and overhead are similar on paper, will require
	testing to see which is the best fit for our implementation
\end{frame}
\begin{frame}
	\frametitle{Stumbling Blocks}
	Stumbling blocks go here
\end{frame}

\begin{frame}
\end{frame}
\begin{frame}
	\frametitle{References}
		\bibliographystyle{IEEEtran}
		\bibliography{pres}
	\end{frame}
\end{document}
